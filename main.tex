%%----------------------------------------------------------------------------------------
%	PACKAGES AND OTHER DOCUMENT CONFIGURATIONS
%----------------------------------------------------------------------------------------

\documentclass[
		twoside,openright,titlepage,numbers=noenddot,headinclude,%1headlines,
                footinclude=true,cleardoublepage=empty,
                BCOR=10mm,paper=a4,fontsize=10pt, % Binding correction, paper type and font size
                ngerman,american, % Languages
                ]{scrreprt}

% Includes the file which contains all the document configurations and packages - make sure to edit this file
\input{classicthesis-config}
\usepackage{semantic}
\usepackage{paralist}
\usepackage{etoolbox}
\usepackage{stmaryrd}
\usepackage{amssymb}
\usepackage{amsmath}
\usepackage{amsthm}
%\usepackage{catchfilebetweentags}

\newcounter{mythmcounter}
\newcounter{mydefcounter}
\newcounter{myexpcounter}
\newtheorem{example}[myexpcounter]{Example}
\newtheorem{remark}[mythmcounter]{Remark}
\newtheorem{definition}[mydefcounter]{Definition}
\newtheorem{theorem}[mythmcounter]{Theorem}
\newtheorem{lemma}[mythmcounter]{Lemma}
\newtheorem{proposition}[mythmcounter]{Proposition}
\newtheorem*{lemma*}{Lemma}

% Formatting for source code & types
\definecolor{cmtclr}{rgb}{0.0,0.6,0.0}
\definecolor{kvdclr}{rgb}{0.0,0.0,0.6}
\definecolor{strclr}{rgb}{0.5,0.1,0.0}
\definecolor{prepclr}{rgb}{0.0,0.0,0.0}

\newcommand{\sep}[0]{\; | \;}
\newcommand{\kvd}[1]{\textnormal{\textcolor{kvdclr}{\ttfamily #1}}}
\newcommand{\str}[1]{\textnormal{\textcolor{strclr}{\ttfamily "#1"}}}
\newcommand{\strf}[1]{\textnormal{\textcolor{strclr}{\ttfamily #1}}}
\newcommand{\prepk}[1]{\textnormal{\textcolor{prepclr}{\bfseries\sffamily #1}}}
\newcommand{\prepi}[1]{\textnormal{\textcolor{prepclr}{\ttfamily #1}}}
\newcommand{\ident}[1]{\textnormal{\sffamily #1}}
\newcommand{\cmt}[1]{\textit{\sffamily\textcolor{cmtclr}{#1}}}

\newcommand{\ctyp}[2]{C^{#1}#2}
\newcommand{\mtyp}[2]{M^{#1}#2}
\newcommand{\subst}[3]{#1[#2 \leftarrow #3]}

% Typing rule and typing statement (single line)
\newcommand{\tyrule}[3]{ \inference[\footnotesize{(\emph{#1})}~~]{#2}{#3} }
\newcommand{\tyruler}[3]{ \inference{#2}{#3}~~\footnotesize{(\emph{#1})} }

% Coeffect algebra
\definecolor{aclr}{rgb}{0.6,0.3,0.0}
\newcommand{\aclrd}[1]{\textcolor{aclr}{#1}}

% Coeffect algebra
\definecolor{cclr}{rgb}{0.0,0.5,0.0}
\newcommand{\cclrd}[1]{\textcolor{cclr}{#1}}

% Red color
\definecolor{rclr}{rgb}{0.75,0.0,0.0}
\newcommand{\rclrd}[1]{\textcolor{rclr}{#1}}

\newcommand{\czero}{ \textcolor{cclr}{ \mathsf{ign} } }
\newcommand{\cunit}{ \textcolor{cclr}{ \mathsf{use} } }
\newcommand{\cseq}{ \textcolor{cclr}{ \circledast }}
\newcommand{\cpar}{ \textcolor{cclr}{\oplus} }
\newcommand{\czip}{ \textcolor{cclr}{\wedge} } % only used by flat coeffects
\newcommand{\cleq}{ \textcolor{cclr}{\leq} }
\newcommand{\cgeq}{ \textcolor{cclr}{\geq} }

\newcommand{\atimes}{ \textcolor{aclr}{\times} }
\newcommand{\aseq}{ \textcolor{aclr}{\circledast}} % Should be the same as \cseq
\newcommand{\aparstr}{ \textcolor{aclr}{\oplus}} % Should be the same as \cseq

% Unified

\newcommand{\SH}{ {\textcolor{sclr}{S}} }
\newcommand{\SP}{ {\textcolor{sclr}{P}} }
\newcommand{\SHP}{ {\textcolor{sclr}{S\triangleleft P}} }
\newcommand{\sempty}{ {\textcolor{sclr}{ \hat{0}}} }
\newcommand{\sunit}{ {\textcolor{sclr}{\hat{1}}} }
\newcommand{\stimes}{ {\textcolor{sclr}{\diamond}} }
\newcommand{\sflat}{\star}

\newcommand{\azero}{ \textcolor{aclr}{\bot} }
\newcommand{\aunit}{ \alift{\cunit} }
\newcommand{\aweak}{ \alift{\czero} }

\newcommand{\azip}{ \textcolor{aclr}{ \mathbin{\rotatebox[origin=c]{-90}{$\ltimes$}} } }
\newcommand{\apar}{ \textcolor{aclr}{ \mathbin{\rotatebox[origin=c]{90}{$\ltimes$}} } }


\newcommand{\xtimes}{\hat{\times}}

\newcommand{\ctx}{\textit{ctx}}


% Ordinary context where colouring is done by hand
\newcommand{\coctx}[2]{ #1 \,\text{\scriptsize @}\, {{\text{$#2$}}}  }

% Simple coeffect systems in the introduction
\newcommand{\cons}{\hspace{-0.15em}\times\hspace{-0.25em}} %\sctxclrd{+\hspace{-0.55em}+}}

\newcommand{\alift}[1]{ \textcolor{aclr}{\langle} #1 \textcolor{aclr}{\rangle}}

\definecolor{sclr}{rgb}{0.2,0,0.8}
\newcommand{\sclrd}[1]{ \textcolor{sclr}{#1} }

\newcommand{\C}{ {\textcolor{cclr}{\mathcal{C}}} }
\newcommand{\M}{ {\textcolor{cclr}{\mathcal{M}}} }

\newcommand{\alen}[1]{{\aclrd{\textit{len}(}#1\aclrd{)}}}
\newcommand{\slen}[1]{{\sclrd{\textit{len}(}#1\sclrd{)}}}
\newcommand{\xlen}[1]{{\textit{len}(#1)}}

\newcommand{\reduce}{\longrightarrow}


\newcommand{\restr}[2]{#1|_{#2}}
\newcommand{\narrow}[1]{\hspace{-0.5em} #1 \hspace{-0.65em}~}


\definecolor{todoclr}{rgb}{0.8,0,0.5}
\newcommand{\todo}[1]{\textcolor{todoclr}{ \begin{quotation} \noindent\textbf{TODO:} #1 \end{quotation} }}

\newcommand{\sem}[1]{\llbracket #1 \rrbracket}
\newcommand{\catc}{\mathcal{C}}
\newcommand{\idf}[1]{ {\textnormal{\sffamily id}}_{#1} }


\newcommand{\figcaption}[1]{
  \vspace{0.5em}
  \noindent\makebox[\linewidth]{\rule{\textwidth}{0.5pt}}
  \vspace{-2em}
  \caption{#1}
}

% For definining semantics of things over a typing derivation
\newcommand{\semdef}[2]{
\begin{array}{rcl}
  ~ & ~ & ~ \\ \cline{1-1}\cline{3-3}
  \sem{#1} & = & #2 \\
\end{array}
}
\newcommand{\semdeff}[4]{
\begin{array}{rcl}
  \sem{#1} & = & #3 \\ \cline{1-1}\cline{3-3}
  \sem{#2} & = & #4 \\
\end{array}
}
\newcommand{\mdeff}[4]{
\begin{array}{rcl}
  #1 & = & #3 \\ \cline{1-1}\cline{3-3}
  #2 & = & #4 \\
\end{array}
}
\newcommand{\emdeff}[4]{
\begin{array}{rcl}
  \sem{#1} & = & #3 \\ \cline{1-1}\cline{3-3}
  #2 & = & #4 \\
\end{array}
}
\newcommand{\semdefff}[6]{
\begin{array}{rcl}
  \sem{#1} & = & #4 \\
  \sem{#2} & = & #5 \\ \cline{1-1}\cline{3-3}
  \sem{#3} & = & #6 \\
\end{array}
}
\newcommand{\emdefff}[6]{
\begin{array}{rcl}
  \sem{#1} & = & #4 \\
  \sem{#2} & = & #5 \\ \cline{1-1}\cline{3-3}
  #3 & = & #6 \\
\end{array}
}

% Half line spacing around align
\newcommand{\zerodisplayskips}{%
  \setlength{\abovedisplayskip}{0.5em}
  \setlength{\belowdisplayskip}{0.5em}
  \setlength{\abovedisplayshortskip}{0.5em}
  \setlength{\belowdisplayshortskip}{0.5em}}
\appto{\normalsize}{\zerodisplayskips}
\appto{\small}{\zerodisplayskips}
\appto{\footnotesize}{\zerodisplayskips}

% half line spacing around compactitem
\setlength{\pltopsep}{0.5em}
\setlength{\plitemsep}{0.3em}


\begin{document}

\frenchspacing % Reduces space after periods to make text more compact

\raggedbottom % Makes all pages the height of the text on that page

\selectlanguage{american} % Select your default language - e.g. american or ngerman

%\renewcommand*{\bibname}{new name} % Uncomment to change the name of the bibliography
%\setbibpreamble{} % Uncomment to include a preamble to the bibliography - some text before the reference list starts

\pagenumbering{roman} % Roman page numbering prior to the start of the thesis content (i, ii, iii, etc)

\pagestyle{plain} % Suppress headers for the pre-content pages

%----------------------------------------------------------------------------------------
%	PRE-CONTENT THESIS PAGES
%----------------------------------------------------------------------------------------

%% Title Page

\begin{titlepage}

\begin{addmargin}[-1cm]{-3cm}
\begin{center}
\large

\hfill
\vfill

\begingroup
\color{Maroon}\spacedallcaps{Context-aware programming languages} \\ \bigskip 
\endgroup

\spacedlowsmallcaps{Tomas Petricek} 

\vfill
\vspace{7em}

December 2014 \\
\medskip
Clare Hall, University of Cambridge

\vspace{2em}


{\small This dissertation is submitted for the degree of Doctor of Philosophy.}


\vfill

\end{center}
\end{addmargin}

\end{titlepage} % Main title page

%\cleardoublepage% Declaration

\refstepcounter{dummy}
\pdfbookmark[0]{Declaration}{declaration} % Bookmark name visible in a PDF viewer

\chapter*{Declaration} % Declaration section text

\thispagestyle{empty}

This dissertation is my own work and includes nothing which is the outcome of work done in collaboration
with others except where specifically indicated in the text.
This dissertation does not exceed the regulation length of 60,000 words, including tables and footnotes. % Publications from the thesis page

%\cleardoublepage% Abstract

\pdfbookmark[1]{Abstract}{Abstract} % Bookmark name visible in a PDF viewer

\begingroup
\let\clearpage\relax
\let\cleardoublepage\relax
\let\cleardoublepage\relax

\chapter*{Abstract} % Abstract name

Static type systems are the most common form of program verification. In their most basic form, type
systems guarantee that a computation always returns a value of some type. The work on monads
and effect systems extends type systems to guarantee properties about \emph{effectful 
computations} that perform side-effects like non-termination, I/O operations or exceptions.

In this thesis we extend type systems to guarantee properties about \emph{context-dependent
computations}. Such computations are increasingly common in modern software. For example, applications
accessing data need to guarantee that the right data is available and distributed systems need to 
guarantee that resources are available on specific nodes.

We develop three \emph{coeffect} type systems that are capable of capturing different context-dependent
properties. All three systems are based on the unifying semantic structure called \emph{indexed
comonads}, but they use it differently. 

Our \emph{flat coeffect calculus} tracks one annotation about the entire context. It can be used to
keep a set of required resources or a set of supported platforms. Our \emph{structural coeffect 
calculus} tracks one annotation for each variable. It can be used to track liveness of variables
or number of past values needed in data-flow computations. Finally, our \emph{coeffect meta-language}
embeds contextual values directly into a language. It can be used to capture context-dependent 
properties of meta-programming languages and mobile values in distributed computation.

\endgroup			

\vfill % Abstract page

\pagestyle{scrheadings} % Show chapter titles as headings

\cleardoublepage\include{text/Contents} % Contents, list of figures/tables/listings and acronyms

\pagenumbering{arabic} % Arabic page numbering for thesis content (1, 2, 3, etc)
%\setcounter{page}{90} % Uncomment to manually start the page counter at an arbitrary value (for example if you wish to count the pre-content pages in the page count)

\cleardoublepage % Avoids problems with pdfbookmark

%----------------------------------------------------------------------------------------
%	THESIS CONTENT - CHAPTERS
%----------------------------------------------------------------------------------------

% \ctparttext{You can put some informational part preamble text here. Illo principalmente su nos. Non message \emph{occidental} angloromanic da. Debitas effortio simplificate sia se, auxiliar summarios da que, se avantiate publicationes via. Pan in terra summarios, capital interlingua se que. Al via multo esser specimen, campo responder que da. Le usate medical addresses pro, europa origine sanctificate nos se.} % Text on the Part 1 page describing  the content in Part 1

% \part{Some Kind of Manual} % First part of the thesis

% \chapter{Introduction} 
\label{ch:introduction} 

%----------------------------------------------------------------------------------------

Modern software applications behave differently depending on the environment or context 
in which they execute. They often run in increasingly rich environments that provide 
resources (e.g. database or GPS sensor) and are gradually more diverse (e.g. multiple 
versions of different mobile platforms). Web applications are split between client, server 
and mobile components; mobile applications must be aware of the context and of the platform 
while the ``internet of things'' makes the environments even more heterogeneous; 
applications that access rich data sources need to propagate security policies and provenance 
information about the data.

Writing such context-aware applications is a fundamental problem of modern software 
engineering. The state of the art relies on ad-hoc approaches -- using hand-written conditions 
or pre-processors for conditional compilation. Common problems that developers face include: 

\begin{itemize}
\item \textbf{System capabilities.} When writing code that is cross-compiled to multiple 
  targets (e.g. SQL, CUDA or JavaScript) the compilation often occurs at runtime and developers 
  have no guarantee that it will succeed until the program is executed.
  
\item \textbf{Platform versioning.} When developing application for multiple versions of a 
  system (e.g. Android), developers rely on lazy loading at runtime or use conditional compilation 
  using \ident{\#if}. The former delays errors to runtime, while the latter requires building all 
  possible configurations to discover simple compile-time errors.
  
\item \textbf{Resources \& data availability.} When programming applications that access 
  resources or data provided by the environment (e.g. specific database table, GPS sensor), 
  the program typically performs dynamic check for the resource availability. However, this 
  is not checked in any way -- we have no easy way to tell what happens when the resource is 
  not available (e.g. is there a fallback strategy or not?)

\item \textbf{Meta-programming.} In distributed client-server programming, we often need to 
  manipulate code using meta-programming techniques on the server and then pass a closed 
  expression to client, translated to another language (JavaScript). We need to be able to 
  guarantee that such expressions are well-formed before they are translated.
\end{itemize}

The issues discussed above pose an important problem to current and future state of computing.
In this thesis, we develop a unifying mathematical model that allows us to capture the properties
discussed above. We use the model in three different ways to capture different notions of 
context-dependence.

In the rest of the chapter, we first look at the distinguishing factor of two of the presented
coeffect systems (how they differ from other existing work). Then we look at a number of concrete 
problems that are discussed in greater details in later chapters. We use the examples to demonstrate 
the real-world problems that motivate our work. (Theoretical and other motivations are discussed 
later in Chapter~\ref{ch:related-work}).

%===================================================================================================

\section{Context and lambda abstraction}

Our work on context-aware programming languages connects two directions in existing research on the 
theory of programming languages. On one side, effect systems \cite{effects-gifford} and monadic 
computations \cite{monad-notions,monads-effects-marriage} provide a detailed and established method 
for tracking what effects programs have -- that is, how they affect the environment where they execute. 
On the other side, the work on comonadic notions of computations \cite{comonads-notions} shows how 
to use the mathematical dual of monads -- comonads -- to give categorical semantics of 
context-dependent computations.

Effect systems introduced track actions such as memory operations or communication. They are described 
by typing judgments of a form $\Gamma \vdash e : \alpha, \sigma$ where $\Gamma$ is the context of a 
program (typically available variables), $e$ is the expression (program) itself, $\alpha$ is the type of 
values returned by the program (e.g. integer or boolean) and $\sigma$ is a set of possible effects. The 
judgment states that, given the context $\Gamma$, an expression has a type $\alpha$ and can only perform 
effects specified by the set $\sigma$. Wadler and Thiemann \cite{monads-effects-marriage} explain how 
this shapes effect analysis of a lambda abstraction -- that is, how effect systems analyze the effects 
associated with a definition of a function:
%
\begin{quote} 
\emph{In the rule for abstraction, the effect is empty because evaluation immediately
returns the function, with no side effects. The effect on the function arrow
is the same as the effect for the function body, because applying the function will
have the same side effects as evaluating the body.}
\end{quote}
%
This means that, when a programmer defines a function, the system records that executing the 
function will perform the effects of the body of the function. However, simply defining a function
is an effect-free computation. Tate \cite{effects-producer-semantics} calls such effect systems
\emph{producer} effect systems and generalizes the idea of function to more general ``thunking'':
%
\begin{quote}
\emph{We will define an effect as a producer effect if all computations with that effect 
can be thunked as "pure" computations for a domain-specific notion of purity.}
\end{quote}
%
In contrast to the static analysis of (producer) effect systems, the analysis of 
\emph{context-dependence} does not match this pattern. In the systems we consider, lambda 
abstraction places requirements on both the \emph{call-site} (latent requirements) and the 
\emph{declaration-site} (immediate requirements), resulting in different syntactic properties. 
We informally discuss three examples first that demonstrate how contextual requirements propagate. 

%===================================================================================================

\section{Why coeffects matter}

This section gives three examples of context-dependent computations whose properties can be captured
by the tree calculi presented in this thesis. We look at an example of the \emph{flat coeffect calculus} 
(Chapter~\ref{ch:flat-coeffects}), \emph{structural coeffect calculus} (Chapter~\ref{ch:structural-coeffects})
and \emph{coeffect meta-language} (Chapter~\ref{ch:coeffect-metalanguage}).

%---------------------------------------------------------------------------------------------------

\subsection{Flat coeffect calculus}

The flat coeffect calculus associates a single piece of information with the context. As an example,
we look at a simple distributed programming language that includes the concept of \emph{resources}.
A resource may be accessed using a special construct $\kvd{access}~\ident{Res}$. The following 
example shows a function taht lists recent events -- it accesses the resource \ident{News} 
(representing a database) and a resource \ident{Clock} (with the current time):
%
\begin{equation*}
\begin{array}{l}
\kvd{let}~\ident{recentEvents} = \lambda () \rightarrow\\
\quad\kvd{let}~\ident{db} = \kvd{access}~\ident{News}~\kvd{in}\\
\quad\ident{query}~\ident{db}~\str{"SELECT * WHERE Date > \%1"}~(\kvd{access}~\ident{Clock})
\end{array}
\end{equation*}
%
Consider a scenario where the function is \emph{declared} on the server-side and then
transferred and \emph{executed} on the client-side. The resource \ident{News} represents a 
database that is only available on the server-side and so the function needs to keep a remote
reference to the server. However, the \ident{Clock} resource may (or may not) be available on the
client-side. If the resource is available on the client, then it may be re-bound and the function
will use the current client's data -- for example, to accommodate for time-zone changes.

This example demonstrates how lambda abstraction behaves for context-dependent computations. The 
context requirement of the function body is a set of resources $\{\ident{Clock}, \ident{News}\}$.
The context requirements are split between the \emph{declaration-site} and \emph{call-site}.
However, there are multiple possible splittings. The splitting $\{\ident{News}\} \cup \{\ident{Clock}\}$
models the case when the database is accessed from the server, but time is taken from the client,
while the splitting $\{\ident{Clock}, \ident{News}\} \cup \{\}$ corresponds to the case when both
resources are accessed from the server.

%---------------------------------------------------------------------------------------------------

\subsection{Structural coeffect calculus}

\newcommand{\pastval}[2]{\ident{#1}_{[#2]}}
\newcommand{\dnat}{\ident{stream}}

The calculus used in the previous section annotates the entire context with a single value --  such
as the set of required resources. However, sometimes it is desirable to annotate not the entire 
context, but individual variables of the context.

As an example, consider a language that allows us to get a value of a variable (representing
some changing data-source) \ident{x} versions back using the syntax $\pastval{a}{x}$. 
To track information about individual variables, we use a product-like operation $\times$ on tags 
to mirrors the product structure of variables. For example:
%
\begin{equation*}
\begin{array}{l}
(\ident{a}:\dnat, \ident{b}:\dnat) ~@~ 5 \times 10
  \vdash
    \pastval{a}{5}+\pastval{b}{10}: \ident{nat}
\end{array}
\end{equation*}
%
The annotations $5 \times 10$ corresponds to the free-variable context $\ident{a}, \ident{b}$, denoting
that we need at most 5 and 10 past values of \ident{a} and \ident{b}, respectively. If we substitute 
\ident{c} for both \ident{a} and \ident{b}, we need to combine the context-requirements and take the
maximum of the requirements of the individual variables:
%
\begin{equation*}
\begin{array}{l}
(\ident{c}:\dnat) ~@~ \ident{max}(5, 10)
  \vdash
    \pastval{c}{5}+\pastval{c}{10}: \ident{nat}
\end{array}
\end{equation*}
%
This version of the calculus removes the non-determinism of the lambda abstraction from the previous
version. As we associate information with individual variables, lambda abstraction creates a function
that requires the context requirements associated with the variable that is being abstracted over.

For example, if we wrapped the earlier example above in a function taking \ident{a} (and using 
\ident{b} from the declaration-site) then the function would have context requirements $5$ -- that
is the number associated with the variable \ident{a}.

%---------------------------------------------------------------------------------------------------

\subsection{Coeffect meta-language}

\todo{Give some example that uses the coeffects metalanguage style. This will probably be 
meta-programming with open expressions (similarly to what Pfenning and Nanevski do with 
contextual modal type theory).}

%---------------------------------------------------------------------------------------------------

\section{Why context matters}

We claimed earlier that context-dependent comptuations are becoming increasingly common and our
work is focused on annotating the (free-variable) context with additional information.
The importance of context can be also demonstrated by looking at a technology that focuses 
solely on making the context richer -- F\# type providers.

\todo{Say more about type providers -- they extend the context $\Gamma$ so that it can be 
lazily loaded and it can be huge. Potentially, it could be also annotated with additional
meta-data...}

%===================================================================================================

\section{Contributions}

\todo{We present three calculi that model common notions of context-dependence and can be used
as basis for developing context-aware programming langauges with static type systems.}
 % Chapter 1

% \cleardoublepage % Empty page before the start of the next part

%------------------------------------------------

% \ctparttext{You can put some informational part preamble text here. Illo principalmente su nos. Non message \emph{occidental} angloromanic da. Debitas effortio simplificate sia se, auxiliar summarios da que, se avantiate publicationes via. Pan in terra summarios, capital interlingua se que. Al via multo esser specimen, campo responder que da. Le usate medical addresses pro, europa origine sanctificate nos se.} % Text on the Part 2 page describing the content in Part 2

% \part{The Showcase} % Second part of the thesis

%\chapter{Introduction} 
\label{ch:introduction} 

%----------------------------------------------------------------------------------------

Modern software applications behave differently depending on the environment or context 
in which they execute. They often run in increasingly rich environments that provide 
resources (e.g. database or GPS sensor) and are gradually more diverse (e.g. multiple 
versions of different mobile platforms). Web applications are split between client, server 
and mobile components; mobile applications must be aware of the context and of the platform 
while the ``internet of things'' makes the environments even more heterogeneous; 
applications that access rich data sources need to propagate security policies and provenance 
information about the data.

Writing such context-aware applications is a fundamental problem of modern software 
engineering. The state of the art relies on ad-hoc approaches -- using hand-written conditions 
or pre-processors for conditional compilation. Common problems that developers face include: 

\begin{itemize}
\item \textbf{System capabilities.} When writing code that is cross-compiled to multiple 
  targets (e.g. SQL, CUDA or JavaScript) the compilation often occurs at runtime and developers 
  have no guarantee that it will succeed until the program is executed.
  
\item \textbf{Platform versioning.} When developing application for multiple versions of a 
  system (e.g. Android), developers rely on lazy loading at runtime or use conditional compilation 
  using \ident{\#if}. The former delays errors to runtime, while the latter requires building all 
  possible configurations to discover simple compile-time errors.
  
\item \textbf{Resources \& data availability.} When programming applications that access 
  resources or data provided by the environment (e.g. specific database table, GPS sensor), 
  the program typically performs dynamic check for the resource availability. However, this 
  is not checked in any way -- we have no easy way to tell what happens when the resource is 
  not available (e.g. is there a fallback strategy or not?)

\item \textbf{Meta-programming.} In distributed client-server programming, we often need to 
  manipulate code using meta-programming techniques on the server and then pass a closed 
  expression to client, translated to another language (JavaScript). We need to be able to 
  guarantee that such expressions are well-formed before they are translated.
\end{itemize}

The issues discussed above pose an important problem to current and future state of computing.
In this thesis, we develop a unifying mathematical model that allows us to capture the properties
discussed above. We use the model in three different ways to capture different notions of 
context-dependence.

In the rest of the chapter, we first look at the distinguishing factor of two of the presented
coeffect systems (how they differ from other existing work). Then we look at a number of concrete 
problems that are discussed in greater details in later chapters. We use the examples to demonstrate 
the real-world problems that motivate our work. (Theoretical and other motivations are discussed 
later in Chapter~\ref{ch:related-work}).

%===================================================================================================

\section{Context and lambda abstraction}

Our work on context-aware programming languages connects two directions in existing research on the 
theory of programming languages. On one side, effect systems \cite{effects-gifford} and monadic 
computations \cite{monad-notions,monads-effects-marriage} provide a detailed and established method 
for tracking what effects programs have -- that is, how they affect the environment where they execute. 
On the other side, the work on comonadic notions of computations \cite{comonads-notions} shows how 
to use the mathematical dual of monads -- comonads -- to give categorical semantics of 
context-dependent computations.

Effect systems introduced track actions such as memory operations or communication. They are described 
by typing judgments of a form $\Gamma \vdash e : \alpha, \sigma$ where $\Gamma$ is the context of a 
program (typically available variables), $e$ is the expression (program) itself, $\alpha$ is the type of 
values returned by the program (e.g. integer or boolean) and $\sigma$ is a set of possible effects. The 
judgment states that, given the context $\Gamma$, an expression has a type $\alpha$ and can only perform 
effects specified by the set $\sigma$. Wadler and Thiemann \cite{monads-effects-marriage} explain how 
this shapes effect analysis of a lambda abstraction -- that is, how effect systems analyze the effects 
associated with a definition of a function:
%
\begin{quote} 
\emph{In the rule for abstraction, the effect is empty because evaluation immediately
returns the function, with no side effects. The effect on the function arrow
is the same as the effect for the function body, because applying the function will
have the same side effects as evaluating the body.}
\end{quote}
%
This means that, when a programmer defines a function, the system records that executing the 
function will perform the effects of the body of the function. However, simply defining a function
is an effect-free computation. Tate \cite{effects-producer-semantics} calls such effect systems
\emph{producer} effect systems and generalizes the idea of function to more general ``thunking'':
%
\begin{quote}
\emph{We will define an effect as a producer effect if all computations with that effect 
can be thunked as "pure" computations for a domain-specific notion of purity.}
\end{quote}
%
In contrast to the static analysis of (producer) effect systems, the analysis of 
\emph{context-dependence} does not match this pattern. In the systems we consider, lambda 
abstraction places requirements on both the \emph{call-site} (latent requirements) and the 
\emph{declaration-site} (immediate requirements), resulting in different syntactic properties. 
We informally discuss three examples first that demonstrate how contextual requirements propagate. 

%===================================================================================================

\section{Why coeffects matter}

This section gives three examples of context-dependent computations whose properties can be captured
by the tree calculi presented in this thesis. We look at an example of the \emph{flat coeffect calculus} 
(Chapter~\ref{ch:flat-coeffects}), \emph{structural coeffect calculus} (Chapter~\ref{ch:structural-coeffects})
and \emph{coeffect meta-language} (Chapter~\ref{ch:coeffect-metalanguage}).

%---------------------------------------------------------------------------------------------------

\subsection{Flat coeffect calculus}

The flat coeffect calculus associates a single piece of information with the context. As an example,
we look at a simple distributed programming language that includes the concept of \emph{resources}.
A resource may be accessed using a special construct $\kvd{access}~\ident{Res}$. The following 
example shows a function taht lists recent events -- it accesses the resource \ident{News} 
(representing a database) and a resource \ident{Clock} (with the current time):
%
\begin{equation*}
\begin{array}{l}
\kvd{let}~\ident{recentEvents} = \lambda () \rightarrow\\
\quad\kvd{let}~\ident{db} = \kvd{access}~\ident{News}~\kvd{in}\\
\quad\ident{query}~\ident{db}~\str{"SELECT * WHERE Date > \%1"}~(\kvd{access}~\ident{Clock})
\end{array}
\end{equation*}
%
Consider a scenario where the function is \emph{declared} on the server-side and then
transferred and \emph{executed} on the client-side. The resource \ident{News} represents a 
database that is only available on the server-side and so the function needs to keep a remote
reference to the server. However, the \ident{Clock} resource may (or may not) be available on the
client-side. If the resource is available on the client, then it may be re-bound and the function
will use the current client's data -- for example, to accommodate for time-zone changes.

This example demonstrates how lambda abstraction behaves for context-dependent computations. The 
context requirement of the function body is a set of resources $\{\ident{Clock}, \ident{News}\}$.
The context requirements are split between the \emph{declaration-site} and \emph{call-site}.
However, there are multiple possible splittings. The splitting $\{\ident{News}\} \cup \{\ident{Clock}\}$
models the case when the database is accessed from the server, but time is taken from the client,
while the splitting $\{\ident{Clock}, \ident{News}\} \cup \{\}$ corresponds to the case when both
resources are accessed from the server.

%---------------------------------------------------------------------------------------------------

\subsection{Structural coeffect calculus}

\newcommand{\pastval}[2]{\ident{#1}_{[#2]}}
\newcommand{\dnat}{\ident{stream}}

The calculus used in the previous section annotates the entire context with a single value --  such
as the set of required resources. However, sometimes it is desirable to annotate not the entire 
context, but individual variables of the context.

As an example, consider a language that allows us to get a value of a variable (representing
some changing data-source) \ident{x} versions back using the syntax $\pastval{a}{x}$. 
To track information about individual variables, we use a product-like operation $\times$ on tags 
to mirrors the product structure of variables. For example:
%
\begin{equation*}
\begin{array}{l}
(\ident{a}:\dnat, \ident{b}:\dnat) ~@~ 5 \times 10
  \vdash
    \pastval{a}{5}+\pastval{b}{10}: \ident{nat}
\end{array}
\end{equation*}
%
The annotations $5 \times 10$ corresponds to the free-variable context $\ident{a}, \ident{b}$, denoting
that we need at most 5 and 10 past values of \ident{a} and \ident{b}, respectively. If we substitute 
\ident{c} for both \ident{a} and \ident{b}, we need to combine the context-requirements and take the
maximum of the requirements of the individual variables:
%
\begin{equation*}
\begin{array}{l}
(\ident{c}:\dnat) ~@~ \ident{max}(5, 10)
  \vdash
    \pastval{c}{5}+\pastval{c}{10}: \ident{nat}
\end{array}
\end{equation*}
%
This version of the calculus removes the non-determinism of the lambda abstraction from the previous
version. As we associate information with individual variables, lambda abstraction creates a function
that requires the context requirements associated with the variable that is being abstracted over.

For example, if we wrapped the earlier example above in a function taking \ident{a} (and using 
\ident{b} from the declaration-site) then the function would have context requirements $5$ -- that
is the number associated with the variable \ident{a}.

%---------------------------------------------------------------------------------------------------

\subsection{Coeffect meta-language}

\todo{Give some example that uses the coeffects metalanguage style. This will probably be 
meta-programming with open expressions (similarly to what Pfenning and Nanevski do with 
contextual modal type theory).}

%---------------------------------------------------------------------------------------------------

\section{Why context matters}

We claimed earlier that context-dependent comptuations are becoming increasingly common and our
work is focused on annotating the (free-variable) context with additional information.
The importance of context can be also demonstrated by looking at a technology that focuses 
solely on making the context richer -- F\# type providers.

\todo{Say more about type providers -- they extend the context $\Gamma$ so that it can be 
lazily loaded and it can be huge. Potentially, it could be also annotated with additional
meta-data...}

%===================================================================================================

\section{Contributions}

\todo{We present three calculi that model common notions of context-dependence and can be used
as basis for developing context-aware programming langauges with static type systems.}

%\chapter{Pathways to coeffects}
\label{ch:pathways}

There are many different directions from which the concept of \emph{coeffects} can be approached
and, indeed, discovered. In the previous chapter, we motivated it by practical applications, but
coeffects also naturally arise as an extension to a number of programming language theories.
Thanks to the Curry-Howard-Lambek correspondence, we can approach coeffects from the perspective of
type theory, logic and also category theory. This chapter gives an overview of the most
important directions.

We start (Section~\ref{sec:path-binding}) by discussing how coeffects arise from the most common
notion of context-dependence -- variable binding. Next, we look at coeffects as the dual of effect
systems (Section~\ref{sec:path-eff}) and we extend the duality to category theory, looking at
\emph{comonads} (Section~\ref{sec:path-sem}). We also consider type systems inspired by linear
and bunched logic, which are closely related to our structural coeffects (Section~\ref{sec:path-logic}).
Finally, we also consider practical motivations for context-aware programming (Section~\ref{sec:path-cop}).

This chapter serves two purposes. Firstly, it provides a high-level overview of the  related work,
although technical details are often postponed until later. Secondly it recasts existing ideas in
a way that naturally leads to the coeffect systems developed later in the thesis. For this reason,
we are not always faithful to the referenced work. We present the work through the coeffect view
and so we sometimes focus on aspects that authors consider unimportant or we present the work
differently than originally intended. When we do so, this is explicitly stated in the text.


%===================================================================================================
%
%    ######
%    #     # # #    # #####  # #    #  ####
%    #     # # ##   # #    # # ##   # #    #
%    ######  # # #  # #    # # # #  # #
%    #     # # #  # # #    # # #  # # #  ###
%    #     # # #   ## #    # # #   ## #    #
%    ######  # #    # #####  # #    #  ####
%
%===================================================================================================

\section{Coeffects via static and dynamic binding}
\label{sec:path-binding}

Accessing a variable is arguably the simplest form of context-dependence, to the extent that we
do not normally think of variables as a notion of context. However, variables fit well with our
earlier description of context in programming: a block of code that accesses a variable can only
be executed in an environment where the variable is available.

In this section, we look at variable binding through the perspective of context-requirements.
We discuss ordinary variable binding and Haskell's implicit parameters \cite{app-implicit-parameters},
which provide an interesting point in the design space. Implicit parameters give an example of
an ambiguity that arises more generally in context-aware programming, as well as one way of
resolving it through \emph{type-directed semantics}.

For a more general context-aware programming example, consider a program
running on a laptop that accesses an implicit parameter representing a printer. When printing, the
text may appear on my home printer (corresponding to static binding), or it may appear on the
nearest printer based on my physical location (corresponding to dynamic binding).

%---------------------------------------------------------------------------------------------------

\subsection{Variable binding}
\label{sec:path-binding-var}

Variable access represents a form of context-dependence. For example, an expression $x+y$ can be
only evaluated if the environment provides values for variables $x$ and $y$. A variable
\emph{requirement} can be satisfied in two standard ways that are characterized as \emph{dynamic}
and \emph{static} (or lexical) variable binding. Consider the following simple program:
%
\begin{equation*}
\begin{array}{l}
\kvd{let}~\ident{x}=10~\kvd{in}\\[-0.25em]
\kvd{let}~\ident{f}=\lambda \ident{y}\rightarrow \ident{x}+\ident{y}~\kvd{in}\\[-0.25em]
\kvd{let}~\ident{x}=5~\kvd{in}\\
\ident{f}~0
\end{array}
\end{equation*}
%
The program can be evaluated in two ways, depending on the variable binding mechanism:
%
\begin{itemize}
\item \textsc{Static (lexical) binding.} In a language with static binding (such as ML or Java),
  the variable \ident{x} inside the body of the lambda function is statically bound to the
  declaration in the lexical scope -- that is, the variable on the second line -- and the
  expression evaluates to 10.

\item \textsc{Dynamic binding.} In a language with dynamic binding (some variants of LISP), the
  variable value is dynamically bound to the topmost value for the available kept on the stack 
  during program execution. Thus, the \ident{x} variable inside the lambda function refers to
  \ident{x} defined on line 5 of the sample and the expression evaluates to 5.
\end{itemize}

\noindent
When we view variable access as a context requirement, we can see that the body of the function
($\ident{x}+\ident{y}$) requires a context that provides values for variables \ident{x} and
\ident{y}. In static binding, the context demands of the body can be placed on the scope in
which the function is defined (declaration site). In dynamic binding, the requirements are
\emph{delayed} and are placed on the scope in which the function is called (call site).

In static and dynamic scoping, all variable requirements are always placed on one site. However,
those are not the only two options. It is conceivable that a language would use a mechanism that
splits variable requirements differently and combines aspects of dynamic and static binding. For
example, the language could use static binding by default, but resort to dynamic binding if a
variable is not available in the lexical scope. One such system is implicit parameters
discussed in the next section.

Languages with static scoping resolve variable bindings at compile-time. In contrast,
languages with dynamic variable binding cannot resolve variables at compile-time. They 
typically perform runtime checks -- if a program attempts to access a variable that is not 
available in the environment, a runtime error occurs. Implementing static checking for dynamic
binding is also possible, but it requires a more sophisticated type system 
(Section~\ref{sec:path-effects-coeff}), while implementing static binding \emph{without} checking 
would be cumbersome and so dynamically scoped languages are often dynamically typed.

%---------------------------------------------------------------------------------------------------

\subsection{Implicit parameter binding}
\label{sec:path-binding-impl}

Haskell uses static binding for ordinary variables, but GHC additionally provides a feature named
\emph{implicit parameters} \cite{app-implicit-parameters} that adds a special kind of variables,
written as \ident{?param}, which use a particular combination of static and dynamic binding.

The following two examples are variations on the one discussed in Section~\ref{sec:path-binding-var},
obtained by replacing a variable \ident{x} with an implicit parameter \ident{?x}. On the left, the
implicit parameter is declared both in the static scope and in the dynamic scope. On the right, the
implicit parameter is available only in the dynamic scope:

\begin{equation*}
\begin{array}{l}
\kvd{let}~\ident{f}=\\[-0.25em]
\qquad\kvd{let}~\ident{?x}=10~\kvd{in}\\[-0.25em]
\qquad\lambda \ident{y}\rightarrow \ident{?x}+\ident{y}~\kvd{in}\\[-0.25em]
\kvd{let}~\ident{?x}=5~\kvd{in}\\
\ident{f}~0
\end{array}\hspace{5em}\begin{array}{l}
\kvd{let}~\ident{f}=\\[-0.25em]
\qquad\lambda \ident{y}\rightarrow \ident{?x}+\ident{y}~\kvd{in}\\[-0.25em]
\kvd{let}~\ident{?x}=5~\kvd{in}\\
\ident{f}~0
\end{array}
\end{equation*}
%
The binding rules for Haskell's implicit parameters can be summarized as ``\emph{static binding when possible,
dynamic binding when needed}''. If an implicit parameter is available in static scope, then the value
is statically bound and the context requirement is satisfied using the declaration site context.
Otherwise, the context requirement is delayed and has to be satisfied at the call site.
In the example on the left, \ident{?x} is bound to $10$ and so the function \ident{f} has no
delayed context demands and thus the expression evaluates to $10$. On the right,
the context demand for \ident{?x} is \emph{delayed} and is satisfied via dynamic binding when calling
the function. The expression evaluates to $5$.

In Haskell, the type system checks that bindings for all required implicit parameters are
available and so no runtime errors can occur. The type of the function \ident{f} on the left is
$\kvd{int}\rightarrow\kvd{int}$, while the type of the \ident{f} function on the right is
$\{\ident{?x}:\kvd{int}\}\Rightarrow\kvd{int}\rightarrow\kvd{int}$.
The part before $\Rightarrow$ specifies the required implicit parameters that need to be available
in the environment when calling the function. It is worth noting that the syntax is similar to the
one used by type-class constraints. Those can be viewed as context demands too
(Section~\ref{sec:applications-flat-impl}).

\paragraph{Thesis perspective.}
The three different binding mechanisms discussed so far can be seen as different ways of splitting
context demands of a particular kind into static and dynamic parts. Dynamic and static binding
represent the opposite ends of the design spectrum and Haskell's implicit parameters are an
interesting point inside the wider spectrum.

In this thesis, we consider various notions of context, using implicit parameters as one
of several motivating examples. Implicit parameters are a particularly valuable example, because they
clearly illustrate the ambiguity inherent in context-aware programs -- the context demands of
a function can be satisfied using the context available at declaration site or using the context
available at the call site. Recall our earlier printer example -- a language that
provides access to resources in the context needs to provide enough flexibility in handling such
ambiguities, be it implicit parameter values available in scope or printers available in the
physical environment.

We aim to find a description of context-aware languages that does not make ad-hoc decisions
about how context demands are split between the declaration site and the call site. While
Haskell's solution for implicit parameters might be the most reasonable one for that particular
case, this thesis argues that other notions of context require different domain-specific choices
and the general framework of context-aware programming should make that possible.

%---------------------------------------------------------------------------------------------------

\subsection{Resolving ambiguity}
\label{sec:path-binding-amb}

In many practical programming languages, the value and semantics of an expression depends on its
type derivation. In order to assign unique semantics to an expression, the choice is typically
hidden behind a mechanism that selects one preferred type derivation.

This mechanism serves as an inspiration for our approach to resolving ambiguity inherent in
context-aware programs. This section discusses a brief example using the F\# language \cite{app-fsharp},
before revisiting the implicit parameters example.

Consider an F\# lambda expression $(\lambda\ident{x}\rightarrow\ident{x.Length})$, which takes
an object $\ident{x}$ and returns the value of its \ident{Length} property. F\# is a nominally-typed
language meaning that, in isolation, the function has an ambiguous meaning\footnote{In contrast,
in a structurally-typed language, the function would have a unique typing in isolation. In OCaml,
the type would be $\langle \ident{Length}:\ident{'a}\rangle \rightarrow \ident{'a}$.}. It can be
a function taking an array, it can be a function taking a string, or it can be a function taking
one of the other .NET types that are equipped with the \ident{Length} property.

The semantics of the function depends on the typing derivation. For example, for arrays, it is
compiled using the \ident{ldlen} intermediate language (IL) instruction, while or strings, it is
compiled using \ident{call} instruction (calling the property getter). In F\#, the compiler chooses
an appropriate typing derivation. For example:
%
\begin{equation*}
\begin{array}{l}
  \lbrack \str{hello};\str{world} \rbrack ~|\hspace{-0.25em}>\;\ident{List.map}~(\lambda\ident{s}\rightarrow\ident{s.Length})\\
  \lbrack \ident{Array.empty};~\ident{Array.create}~\num{100}~\num{0}\; \rbrack ~|\hspace{-0.25em}>\;\ident{List.map}~(\lambda\ident{s}\rightarrow\ident{s.Length})\\
\end{array}
\end{equation*}
%
The $|\hspace{-0.25em}>$ operator passes the value on the left to the function on the right. In
the first case, the compiler infers that the type of the input is a list of strings and so the
type of the lambda function becomes $\ident{string}\rightarrow\ident{int}$. In the second case,
the list contains two arrays (an empty array and an array containing one hundred zero values) and so
the type of the lambda function is $\ident{int}\lbrack\,\rbrack\rightarrow\ident{int}$.
The important points about the example are:
%
\begin{enumerate}
\item The semantics of the function $(\lambda\ident{x}\rightarrow\ident{x.Length})$
  depends on its type. For arrays, it is compiled using a special IL instruction, while for
  strings, it calls a property getter.

\item The compiler chooses an appropriate typing derivation. In the above case, this is done
  based on the context in which the expression appears, but other options are possible
  (in some cases, there is a \emph{default} resolution; often, the compiler requires an
  explicit typing annotation).

\item An expression without a type derivation does not have semantics. For example, given
  $\ident{List.map}~(\lambda\ident{s}\rightarrow\ident{s.Length})$, the F\# compiler fails to
  infer a type; the expression is not well-typed and does not have a semantics.\footnote{Alternatively,
  the compiler could choose default typing among multiple options. The F\# compiler does this
  for the $+$ operator, which can be used on \ident{float} and \ident{int} types, but the compiler
  chooses \ident{int} as the default.}
\end{enumerate}
%
The function $\lambda\ident{y}\rightarrow\ident{?x}+\ident{y}$ in Haskell also has multiple
possible typing derivations and its semantics varies depending on its type. If the lexical
scope contains a binding for $\ident{?x}$, the function type is $\kvd{int}\rightarrow\kvd{int}$
and it captures the value from the lexical scope. Otherwise, the type of the function is
$\{\ident{?x}:\kvd{int}\}\Rightarrow\kvd{int}\rightarrow\kvd{int}$ and it reads the parameter
value from a hidden dictionary that is passed together with the input from the call site.

\paragraph{Thesis perspective.}
Just like the F\# function in the above example, certain expressions in context-aware languages
developed in this thesis have multiple valid typing derivations and their semantics depends on the
type. In F\#, the compiler determines a unique typing derivation based on other parts of the program
(if type is not uniquely determined, it either chooses a default or fails). In our languages, we
also determine a unique typing derivation. However, rather than relying on type information from
other parts of the program, we explicitly define an algorithm that chooses the preferred unique
derivation (Section~\ref{sec:flat-unique} and Section~\ref{sec:struct-unique}).

This approach decouples two important aspects of context-aware programming and lets us study them
independently -- the semantics of context-aware programs and the domain-specific way of resolving
ambiguities in how context demands are satisfied. In our treatment of implicit parameters, we consider
multiple typing derivations (representing a range with static and dynamic scoping at opposite ends),
but we uniquely choose one preferred typing (mimicking the behaviour of GHC for implicit parameters).



%===================================================================================================
%
%    #######
%    #       ###### ###### ######  ####  #####     ####  #   #  ####  ##### ###### #    #  ####
%    #       #      #      #      #    #   #      #       # #  #        #   #      ##  ## #
%    #####   #####  #####  #####  #        #       ####    #    ####    #   #####  # ## #  ####
%    #       #      #      #      #        #           #   #        #   #   #      #    #      #
%    #       #      #      #      #    #   #      #    #   #   #    #   #   #      #    # #    #
%    ####### #      #      ######  ####    #       ####    #    ####    #   ###### #    #  ####
%
%===================================================================================================

\section{Coeffects via type and effect systems}
\label{sec:path-eff}

Introduced by Gifford and Lucassen \cite{effects-gifford,effects-polymorphic}, type and effect
systems have been designed to track effectful operations performed by computations. Examples
include tracking of reading and writing from and to memory locations \cite{effects-talpin-et-al},
communication in message-passing systems \cite{effects-messagepassing} and atomicity in concurrent
applications \cite{effects-atomicity}.

Type and effect systems are usually specified as judgements of the form $\Gamma \vdash e : \tau, \cclrd{r}$,
meaning that the expression $e$ has a type $\tau$ in a (free-variable) context $\Gamma$ and
additionally may have effects described by $\cclrd{r}$. Effect systems are typically added to a
language that already supports effectful operations as a way of increasing the safety -- the type
and effect system provides stronger guarantees than a plain type system. Filinski
\cite{effects-comprehensive} refers to this approach as \emph{descriptive}\footnote{In contrast
to \emph{prescriptive} effect systems that implement computational effects in a pure language
-- such as monads in Haskell.}.

%---------------------------------------------------------------------------------------------------

\begin{figure}[t]
\begin{equation*}
\tyrule{var}
  {x\!:\!\tau \in \Gamma }
  {\Gamma \vdash x : \tau, \cclrd{\emptyset} }
\end{equation*}
\begin{equation*}
\tyrule{write}
  {\Gamma \vdash e : \tau, \cclrd{r} & l:\ident{ref}_\rho~\tau\in \Gamma}
  {\Gamma \vdash l \leftarrow e : \ident{unit}, \cclrd{r} \cup \cclrd{\{\ident{write}(\rho)\}} }
\end{equation*}
\begin{equation*}
\tyrule{abs}
  {\Gamma, x\!:\!\tau_1 \vdash e : \tau_2, \cclrd{r} }
  {\Gamma \vdash \lambda x.e : \tau_1 \xrightarrow{\cclrd{r}} \tau_2, \cclrd{\emptyset} }
\end{equation*}
\begin{equation*}
\tyrule{app}
  {\Gamma \vdash e_1 : \tau_1 \xrightarrow{\cclrd{r}} \tau_2, \cclrd{s} &
   \Gamma \vdash e_2 : \tau_1, \cclrd{t} }
  {\Gamma \vdash e_1~e_2 : \tau_2, \cclrd{r} \cup \cclrd{s} \cup \cclrd{t} }
\end{equation*}

\figcaption{Simple effect system}
\label{fig:path-eff}
\end{figure}

%---------------------------------------------------------------------------------------------------

\subsection{Simple effect system.}
The structure of a simple effect system\footnote{Most work on effect systems uses $\sigma$ or $F$ for
effect annotations. We use letters $\cclrd{r}, \cclrd{s}, \cclrd{t}$ and also distinguish effect
or coeffect annotations by colour.} is demonstrated in Figure~\ref{fig:path-eff}. The example
shows typing rules for a simply typed lambda calculus with an additional (effectful) operation
$l \leftarrow e$ that writes the value of $e$ to a mutable location $l$. The type of locations
($\ident{ref}_\rho~\tau$) is annotated with a \emph{memory region} $\rho$ of the location $l$.
The effects tracked by the type and effect system over-approximate the actual effects and memory
regions provide a convenient way to build such over-approximation. The effects are
represented as a set of effectful actions that an expression may perform and the effectful action
(\emph{write}) adds a primitive effect $\ident{write}(\rho)$.

The remaining rules are shared by a majority of effect systems. Variable access (\emph{var})
has no effects, application (\emph{app}) combines the effects of both expressions, together with
the latent effects of the function to be applied. Finally, lambda abstraction (\emph{abs}) is a
pure computation that turns the \emph{actual} effects of the body into \emph{latent} effects of
the created function.

%---------------------------------------------------------------------------------------------------

\subsection{Simple coeffect system.}
\label{sec:path-effects-coeff}

When writing the judgements of coeffect systems, we want to emphasize the fact that coeffect
systems talk about \emph{context} rather than \emph{results}. For this reason, we write the
judgements in the form $\coctx{\Gamma}{\cclrd{r}} \vdash e : \tau$, associating the additional
information with the context (left-hand side) of the judgement rather than with the result
(right-hand side) as in $\Gamma \vdash e : \tau, \cclrd{r}$. This change alone would not be
very interesting -- we simply used different syntax to write a predicate with four arguments.
The more interesting difference is how the lambda abstraction rule looks.

The language in Figure~\ref{fig:path-coeff} extends simple lambda calculus with resources and
with a construct $\kvd{access}~e$ that obtains the resource specified by the expression $e$.
Most of the typing rules correspond to those of effect systems. Variable access (\emph{var})
has no context demands, application (\emph{app}) combines context demands of the two
sub-expressions and latent context-requirements of the function.
The (\emph{abs}) rule is different than the corresponding rule for effect systems -- the
resource requirements of the body $\cclrd{r} \cup \cclrd{s}$ are split between the \emph{immediate
context-requirements} associated with the current context $\coctx{\Gamma}{\cclrd{r}}$ and the
\emph{latent context-requirements} of the function.

This is where context-aware languages permit multiple valid typing derivations as discussed in
Section~\ref{sec:path-binding-amb}. In the example here, a resource can be captured
when a function is declared (e.g.~when it is constructed on the server-side where database access
is available), or when a function is called (\eg~when a function created on server-side requires
access to current time-zone, it can use the resource available on the client-side). In other words,
resources in this example support both static (lexical) and dynamic scoping. Out of the multiple
valid typing derivation, we would choose one -- for example, capturing only those server-side
resources that are not available on the client-side\footnote{This can be characterized as
``\emph{dynamic binding when possible, static binding when needed}'' and it is, quite curiously, the
opposite choice than the one used by Haskell's implicit parameters.}. We discuss this
system in detail in Section~\ref{sec:applications-flat-impl}.

%---------------------------------------------------------------------------------------------------

\begin{figure}[t]
\begin{equation*}
\tyrule{var}
  {x\!:\!\tau \in \Gamma }
  {\coctx{\Gamma}{\cclrd{\emptyset}} \vdash x : \tau }
\end{equation*}
\begin{equation*}
\tyrule{access}
  {\coctx{\Gamma}{\cclrd{r}} \vdash e : \ident{res}_\rho~\tau }
  {\coctx{\Gamma}{\cclrd{r} \cup \cclrd{ \{\ident{access}(\rho)\} }} \vdash \kvd{access}~e : \tau }
\end{equation*}
\begin{equation*}
\tyrule{abs}
  {\coctx{(\Gamma, x\!:\!\tau_1)}{\cclrd{r} \cup \cclrd{s}} \vdash e : \tau_2 }
  {\coctx{\Gamma}{\cclrd{r}} \vdash \lambda x.e : \tau_1 \xrightarrow{\cclrd{s}} \tau_2}
\end{equation*}
\begin{equation*}
\tyrule{app}
  {\Gamma \vdash e_1 : \tau_1 \xrightarrow{\cclrd{r}} \tau_2, \cclrd{s} \\
   \Gamma \vdash e_2 : \tau_1, \cclrd{t} }
  {\Gamma \vdash e_1~e_2 : \tau_2, \cclrd{r}\cup\cclrd{s}\cup\cclrd{t}  }
\end{equation*}

\figcaption{Simple coeffect system}
\label{fig:path-coeff}
\end{figure}



%===================================================================================================
%
%     #####
%    #     # ###### #    #   ##   #    # ##### #  ####   ####
%    #       #      ##  ##  #  #  ##   #   #   # #    # #
%     #####  #####  # ## # #    # # #  #   #   # #       ####
%          # #      #    # ###### #  # #   #   # #           #
%    #     # #      #    # #    # #   ##   #   # #    # #    #
%     #####  ###### #    # #    # #    #   #   #  ####   ####
%
%===================================================================================================

\section{Coeffects via language semantics}
\label{sec:path-sem}

Another pathway to coeffects leads through the semantics of effectful and context-dependent
computations. In a pioneering work, Moggi \cite{monad-notions} showed that effects (including
partiality, exceptions, non-determinism and I/O) can be modelled using the category theoretic
notion of \emph{monad}.

When using monads, we distinguish effect-free values $\tau$ from programs, or
computations $\mtyp{}{\tau}$. The \emph{monad} $\mtyp{}{}$ abstracts the \emph{notion of
computation} and provides a way of constructing and composing effectful computations:
%
\begin{definition}
A \emph{monad} over a category $\catc$ is a triple $(M, \ident{unit}, \ident{bind})$ where:
\begin{compactitem}
\item $M$ is a mapping on objects (types) $M : \catc \rightarrow \catc$
\item $\ident{unit}$ is a mapping $\alpha \rightarrow \mtyp{}{\alpha}$
\item $\ident{bind}$ is a mapping $(\alpha \rightarrow \mtyp{}{\beta})
  \rightarrow (\mtyp{}{\alpha} \rightarrow \mtyp{}{\beta})$
\end{compactitem}
such that, for all $f:\alpha \rightarrow \mtyp{}{\beta}$ and $g:\beta \rightarrow \mtyp{}{\gamma}$:
\begin{align}
\tag{\emph{left identity}}
  \ident{bind}~\ident{unit} &= \idf{}
  \\
\tag{\emph{right identity}}
  \ident{bind}~f \circ \ident{unit} &= f
  \\
\tag{\emph{associativity}}
  \ident{bind}~(\ident{bind}~g \circ f) &= (\ident{bind}~f) \circ (\ident{bind}~g)
\end{align}
\end{definition}

\noindent
Without providing much details, we note that well known examples of monads include the partiality
monad ($\mtyp{}{\alpha} = \alpha + {\bot}$) also corresponding to the \ident{Maybe} type in
Haskell and list monad ($\mtyp{}{\tau} = 1 + (\tau \times \mtyp{}{\tau})$).
In programming language semantics, monads can be used in two distinct ways.

%---------------------------------------------------------------------------------------------------

\subsection{Effectful languages and meta-languages}
\label{sec:path-sem-langs}

Moggi uses monads to define two formal systems. In the first formal system, a monad is used to model
the \emph{language} itself. This means that the semantics of a language is given in terms of a
one specific monad and the semantics can be used to reason about programs in that language. To quote
\emph{``When reasoning about programs one has only one monad, because the programming language is
fixed, and the main aim is to prove properties of programs''} \cite[p. 5]{monad-notions}.

In the second formal system, monads are added to the programming language as type constructors,
together with additional constructs corresponding to monadic \ident{bind} and \ident{unit}.
A single program can use multiple monads, but the key benefit is the ability to reason
about multiple languages. To quote \emph{``When reasoning about programming languages one has different
monads, one for each programming language, and the main aim is to study how they relate to each
other''} \cite[p. 5]{monad-notions}.

In this thesis, we generally follow the first approach -- this means that we work with an existing
programming language without needing to add additional constructs corresponding to the primitives
of our semantics (the alternative is discussed in Section~\ref{sec:further-meta}). To clarify the
difference, the following two sections show a minimal example of both formal systems. We follow
Moggi and start with language where judgements have the form $x\!:\!\tau_1 \vdash e : \tau_2$ with
exactly one variable\footnote{This simplifies the examples as we do not need \emph{strong} monad,
but that is an orthogonal issue to the distinction between language semantics and meta-language.}.

%---------------------------------------------------------------------------------------------------

\paragraph{Language semantics.} When using monads to provide semantics of a language, we do not
need to extend the language in any way -- we assume that the language already contains the
effectful primitives (such as the assignment operator $x \leftarrow e$). A judgement
of the form $x\!:\!\tau_1 \vdash e : \tau_2$ is interpreted as a morphism $\tau_1 \rightarrow \mtyp{}{\tau_2}$,
meaning that any expression is interpreted as an effectful computation. The semantics of variable
access and the application of a primitive function $f$ is interpreted as follows:
%
\begin{equation*}
\begin{array}{rcl}
 \sem{x\!:\!\tau_1 \vdash x : \tau_1} &=& \ident{unit}_\mtyp{}{}\\
 \sem{x\!:\!\tau_1 \vdash f~e : \tau_3} &=& (\ident{bind}_\mtyp{}{}~f) \circ \sem{e}\\
\end{array}
\end{equation*}
%
Variable access is an effect-free computation, that returns the value of the variable, wrapped
using $\ident{unit}_\mtyp{}{}$. In the second rule, we assume that $e$ is an expression using
the variable $x$ and producing a value of type $\tau_2$ and that $f$ is a (primitive) function
$\tau_2 \rightarrow \mtyp{}{\tau_3}$. The semantics lifts the function $f$ using $\ident{bind}_\mtyp{}{}$
to a function $\mtyp{}{\tau_2} \rightarrow \mtyp{}{\tau_3}$ which is compatible with the
interpretation of the expression $e$.

%---------------------------------------------------------------------------------------------------

\paragraph{Meta-language interpretation.} When designing a meta-language based on monads, we need to
extend the lambda calculus with additional type(s) and expressions that correspond to monadic
primitives:
%
\begin{align*}
\tau &:= \ident{num} \sep \tau_1 \rightarrow \tau_2 \sep \mtyp{}{\tau} \\
   e &:= x \sep f~e \sep \kvd{return}_\mtyp{}{}~e \sep \kvd{let}_\mtyp{}{}~x \Leftarrow e_1~\kvd{in}~e_2
\end{align*}
%
The types consist of the primitive type, function type and a type constructor that
represents monadic computations. Thus the expressions in the language can create both
effect-free values, such as $\tau$ and computations $\mtyp{}{\tau}$. The additional expression
$\kvd{return}_\mtyp{}{}$ is used to create a monadic computation (with no effects) from a
value and $\kvd{let}_\mtyp{}{}$ sequences effectful computations. In the semantics,
monads are not needed to interpret variable access and application, they are only used in the
semantics of additional (monadic) constructs:
%
\begin{equation*}
\begin{array}{rcl}
\sem{x\!:\!\tau \vdash x : \tau} &=& \idf{}\\
\sem{x\!:\!\tau_1 \vdash f~e : \tau_3} &=& f \circ \sem{e}\\
\sem{x\!:\!\tau_1 \vdash \kvd{return}_\mtyp{}{}~e : \mtyp{}{\tau_2}} &=& \ident{unit}_\mtyp{}{} \circ \sem{e}\\
\sem{x\!:\!\tau_1 \vdash \kvd{let}_\mtyp{}{}~y \Leftarrow e_1~\kvd{in}~e_2 : \mtyp{}{\tau_3}} &=&
  \ident{bind}_\mtyp{}{}~\sem{e_2} \circ \sem{e_1}
\end{array}
\end{equation*}

\noindent
In this system, the interpretation of variable access becomes a simple identity function and
application is just composition. Monadic computations are constructed explicitly using
$\kvd{return}_\mtyp{}{}$ (interpreted as $\ident{unit}_\mtyp{}{}$) and they are also sequenced
explicitly using the $\kvd{let}_\mtyp{}{}$ construct. As noted by Moggi, the first formal system
can be easily translated to the latter by inserting appropriate monadic constructs.

Moggi regards the meta-language system as more fundamental, because \emph{``its models are more
general''}. This is a valid and reasonable perspective. Yet, we follow the first style,
precisely because it is \emph{less general}. Our aim is to develop concrete context-aware
programming languages (together with their type systems and semantics) rather than to build a
general framework for reasoning about languages with contextual properties.

%---------------------------------------------------------------------------------------------------

\subsection{Marriage of effects and monads}
\label{sec:path-sem-effects}

The work on effect systems and monads both tackle the same problem -- representing and tracking of
computational effects. The two lines of research have been joined by Wadler and Thiemann
\cite{monads-effects-marriage}. This requires extending the categorical structure. A monadic
computation $\tau_1 \rightarrow \mtyp{}{\tau_2}$ means that the computation has \emph{some}
effects while the judgement $x\!:\!\tau_1 \vdash e : \tau_2, \cclrd{r}$ specifies \emph{what} effects
the computation has.

To solve this mismatch, Wadler and Thiemann use a \emph{family} of monads $\mtyp{\cclrd{r}}{\tau}$
with an annotation that specifies the effects that may be performed by the computation. In their
system, an effectful function $\tau_1 \xrightarrow{\cclrd{r}} \tau_2$ is modelled as a pure
function returning monadic computation $\tau_1 \rightarrow \mtyp{\cclrd{r}}{\tau_2}$. Similarly, the
semantics of a judgement $x\!:\!\tau_1 \vdash e : \tau_2, \cclrd{r}$ can be given as a function
$\tau_1 \rightarrow \mtyp{\cclrd{r}}{\tau_2}$.
The precise nature of the family of monads has been later called \emph{indexed monads} by Tate
\cite{effects-producer-semantics} and further developed by Atkey \cite{monads-parameterised-notions}
in his work on \emph{parameterized monads} and Katsumata \cite{monads-parametric}.

\paragraph{Thesis perspective.}
The key takeaway for this thesis from the outlined line of research is that, if we want to develop a
language with type system that captures context-dependent properties of programs more precisely,
the semantics of the language also needs to be a more fine-grained structure (akin to indexed
monads). While monads have been used to model effects, an existing research links context-dependence
with \emph{comonads} -- the categorical dual of monads.

%---------------------------------------------------------------------------------------------------

\subsection{Context-dependent languages and meta-languages}
\label{sec:path-sem-contextdep}

The theoretical parts of this thesis extend the work of Uustalu and Vene who use comonads
to give the semantics of dataflow computations \cite{comonads-dataflow} and more generally,
notions of \emph{context-dependent computations} \cite{comonads-notions}. The computations discussed
in the latter work include streams, arrays and containers. This is a more diverse set of examples,
but they all mostly represent forms of collections. Ahman et al. \cite{comonads-containers} discuss
the relation between comonads and \emph{containers} \cite{types-containers} in more details.

The utility of comonads has been explored by a number of authors before. Brookes and Geva
\cite{comonads-computational} use \emph{computational} comonads for intensional semantics\footnote{The
structure of a computational comonad has been also used by the author of this thesis to abstract
evaluation order of monadic computations \cite{comonads-malias}.}. In functional programming,
Kieburtz \cite{comonads-and-codata} proposed to use comonads for stream programming, but also
handling of I/O and interoperability.

Biermann and de Paiva used comonads to model the necessity modality $\square$ in intuitionistic
modal S4 \cite{logic-intuitionistic-modal}, linking programming languages derived from modal
logics to comonads. One such language has been reconstructed by Pfenning and Davies
\cite{logic-modal-reconstruction}. Nanevski et al. extend this work to Contextual Modal Type
Theory (CMTT) \cite{logic-cmtt}, which again shows the importance of comonads for
\emph{context-dependent} computations.

While Uustalu and Vene use comonads to define the \emph{language semantics} (the first style
of Moggi), Nanevski, Pfenning and Davies use comonads as part of meta-language, in the form
of $\square$ modality, to reason about context-dependent computations (the second style of
Moggi). Before looking at the details, we use the following definition of comonad:
%
\begin{definition}
A \emph{comonad} over a category $\catc$ is a triple $(C, \ident{counit}, \ident{cobind})$ where:
\begin{compactitem}
\item $C$ is a mapping on objects (types) $C : \catc \rightarrow \catc$
\item $\ident{counit}$ is a mapping $\ctyp{}{\alpha} \rightarrow \alpha$
\item $\ident{cobind}$ is a mapping $(\ctyp{}{\alpha} \rightarrow \beta)
  \rightarrow (\ctyp{}{\alpha} \rightarrow \ctyp{}{\beta})$
\end{compactitem}
%
such that, for all $f:\ctyp{}{\alpha} \rightarrow \beta$ and $g:\ctyp{}{\beta} \rightarrow \gamma$:

\begin{align}
\tag{\emph{left identity}}
  \ident{cobind}~\ident{counit} &= \idf{}
  \\
\tag{\emph{right identity}}
  \ident{counit} \circ \ident{cobind}~f &= f
  \\
\tag{\emph{associativity}}
  \ident{cobind}~(g \circ \ident{cobind}~f) &= (\ident{cobind}~g) \circ (\ident{cobind}~f)
\end{align}
\end{definition}

\noindent
The definition is dual to a monad. Intuitively, the $\ident{counit}$
operation extracts a value $\alpha$ from a value that carries additional context $\ctyp{}{\alpha}$.
The $\ident{cobind}$ operation turns a context-dependent function
$\ctyp{}{\alpha} \rightarrow \beta$ into a function that takes a value with context, applies
the context-dependent function to value(s) in the context and then propagates the context. The
next section makes this intuitive definition more concrete. More detailed discussion about
comonads can be found in Orchard's PhD thesis \cite{comonads-dom-thesis}.

%---------------------------------------------------------------------------------------------------

\paragraph{Language semantics.}
To demonstrate the approach of Uustalu and Vene, we consider the non-empty list comonad
$\ctyp{}{\tau} = \tau + (\tau \times \ctyp{}{\tau})$. A value of the type is either
the last element $\tau$ or an element followed by another non-empty list $\tau \times \ctyp{}{\tau}$
(consisting of the head $\tau$ and the tail $\ctyp{}{\tau}$). Note that the list must be non-empty,
otherwise \ident{counit} would not be a complete function (it would be undefined on empty list). In
the following, we write $(l_1, \ldots, l_n)$ for a list of $n$ elements:
%
\begin{equation*}
\begin{array}{rcl}
\ident{counit}~(l_1, \ldots, l_n) &=& l_1\\
\ident{cobind}~f~(l_1, \ldots, l_n) &=& (f (l_1, \ldots, l_n), f (l_2, \ldots, l_n), \ldots, f (l_n))
\end{array}
\end{equation*}
%
The \ident{counit} operation returns the current (first) element of the (non-empty) list.
The \ident{cobind} operation creates a new list by applying the context-dependent function $f$
to the entire list, to the suffix of the list, to the suffix of the suffix and so on. Interestingly,
it preserves the \emph{shape} of the list as it turns a list of $n$ elements into another list
of $n$ elements.

In causal dataflow, we can interpret the list as a list consisting of past values, with the
current value in the head. Then, the $\ident{cobind}$ operation calculates the current value
of the output based on the current and all past values of the input; the second element is
calculated based on all past values and the last element is calculated based just on the initial
input $(l_n)$. In addition to the operations of comonad, the model also uses some operations that
are specific to causal dataflow:
%
\begin{equation*}
\begin{array}{rcl}
\ident{prev}~(l_1, \ldots, l_n) &=& (l_2, \ldots, l_n)\\
\end{array}
\end{equation*}
%
The operation drops the first element from the list. In the dataflow interpretation, this means
that it returns the previous state of a value.

Now, consider a simple dataflow language with single-variable contexts, variables,
primitive built-in functions and a construct $\kvd{prev}~e$ that returns the previous
value of the computation $e$. We omit the typing rules, but they are simple -- assuming $e$
has a type $\tau$, the expression $\kvd{prev}~e$ has also type $\tau$. The fact that
the language models dataflow and values are lists (of past values) is a matter of semantics,
which is defined as follows:

\begin{equation*}
\begin{array}{rcl}
\sem{x\!:\!\tau \vdash x : \tau} &=& \ident{counit}_\ctyp{}{}\\
\sem{x\!:\!\tau_1 \vdash f~e : \tau_3} &=& f \circ (\ident{cobind}_\ctyp{}{} ~\sem{e})\\
\sem{x\!:\!\tau_1 \vdash \kvd{prev}~e : \tau_2} &=& \ident{prev} \circ (\ident{cobind}_\ctyp{}{} ~\sem{e})\\
\end{array}
\end{equation*}
%
The semantics follows that of effectful computations using monads. A variable access is interpreted
using $\ident{counit}_\ctyp{}{}$ (extract the variable value); composition
uses $\ident{cobind}_\ctyp{}{}$ to propagate the context to the function $f$ and $\kvd{prev}$
is interpreted using the primitive $\ident{prev}$ (which takes a list and returns a list).
For example, the judgement $x\!:\!\tau \vdash \kvd{prev}~(\kvd{prev}~x) : \tau$ represents an
expression that expects context with variable $x$ and returns a stream of values before the
previous one. The semantics of the term expresses this behaviour:
$(\ident{prev} \circ \ident{prev} \circ (\ident{cobind}_\ctyp{}{}~\ident{counit}_\ctyp{}{}))$.
Note that the first operation is simply an identity function thanks to the comonad laws discussed
earlier.

In the outline presented here, we ignored lambda abstraction. Similarly to monadic semantics,
where lambda abstraction requires a \emph{strong} monad, the comonadic semantics also requires
additional structure called \emph{symmetric (semi)monoidal} comonads. This structure is
responsible for the splitting of context-requirements in lambda abstraction. Note that this is
what happens in the unusual (\emph{abs}) rule in Figure~\ref{fig:path-coeff}, which distinguishes
coeffect systems from effect systems.

We return to this topic when discussing lambda abstraction in Section~\ref{sec:applications-structure-lam}
and semantics of flat coeffect systems in Section~\ref{sec:semantics-theory}.

%---------------------------------------------------------------------------------------------------

\paragraph{Meta-language interpretation.} To demonstrate the approach that employs comonads
as part of a meta-language, we look at an example inspired by the work of Pfenning et al.
\cite{logic-modal-reconstruction,logic-cmtt}. We do not attempt to provide a precise overview of
their work. The main purpose of the following discussion is to provide a different intuition behind
comonads, and to present an example of a language that includes comonad as a type constructor,
together with language primitives corresponding to comonadic operations\footnote{In fact,
Pfenning et al. never mention comonads explicitly. This is done in later work by Gabbay et al.
\cite{logic-cmtt-semantics},  but the connection between the language and comonads
is not as direct as in case of monadic or comonadic semantics covered in the previous section.}.

In languages inspired by modal logics, types can have the form $\square \tau$. In the work of
Pfenning and Davies, this is the type of a term that is provable with no assumptions. In the
ML5 language by Murphy et al. \cite{app-distributed-ml5,logic-distributed-calculus}, the
$\square \tau$ type means \emph{mobile code}, that is code that can be evaluated at any node of a
distributed system (the evaluation corresponds to the axiom $\square \tau \rightarrow \tau$).
Finally, Davies and Pfenning \cite{logic-modal-staged} consider staged computations and interpret
$\square \tau$ as a type of unevaluated expressions of type $\tau$ (with no free variables).

In Contextual Modal Type Theory, the modality $\square$ is further annotated with the free variables
of the (unevaluated) expression. We write $\square^{\cclrd{\Psi}} \tau$ for a type of expressions
that requires a context $\Psi$. The type is a comonadic counterpart to \emph{indexed monads} used by
Wadler and Thiemann when linking monads and effect systems and, indeed, it gives rise to a language
that tracks context-dependence of computations in a type system.

In staged computation, the type $\ctyp{\cclrd{\Psi}}{\tau}$ represents an expression
that requires the context $\Psi$ (i.e.~the expression is an open term that requires variables $\Psi$).
The Figure~\ref{fig:modal-meta} shows two typing rules for such language. The rules directly
correspond to the two operations of a comonad and can be interpreted as follows:

\begin{itemize}
\item (\emph{eval}) corresponds to $\ident{counit} : \ctyp{\cclrd{\emptyset}}{\alpha} \rightarrow \alpha$.
  It indicates that we can evaluate a closed (unevaluated) term and obtain a value. Interestingly, the
  rule requires a specific context annotation (empty set of free variables).
  It is not possible to evaluate an open term.

\item (\emph{letbox}) corresponds to $\ident{cobind} : (\ctyp{\cclrd{\Psi}}{\alpha} \rightarrow \beta)
  \rightarrow \ctyp{\cclrd{\Psi}, \cclrd{\Phi}}{\alpha} \rightarrow \ctyp{\cclrd{\Phi}}{\beta}$.
  Given a term which requires variable context $\cclrd{\Psi}, \cclrd{\Phi}$
  (expression $e_1$) and a function that turns a term needing $\cclrd{\Psi}$ into an evaluated
  value (expression $e_2$), we can construct a term that requires just $\cclrd{\Phi}$.
\end{itemize}

\noindent
The fact that the (\emph{eval}) rule requires a specific context is an interesting relaxation
from ordinary comonads where \ident{counit} needs to be defined for all values. Here, the indexed
\ident{counit} operation needs to be defined \emph{only} on values annotated with $\emptyset$.

The annotated \ident{cobind} operation that corresponds to (\emph{letbox}). An interesting aspect
is that it propagates the context-requirements ``backwards''. The input expression (second parameter)
requires a combination of contexts that are required by the two components -- those required by the
input of the function (first argument) and those required by the resulting expression (result).
This is another key aspect that distinguishes coeffects from effect systems. We return back to
the meta-language approach of embedding comonads in Section~\ref{sec:further-meta}.

%---------------------------------------------------------------------------------------------------

\begin{figure}
\begin{equation*}
\inference[(eval)]
  {\Gamma \vdash e : \square^{\cclrd{\emptyset}}{\tau}}
  {\Gamma \vdash !e : \tau}
\end{equation*}
\begin{equation*}
\inference[(letbox)]
  { \Gamma \vdash e_1 : \square^{\cclrd{\Phi}, \cclrd{\Psi}}{\tau_1} &
    \Gamma, x\!:\!\square^{\cclrd{\Phi}}{\tau_1} \vdash e_2 : \tau_2 }
  { \Gamma \vdash \kvd{let~box}~x=e_1~\kvd{in}~e_2 : \square^{\cclrd{\Psi}}{\tau_2}}
\end{equation*}

\figcaption{Typing for a comonadic language with contextual staged computations}
\label{fig:modal-meta}
\end{figure}

%---------------------------------------------------------------------------------------------------

\paragraph{Thesis perspective.}
As mentioned earlier, we are interested in designing context-dependent languages and so we
use comonads for \emph{language semantics}. Uustalu and Vene present a semantics of
context-dependent computations in terms of comonads. We provide the rest of the story known
from the marriage of monads and effects. We develop coeffect calculus with a type system that
tracks the context demands more precisely (by annotating the types) and we add indexing
to comonads and link the two by giving a formal semantics. The indexing allows us to capture
applications that do not fit into the model provided by plain comonads.

The \emph{meta-language} approach of Pfenning et al. is closely related to
our work. Most importantly, Contextual Modal Type Theory (CMTT) uses indexed $\square$ modality
which corresponds to indexed comonads (in a similar way in which effect systems correspond to
indexed monads). The relation between CMTT and comonads has been suggested by
Gabbay et al. \cite{logic-cmtt-semantics}, but the meta-language employed by CMTT does not
directly correspond to comonadic operations. For example, our (\emph{letbox}) typing rule from
Figure~\ref{fig:modal-meta} is not a primitive of CMTT and would correspond to
$\ident{box}(\cclrd{\Psi}, \ident{letbox}(e_1, x, e_2))$. Nevertheless, the indexing in
CMTT provides a useful hint for adding indexing to the work of Uustalu and Vene.


%===================================================================================================
%
%     #####
%    #     # #    # #####   ####  ##### #####  #    #  ####  ##### #    # #####    ##   #
%    #       #    # #    # #        #   #    # #    # #    #   #   #    # #    #  #  #  #
%     #####  #    # #####   ####    #   #    # #    # #        #   #    # #    # #    # #
%          # #    # #    #      #   #   #####  #    # #        #   #    # #####  ###### #
%    #     # #    # #    # #    #   #   #   #  #    # #    #   #   #    # #   #  #    # #
%     #####   ####  #####   ####    #   #    #  ####   ####    #    ####  #    # #    # ######
%
%===================================================================================================

\section{Coeffects via substructural and bunched logics}
\label{sec:path-logic}

In the coeffect system for tracking resource usage outlined earlier, we associated additional
contextual information (set of available resources) with the variable context of the typing
judgement: $\coctx{\Gamma}{\cclrd{r}} \vdash e : \tau$. In other words, our work focuses on
what is happening on the left hand side of $\vdash$.

In the case of resources, the additional information about the context is added to the
variable context (as a product), but we will later look at contextual properties that affect
how variables are represented. More importantly, \emph{structural coeffects} link additional
information to individual variables in the context, rather than the context as a whole.

In this section, we look at type systems that reconsider $\Gamma$ in a number of ways.
First of all, substructural type systems \cite{substruct-attpl-intro} restrict the use of variables
in the language. Most famously linear type systems introduced by Wadler \cite{substruct-linear-change}
can guarantee that a variable is used exactly once. This has interesting implications for memory
management and I/O.

In bunched typing developed by O'Hearn \cite{substruct-bunched}, the variable context is a tree
formed by multiple different constructors (e.g.~one that allows sharing and one that does not).
Most famously, bunched typing has contributed to the development of separation logic
\cite{substruct-separation-logic} (starting a fruitful line of research in software verification),
but it is also interesting on its own.

%---------------------------------------------------------------------------------------------------


\begin{figure}
\begin{equation*}
\tyrule{exchange}
  {\Gamma, x\!:\!\tau_1, y\!:\!\tau_2 \vdash e : \gamma}
  {\Gamma, y\!:\!\tau_2, x\!:\!\tau_1 \vdash e : \gamma}
\end{equation*}
\begin{equation*}
\tyrule{weakening}
  {\Gamma, \Delta \vdash e : \gamma}
  {\Gamma, x\!:\!\tau, \Delta \vdash e : \gamma}
\end{equation*}
\begin{equation*}
\tyrule{contraction}
  {\Gamma, x\!:\!\tau_1, y\!:\!\tau_1, \Delta \vdash e : \tau_2}
  {\Gamma, x\!:\!\tau_1, \Delta \vdash \subst{e}{y}{x} : \tau_2}
\end{equation*}

\figcaption{Exchange, weakening and contraction typing rules}
\label{fig:substructural-rules}
\end{figure}

%---------------------------------------------------------------------------------------------------

\subsection{Substructural type systems.}

Traditionally, $\Gamma$ is viewed as a set of assumptions and typing rules admit (or explicitly
include) three transformations that manipulate the variable contexts which are shown in
Figure~\ref{fig:substructural-rules}. The (\emph{exchange}) rule allows reordering of variables
(which is implicit when assumptions are treated as set); (\emph{weakening}) makes it possible
to discard an assumption -- this has the implication that a variable may be declared but never
used. Finally, (\emph{contraction}) makes it possible to use a single variable multiple times
(in the rule, this is done explicitly by joining multiple variables into a single one using
substitution).

In substructural type systems, the assumptions are typically treated as a list. As a result,
they have to be manipulated explicitly. Different systems allow different subsets of the rules.
For example, \emph{affine} systems allows exchange and weakening, leading to a system where
variable may be used at most once; in \emph{linear} systems, only exchange is permitted and so
every variable has to be used exactly once.

When tracking context-dependent properties associated with individual variables, we need to
be more explicit in how variables are used. Substructural type systems provide a way to do this.
Even if we allow all three operations, we can use a variation on the three rules (exchange,
weakening and contraction) to track which variables are used and how (and to track additional
contextual information about variables).

%---------------------------------------------------------------------------------------------------

\subsection{Bunched type systems.}
Bunched typing makes one more refinement to how $\Gamma$ is treated. Rather than having a list
of assumptions, the context becomes a tree that contains variable typings (or special identity
values) in the leaves and has multiple different types of nodes. The context can be defined,
for example, as follows:
%
\begin{equation*}
\Gamma, \Delta, \Sigma := x\!:\!\alpha \sep I \sep \Gamma, \Gamma \sep 1 \sep \Gamma; \Gamma
\end{equation*}
%
The values $I$ and $1$ represent two kinds of ``empty'' contexts. More interestingly, non-empty
variable contexts may be constructed using two distinct constructors -- $\Gamma, \Gamma$ and
$\Gamma; \Gamma$ -- that have different properties. In particular, weakening and contraction is
only allowed for the $;$ constructor, while exchange is allowed for both.

The structural rules for bunched typing are shown in Figure~\ref{fig:substructural-bunched}.
The syntax $\Gamma(\Delta)$ is used to mean an assumption tree that contains $\Delta$ as a
sub-tree and so, for example, (\emph{exchange1}) can switch the order of contexts anywhere in the
tree. The remaining rules are similar to the rules of linear logic.

One important note about bunched typing is that it requires a different interpretation. The omission
of weakening and contraction in linear logic means that variable must be used exactly once.
In bunched typing, variables may still be duplicated, but only using the ``;'' separator.
The type system can be interpreted as specifying whether a variable may be shared between the
body of a function and the context where a function is declared.

The system introduces two
distinct function types $\tau_1 \rightarrow \tau_2$ and $\tau_1~ \textendash\!\!\!\ast \tau_2$
(corresponding to ``;'' and ``,'' respectively). The key property is that only the first kind
of functions can share variables with the context where a function is declared, while the second
restricts such sharing. We do not attempt to give a detailed description here as it is not
immediately related to coeffects -- for more information, refer to O'Hearn's introduction
\cite{substruct-bunched}.

\begin{figure}
\begin{equation*}
\tyrule{exchange1}
  {\Gamma(\Delta, \Sigma) \vdash e : \alpha}
  {\Gamma(\Sigma, \Delta) \vdash e : \alpha}
\end{equation*}
\begin{equation*}
\tyrule{exchange2}
  {\Gamma(\Delta; \Sigma) \vdash e : \alpha}
  {\Gamma(\Sigma; \Delta) \vdash e : \alpha}
\end{equation*}
\begin{equation*}
\tyrule{weakening}
  {\Gamma(\Delta) \vdash e : \alpha}
  {\Gamma(\Delta; \Sigma) \vdash e : \alpha}
\end{equation*}
\begin{equation*}
\tyrule{contraction}
  {\Gamma(\Delta; \Sigma) \vdash e : \alpha}
  {\Gamma(\Delta) \vdash \subst{e}{\Sigma}{\Delta} : \alpha}
\end{equation*}
\figcaption{Exchange, weakening and contraction rules for bunched typing}
\label{fig:substructural-bunched}
\end{figure}

%---------------------------------------------------------------------------------------------------

\paragraph{Thesis perspective.}

From the perspective of substructural and bunched types, our work can be viewed as annotating
bunches. Such annotations then specify additional information about the context -- or, more
specifically, about the sub-tree of the context. Although this is not the exact definition used in
Chapter~\ref{ch:structural}, we could define contexts as follows:
%
\begin{equation*}
\Gamma, \Delta, \Sigma := x\!:\!\alpha \sep 1 \sep \Gamma, \Gamma \sep \coctx{\Gamma}{\cclrd{r}}
\end{equation*}
%
Now we can not only annotate an entire context with some information (as in the simple coeffect
system for tracking resources that used judgements of a form $\cclrd{\Gamma}{\cclrd{r}} \vdash e : \tau$).
We can also annotate individual components. For example, a context containing variables $x,y,z$
where only $x$ is used could be written as $\coctx{(x\!:\!\tau_1)}{\ident{\cclrd{used}}}, \coctx{(y\!:\!\tau_2, z\!:\!\tau_3)}{\ident{\cclrd{unused}}}$.

For the purpose of this introduction, we ignore important aspects such as how are nested annotations
interpreted. The main goal is to show that coeffects can be easily viewed as an extension to the
work on bunched logic. Aside from this principal connection, \emph{structural coeffects} also
use some of the proof techniques from the work on bunched logics.



%===================================================================================================
%
%     #####  ####### ######
%    #     # #     # #     #
%    #       #     # #     #
%    #       #     # ######
%    #       #     # #
%    #     # #     # #
%     #####  ####### #
%
%===================================================================================================

\section{Context oriented programming}
\label{sec:path-cop}

The importance of context-aware computations is perhaps most obvious when considering mobile
application, client/server web applications or even the internet of things. A pioneering work
in the area using functional languages has been done by Serrano \cite{app-hop-diffuse,app-hop-lang}
(which also inspired the motivating example presented in Chapter~\ref{ch:intro}). His HOP language supports
cross-compilation and programs execute in different contexts. However, HOP is not statically
type checked.

In the software engineering community, a number of authors have addressed the
problem of context-aware computations. Hirschfeld et al. propose \emph{Context-Oriented Programming}
(COP) as a methodology \cite{app-cop-method}. The COP paradigm has been later implemented by
programming language features. Costanza \cite{app-cop-contextl} develops a domain-specific LISP-like
language ContextL and Bardram \cite{app-cop-javafwk} proposes a Java framework for COP.

Finally, the subject of context-awareness has also been addressed in work focusing on the development
of mobile applications \cite{app-cop-mobile,app-cop-mobile2}. Here, the \emph{context} focuses more
on concrete physical context (obtained from the device sensors) than context as an abstract
language feature.

We approach the problem from a different perspective, building on the tradition of
statically-typed functional programming languages, focusing on type systems as the primary way
of capturing contextual properties.


%===================================================================================================
%
%     #####
%    #     # #    # #    # #    #   ##   #####  #   #
%    #       #    # ##  ## ##  ##  #  #  #    #  # #
%     #####  #    # # ## # # ## # #    # #    #   #
%          # #    # #    # #    # ###### #####    #
%    #     # #    # #    # #    # #    # #   #    #
%     #####   ####  #    # #    # #    # #    #   #
%
%===================================================================================================

\section{Summary}

This chapter presented four different pathways leading to the idea of coeffects. We also
introduced the most important related work, although presenting related work was not the
primary goal of the chapter. The primary goal was to present the idea of coeffects as a logical
follow up to a number of research directions. For this reason, we highlighted only certain aspects
of the discussed related work -- the remaining aspects as well as important technical details are
covered throughout the thesis.

The first pathway follows as a generalization of static and dynamic variable binding. Variable
binding can be seen as the most primitive form of context-dependence and coeffects provide a
generalization that can capture different binding mechanisms in a unified way. In the second
pathway, we looked at the dual of well-known work on effect systems. However, this is not simply
a syntactic transformation. As we further discuss in the next chapter, coeffect systems treat lambda
abstraction differently. The third pathway follows by extending comonadic semantics of context-dependent
computations with indexing and building a type system analogous to effect system from the ``marriage of
effects and monads''. Finally, the fourth pathway starts with substructural type systems. Coeffect
systems naturally arise by annotating bunches in bunched logics with additional information. In this
thesis, we mostly follow the first two approaches.

\chapter{Context-aware systems}
\label{ch:applications}

Software developers as well as programming language researchers choose abstractions based not
just on how appropriate they are. Other factors include social aspects -- how well is the
abstraction known, how well is it documented and whether it is a standard tool of the
\emph{research programme}\footnote{A research programme, as introduced by Lakatos \cite{philosophy-lakatos},
is a network of scientists sharing the same basic assumptions and techniques.} that the researcher
unconsciously subscribes to.

For tracking of effects, such \emph{standard tools} are well known. When faced with an effectful
computation, programming language designers immediately pick monads. For context-aware computations,
there are no standard tools. Thus contextual properties may, at first, appear as a set of disconnected
examples. Existing systems that capture contextual properties use a wide range of methods including
special-purpose type systems, approaches arising from modal logic S4, as well as techniques based
on abstractions designed for other purpose, most frequently monads.

This chapter reviews some of the existing context-aware programming abstractions and presents them
in a uniform way. We start with disconnected examples, but at the end, we will see that they
share a common pattern\footnote{The different properties captured by monads may appear similarly
disconnected at first!}.

\paragraph{Chapter structure and contributions}
\begin{itemize}
\item We characterize contextual properties -- the Section~\ref{sec:applications-structure}
  explains what is a \emph{coeffect} and contrasts it with a better known notion of
  \emph{effect}. It explains what is the nature of properties that can be tracked using
  coeffect systems presented in this thesis.

\item We describe a number of simple calculi for tracking a wide range of contextual properties.
  The systems are adapted from diverse sources (type systems, static analyses, logics) and apply to
  various domains (cross-compilation, liveness, distributed computing, dataflow, security), but
  share a common structure.

\item The uniform presentation of the systems is the key contribution of this chapter. We distinguish
  between \emph{flat coeffect} systems (Section~\ref{sec:applications-flat}) and \emph{structural coeffect}
  systems (Section~\ref{sec:applications-structural}). This common structure is precisely
  captured by the two \emph{coeffect calculi} in the upcoming chapters.

\item In addition, the coeffect systems for tracking the number of accessed past values in
  dataflow languages (Sections~\ref{sec:applications-flat-dataflow} and \ref{sec:applications-structural-dataflow})
  presents novel results and can be used to optimize dataflow programs.
\end{itemize}



% ===================================================================================================
%
% 	  ###    #                           #
% 	 #   #   #                           #
% 	 #      ####   # ##   #   #   ###   ####   #   #  # ##    ###
% 	  ###    #     ##  #  #   #  #   #   #     #   #  ##  #  #   #
% 	     #   #     #      #   #  #       #     #   #  #      #####
% 	 #   #   #  #  #      #  ##  #   #   #  #  #  ##  #      #
% 	  ###     ##   #       ## #   ###     ##    ## #  #       ###
%
% ===================================================================================================

\section{Structure of coeffect systems}
\label{sec:applications-structure}

When introducing coeffect systems in Section~\ref{sec:path-effects-coeff}, we related coeffect systems
with effect systems. Effect systems track how a program affects the environment, or, in other words
capture some \emph{output impurity}. In contrast, coeffect systems track what a program requires from
the execution envionment, or \emph{input impurity}.

Effect systems generally use judgements of the form $\Gamma \vdash e : \tau \;\&\; \cclrd{\sigma}$,
associating effects $\sigma$ with the output type. We write coeffect
systems using judgements of the form $\coctx{\Gamma}{\cclrd{\sigma}} \vdash e : \tau$, associating
the context requirements with $\Gamma$. Thus, we extend the traditional notion of free-variable
context $\Gamma$ with richer notions of context. This notation emphasizes the right intuition,
but there are more important differences between effects and coeffects.

% --------------------------------------------------------------------------------------------------

\subsection{Effectful lambda abstraction}
\label{sec:applications-structure-lam}

The difference between effects and coeffects becomes apparent when we consider lambda abstraction.
The typical lambda abstraction rule for effect systems looks as (\emph{effect}) in
Figure~\ref{fig:applications-abs}. Wadler and Thiemann~\cite{monads-effects-marriage} explain how
the effect analysis works as follows:
%
\begin{quote}
\emph{In the rule for abstraction, the effect is empty because evaluation immediately
returns the function, with no side effects. The effect on the function arrow
is the same as the effect for the function body, because applying the function will
have the same side effects as evaluating the~body.}
\end{quote}
%
This is the key property of \emph{output impurity}. The effects are only produced when the
function is evaluated and so the effects of the body are attached to the function. A recent
work by Tate~\cite{effects-producer-semantics} uses the term \emph{producer} effect systems
for such standard systems and characterises them as follows:
%
\begin{quote}
\emph{Indeed, we will define an effect as a producer effect if all computations with that
effect can be thunked as ``pure'' computations for a domain-specific notion of purity.}
\end{quote}
%
The thunking is typically performed by a lambda abstraction -- given an effectful expression
$e$, the function $\lambda x.e$ is an effect free value (thunk) that delays all effects.
As shown in the next section, contextual properties do not follow this pattern.

% --------------------------------------------------------------------------------------------------

\begin{figure}
\begin{equation*}
\tyrule{pure}
  { \Gamma, x\!:\!\tau_1 \vdash e : \tau_2}
  { \Gamma \vdash \lambda x.e : \tau_1 \rightarrow \tau_2}
\end{equation*}
\begin{equation*}
\tyrule{effect}
  { \Gamma, x\!:\!\tau_1 \vdash e : \tau_2 \,\&\, \cclrd{\sigma} }
  { \Gamma \vdash \lambda x.e : \tau_1 \xrightarrow{\cclrd{\sigma}} \tau_2 \,\&\, \cclrd{\emptyset} }
\end{equation*}

\figcaption{Lambda abstraction for pure and effectful computations}
\label{fig:applications-abs}
\end{figure}

% --------------------------------------------------------------------------------------------------

\subsection{Notions of context}

We look at three notions of context. The first is the standard free-variable context in
$\lambda$-calculus. This is well understood and we use it to demonstrate how contextual
properties behave. Then we consider two notions of context introduced in this thesis --
\emph{flat coeffects} refer to overall properties of the environment and \emph{structural coeffects}
refer to properties attached to individual variables. We could track properties associated
with values in data structures (\eg~fields of a tuple), but this is left as future work.

\paragraph{Variable coeffects.}

As discussed in Section~\ref{sec:path-binding}, variable access can be seen as a basic form
of context requirement in $\lambda$-calculus. The expression $x$ is typeable only in a context
that contains $x:\tau$ for some type $\tau$.

In lexically scoped languages, lambda abstraction (\emph{pure}), as shown in
Figure~\ref{fig:applications-abs}, splits the free-variable context of an expression into two parts.
At runtime, the value of the parameter has to be provided by the \emph{call site} (dynamic scope)
and the remaining values are provided by the \emph{declaration site} (lexical scope). In the type
checking, the splitting is determined syntactically. The notation $\lambda x.e$ \emph{names} the
variable whose value comes from the call site.

Flat and structural coeffects also split context-requirements between the declaration site and
the call site. The flat and structural coeffects capture two different ways of doing this.

\paragraph{Flat coeffects.}

In Section~\ref{sec:intro-context-example}, we used \emph{resources} in a distributed system as an
example of flat coeffects. These could be, for example, a database, GPS sensor or access to the current
time. We also outlined that such context requirements can be tracked as part of the typing assumption,
for example, say we have an expression $e$ that requires GPS coordinates and the current time.
The variable context of such expression will be annotated with a set of required resources,
\ie~$\coctx{\Gamma}{ \cclrd{\ident{\{ gps, time \}}} }$.

The interesting case is when we construct a lambda function $\lambda x.e$, marshall it and
send it to another node. In systems such as Acute \cite{app-distributed-acute}, the context
requirements can be satisfied in a number of ways. When the same resource is available at the target
machine (\eg~current time), we can transfer the function with a context requirement and \emph{rebind}
the resource. However, if the resource is not available (\eg.~GPS on the server), we need to a
capture \emph{remote reference}.

In the example discussed here, $\lambda x.e$ would require GPS sensor from the declaration
site (lexical scope) where the function is declared, which is attached to the current context
as $\coctx{\Gamma}{ \cclrd{\ident{\{ gps \}}} }$. The current time is required from the caller
of the function. So, the context requirement on the call site (dynamic scope) will be
$\cclrd{r}=\cclrd{\footnotesize\ident{\{ time \}}}$. In coeffect systems, we attach this information
to the function, writing $\tau_1 \xrightarrow{ \cclrd{r} } \tau_2$.

We look at resources in distributed programming in more details in Section~\ref{sec:applications-flat-distr}.
The important point here is that in flat coeffect systems, contextual requirements are
\emph{split} between the call site and declaration site. Furthermore, there is no syntactic
structure that determines how the requirements are split. As mentioned in Section~\ref{sec:path-binding-amb},
we decouple the definition of semantics from the domain-specific choice that determines how
context requirements are satisfied. We capture the choice in the type and give semantics over a
\emph{typing derivation}. A domain-specific algorithm then chooses the desirable typing -- for
example, by preferring resources available on the client over resources available on the server.

\paragraph{Structural coeffects.}
On the one hand, variable context provides a \emph{fine-grained tracking} mechanism of how context
(variables) are used. On the other hand, flat coeffects let us track \emph{additional information} about
the context. The purpose of \emph{structural coeffects} is to reconcile the two and to provide a way
for fine-grained tracking of additional information linked to variables in programs. Structural
coeffects follow the lexical scoping structure determined by the typing rules.

In Section~\ref{sec:intro-why-array}, we used an example of tracking array access patterns. For every
variable, the additional coeffect annotation keeps a range of indices that may be accessed relatively
to the current cursor. For example, consider an expression
$x[\kvd{cursor}] = y[\kvd{cursor}-1] + y[\kvd{cursor}+1]$.

Here, the variable context $\Gamma$ contains two variables, both of type \ident{Arr}. This means
$\Gamma = x\!:\!\ident{Arr},\, y\!:\!\ident{Arr}$. For simplicity, we treat \kvd{cursor} as a
language primitive. The coeffect annotations will be $(0,0)$ for $x$ and $(-1,1)$ for $y$,
denoting that we access only the current value in $x$, but we need access to both left and right
neighbours in the $y$ array. In order to unify the flat and structural notions, we attach this information
as a \emph{vector} of annotations associated with a \emph{vector} of variable and write:
$\coctx{x\!:\!\ident{Arr},\, y\!:\!\ident{Arr}}{ \aclrd{\alift{ (0,0), (-1,1) }} }$.
The unification is outlined in Section~\ref{sec:further-unified}.

In structural systems, the splitting of context is determined by the name (variable) binding.
For example, consider a function that takes $y$ and contains the above body:
$\lambda y.x[\kvd{cursor}] = y[\kvd{cursor}-1] + y[\kvd{cursor}+1]$. Here, the declaration site
contains $x$ and needs to provide access at least within a range $(0,0)$. The call site provides
a value for $y$, which needs to be accessible at least within $(-1, 1)$. In this way, structural
coeffects remove the ambiguity arising from the splitting of requirements in flat coeffect systems.

Before looking at concrete flat and structural systems, we briefly overview
some notation used in this thesis. Structural coeffects keep annotations as \emph{vectors} and
use a number of operations related to scalars and vectors.

% --------------------------------------------------------------------------------------------------

\subsection{Scalars and vectors}
\label{sec:applications-strucutre-vec}

The $\lambda$-calculus is asymmetric. It maps a context with \emph{multiple} variables to a
\emph{single} result. An expression with $n$ free variables of types $\tau_i$ can be modelled by a function
$\tau_1 \times \ldots \times \tau_n \rightarrow \tau$ with a product on the left, but a single value
on the right. In both effect systems and coeffect systems, we write the annotation as part of
the function arrow. However, in the underlying categorical model, effects are attached to the result
$\tau$, while coeffects are attached to the context $\tau_1 \times \ldots \times \tau_n$.

Structural coeffects have one annotation per each variable. Thus, the annotation consists
of multiple values -- one belonging to each variable. To distinguish between the overall annotation
and individual (per-variable) annotations, we call the overall coeffect a \emph{vector} consisting of
\emph{scalar} coeffects. This asymmetry also explains why coeffects are not trivially dual to
effects.

It is useful to clarify how vectors are used in this thesis. Suppose we have a set $\C$ of
\emph{scalars} ranged over by $\cclrd{r},\cclrd{s},\cclrd{t}$. A vector $\aclrd{R}$
over $\C$ is a tuple $\alift{ \cclrd{r_1}, \ldots, \cclrd{r_n} }$ of scalars.
We use bold face letters like $\aclrd{\textbf{r}}, \aclrd{\textbf{s}}, \aclrd{\textbf{t}}$ for
vectors and normal face $\cclrd{r},\cclrd{s},\cclrd{t}$ for scalars. We also say that a
\emph{shape} of a vector $\slen{\aclrd{\textbf{r}}}$ (or more generally any container)
determines the set of \emph{positions} in a vector. So, a vector of a shape (length) $n$ has positions
$\{ 1, 2, \ldots, n \}$. We discuss containers and shapes further in Section~\ref{sec:further-unified} and also
discuss how our use relates to containers of Abbott, Altenkirch and Ghani \cite{types-containers}.

Just as in the usual pointwise multiplication of a vector by a scalar, we lift any binary operation on scalars into a
scalar-vector one. For a binary operation on scalars $\circ : \C \times \C \rightarrow \C$, we define
 $\cclrd{s} \circ \aclrd{\textbf{r}} = \alift{ \cclrd{s}\circ\cclrd{r_1}, \ldots, \cclrd{s}\circ\cclrd{r_n}}$.
Relations on scalars can be also lifted to vectors. Given two vectors $\aclrd{\textbf{r}}, \aclrd{\textbf{s}}$ of the
same shape with positions $\{ 1, \ldots, n \}$ and a relation $\varpropto\, \subseteq \C \times \C$ we define
$\aclrd{\textbf{r}} \varpropto \aclrd{\textbf{s}} \Leftrightarrow (\cclrd{r_1} \varpropto \cclrd{s_1}) \wedge \ldots \wedge (\cclrd{r_n} \varpropto \cclrd{s_n}) $
Finally, we often concatenate vectors, for example, when joining two variable contexts.
Given vectors $\aclrd{\textbf{r}}, \aclrd{\textbf{s}}$ with (possibly different) shapes $\{ 1, \ldots, n \}$ and
$\{ 1, \ldots, m \}$, the associative operation for concatenation $\times$ is defined as
$\aclrd{\textbf{r}}\times\aclrd{\textbf{s}} = \alift{\cclrd{r_1},\ldots,\cclrd{r_n},\cclrd{s_1},\ldots,\cclrd{s_m}}$.

We note that an environment $\Gamma$ containing $n$ uniquely named, typed variables is also a vector,
but we continue to write `$,$' for the product, so $\Gamma_1, x\!:\!\tau, \Gamma_2$ should
be seen as $\Gamma_1 \times \langle x\!:\!\tau\rangle \times \Gamma_2$.




% ===================================================================================================
%
% 	 #####   ##            #
% 	 #        #            #
% 	 #        #     ###   ####
% 	 ####     #        #   #
% 	 #        #     ####   #
% 	 #        #    #   #   #  #
% 	 #       ###    ####    ##
%
% ===================================================================================================

\section{Flat coeffect systems}
\label{sec:applications-flat}

In flat coeffect systems, the additional contextual information are independent of lexically scoped
variables. As such, flat coeffects capture properties where the execution environment provides some
additional data, resources or information about the execution context.

As mentioned in the introduction, coeffect systems in this chapter may appear as a disconnected
set of examples at first. Indeed, this section covers a diverse set of calculi including
Haskell's implicit parameters (Section~\ref{sec:applications-flat-impl}), distributed computing and
cross-compilation (Section~\ref{sec:applications-flat-distr}), liveness analysis
(Section~\ref{sec:applications-flat-live}) and dataflow (Section~\ref{sec:applications-flat-dataflow}).

For three of the examples, we present a type system and a simple semantics (given inductively over
the typing derivation). We informally discuss how preferred typing derivation is chosen (to resolve
the inherent ambiguity), but leave details to later chapters. Although the examples
are not new, our novel presentation of the systems (and the fact that they appear side-by-side)
makes it possible to see that they share a common structure. The structure is captured by
a unified \emph{flat coeffect calculus} in Chapter~\ref{ch:flat}.

%---------------------------------------------------------------------------------------------------

\subsection{Implicit parameters and type classes}
\label{sec:applications-flat-impl}

Haskell provides two examples of flat coeffects -- type class constraints and implicit parameter
constraints \cite{app-type-classes,app-implicit-parameters}. Both of the features introduce additional
\emph{constraints} on the context requiring that the environment provides certain operations for
a type (type classes) or that it provides values for named implicit parameters.
In the Haskell type system, constraints $C$ are attached to the types of top-level declarations,
such as let-bound functions. The Haskell notation $\Gamma \vdash e : C \Rightarrow \tau$
corresponds to our notation $\coctx{\Gamma}{C} \vdash e : \tau$.

In this section, we present a type system for implicit parameters in terms of the coeffect typing
judgement. We briefly consider type classes, but do not give a full type system.

\paragraph{Implicit parameters.}
As discussed in Section~\ref{sec:path-binding-impl}, implicit parameters are a special kind of
variables that support dynamic scoping. They make it possible to parameterise a computation
(involving a long chain of function calls) without passing parameters explicitly as additional
arguments of all involved functions.

The dynamic scoping means that if a function uses a parameter $\ident{?param}$ then the caller of the
function must set a value of $\ident{?param}$ before calling the function. However, implicit
parameters also support lexical scoping. If the parameter $\ident{?param}$ is available in the
lexical scope where a function is defined, then the function will not require a value from the caller.

A simple language with support for implicit parameters has an expression $\ident{?param}$ to read a
parameter and an expression\footnote{Haskell uses $\kvd{let}~\ident{?p} = e_1~\kvd{in}~e_2$, but we use a
different keyword to avoid confusion.} $\kvd{letdyn}~\ident{?param} = e_1~\kvd{in}~e_2$ that sets a
parameter $\ident{?param}$ to the value of $e_1$ and evaluates $e_2$ in a context containing
$\ident{?param}$.

The fact that implicit parameters support both lexical and dynamic scoping becomes interesting
when we consider nested functions. The following function does some pre-processing and then returns a
function that builds a formatted string based on two implicit parameters $\ident{?width}$ and
$\ident{?size}$:

\begin{equation*}
\begin{array}{l}
\kvd{let}~\ident{format} = \lambda \ident{str}~\rightarrow \\[-0.25em]
\quad \kvd{let}~\ident{lines} = \ident{formatLines}~\ident{str}~\ident{?width}~\kvd{in}\\[-0.25em]
\quad (\lambda \ident{rest}~\rightarrow~\ident{append}~
         \ident{lines}~\ident{rest}~\ident{?width}~\ident{?size})
\end{array}
\end{equation*}
%
The body of the outer function accesses the parameter $\ident{?width}$, so it certainly requires a context
$\{ \ident{?width} \}$. The nested function (returned as a result) uses the parameter
$\ident{?width}$, but in addition also uses $\ident{?size}$. Where should the parameters used by the
nested function come from?

To keep examples in this chapter uniform, we do not use the Haskell notation and instead
write $\tau_1 \xrightarrow{r} \tau_2$ for a function that requires implicit parameters specified by $r$.
We also assume that are implicit parameters are of type $\ident{num}$, so the annotation can be a
simple set of names (rather than mapping from names to types). In a purely dynamically scoped system,
implicit parameters would have to be defined when the user invokes the nested function.
However, implicit parameters behave as a combination of lexical and dynamic scoping. This means
that the nested function can capture the value of $\ident{?width}$ and require just $\ident{?size}$.
The following shows the two options:
%
\begin{equation}
\tag{\emph{dynamic}}
\ident{string} \xrightarrow{ \{ \ident{?width} \} }
  (\ident{string} \xrightarrow{ \{ \ident{?width}, \ident{?size} \} } \ident{string})
\end{equation}
\vspace{-1em}
\begin{equation}
\tag{\emph{mixed}}
\ident{string} \xrightarrow{ \{ \ident{?width} \} }
  (\ident{string} \xrightarrow{ \{ \ident{?size} \} } \ident{string})
\end{equation}
%
This is not a complete list of possible typings, but it demonstrates the options. The (\emph{dynamic})
case requires the parameter \ident{?width} twice (the caller may provide different value, in which
case, the semantics needs to specify which value is preferred). In the (\emph{mixed}) case, the
nested function captures the \ident{?width} parameter available from the declaration site. Using
the latter typing, the function can be called as follows:
%
\begin{equation*}
\begin{array}{l}
 \kvd{let}~\ident{formatHello} = (~\kvd{letdyn}~\ident{?width}=5~\kvd{in}~\ident{format}~\texttt{"Hello"})\\[-0.25em]
 \kvd{in}\,(~\kvd{letdyn}~\ident{?size} = 10~\kvd{in}~\ident{formatHello}~\texttt{"world"}~)
\end{array}
\end{equation*}
%
For different typings of \ident{format}, different ways of calling it are valid. This illustrates
the point made in Section~\ref{sec:applications-structure-lam} -- flat coeffect programs have
multiple typing derivations and the semantics depends on the domain-specific choice of preferred
typing. The following section shows how this looks in the type system for implicit parameters.

% --------------------------------------------------------------------------------------------------

\begin{figure}[t]
\begin{equation*}
\tyrule{var}
  {x : \tau \in \Gamma}
  {\coctx{\Gamma}{\cclrd{\emptyset}} \vdash x : \tau }
\end{equation*}
\begin{equation*}
\tyrule{param}
  {}
  {\coctx{\Gamma}{\cclrd{ \{\ident{?param}:\tau \} }} \vdash \ident{?param} : \tau }
\end{equation*}
\begin{equation*}
\tyrule{sub}
  {\coctx{\Gamma}{\cclrd{r'}} \vdash e : \tau }
  {\coctx{\Gamma}{\cclrd{r}} \vdash e : \tau }\quad\quad(\cclrd{r'} \subseteq \cclrd{r})
\end{equation*}
\begin{equation*}
\tyrule{app}
  {\coctx{\Gamma}{\cclrd{r}} \vdash e_1 : \tau_1 \xrightarrow{\cclrd{t}} \tau_2 &
   \coctx{\Gamma}{\cclrd{s}} \vdash e_2 : \tau_1 }
  {\coctx{\Gamma}{\cclrd{r} \cup \cclrd{s} \cup \cclrd{t}} \vdash e_1~e_2 : \tau_2}
\end{equation*}
\begin{equation*}
\tyrule{let}
  { \coctx{\Gamma}{\cclrd{r}} \vdash e_1 : \tau_1 &
    \coctx{\Gamma, x:\tau_1}{\cclrd{s}} \vdash e_2 : \tau_2}
  {\coctx{\Gamma}{\cclrd{r \cup s}} \vdash \kvd{let}~x=e_1~\kvd{in}~e_2 : \tau_2 }
\end{equation*}
\begin{equation*}
\tyrule{abs}
  {\coctx{\Gamma, x:\tau_1}{\cclrd{r} \cup \cclrd{s}} \vdash e : \tau_2}
  {\coctx{\Gamma}{\cclrd{r}} \vdash \lambda x.e : \tau_1 \xrightarrow{\cclrd{s}} \tau_2 }
\end{equation*}
\begin{equation*}
\tyrule{letdyn}
  { \coctx{\Gamma}{\cclrd{r}} \vdash e_1 : \tau_1 &
    \coctx{\Gamma}{\cclrd{s}} \vdash e_2 : \tau_2}
  {\coctx{\Gamma}{\cclrd{r \cup (s \setminus \{ \ident{?p}:\tau_1 \})}} \vdash \kvd{letdyn}~\ident{?p}=e_1~\kvd{in}~e_2 : \tau_2 }
\end{equation*}

\figcaption{Coeffect rules for tracking implicit parameters}
\label{fig:applications-flat-impl}
\vspace{-1em}
\end{figure}

% --------------------------------------------------------------------------------------------------

\paragraph{Type system.}

Figure~\ref{fig:applications-flat-impl} shows a type system that tracks the set of expression's
implicit parameters. The type system uses judgements of the form $\coctx{\Gamma}{\cclrd{r}} \vdash e : \tau$
meaning that an expression $e$ has a type $\tau$ in a free-variable context $\Gamma$ with a set
of implicit parameters specified by $\cclrd{r}$. The annotations $\cclrd{r},\cclrd{s},\cclrd{t}$ are
finite partial functions mapping implicit parameter names to types, \ie~$\cclrd{r},\cclrd{s},\cclrd{t} \subseteq
\ident{Names}$. The expressions include \ident{?param} to read implicit
parameter and \kvd{letdyn} to bind an implicit parameter. The types are standard, but functions are
annotated with the set of implicit parameters that must be available on the call site, \ie~
$\tau_1 \xrightarrow{\cclrd{s}} \tau_2$.

Accessing an ordinary variable (\emph{var}) does not require any implicit parameters. The rule that
introduces primitive context requirements is (\emph{param}). Accessing a parameter \ident{?param}
requires it to be available in the context. The context may provide more (unused)
implicit parameters thanks to the subcoeffecting rule (\emph{sub}).

When we read the rules from the top to the bottom, application (\emph{app}) and let binding
(\emph{let}) simply union the context requirements of the sub-expressions. However, lambda abstraction
(\emph{abs}) is where the example differs from effect systems. The implicit parameters required by
the body $\cclrd{r} \cup \cclrd{s}$ can be freely split between the declaration site ($\coctx{\Gamma}{\cclrd{r}}$)
and the call site ($\tau_1 \xrightarrow{\cclrd{s}} \tau_2$) and thus an expression may have multiple
valid typing derivations. Finally, (\emph{letdyn}) removes the bound parameter from the set of requirements.

The union operation $\cup$ is not a disjoint union, which means that the values for implicit
parameters can also be provided by both sites. For example, consider a function with a body
$\ident{?a} + \ident{?b}$. Assuming that the function takes and returns $\ident{int}$, the following
list shows 4 out of 9 possible valid typing. Full typing derivations can be found in Appendix~\ref{sec:appendixa-implicit}:
%
\label{pg:applications-flat-paramsex}
\begin{equation*}
\begin{array}{rcllllr}
\coctx{\Gamma}{\cclrd{ \{ \ident{?a}:\ident{int} \} }} &\vdash& \lambda x.\ident{?a} + \ident{?b} &:&
  \ident{int} \xrightarrow{\cclrd{ \{ \ident{?b}:\ident{int} \} }} \ident{int} &\qquad\qquad&(1) \\
\coctx{\Gamma}{\cclrd{ \{ \ident{?b}:\ident{int} \} }} &\vdash& \lambda x.\ident{?a} + \ident{?b} &:&
  \ident{int} \xrightarrow{\cclrd{ \{ \ident{?a}:\ident{int} \} }} \ident{int} &&(2)\\
\coctx{\Gamma}{\cclrd{ \{\ident{?a}:\ident{int} \} }} &\vdash& \lambda x.\ident{?a} + \ident{?b} &:&
  \ident{int} \xrightarrow{\cclrd{ \{\ident{?a}:\ident{int}, \ident{?b}:\ident{int} \} }} \ident{int} &&(3)\\
\coctx{\Gamma}{\cclrd{ \emptyset }} &\vdash& \lambda x.\ident{?a} + \ident{?b} &:&
  \ident{int} \xrightarrow{\cclrd{ \{\ident{?a}:\ident{int}, \ident{?b}:\ident{int} \} }} \ident{int} &&(4)
\end{array}
\end{equation*}
%
The first two examples demonstrate that the system does not have the principal typing property.
Both ($1$) and ($2$) are valid typings and they may both be desirable in certain contexts where
the function is used.

The next typing derivation ($3$) requires the parameter \ident{?a} from both the declaration site and
the call site. This means that, at runtime, two values will be available. Our semantics for the
system describes \emph{dynamic rebinding}, meaning that when the caller provides a value for a
parameter that is already specified by the declaration site, the new value hides the old one. This
means that only the value from the call site is actually used. This ($4$) gives a more precise
typing for this situation.

%---------------------------------------------------------------------------------------------------

\begin{figure}[t]
\begin{equation*}
\hspace{0em}
\begin{array}{ll}
\hspace{4.5em}\semdef
  {\coctx{\Gamma}{\cclrd{r}} \vdash x_i : \tau_i}
  {\lambda ((x_1, \ldots, x_n), \_) . x_i}
& (\emph{var})
\\[0.0em]
\hspace{3.5em}\semdef
  {\coctx{\Gamma}{\cclrd{r}} \vdash \ident{?p} : \ident{num}}
  {\lambda (\_, f) . f~\ident{?p}}
& (\emph{param})
\\[1.5em]
\hspace{5em}\semdeff
  {\coctx{\Gamma}{\cclrd{r'}} \vdash e : \tau}
  {\coctx{\Gamma}{\cclrd{r}} \vdash e : \tau}
  {f}
  {\lambda (x, g) . f~(x, \restr{g}{\cclrd{r'}})}
& (\emph{sub})
\\[1.5em]
\hspace{0em}\emdeff
  {\coctx{\Gamma,y:\tau_1}{\cclrd{r}\cup\cclrd{s}} \vdash e : \tau_2}
  {\begin{array}{l}\sem{\coctx{\Gamma}{\cclrd{r}} \vdash \lambda y. e : \tau_1 \xrightarrow{\cclrd{s}} \tau_2}\\[-0.25em]~\end{array}}
  {f}
  {\hspace{-0.5em}\begin{array}{l}
  \lambda ((x_1, \ldots, x_n), g_1) . \lambda (y, g_2) .\\[-0.25em]
  \quad f~((x_1, \ldots, x_n, y), g_1 \uplus g_2)
  \end{array}\hspace{-1em}~}
& (\emph{abs})
\\[1.5em]
\hspace{-0.8em}\emdefff
  {\coctx{\Gamma}{\cclrd{r}} \vdash e_1 : \tau_1 \xrightarrow{\cclrd{t}} \tau_2}
  {\coctx{\Gamma}{\cclrd{s}} \vdash e_2 : \tau_1}
  {\begin{array}{l}\sem{\coctx{\Gamma}{\cclrd{r}\cup\cclrd{s}\cup\cclrd{t}} \vdash e_1~e_2 : \tau_2}\\[-0.25em]~\end{array}}
  {f_1}
  {f_2}
  {\hspace{-0.5em}\begin{array}{l}
  \lambda (x, g) . \\[-0.25em]
  \quad(f_1~(x, \restr{g}{\cclrd{r}}))~(f_2~(x, \restr{g}{\cclrd{s}} ), \restr{g}{\cclrd{t}})
  \end{array}\hspace{-2em}~}
& (\emph{app})
\\[1.5em]
\hspace{-3.3em}\emdefff
  {\coctx{\Gamma}{\cclrd{r}} \vdash e_1 : \ident{num}}
  {\coctx{\Gamma}{\cclrd{s}} \vdash e_2 : \tau_2}
  {\begin{array}{l}\sem{\coctx{\Gamma}{\cclrd{r \cup (s \setminus \{ \ident{?p} \})}} \vdash\\[-0.25em]
    \qquad\kvd{letdyn}~\ident{?p}=e_1~\kvd{in}~e_2 : \tau_2}\end{array}}
  {f_1}
  {f_2}
  {\hspace{-0.5em}\begin{array}{l}
  \lambda (x, g) . f_2~(x, \restr{g}{\cclrd{s} \setminus \cclrd{\{ \ident{?p} \}}} \,\uplus \\[-0.25em]
    \qquad \{ \ident{?p} \mapsto (f_1~(x, \restr{g}{\cclrd{r}})) \} )
  \end{array}\hspace{-1em}~}
& (\emph{letdyn})
\end{array}
\end{equation*}

\vspace{1em}
Assuming the following auxiliary definitions:
\begin{equation*}
\begin{array}{rcl}
 \restr{f}{\cclrd{r}} &=& \{ (p,v) \;|\; (p,v) \in f, \;p \in \cclrd{r} \}\\[-0.25em]
 f \uplus g &=& \restr{f}{\, \textit{dom}(f) \setminus \textit{dom}(g)} \cup g
\end{array}
\end{equation*}

\figcaption{Semantics of a language with implicit parameters}
\label{fig:applications-flat-implsem}
\vspace{-0.5em}
\end{figure}

%---------------------------------------------------------------------------------------------------

\paragraph{Semantics.}
Implicit parameters can be implemented by passing around a hidden dictionary that provides values
to the implicit parameters. Accessing a parameter then becomes a lookup in the dictionary and
the new \kvd{letdyn} construct extends the dictionary. To elucidate how such hidden dictionaries
are propagated through the program when using lambda abstractions and applications, we present a
simple semantics for implicit parameters. The goal here is not to prove properties of the language,
but simply to provide a better explanation. A detailed semantics in terms of indexed comonads is
shown in Chapter~\ref{ch:semantics}.

Given an expression $e$ of type $\tau$ that requires free variables $\Gamma$ and implicit parameters
$\cclrd{r}$, the semantics is a function that takes a product of variables from $\Gamma$ together
with a dictionary of implicit parameters and returns $\tau$:
%
\begin{equation*}
\sem{\coctx{x_1\!:\!\tau_1, \ldots, x_n\!:\!\tau_n}{ \cclrd{r} } \vdash e : \tau}
  ~:~ (\tau_1 \times \ldots \times \tau_n) \times (\cclrd{r} \rightarrow \ident{num}) \rightarrow \tau
\end{equation*}
%
The dictionary is represented as a function from $\cclrd{r}$ to $\ident{num}$. This means that it
provides a $\ident{num}$ value for all implicit parameters that are required according to the typing.
Note that the domain of the function is not the set of all possible implicit parameter names, but
only the finite subset of names that are required according to the typing.

The dictionary is also attached to the inputs of all functions. That is, a function $\tau_1 \xrightarrow{\cclrd{s}} \tau_2$
is interpreted by a function that takes $\tau_1$ together with a dictionary that defines values for
implicit parameters in $\cclrd{s}$:
%
\begin{equation*}
\begin{array}{l}
 \sem{\tau_1 \xrightarrow{\cclrd{s}} \tau_2} = \tau_1 \times (\cclrd{s} \rightarrow \ident{num}) \rightarrow \tau_2
\end{array}
\end{equation*}
%
The definition of the semantics is shown in Figure~\ref{fig:applications-flat-implsem}. We use a
notation that emphasizes the fact that the semantics is given over a typing derivation. On the
left-hand side of $=$, we show the applied typing rule. The right-hand side of $=$ then shows the
semantic functions assigned to the individual assumptions and the resulting semantics for the
consequent.

The (\emph{var}) and (\emph{param}) rules are simple -- they project the appropriate variable and
implicit parameter, respectively. When an expression requires implicit parameters $\cclrd{r}$, the
semantics always provides a dictionary defined \emph{exactly} on $\cclrd{r}$. To achieve this, the
(\emph{sub}) rule restricts the function to $\cclrd{r'}$ (which is valid because $\cclrd{r'} \subseteq \cclrd{r}$).

The most interesting rules are (\emph{abs}) and (\emph{app}). In abstraction, we get two dictionaries
$g_1$ and $g_2$ (from the declaration site and call site, respectively), which are combined and passed
to the body of the function. The semantics prefers values from the call site, which is captured by
the $\uplus$ operation. In application, we first evaluate the expression $e_1$, then $e_2$ and finally
call the returned function. The three calls use (possibly overlapping) restrictions of the dictionary
as required by the static types.

Finally, the (\emph{letdyn}) rule specifies the semantics of the \kvd{letdyn} construct, which
assigns a value to an implicit parameter. This is similar to (\emph{app}), because it needs to
evaluate the sub-expression first. After evaluating $e_1$, the result is added to the dictionary
using $\uplus$. The semantics of ordinary let binding is omitted, because let binding can be
treated as a syntactic sugar for $(\lambda x.e_2)~e_1$.

Without providing a proof here, we note that the semantics is sounds with respect to the type
system -- when evaluating an expression, it provides it with a dictionary that is guaranteed to
contain values for all implicit parameters that may be accessed. This can be easily checked by
examining the semantic rules (and noting that the restriction and union always provide the
expected set of parameters). This idea is captured more formally by the soundness proof for the
operational semantics given in Chapter~\ref{ch:semantics}.

\paragraph{Monadic semantics.}
Implicit parameters are related to the \emph{reader monad}. The type
$\tau_1 \times (\cclrd{r} \rightarrow \ident{num}) \rightarrow \tau_2$ is equivalent to
$\tau_1 \rightarrow ((\cclrd{r} \rightarrow \ident{num}) \rightarrow \tau_2)$ through currying. Thus, we can
express the function as $\tau_1 \rightarrow M\tau_2$ for $M\tau = (\cclrd{r} \rightarrow \ident{num}) \rightarrow \tau$.
Indeed, the reader monad can be used to model dynamic scoping. However, there is an important distinction
from implicit parameters. The usual monadic semantics models fully dynamic scoping, while implicit
parameters combine lexical and dynamic scoping.

When using the usual monadic semantics based on the reader monad, the semantics of the (\emph{abs})
rule would be modified as follows:
%
\begin{equation*}
\emdeff
  {\coctx{\Gamma,y:\tau_1}{\cclrd{r}} \vdash e : \tau_2}
  {\begin{array}{l}\sem{\coctx{\Gamma}{\cclrd{\emptyset}} \vdash \lambda y. e : \tau_1 \xrightarrow{\cclrd{t}} \tau_2}\\[-0.25em]~\end{array}}
  {f}
  {\hspace{-0.5em}\begin{array}{l}
  \lambda ((x_1, \ldots, x_n), \_) . \lambda (y, g) .\\[-0.25em]
  \quad f~((x_1, \ldots, x_n, y), g)
  \end{array}\hspace{-1em}~}
\end{equation*}
%
Note that the declaration site dictionary
is ignored and the body is called with only the dictionary provided by the call site. This is
a consequence of the fact that monadic functions are always pure values created using monadic
\emph{unit}, which turns a function $\tau_1 \rightarrow \mtyp{\cclrd{r}}{\tau_2}$ into a monadic
computation with no side-effects $\mtyp{\cclrd{\emptyset}}{\tau_1 \rightarrow \mtyp{\cclrd{r}}{\tau_2}}$.

As we discuss later in Section~\ref{sec:semantics-related-monads}, the reader monad can be extended
to model rebinding. However, later examples in this chapter, such as liveness in
Section~\ref{sec:applications-flat-live} show that other context-aware computations
cannot be captured by \emph{any} monad.

%---------------------------------------------------------------------------------------------------

\paragraph{Type classes.}
Another type of constraints in Haskell that is closely related to implicit parameters are
\emph{type class} constraints \cite{app-type-classes}. They provide a principled form of ad-hoc
polymorphism (overloading). When code uses an overloaded operation (\eg~comparison or numeric
operators) a constraint is placed on the context in which the operation is used. For example:
%
\begin{equation*}
\begin{array}{l}
 \ident{twoTimes}~::~\ident{Num}~\alpha \Rightarrow \alpha \rightarrow \alpha \\[-0.25em]
 \ident{twoTimes}~x=x+x
\end{array}
\end{equation*}
%
The constraint $\ident{Num}~\alpha$ on the function type arises from the use of the $+$ operator.
Similarly to implicit parameters, type classes can be implemented using a hidden dictionary. In
the above case, the function \ident{twoTimes} takes an additional dictionary that provides an operation
$+$ of type $\alpha \times \alpha \rightarrow \alpha$.

Type classes could be modelled as a coeffect system. The type system would annotate the context
with a set of required type classes. The typing of the body of \ident{twoTimes} would look as
follows:
%
\begin{equation*}
\coctx{x\!:\!\alpha}{ \cclrd{\{ \ident{Num}_\alpha \}} } \vdash x + x : \alpha
\end{equation*}
%
Similarly, the semantics of a language with type class constraints can be defined in a way
similar to implicit parameters. The interpretation of the body is a function that takes $\alpha$
together with a hidden dictionary of operations: $\alpha \times \ident{Num}_\alpha \rightarrow \alpha$.

Type classes and implicit parameters show two important points about flat coeffect systems.
First, the context requirements are associated with some \emph{scope}, such as the body
of a function. Second, they are associated with the input. To call a function that takes an
implicit parameter or has a type-class constraint, the caller needs to pass a (hidden) parameter
together with the function inputs.

\paragraph{Summary.}
Implicit parameters are the simplest example of a system where function abstraction does not
delay all impurities of the body. Here, the term ``delay'' refers to the fact that some
implicit parameters may be captured (from the declaration site) at the time when the function is
defined, but before it is executed. As discussed in Section~\ref{sec:applications-structure-lam},
this is the defining feature of \emph{coeffect} systems.

In this section, we have seen how this affects both the type system and the semantics of the
language. In the type system, the (\emph{abs}) rule places context-requirements on both the
declaration site and the call site. For implicit parameters, this rule introduces non-determinism
in the type-inference, because the parameters can be split arbitrarily. However, as we show in the
next section, this is not always the case. Semantically, lambda abstraction \emph{merges} two
parts of context (implicit parameter dictionaries) that are provided by the call site and
declaration site.

%----------------------------------------------------------------------------------------------------

\subsection{Distributed computing}
\label{sec:applications-flat-distr}

Distributed programming was used as one of the motivating examples for coeffects in
Chapter~\ref{ch:intro}. This section explores the use case. We look at rebindable resources and
cross-compilation. The structure of both is very similar to implicit parameters and type
class constraints, but they demonstrate that there is a broader use for coeffect systems.

% --------------------------------------------------------------------------------------------------

\paragraph{Rebindable resources.}

The need for parameters that support dynamic scoping also arises in distributed computing.
To quote an example discussed by Bierman et al. \cite{app-distributed-rebinding}: \emph{``Dynamic
binding is required in various guises, for example when a marshalled value is received from the
network, containing identifiers that must be rebound to local resources.''}

Rebindable parameters are identifiers that refer to some specific resource. When a function value
is marshalled and sent to another machine, rebindable resources can be handled in two ways.
If the resource is available on the target machine, the parameter may be \emph{rebound} to
the resource on the new machine. This is captured by the dynamic scoping rule. If the
resource is not available on the target machine, the resource is either marshalled or a \emph{remote
reference} is created. This is captured by the lexical scoping rule.

A practical language that supports rebindable resources is for example Acute \cite{app-distributed-acute}.
In the following example, we use the construct $\kvd{access}~\ident{Res}$ to represent
access to a rebindable resource named $\ident{Res}$. The following simple function accesses
a database together with a current date; then it filters from the database based on the date:
%
\begin{equation*}
\begin{array}{l}
 \kvd{let}~\ident{recentNews} = \lambda () \rightarrow\\[-0.25em]
 \quad\kvd{let}~\ident{db} = \kvd{access}~\ident{News}~\kvd{in}\\[-0.25em]
 \quad\ident{query}~\ident{db}~\str{SELECT * WHERE Date > \%1}~(\kvd{access}~\ident{Clock})
\end{array}
\end{equation*}
%
When \ident{recentNews} is created on the server and sent to the client, a remote reference to
the database (available only on the server) must be captured. If the client device supports a
clock, then \ident{Clock} can be locally \emph{rebound}, \eg, to accommodate time-zone changes.
Otherwise, the date and time needs to be obtained from the server too.

~

The type system and semantics for rebindable resources are essentially the same as those for
implicit parameters. Primitive requirements are introduced by the $\kvd{access}$ keyword.
Lambda abstraction splits the requirements between declaration site (capturing remote reference)
and call site (representing rebinding). For this reason, we do not discuss the system in details
and instead look at other uses.

%---------------------------------------------------------------------------------------------------

\begin{figure}
\begin{equation*}
\begin{array}{l}
\cmt{// Checks that input is valid; can run on both server and client}\\[-0.25em]
\kvd{let}~\ident{validateInput} = \lambda \ident{name} \rightarrow\\[-0.25em]
\quad\ident{name} \neq \str{} ~~\&\&~~ \ident{forall~isLetter~name}
\\[0.5em]
\cmt{// Searches database for a product; must run on the server-side}\\[-0.25em]
\kvd{let}~\ident{retrieveProduct} = \lambda \ident{name} \rightarrow\\[-0.25em]
\quad\kvd{if}~\ident{validateInput name}~\kvd{then}~\ident{Some}(\ident{queryProductDb~name})\\[-0.25em]
\quad\kvd{else}~\ident{None}
\\[0.5em]
\cmt{// Client-side function to show price or error (for invalid inputs)}\\[-0.25em]
\kvd{let}~\ident{showPrice} = \lambda \ident{name} \rightarrow\\[-0.25em]
\quad\kvd{if}~\ident{validateInput name}~\kvd{then}\\[-0.25em]
\quad\quad\kvd{match}~(\kvd{remote}~\ident{retrieveProduct}())~\kvd{with}\\[-0.25em]
\quad\quad|~\ident{Some}~\ident{p} \rightarrow \ident{showPrice}~(\ident{getPrice}~\ident{p})\\[-0.25em]
\quad\quad|~\ident{None}~\rightarrow \ident{showError}~\str{Invalid input on the server}\\[-0.25em]
\quad\kvd{else}~\ident{showError}~\str{Invalid input on the client}
\end{array}
\end{equation*}

\figcaption{Sample client-server application with input validation}
\label{fig:applications-flat-distr}
\end{figure}

%---------------------------------------------------------------------------------------------------

\paragraph{Cross-compilation.}
A related issue with distributed programming is the need to target increasing number of diverse
platforms. Modern applications often need to run on multiple platforms (iOS, Android, Windows or
as JavaScript) or multiple versions of the same platform. Many programming languages are capable
of targeting multiple different platforms. For example, functional languages that can be compiled
to native code and JavaScript include, among others, F\#, Haskell and OCaml \cite{app-ocaml-js}.

Links \cite{app-distributed-links}, F\# WebTools and WebSharper \cite{app-fsharp-webapps,app-fsharp-webtools},
ML5 and QWeSST \cite{app-distributed-ml5, app-distributed-qwesst} and Hop \cite{app-hop-lang} go
further and allow including code for multiple distinct platforms in a single source file.
A single program is then automatically split and compiled to multiple target runtimes. This
posses additional challenges -- it is necessary to check where each part of the program can run
and statically guarantee that it will be possible to compile code to the required target
platform (safe \emph{multi-targetting}).

We demonstrate the problem by looking at input validation. In applications that communicate over
an unsecured HTTP channel, user input needs to be validated interactively on the client-side (to
provide immediate response) and then again on the server-side (to guarantee safety).

Consider the client-server example in Figure~\ref{fig:applications-flat-distr}. The
\ident{retrieveProduct} function represents the server-side, while \ident{showPrice} is called
on the client-side and performs a remote call to the server-side function (how this is implemented
is not our concern here). To ensure that the input is valid \emph{both} functions call
\ident{validateInput} -- however, this is fine, because \ident{validateInput} uses only basic
functions and language features that can be cross-compiled to both client-side and server-side.

In Links \cite{app-distributed-links}, functions can be annotated as client-side, server-side
and database-side. F\# WebTools \cite{app-fsharp-webtools} supports cross-compiled (mixed-side)
functions similar to \ident{validateInput}. However, these are single-purpose language features
and they are not extensible. A practical implementation needs to be able to capture multiple
different patterns -- sets of environments (client, server, mobile) for distributed computing,
but also Android API level \cite{app-android-multitarget} to cross-compile for multiple versions
of the same platform.

% --------------------------------------------------------------------------------------------------

\begin{figure}[t]

a.) Set-based type system for cross-compilation, inspired by Links \cite{app-distributed-links}

\begin{equation*}
\tyrule{sub}
  {\coctx{\Gamma}{\cclrd{r'}} \vdash e : \tau }
  {\coctx{\Gamma}{\cclrd{r}} \vdash e : \tau }\quad\quad(\cclrd{r'} \supseteq \cclrd{r})
\end{equation*}
\begin{equation*}
\tyrule{app}
  {\coctx{\Gamma}{\cclrd{r}} \vdash e_1 : \tau_1 \xrightarrow{\cclrd{t}} \tau_2 &
   \coctx{\Gamma}{\cclrd{s}} \vdash e_2 : \tau_1 }
  {\coctx{\Gamma}{\cclrd{r} \cap \cclrd{s} \cap \cclrd{t}} \vdash e_1~e_2 : \tau_2}
\end{equation*}
\begin{equation*}
\tyrule{abs}
  {\coctx{\Gamma, x:\tau_1}{\cclrd{r} \cup \cclrd{s}} \vdash e : \tau_2}
  {\coctx{\Gamma}{\cclrd{r}} \vdash \lambda x.e : \tau_1 \xrightarrow{\cclrd{s}} \tau_2 }
\end{equation*}
\vspace{0.5em}

b.) Version-based type system, inspired by Android API level \cite{app-android-multitarget}

\begin{equation*}
\tyrule{sub}
  {\coctx{\Gamma}{\cclrd{r'}} \vdash e : \tau }
  {\coctx{\Gamma}{\cclrd{r}} \vdash e : \tau }\quad\quad(\cclrd{r'} \leq \cclrd{r})
\end{equation*}
\begin{equation*}
\tyrule{app}
  {\coctx{\Gamma}{\cclrd{r}} \vdash e_1 : \tau_1 \xrightarrow{\cclrd{t}} \tau_2 &
   \coctx{\Gamma}{\cclrd{s}} \vdash e_2 : \tau_1 }
  {\coctx{\Gamma}{\ident{max} \{\cclrd{r}, \cclrd{s}, \cclrd{t} \}} \vdash e_1~e_2 : \tau_2}
\end{equation*}
\begin{equation*}
\tyrule{abs}
  {\coctx{\Gamma, x:\tau_1}{\cclrd{r}} \vdash e : \tau_2}
  {\coctx{\Gamma}{\cclrd{r}} \vdash \lambda x.e : \tau_1 \xrightarrow{\cclrd{r}} \tau_2 }
\end{equation*}

\figcaption{Two variants of coeffect typing rules for cross-compilation}
\label{fig:applications-flat-cross}
\vspace{-1em}
\end{figure}

% --------------------------------------------------------------------------------------------------

\paragraph{Type systems.}

Cross-compilation is similar to resource tracking (and thus to the tracking of implicit
parameters), but it demonstrates a couple of new ideas that are important for flat coeffect
systems. Unlike with implicit parameters, we will not give a full type system in this
section, but we briefly look at two examples that explore the range of possibilities.

In the first system, shown in Figure~\ref{fig:applications-flat-cross} (a), the coeffect annotations
are sets of execution environments, \ie~$\cclrd{r}, \cclrd{s}, \cclrd{t} \subseteq \{ \ident{client},
\ident{server}, \ident{database} \}$. Subcoeffecting (\emph{sub}) lets us ignore some of the supported
execution environments; application (\emph{app}) can be only executed in the \emph{intersection} of the
environments required by the two expressions and the function value.

Subcoeffecting and application are trivially dual to the rules for implicit parameters. We just track
supported environments using intersection as opposed to tracking required parameters using union.
However, this symmetry does not hold for lambda abstraction (\emph{abs}), which still uses \emph{union}.
This models the case when there are two ways of executing the function:
%
\begin{itemize}
\item The function is represented as executable code for an call site environment
  and is executed there, possibly after it is marshalled and transferred to another machine.
\vspace{-0.5em}
\item The function body is compiled for the declaration site environment; the value
  that is returned is a remote reference to the code and function calls are performed as remote invocations.
\end{itemize}
%
This example ignores important considerations -- for example, it is likely desirable to make this
difference explicit (\eg~using explicit wrapping of unevaluated expressions) and the implementation
also needs to be clarified. For a system that does this, see \eg~ML5 \cite{app-distributed-ml5}).
The key point of our brief example is that the algebraic structure of coeffect annotations may be more
complex and use, for example, $\cap$ for application and $\cup$ for abstraction.

The second system, shown in Figure~\ref{fig:applications-flat-cross} (b) is inspired by the API
level requirements in Android. Coeffect annotations are simply numbers representing the level
($\cclrd{r}, \cclrd{s}, \cclrd{t} \in \mathbb{N}$). Levels are ordered increasingly, so we can
always require higher level (\emph{sub}). The requirement on function application (\emph{app}) is
the highest level of the levels required by the sub-expressions and the function. The system uses
yet another variant of lambda abstraction (\emph{abs}). The requirements of the body are duplicated
and placed on \emph{both} the declaration site and the call site.

The ML5 language \cite{app-distributed-ml5} mentioned above served as an inspiration for our example.
It tracks execution environments using modalities of modal S4 to represent the environment -- this
approach is similar to coeffects, both from the practical perspective, but also through deeper
theoretical links. However, it is based on the \emph{meta-language} style of embedding modalities
rather than on the \emph{language-semantics} style (see~Section~\ref{sec:path-sem-langs}). We
return to this topic in Section~\ref{sec:further-meta}.

% --------------------------------------------------------------------------------------------------

\subsection{Liveness analysis}
\label{sec:applications-flat-live}
Our next example shows the idea of coeffects from a different perspective. Rather than
keeping additional information independent of the variable context, we track properties about how
variables are used. Nevertheless, we still look at the left-hand side of $\vdash$ and the structure
of the typing rules and semantics will be very similar.

~

\emph{Live variable analysis} (LVA) \cite{app-modern-compiler} is a standard technique in compiler
theory. It detects whether a free variable of an expression may be used by a program during its
evaluation (it is \emph{live}) or whether it is definitely not needed (it is \emph{dead}). As an
optimization, compiler can remove bindings to dead variables as they are never accessed. Wadler
\cite{app-strictness-absecnce} describes the property of a variable that is dead as the
\emph{absence} of a variable.

\paragraph{Flat liveness analysis.}
In this section, we discuss a restricted form of liveness analysis. We do not track liveness of
\emph{individual} variables, but of the \emph{entire} variable context. This is not practically
useful, but it provides an interesting insight into how flat coeffects work. A per-variable liveness
analysis can be captured using structural coeffects and is discussed in Section~\ref{sec:applications-struct-live}.
Consider the following two examples:
%
\begin{equation*}
\begin{array}{l}
\kvd{let}~\ident{constant42} = \lambda \ident{x} \rightarrow 42\\
\kvd{let}~\ident{constant} = \lambda \ident{value} \rightarrow \lambda \ident{x} \rightarrow \ident{value}
\end{array}
\end{equation*}
%
The body of the first function is just a constant $42$ and so the context of the body is marked
as \emph{dead}. The parameter (call site) of the function is not used and can also be marked as dead.
Similarly, no variables from the declaration site are used and so they are also marked as dead.

In contrast, the body of the second function accesses a variable \ident{value} and so the body
of the function is marked as \emph{live}. In the flat system, we do not track \emph{which}
variable was used and so we have to mark both the call site and the declaration site as live (this will
be refined in a structural version).

\paragraph{Forward vs. backward \& may vs. must.}
Static analyses can be classified as either \emph{forward} or \emph{backward} (depending on how they
propagate information) and as either \emph{must} or \emph{may} (depending on what properties they
guarantee). Liveness is a \emph{backward} analysis -- the requirements are propagated from variables
to their declarations. The distinction between \emph{must} and \emph{may} is apparent when we look
at an example with conditionals:
%
\begin{equation*}
\begin{array}{l}
\kvd{let}~\ident{defaultArg}~= \lambda \ident{cond} \rightarrow \lambda \ident{input} \rightarrow\\[-0.25em]
\quad\kvd{if}~\ident{cond}~\kvd{then}~42~\kvd{else}~\ident{input}
\end{array}
\end{equation*}
%
Liveness analysis is a \emph{may} analysis meaning that it marks variable as live when it
\emph{may} be used and as dead if it is \emph{definitely} not used. This means that the variable
\ident{input} is \emph{live} in the example above. A \emph{must} analysis would mark the variable
only if it was used in both of the branches (this is sometimes called \emph{neededness} or
\emph{very busy} variable/expression).

The distinction between \emph{may} and \emph{must} analyses demonstrates the importance of
interaction between contextual properties and certain language constructs such as conditionals.

% --------------------------------------------------------------------------------------------------

\begin{figure}[t]
\begin{equation*}
\tyrule{var}
  {x : \tau \in \Gamma}
  {\coctx{\Gamma}{\cclrd{\cclrd{\ident{L}} }} \vdash x : \tau }
\end{equation*}
\begin{equation*}
\tyrule{num}
  {~}
  {\coctx{\Gamma}{ \cclrd{\cclrd{\ident{D}}} } \vdash n : \ident{num} }
\end{equation*}
\begin{equation*}
\tyrule{sub}
  {\coctx{\Gamma}{\cclrd{r'}} \vdash e : \tau }
  {\coctx{\Gamma}{\cclrd{r}} \vdash e : \tau }\quad\quad(\cclrd{r'} \,\cclrd{\sqsubseteq}\, \cclrd{r})
\end{equation*}
\begin{equation*}
\tyrule{app}
  {\coctx{\Gamma}{\cclrd{r}} \vdash e_1 : \tau_1 \xrightarrow{\cclrd{t}} \tau_2 &
   \coctx{\Gamma}{\cclrd{s}} \vdash e_2 : \tau_1 }
  {\coctx{\Gamma}{\cclrd{r} \,\cclrd{\sqcup}\, (\cclrd{s} \,\cclrd{\sqcap}\, \cclrd{t})} \vdash e_1~e_2 : \tau_2}
\end{equation*}
\begin{equation*}
\tyrule{let}
  { \coctx{\Gamma}{\cclrd{r}} \vdash e_1 : \tau_1 &
    \coctx{\Gamma, x:\tau_1}{\cclrd{s}} \vdash e_2 : \tau_2}
  {\coctx{\Gamma}{\cclrd{s}} \vdash \kvd{let}~x=e_1~\kvd{in}~e_2 : \tau_2 }
\end{equation*}
\begin{equation*}
\tyrule{abs}
  {\coctx{\Gamma, x:\tau_1}{\cclrd{r}} \vdash e : \tau_2}
  {\coctx{\Gamma}{\cclrd{r}} \vdash \lambda x.e : \tau_1 \xrightarrow{\cclrd{r}} \tau_2 }
\end{equation*}
\vspace{-0.9em}

\figcaption{Coeffect rules for tracking whole-context liveness}
\label{fig:applications-flat-liveness}
\vspace{-1.2em}
\end{figure}

% --------------------------------------------------------------------------------------------------

\paragraph{Type system.}
A type system that captures whole-context liveness annotates the context with value of a
two-point lattice $\mathcal{L} = \{ \cclrd{\ident{L}}, \cclrd{\ident{D}} \}$. The annotation \cclrd{\ident{L}} marks
the context as \emph{live} and \cclrd{\ident{D}} stands for a \emph{dead} context.
Figure~\ref{fig:applications-flat-livealg} (a) defines the ordering $\cclrd{\sqsubseteq}$, meet $\cclrd{\sqcup}$ and join
operations $\cclrd{\sqcap}$ of the lattice.

The typing rules for tracking whole-context liveness are shown in Figure~\ref{fig:applications-flat-liveness}.
The language now includes numerical constants $n$. Accessing a constant (\emph{num}) annotates
the context as dead using \cclrd{\ident{D}}. This contrasts with variable access (\emph{var}), which marks the
context as live using \cclrd{\ident{L}}. A dead context (definitely not needed) can be treated as live context
using the (\emph{sub}) rule. This captures the \emph{may} nature of the analysis.

The (\emph{app}) rule is best understood by discussing its semantics. The semantics uses
\emph{sequential composition} to compose the semantics of $e_2$ with the function obtained
as the result of $e_1$. However, we need more than just sequential composition. The same input
context is passed to the expression $e_1$ (in order to get the function value) and to a function
obtained by sequential composition (first evaluate the argument $e_2$ and pass the result to the
function value). This is captured by \emph{pointwise composition}.

Consider first \emph{sequential composition} of (semantic) functions $f, g$ annotated with
$\cclrd{r}, \cclrd{s}$. The composed function $g \circ f$ is annotated with $\cclrd{r} \cclrd{\sqcup} \cclrd{s}$
as shown in Figure~\ref{fig:applications-flat-livealg} (b).
The argument of the function $g \circ f$ is live only when the arguments of both $f$ and $g$ are
live ($1$). When the argument of $f$ is dead, but $g$ requires $\tau_2$ ($2$), we can evaluate
$f$ without any input and obtain $\tau_2$, which is then passed to $g$. When $g$ does not require
its argument ($3, 4$), we can just evaluate $g$, without evaluating $f$. Here, the semantics
\emph{implements} the dead code elimination optimization.

Secondly, a \emph{pointwise composition} passes the same argument to $f$ and $h$. The parameter
is live if either the parameter of $f$ or $h$ is live. The pointwise composition is written as
$\langle f, h \rangle$ and it combines annotations using $\cclrd{\sqcap}$ as shown in Figure~\ref{fig:applications-flat-livealg} (c).
Here, the argument is not needed only when both $f$ and $h$ do not need it ($1$). In all other cases,
the parameter is needed and is then used either once ($2,3$) or twice ($4$). The rule for function
application (\emph{app}) combines the two operations. The context $\Gamma$ is live if it is needed by
$e_1$ (which always needs to be evaluated) \emph{or} when it is needed by the function value \emph{and}
by $e_2$.

The (\emph{abs}) rule duplicates the annotation of the body, similarly to the cross-compilation
example in Figure~\ref{fig:applications-flat-cross}. When the body accesses any variables, it
requires both the argument and the variables from declaration site. When it does not use any variables,
it marks both as dead. Finally, the (\emph{let}) rule annotates the composed expression with the
liveness of the expression $e_2$ -- if the context of $e_2$ is live, then it also requires variables
from $\Gamma$; if it is dead, then it does not require $\Gamma$ or $x$.
The (\emph{let}) rule is again just a syntactic sugar for
$(\lambda x.e_2)~e_1$. This follows from the simple observation that
$\cclrd{r} \;\cclrd{\sqcup}\; (\cclrd{s} \;\cclrd{\sqcap}\; \cclrd{r}) = \cclrd{r}$.

% --------------------------------------------------------------------------------------------------

\begin{figure}
a.) The operations of a two-point lattice $\mathcal{L} = \{\cclrd{\ident{L}}, \cclrd{\ident{D}}\}$
where $\cclrd{\ident{D}} \,\cclrd{\sqsubseteq}\, \cclrd{\ident{L}}$ are:
%
\begin{equation*}
\begin{array}{rcl}
\cclrd{\ident{L}} \,\cclrd{\sqcup}\, \cclrd{\ident{L}} &=& \cclrd{\ident{L}}\\
\cclrd{\ident{D}} \,\cclrd{\sqcup}\, \cclrd{\ident{L}} &=& \cclrd{\ident{D}}\\
\end{array}
\qquad
\begin{array}{rcl}
\cclrd{\ident{L}} \,\cclrd{\sqcup}\, \cclrd{\ident{D}} &=& \cclrd{\ident{D}}\\
\cclrd{\ident{D}} \,\cclrd{\sqcup}\, \cclrd{\ident{D}} &=& \cclrd{\ident{D}}
\end{array}
\qquad
\begin{array}{rcl}
\cclrd{\ident{L}} \,\cclrd{\sqcap}\, \cclrd{\ident{L}} &=& \cclrd{\ident{L}}\\
\cclrd{\ident{D}} \,\cclrd{\sqcap}\, \cclrd{\ident{L}} &=& \cclrd{\ident{L}}\\
\end{array}
\qquad
\begin{array}{rcl}
\cclrd{\ident{L}} \,\cclrd{\sqcap}\, \cclrd{\ident{D}} &=& \cclrd{\ident{L}}\\
\cclrd{\ident{D}} \,\cclrd{\sqcap}\, \cclrd{\ident{D}} &=& \cclrd{\ident{D}}
\end{array}
\end{equation*}

\vspace{0.5em}
b.) Sequential composition composes annotations using $\cclrd{\sqcup}$:
\begin{equation*}
\begin{array}{llll}
f : \tau_1 \xrightarrow{\cclrd{r}} \tau_2 \qquad\qquad&
g : \tau_2 \xrightarrow{\cclrd{s}} \tau_3 \qquad\qquad&
g \circ f : \tau_1 \xrightarrow{\cclrd{r} \cclrd{\sqcup} \cclrd{s}} \tau_3 \qquad\qquad& ~
\\[0.75em]
f : \tau_1 \xrightarrow{\cclrd{\cclrd{\ident{L}}}} \tau_2 &
g : \tau_2 \xrightarrow{\cclrd{\cclrd{\ident{L}}}} \tau_3 &
g \circ f : \tau_1 \xrightarrow{\cclrd{ \cclrd{\ident{L}} }} \tau_3 & (1)
\\[-0.10em]
f : \tau_1 \xrightarrow{\cclrd{\cclrd{\ident{D}}}} \tau_2 &
g : \tau_2 \xrightarrow{\cclrd{\cclrd{\ident{L}}}} \tau_3 &
g \circ f : \tau_1 \xrightarrow{\cclrd{ \cclrd{\ident{D}} }} \tau_3 & (2)
\\[-0.10em]
f : \tau_1 \xrightarrow{\cclrd{\cclrd{\ident{L}}}} \tau_2 &
g : \tau_2 \xrightarrow{\cclrd{\cclrd{\ident{D}}}} \tau_3 &
g \circ f : \tau_1 \xrightarrow{\cclrd{ \cclrd{\ident{D}} }} \tau_3 & (3)
\\[-0.10em]
f : \tau_1 \xrightarrow{\cclrd{\cclrd{\ident{D}}}} \tau_2 &
g : \tau_2 \xrightarrow{\cclrd{\cclrd{\ident{D}}}} \tau_3 &
g \circ f : \tau_1 \xrightarrow{\cclrd{ \cclrd{\ident{D}} }} \tau_3 & (4)
\end{array}
\end{equation*}

\vspace{0.5em}
c.) Pointwise composition composes annotations using $\cclrd{\sqcap}$:
\begin{equation*}
\begin{array}{llll}
f : \tau_1 \xrightarrow{\cclrd{r}} \tau_2 \qquad\qquad&
h : \tau_1 \xrightarrow{\cclrd{s}} \tau_3 \qquad\qquad&
\langle f, h \rangle : \tau_1 \xrightarrow{\cclrd{r} \cclrd{\sqcap} \cclrd{s}} \tau_2 \times \tau_3 \qquad&~
\\[0.75em]
f : \tau_1 \xrightarrow{\cclrd{ \cclrd{\ident{D}} }} \tau_2 &
h : \tau_1 \xrightarrow{\cclrd{ \cclrd{\ident{D}} }} \tau_3 &
\langle f, h \rangle : \tau_1 \xrightarrow{\cclrd{ \cclrd{\ident{D}} }} \tau_2 \times \tau_3 & (1)
\\[-0.10em]
f : \tau_1 \xrightarrow{\cclrd{ \cclrd{\ident{D}} }} \tau_2 &
h : \tau_1 \xrightarrow{\cclrd{ \cclrd{\ident{L}} }} \tau_3 &
\langle f, h \rangle : \tau_1 \xrightarrow{\cclrd{ \cclrd{\ident{L}} }} \tau_2 \times \tau_3 & (2)
\\[-0.10em]
f : \tau_1 \xrightarrow{\cclrd{ \cclrd{\ident{L}} }} \tau_2 &
h : \tau_1 \xrightarrow{\cclrd{ \cclrd{\ident{D}} }} \tau_3 &
\langle f, h \rangle : \tau_1 \xrightarrow{\cclrd{ \cclrd{\ident{L}} }} \tau_2 \times \tau_3 & (3)
\\[-0.10em]
f : \tau_1 \xrightarrow{\cclrd{ \cclrd{\ident{L}} }} \tau_2 &
h : \tau_1 \xrightarrow{\cclrd{ \cclrd{\ident{L}} }} \tau_3 &
\langle f, h \rangle : \tau_1 \xrightarrow{\cclrd{ \cclrd{\ident{L}} }} \tau_2 \times \tau_3 & (4)
\end{array}
\end{equation*}

\figcaption{Liveness annotations with sequential and pointwise composition}
\label{fig:applications-flat-livealg}
\end{figure}

% --------------------------------------------------------------------------------------------------

\paragraph{Examples.}
Before looking at the semantics, we consider a number of simple examples to demonstrate the
key aspects of the system. Full typing derivations are shown in Appendix~\ref{sec:appendixa-liveness}:
%
\begin{equation*}
\begin{array}{lll}
(\lambda x . 42)~y &~\hspace{1em}~&(1)\\
\ident{twoTimes}~42          &&(2)\\
(\lambda x . x)~42 &&(3)\\
\end{array}
\end{equation*}
%
In the first case ($1$), the context is dead. The function's parameter is dead and so the
overall context is dead, even though the argument uses a variable $y$ -- the semantics evaluates
the function without passing it an actual argument. In the second case ($2$), the function is
a variable that needs to be obtained and so the context is live. In the last case ($3$), the
function accesses a variable and so its declaration site is marked as requiring the context
(\emph{abs}). This is where structural coeffect analysis would be more precise -- the system shown
here cannot capture the fact that $x$ is a bound variable.

%---------------------------------------------------------------------------------------------------

\paragraph{Semantics.}
As showed in the examples, the type system for the liveness coeffect calculus marks the context
of an expression $(\lambda x.42)~y$ as dead. This means that the semantics of the above expression
must not evaluate the argument $y$. In other words, the type system is only sound if the semantics
includes dead code elimination.

To capture dead code elimination in the semantics, we add a special empty value and pass it as an
argument to a function whose argument is not needed, so $(\lambda x.42)$ will be called with
an empty value as argument (because it does not need its argument).

We can represent such empty values using the option type (known as \ident{Maybe} in Haskell).
We use the notation $\tau + 1$ to denote option types. Given a context with variables $x_i$ of
type $\tau_i$, the semantics is a function taking $(\tau_1 \times \ldots \times \tau_n) + 1$.
When the context is live, it will be called with the left value (product of variable assignments);
when the context is dead, it will be called with the right value (containing no information).

However, ordinary option type is not sufficient. We need to capture the fact that the
representation depends on the annotation -- in other words, the type is \emph{indexed} by
the coeffect annotation. The indexing is discussed in details in Section~\ref{sec:semantics-flat-idx}.
For now, it suffices to define the semantics using two separate rules:
%
\begin{equation*}
\begin{array}{rlrcl}
\sem{\coctx{x_1\!:\!\tau_1, \ldots, x_n\!:\!\tau_n}{ \cclrd{\cclrd{\ident{L}}} } \vdash e : \tau}
  &:& (\tau_1 \times \ldots \times \tau_n) &\narrow{\rightarrow}& \tau\\
\sem{\coctx{x_1\!:\!\tau_1, \ldots, x_n\!:\!\tau_n}{ \cclrd{\cclrd{\ident{D}}} } \vdash e : \tau}
  &:& 1 &\narrow{\rightarrow}& \tau
\end{array}
\end{equation*}
%
The semantics of functions is defined similarly. When the argument of a function is live, the function
takes the input value; when the argument is dead, the semantic function takes a unit as its argument:
%
\begin{equation*}
\begin{array}{l}
\sem{\tau_1 \xrightarrow{\cclrd{\cclrd{\ident{L}}}} \tau_2} = \tau_1 \rightarrow \tau_2\\
\sem{\tau_1 \xrightarrow{\cclrd{\cclrd{\ident{D}}}} \tau_2} = 1 \rightarrow \tau_2
\end{array}
\end{equation*}
%
Unlike with implicit parameters, the coeffect system for liveness tracking cannot be modelled
using monads. Any monadic semantics would express functions as $\tau_1 \rightarrow M\, \tau_2$.
Unless laziness is already built-in, there is no way to call such function without
first obtaining a value $\tau_1$. The above semantics makes this possible by taking a unit $1$ when
the argument is not live.

In Figure~\ref{fig:applications-flat-livsem}, we define the semantics directly. We write $()$ for
the only value of type $1$. This appears, for example, in (\emph{const}) which takes $()$ as the
input and returns a constant using a global dictionary $\delta$. In (\emph{var}), the context is live
and so the semantics performs a projection. Subcoeffecting is captured by two rules. A dead context
can be treated as live using (\emph{abs-1}); in other cases, the annotation is not changed (\emph{abs-2}).

Lambda abstraction can be annotated in just two ways. When the body requires context (\emph{abs-1}),
the value of a bound variable $y$ is added to the context $\Gamma$ before passing it to the body.
When the body does not require context (\emph{abs-2}), it is called with $()$ as the input.

For application, there are 8 possible combinations of annotations. The semantics of some of them
is the same, so we only need to show 3 cases. The rules should be read as ML-style pattern matching,
where the last rule handles all cases not covered by the first two. In (\emph{app-1}), we handle the
case when the function $f_2$ does not require its argument -- $x$ is not used and instead, the function
is called with $()$ as the argument. The case (\emph{app-2}) covers the case when the expression
$e_1$ does not require a context, but $e_1$ does. Finally, in (\emph{app-3}), the same input
(which may be either tuple of variables or unit) is propagated uniformly to both $e_1$ and $e_2$.

% --------------------------------------------------------------------------------------------------

\begin{figure}[t]
\begin{equation*}
\hspace{0em}
\begin{array}{ll}
\hspace{3.5em}\semdef
  {\coctx{\Gamma}{\cclrd{\ident{L}}} \vdash x_i : \tau_i}
  {\lambda (x_1, \ldots, x_n) . x_i}
& (\emph{var})
\\[-0.75em]
\hspace{2.75em}\semdef
  {\coctx{\Gamma}{\cclrd{\ident{D}}} \vdash n : \ident{num}}
  {\lambda () . n}
& (\emph{num})
\\[2em]
\hspace{4.25em}\semdeff
  {\coctx{\Gamma}{\cclrd{\ident{D}}} \vdash e : \tau}
  {\coctx{\Gamma}{\cclrd{\ident{L}}} \vdash e : \tau}
  {f}
  {\lambda x . f~()}
& (\emph{sub-1})
\\[1em]
\hspace{4.5em}\semdeff
  {\coctx{\Gamma}{\cclrd{r}} \vdash e : \tau}
  {\coctx{\Gamma}{\cclrd{r}} \vdash e : \tau}
  {f}
  {\lambda x . f~x}
& (\emph{sub-2})
\\[2.5em]
\hspace{-0.55em}\emdeff
  {\coctx{\Gamma,y:\tau_1}{\cclrd{\ident{L}}} \vdash e : \tau_2}
  {\begin{array}{l}\sem{\coctx{\Gamma}{\cclrd{\ident{L}}} \vdash \lambda y. e : \tau_1 \xrightarrow{\cclrd{\ident{L}}} \tau_2}\\[-0.25em]~\end{array}\hspace{-0.5em}~}
  {f}
  {\hspace{-0.5em}\begin{array}{l}\lambda (x_1, \ldots, x_n) . \lambda y .\\[-0.25em]
    \quad f~(x_1, \ldots, x_n, y)\end{array}~~}
& (\emph{abs-1})
\\[1em]
\hspace{-0.5em}\semdeff
  {\coctx{\Gamma,y:\tau_1}{\cclrd{\ident{D}}} \vdash e : \tau_2}
  {\coctx{\Gamma}{\cclrd{\ident{D}}} \vdash \lambda y. e : \tau_1 \xrightarrow{\cclrd{\ident{D}}} \tau_2}
  {f}
  {\lambda () . \lambda () . f~()}
& (\emph{abs-2})
\\[2.5em]
\hspace{0.75em}\semdefff
  {\coctx{\Gamma}{\cclrd{r}} \vdash e_1 : \tau_1 \xrightarrow{\cclrd{\ident{D}}} \tau_2}
  {\coctx{\Gamma}{\cclrd{r}} \vdash e_2 : \tau_1}
  {\coctx{\Gamma}{\cclrd{r}} \vdash e_1~e_2 : \tau_2}
  {f}
  {\_}
  {\lambda x . (f~x)~()}
& (\emph{app-1})
\\[2em]
\hspace{0.75em}\semdefff
  {\coctx{\Gamma}{\cclrd{\ident{L}}} \vdash e_1 : \tau_1 \xrightarrow{\cclrd{\ident{L}}} \tau_2}
  {\coctx{\Gamma}{\cclrd{\ident{D}}} \vdash e_2 : \tau_1}
  {\coctx{\Gamma}{\cclrd{\ident{L}}} \vdash e_1~e_2 : \tau_2}
  {f_1}
  {f_2}
  {\lambda x . (f_1~x)~(f_2~())}
& (\emph{app-2})
\\[2em]
\hspace{-1.6em}\semdefff
  {\coctx{\Gamma}{\cclrd{r}} \vdash e_1 : \tau_1 \xrightarrow{\cclrd{t}} \tau_2}
  {\coctx{\Gamma}{\cclrd{s}} \vdash e_2 : \tau_1}
  {\coctx{\Gamma}{\cclrd{r} \;\cclrd{\sqcup}\; (\cclrd{s} \,\cclrd{\sqcap}\, \cclrd{t})} \vdash e_1~e_2 : \tau_2}
  {f_1}
  {f_2}
  {\lambda x . (f_1~x)~(f_2~x)}
& (\emph{app-3})
\end{array}
\end{equation*}

\figcaption{Semantics that implements dead code elimination for $\lambda$-calculus}
\label{fig:applications-flat-livsem}
\vspace{0.5em}
\end{figure}

% --------------------------------------------------------------------------------------------------

\paragraph{Summary.}
Unlike with implicit parameters, lambda abstraction for liveness analysis does not introduce
non-determinism. It simply duplicates the context requirements. However, this still matches the
property of coeffects that impurities cannot be delayed or thunked and attached just to the
function arrow -- we place requirements on both call site and declaration site.

The semantics of liveness reveals three interesting properties. Firstly, the coeffect calculus for
liveness cannot be modelled as a monadic computation of the form $\tau_1 \rightarrow \mtyp{}{\tau_2}$.
Secondly, the system would not work without the coeffect annotations.
The shape of the semantic function depends on the annotation (the input is either $1$ or $\tau$) and
is \emph{indexed} by the annotation.

Finally, we discussed how the semantics of application arises from \emph{sequential} and
\emph{pointwise} composition. This is an important aspect of coeffect systems -- categorical
semantics typically builds on \emph{sequential} composition, but to model full $\lambda$ calculus
it needs more. For coeffects, we need \emph{pointwise} composition where the same context
is shared by multiple sub-expressions.

% --------------------------------------------------------------------------------------------------

\subsection{Dataflow languages}
\label{sec:applications-flat-dataflow}

We used implicit parameters as our first example, because they show the simplest form of coeffects.
Liveness requires a richer coeffect annotation structure, but the flat version is not practical.
In this section, we look at a system with a structure similar to liveness that is not a toy example.

The Section~\ref{sec:intro-why-array} briefly demonstrated that we can treat array access as an
operation that accesses a context. In case of arrays, the context is neighbourhood of a current
location in the array specified by a cursor. In this section, we make the example more concrete,
using a simpler and better studied programming model, dataflow languages.

Lucid \cite{app-lucid} is a declarative dataflow language designed by Wadge and Ashcroft. In Lucid,
variables represent streams and programs are written as transformations over streams. A function
application $\ident{square}(x)$ represents a stream of squares calculated from the stream of values $x$.

The dataflow approach has been successfully used in domains such as development of real-time embedded
application where many \emph{synchronous languages} \cite{app-synchronous-lang} build on the dataflow
paradigm. The following example is inspired by the Lustre \cite{app-synchronous-lustre} language
and implements a program to count the number of edges on a Boolean stream:
%
\begin{equation*}
\begin{array}{l}
\kvd{let}~\ident{edge} = \ident{false}~\kvd{fby}~(\ident{input}~\&\&~\ident{not}~(\kvd{prev}~\ident{input}))
\\[0.5em]
\kvd{let}~\ident{edgeCount} = \\[-0.25em]
\quad 0~\kvd{fby}~ (~\kvd{if}~\ident{edge}~\kvd{then}~1 + (\kvd{prev}~\ident{edgeCount})\\[-0.25em]
\quad\quad\quad~~~\, \kvd{else}~\kvd{prev}~\ident{edgeCount} ~)
\end{array}
\end{equation*}
%
The construct $\kvd{prev}~x$ returns a stream consisting of previous values of the stream
$x$. The second value of $\kvd{prev}~x$ is first value of $x$ (and the first
value is undefined). The construct $y~\kvd{fby}~x$ returns a stream whose first element is the
first element of $y$ and the remaining elements are values of $x$. Note that in Lucid, the constants
such as \ident{false} and $0$ are constant streams.

Formally, the constructs are defined as follows (writing $x_n$ for $n$-th element of a stream $x$):
%
\[
(\kvd{prev}~x)_n = \left\{
  \begin{array}{ll}
    \ident{nil}     & \; \text{if $n=0$}\\
    x_{n-1} & \; \text{if $n>0$}
  \end{array} \right.
\quad
(y~\kvd{fby}~x)_n = \left\{
  \begin{array}{ll}
    y_0     & \; \text{if $n=0$}\\
    x_n     & \; \text{if $n>0$}
  \end{array} \right.
\]
%
When reading dataflow programs, we do not need to think about variables in terms of streams --
we can see them as simple values. Most of the operations perform calculation just on the
\emph{current} value of the stream. However, the operation \kvd{fby} and \kvd{prev} are different.
They require additional \emph{context} which provides past values of variables
(for \kvd{prev}) and information about the current location in the stream (for \kvd{fby}).

The semantics of Lucid-like languages can be captured using a number of mathematical
structures. Wadge \cite{app-lucid-monads} originally defined a monadic semantics, while Uustalu
and Vene later used comonads \cite{app-dataflow-essence}. In Chapter~\ref{ch:flat}, we extend
the latter approach. The present chapter presents a sketch of a concrete dataflow semantics
defined directly on streams.

In the introductory example with array access patterns, we used coeffects to track the range
of values accessed. In this section, we look at a simpler example -- we only consider the
\kvd{prev} operation and track the maximal number of \emph{past values} needed. This is an
important information for efficient implementation of dataflow languages. When we can guarantee
that at most $x$ past values are accessed, the values can be stored in a pre-allocated buffer
rather than using \eg~on-demand computed lazy streams.

% --------------------------------------------------------------------------------------------------

\begin{figure}[t]
\begin{equation*}
\tyrule{var}
  {x : \tau \in \Gamma}
  {\coctx{\Gamma}{\cclrd{ 0 }} \vdash x : \tau }
\end{equation*}
\begin{equation*}
\tyrule{prev}
  {\coctx{\Gamma}{ \cclrd{n} } \vdash e : \tau}
  {\coctx{\Gamma}{ \cclrd{n}+1} \vdash \kvd{prev}~e : \tau}
\end{equation*}
\begin{equation*}
\tyrule{sub}
  {\coctx{\Gamma}{\cclrd{n'}} \vdash e : \tau }
  {\coctx{\Gamma}{\cclrd{n}} \vdash e : \tau }\quad\quad(\cclrd{n'} \leq \cclrd{n})
\end{equation*}
\begin{equation*}
\tyrule{app}
  {\coctx{\Gamma}{\cclrd{m}} \vdash e_1 : \tau_1 \xrightarrow{\cclrd{p}} \tau_2 &
   \coctx{\Gamma}{\cclrd{n}} \vdash e_2 : \tau_1 }
  {\coctx{\Gamma}{\cclrd{\textnormal{max}}(\cclrd{m}, \cclrd{n+p})} \vdash e_1~e_2 : \tau_2}
\end{equation*}
\begin{equation*}
\tyrule{let}
  { \coctx{\Gamma}{\cclrd{m}} \vdash e_1 : \tau_1 &
    \coctx{\Gamma, x:\tau_1}{\cclrd{n}} \vdash e_2 : \tau_2}
  {\coctx{\Gamma}{\cclrd{n+m}} \vdash \kvd{let}~x=e_1~\kvd{in}~e_2 : \tau_2 }
\end{equation*}
\begin{equation*}
\tyrule{abs}
  {\coctx{\Gamma, x:\tau_1}{\cclrd{n}} \vdash e : \tau_2}
  {\coctx{\Gamma}{\cclrd{n}} \vdash \lambda x.e : \tau_1 \xrightarrow{\cclrd{n}} \tau_2 }
\end{equation*}

\figcaption{Coeffect rules for tracking context-usage in dataflow language}
\label{fig:applications-flat-dataflow}
\end{figure}

% --------------------------------------------------------------------------------------------------

\paragraph{Type system.}
We can use a coeffect type system to track the maximal number of accessed past values. Here,
the context is annotated with a single integer. The current value is always present, so $0$ means
that no past values are needed, but the current value is still available. The typing rules of
the system are shown in Figure~\ref{fig:applications-flat-dataflow}.

Variable access (\emph{var}) annotates the context with $0$; subcoeffecting (\emph{sub}) allows
us to require more values than is actually needed. Primitive context-requirements are introduced
in (\emph{prev}), which increments the number of past values by one. Thus, for example,
$\kvd{prev}~(\kvd{prev}~x)$ requires 2 past values.

The (\emph{app}) rule follows the same intuition as for liveness. It combines \emph{sequential}
and \emph{pointwise} composition of semantic functions. In case of dataflow, the operations
combine annotations using $+$ and \emph{max} operations:
%
\begin{equation*}
\begin{array}{lll}
f : \tau_1 \xrightarrow{\cclrd{m}} \tau_2 \qquad\qquad&
g : \tau_2 \xrightarrow{\cclrd{n}} \tau_3 \qquad\qquad&
g \circ f : \tau_1 \xrightarrow{\cclrd{m+n}} \tau_3 \qquad\qquad
\\
f : \tau_1 \xrightarrow{\cclrd{m}} \tau_2 \qquad\qquad&
h : \tau_1 \xrightarrow{\cclrd{n}} \tau_3 \qquad\qquad&
\langle f, h \rangle : \tau_1 \xrightarrow{\cclrd{\textnormal{max}}(\cclrd{m}, \cclrd{s})} (\tau_2 \times \tau_3) \qquad
\end{array}
\end{equation*}
%
Sequential composition adds the annotations. The function $f$ needs $\cclrd{m}$ past values to
produce a single $\tau_2$ value. To produce two $\tau_2$ values, we thus need $\cclrd{m}+1$ past
values of $\tau_1$; to produce three $\tau_2$ values, we need $\cclrd{m}+2$ past values of $\tau_1$,
and so on. To produce $\ident{n}$ past values that are required as the input of $g$, we need
$\cclrd{m+n}$ past values of type $\tau_1$. The pointwise composition is simpler. It uses
the same stream to evaluate functions requiring $\cclrd{m}$ and $\cclrd{n}$ past values, and so it
needs maximum of the two at most.

In summary, function application (\emph{app}) requires maximum of the values needed to evaluate
$e_1$ and the number of values needed to evaluate the argument $e_2$, sequentially composed with
the function.

In function abstraction (\emph{abs}), the requirements of the body are duplicated on the declaration site
and the call site as in liveness analysis. If the body requires $\cclrd{n}$ past values, it may access
$\cclrd{n}$ values of any variables -- including those available in $\Gamma$, as well as the parameter
$x$. Finally, the (\emph{let}) rule simply adds the two requirements. This corresponds to the sequential
composition operation, but it is also a rule that we obtain by treating let-binding as a syntactic
sugar for $(\lambda x.e_2)~e_1$.

% --------------------------------------------------------------------------------------------------

\paragraph{Example.}
As with the liveness example, the application rule might require more explanation. The following
example is somewhat arbitrary, but it demonstrates the rule well. We assume that \ident{counter}
is a stream of positive integers (starting from zero) and \ident{tick} flips between $0$ and $1$.
The full typing derivation is shown in Appendix~\ref{sec:appendixa-dataflow}:
%
\begin{equation*}
\begin{array}{l}
(\,   \kvd{if}~~(\kvd{prev}~\ident{tick})=0\\[-0.25em]
\,\,\, \kvd{then}~(\lambda x \rightarrow \kvd{prev}~x)\\[-0.25em]
\,\,\, \kvd{else}~(\lambda x \rightarrow x) \,)\quad(\kvd{prev}~\ident{counter})
\end{array}
\end{equation*}
%
The left-hand side of the application returns a function depending on the \emph{previous}
value of \ident{tick}. The resulting stream of functions flips between a function returning
a current value and a function returning the previous value. If the current \ident{tick} is 0, and
the function is applied to a stream $\langle{\ldots,4,3,2,1}\rangle$
(where $1$ is the current value), it yields the stream $\langle{\ldots,4,4,2,2}\rangle$.

To obtain the function, we need one past value from the context (for $\kvd{prev}~\ident{tick}$). The
returned function needs either none or one past value (thus a subtyping rule is required to type
it as requiring one past value). So, the annotations for (\emph{app}) are $\cclrd{m}=1, \cclrd{p}=1$.
The function is called with $\kvd{prev}~\ident{counter}$ as an argument, meaning that the result
is either the first or second past element. Given
$\ident{counter}\hspace{-0.1em}=\hspace{-0.1em}\langle{\ldots,5,4,3,2,1}\rangle$, the argument
is $\langle{\ldots,5,4,3,2}\rangle$ and so the overall result is a stream $\langle{\ldots,5,5,3,3}\rangle$.
From the argument, we get the requirement $\cclrd{n}=1$.

Using the (\emph{app}) rule, we get that the overall number of past elements needed is
$\mathit{max}(1, 1+1) = 2$. This should match the intuition about the code -- when the first function
is applied to the argument, the computation will first access $\kvd{prev}~\ident{tick}$ (using one
past value) and then $\kvd{prev}~(\kvd{prev}~\ident{counter}))$ (using two past values).


\paragraph{Semantics.}
The language discussed in this section is a \emph{causal} dataflow language. This means that
a computation can access \emph{past} values of the stream (but not future values). In the semantics,
we again need richer structure over the input.

Uustalu and Vene \cite{comonads-notions} model causal dataflow computations using a non-empty list
$\ident{NeList}~\tau = \tau \times (\ident{NeList}~\tau + 1)$ over the input. A function $\tau_1 \rightarrow \tau_2$
is thus modelled as $\ident{NeList}~\tau_1 \rightarrow \tau_2$. This model is difficult to implement
efficiently, as it creates unbounded lists of past elements.

The coeffect system tracks maximal number of past values and so we can define the semantics using
a list of fixed length. As with liveness, this is a data structure \emph{indexed} by the coeffect
annotation. We write $\tau^{\cclrd{n}}$ for a list containing $\cclrd{n}$ elements, which can be
also viewed as an $\cclrd{n}$-element product $\tau \times \ldots \times \tau$.

As with the previous examples, our semantics interprets a judgement using a (semantic) function;
functions in the language are modelled as functions taking a list of inputs:
%
\begin{equation*}
\begin{array}{rcl}
\sem{\coctx{x_1\!:\!\tau_1, \ldots, x_n\!:\!\tau_n}{ \cclrd{n} } \vdash e : \tau}
  &:& (\tau_1 \times \ldots \times \tau_n)^{\cclrd{n}+1} \rightarrow \tau\\
\sem{\tau_1 \xrightarrow{\cclrd{n}} \tau_2} &:& \tau_1^{\cclrd{n}+1} \rightarrow \tau_2
\end{array}
\end{equation*}
%
Note that the semantics requires one more value than is the number of past values. This is because
the first value is the current value and has to be always available, even when the annotation is
zero as in (\emph{var}).


%---------------------------------------------------------------------------------------------------

\begin{figure}[t]

\begin{equation*}
\begin{array}{ll}
\hspace{3em}\semdef
  {\coctx{\Gamma}{\cclrd{0}} \vdash x_i : \tau_i}
  {\lambda \langle(x_1, \ldots, x_n)\rangle . x_i}
& (\emph{var})
\\[1.5em]
\hspace{-0.75em}\emdeff
  {\coctx{\Gamma}{\cclrd{n}} \vdash e : \tau}
  {\begin{array}{l}\sem{\coctx{\Gamma}{\cclrd{n}+1} \vdash \kvd{prev}~e : \tau}\\[-0.25em]~\end{array}\hspace{-0.5em}~}
  {f}
  {\hspace{-0.5em}\begin{array}{l}\lambda \langle \mathbf{v}_0, \ldots, \mathbf{v}_{\cclrd{n}+1} \rangle .\\[-0.25em]
    \qquad f~\langle \mathbf{v}_1, \ldots, \mathbf{v}_{ \cclrd{n}+1 }\rangle\end{array}\hspace{-1em}~}
& (\emph{prev})
\\[1.5em]
\hspace{3.25em}\emdeff
  {\coctx{\Gamma}{\cclrd{n'}} \vdash e : \tau}
  {\begin{array}{l}\sem{\coctx{\Gamma}{\cclrd{n}} \vdash e : \tau}\\[-0.25em]~\end{array}\hspace{-0.5em}~}
  {f}
  {\hspace{-0.5em}\begin{array}{l}\lambda \langle \mathbf{v}_0, \ldots, \mathbf{v}_{\cclrd{n}} \rangle .\\[-0.25em]
    \qquad f~\langle \mathbf{v}_0, \ldots, \mathbf{v}_{ \cclrd{n'} }\rangle\end{array}\hspace{-1em}~}
& (\emph{sub})
\\[1.5em]
\hspace{-1.5em}\emdeff
  {\coctx{\Gamma,y:\tau_1}{\cclrd{n}} \vdash e : \tau_2}
  {\begin{array}{l}\sem{\coctx{\Gamma}{\cclrd{n}} \vdash \lambda y. e : \tau_1 \xrightarrow{\cclrd{n}} \tau_2}\\[-0.25em]~\end{array}\hspace{-0.5em}~}
  {f}
  {\hspace{-0.5em}\begin{array}{l}
  \lambda \langle \mathbf{v}_0, \ldots \mathbf{v}_{\cclrd{n}} \rangle . \lambda \langle y_0, \ldots, y_{\cclrd{n}}\rangle .\\[-0.25em]
  \quad f~\langle (\mathbf{v}_0, y_0), \ldots, (\mathbf{v}_{ \cclrd{n}  }, y_{ \cclrd{n} } ) \rangle
  \end{array}\hspace{-1em}~}
& (\emph{abs})
\\[1.5em]
\hspace{0.25em}\emdefff
  {\coctx{\Gamma}{\cclrd{r}} \vdash e_1 : \tau_1 \xrightarrow{\cclrd{t}} \tau_2}
  {\coctx{\Gamma}{\cclrd{s}} \vdash e_2 : \tau_1}
  {\begin{array}{l}\coctx{\Gamma}{\cclrd{\textnormal{max}}(\cclrd{m}, \cclrd{n+p})} \\[-0.25em]\qquad \vdash e_1~e_2 : \tau_2\\[-0.25em]~\\[-0.25em]~\end{array}}
  {f_1}
  {f_2}
  {\hspace{-0.5em}\begin{array}{l}
  \lambda \langle\mathbf{v}_0, \ldots, \mathbf{v}_{ \textnormal{max}(\cclrd{m}, \cclrd{n}+\cclrd{p})} \rangle . \\[-0.1em]
  \quad (f_1~\langle\mathbf{v}_0, \ldots, \mathbf{v}_{ \cclrd{m} }\rangle)\\[-0.1em]
  ~~\qquad\langle\; f_2~\langle\mathbf{v}_0, \ldots, \mathbf{v}_{ \cclrd{n} }\rangle,\ldots,\\[-0.25em]
  ~~\qquad~\;\, f_2~\langle\mathbf{v}_{ \cclrd{p} }, \ldots, \mathbf{v}_{ \cclrd{n}+\cclrd{p} }\rangle~\rangle
  \end{array}\hspace{-0.5em}~}
& (\emph{app})
\\[1.5em]
\end{array}
\end{equation*}

\figcaption{Semantics showing how past values are accessed in a dataflow language}
\label{fig:applications-flat-dfsem}
\end{figure}

%---------------------------------------------------------------------------------------------------

The rules defining the semantics are shown in Figure~\ref{fig:applications-flat-dfsem}. The
semantics of the context is a \emph{list of products}. To make the rules easier to follow, we write
$\langle \mathbf{v}_1, \ldots, \mathbf{v}_n \rangle$ for an $n$-element list containing products.
Products that model the entire context such as $\mathbf{v}_1$ are written in bold. When we access
individual variables, we write $\mathbf{v} = (x_1, \ldots, x_m)$ where $x_i$ denote individual
variables of the context.

In (\emph{var}), the context is a singleton-list containing a product of variables, from which
we project the right one. In (\emph{prev}) and (\emph{sub}), we drop some of the elements from
the history (from the front and end, respectively) and then evaluate the original expression.

Lambda abstractions (\emph{abs}) receives two lists of the same size -- one containing values of
the variables (list of products) from the declaration site $\langle \mathbf{v}_0, \ldots, \mathbf{v}_{\cclrd{n}} \rangle$
and one containing the argument (list of values) provided by the call site $\langle y_0, \ldots, y_{\cclrd{n}} \rangle$.
The semantics applies the well-known \emph{zip} operation on the lists and passes the result to the
body.

Finally, application (\emph{app}) uses the input context in two ways, which gives rise to the
two requirements combined using \emph{max}. First, it evaluates the expression $e_1$ which is
called with the past $\cclrd{m}$ values. The resulting function $g$ is then sequentially composed
with the semantics of $e_2$. To call the function, we need to evaluate $e_2$ repeatedly -- namely,
$\cclrd{p}+1$ times, which results in the overall requirement for $\cclrd{n}+\cclrd{p}$ past values.

\paragraph{Summary.}
Type systems have been used in the context of dataflow languages, for example to check
initialization properties \cite{app-dataflow-init}, but to our knowledge, not for checking
the maximal number of required past values. Thus this section serves not just as an example,
but also shows how coeffects can lead to novel results.

The most interesting point about the dataflow system is that it is remarkably similar to our
earlier liveness example. In the type system, abstraction (\emph{abs}) duplicates the context
requirements and application (\emph{abs}) arises from sequential and pointwise composition.
We capture this striking similarity in Chapter~\ref{ch:flat}. Before doing that, we look at one
more example and then explore the \emph{structural} class of systems.

% --------------------------------------------------------------------------------------------------

\subsection{Permissions and safe locking}
In the implicit parameters and dataflow examples, the context provides additional resources or
values that may be accessed at runtime. However, coeffects can also track \emph{permissions} or
\emph{capabilities} to perform some operation. We can invert the intuition behind liveness and
use it as a trivial example. When the context is live, it contains a \emph{permission} to access
variables. In this section, we briefly consider a system for safe locking of Flanagan and
Abadi~\cite{app-safe-locking} as one, more advanced example. Calculus of capabilities of
Cray et al.~\cite{app-capabilities} is discussed later in Section~\ref{sec:applications-active-ccc}.

\paragraph{Safe locking.}
The system for safe locking prevents race conditions (by only allowing access to mutable state
under a lock) and avoids deadlocks (by imposing strict partial order on locks). The following
program uses a mutable state under a lock:
%
\begin{equation*}
\begin{array}{l}
\kvd{newlock}~l:\rho~\kvd{in}\\[-0.25em]
\kvd{let}~\ident{state}~=~\ident{ref}_\rho~10~\kvd{in}\\[-0.25em]
\kvd{sync}~l~(!\ident{state})
\end{array}
\end{equation*}
%
The declaration \kvd{newlock} creates a lock $l$ protecting memory region $\rho$. We can than
allocate mutable variables in that memory region (second line). An access to one or more mutable
variables is only allowed in scope that is protected by a lock. This is done using the \kvd{sync} keyword,
which locks a lock and evaluates an expression in a context that contains permission to access
memory region of the lock ($\rho$ in the above example).

The type system for safe locking associates a list of acquired locks with the context.
Interestingly, the original presentation of the system by Flanagan and Abadi \cite{app-safe-locking}
uses a coeffect-style judgements of a form $\Gamma; p \vdash e : \tau$ where $p$ is a list of
accessible regions (protected by an acquired lock). Using our notation, the rule for \kvd{sync}
looks as follows:

\noindent
\vspace{-0.5em}
\begin{equation*}
\tyrule{sync}
  { \coctx{\Gamma}{\cclrd{p}} \vdash e_1 : m&
    \coctx{\Gamma}{\cclrd{p} \cup \{m\} } \vdash e_2 : \tau }
  { \coctx{\Gamma}{\cclrd{p}} \vdash \kvd{sync}~e_1~e_2 : \tau }
\end{equation*}
\vspace{-0.5em}

\noindent
The rule requires that $e_1$ yields a value of a singleton type $m$. The type is added as an
indicator of the locked region to the context $\cclrd{p}\cup \{m\}$ which is then used to evaluate
the expression $e_2$.

\paragraph{Summary.}
Despite attaching annotations to the variable context, the system for safe locking uses
effect-style lambda abstraction. Lambda abstraction associates all requirements with the
call site -- a lambda function created under a lock cannot access protected memory available at
the time of creation. It will be executed later and can only access the memory available then.
This suggests that safe locking is better seen as an effect system.

Another interesting aspect is the extension to avoid deadlocks. In that case, the type system
needs to reject programs that acquire locks in an invalid order. One way to model this is to
replace $\cclrd{p} \cup \{m\}$ with a \emph{partial} operation $\cclrd{p} \uplus \{m\}$ which
is only defined when the lock $m$ can be added to the set $\cclrd{p}$. Supporting partial
operations on coeffect annotations is an interesting future extension for coeffect systems.



% ==================================================================================================
%
% 	  ###    #                           #                           ##
% 	 #   #   #                           #                            #
% 	 #      ####   # ##   #   #   ###   ####   #   #  # ##    ###     #
% 	  ###    #     ##  #  #   #  #   #   #     #   #  ##  #      #    #
% 	     #   #     #      #   #  #       #     #   #  #       ####    #
% 	 #   #   #  #  #      #  ##  #   #   #  #  #  ##  #      #   #    #
% 	  ###     ##   #       ## #   ###     ##    ## #  #       ####   ###
%
% ==================================================================================================

\section{Structural coeffect systems}
\label{sec:applications-structural}

In structural coeffect systems, the additional information is associated with individual variables.
This is very often information about how the variables are used, or, in which contexts they are used.
In Chapter~\ref{ch:intro}, we introduced the idea using an example that tracks array access patterns.
Each variable is annotated with a range specifying which elements of the corresponding array
may be accessed.

In this section, we look at three examples in detail -- we revisit liveness and show a practically
useful structural version of the system; we consider an example inspired by linear logic; finally,
we revisit dataflow to get a more precise analysis. Although quite different, the common pattern
among these three examples is somewhat easier to see, because they all track information about
variable usage. We finish the section with a brief outline of several other applications.

%---------------------------------------------------------------------------------------------------

\subsection{Liveness analysis revisited}
\label{sec:applications-struct-live}

The flat system for liveness analysis presented in Section~\ref{sec:applications-flat-live} is
interesting from a theoretical perspective, but it is not practically useful. Here, we
revisit the problem and define a structural system that tracks liveness per-variable.

\paragraph{Structural liveness.}
Recall two examples discussed earlier where the flat liveness analysis marked the whole context
as (syntactically) live, despite the fact part of it was (semantically) dead:
%
\begin{equation*}
\begin{array}{l}
\kvd{let}~\ident{constant} = \lambda y \rightarrow \lambda x \rightarrow y\\[-0.25em]
\kvd{let}~\ident{answer} = (\lambda x \rightarrow x)~42
\end{array}
\end{equation*}
%
In the first case, the variable $x$ is dead, but was marked as live. In the second example, the
declaration site of the \ident{answer} value is dead, but was marked as live. This is because in
both of the expressions, \emph{some} variable is accessed. However, the (\emph{abs}) rule of flat
liveness has no way of determining \emph{which} variables are used by the body -- and, in particular,
whether the accessed variable is the \emph{bound} variable or some of the \emph{free} variables.

As discussed earlier, we can resolve this by attaching a \emph{vector} of liveness annotations to
a \emph{vector} of variables. In the first example, the available variables are $y$ and $x$, so
the variable context $\Gamma$ is a vector $\langle y\!:\!\tau, x\!:\!\tau \rangle$. Only the variable $y$
is used and so the annotated context is: $\coctx{y\!:\!\tau, x\!:\!\tau}{ \alift{\cclrd{\cclrd{\ident{L}}}, \cclrd{\cclrd{\ident{D}}}} }$.
When writing the contexts, we omit angle brackets around variables, but it should still be viewed
as a vector. There are two important points:

\begin{itemize}
\item The fact that variables are now a vector means that we cannot freely re-order them. This
  guarantees that $\coctx{x\!:\!\tau, y\!:\!\tau}{\alift{\cclrd{\cclrd{\ident{L}}}, \cclrd{\cclrd{\ident{D}}}}}$
  can not be confused with $\coctx{y\!:\!\tau, x\!:\!\tau}{\alift{\cclrd{\cclrd{\ident{L}}}, \cclrd{\cclrd{\ident{D}}}}}$.
  We need to define the type system in a way that is similar to substructural systems
  (discussed in Section~\ref{sec:path-logic}) and add explicit rules for manipulating
  the context.

\item We choose to attach a vector of annotations to a vector of variables, rather than attaching
  individual annotations to individual variables. This lets us unify and combine flat and
  structural systems as discussed in Section~\ref{sec:further-unified}, but the alternative is briefly
  explored in Section~\ref{sec:further-meta}.
\end{itemize}

\paragraph{Type system.}
The structural system for liveness uses the same two-point lattice of annotations
$\mathcal{L}=\{ \cclrd{\cclrd{\ident{L}}}, \cclrd{\cclrd{\ident{D}}} \}$ that was used by the flat system. We also use the
$\cclrd{\sqcup}, \cclrd{\sqcap}$ and $\cclrd{\sqsubseteq}$ operators that are defined in Figure~\ref{fig:applications-flat-livealg}.

The rules of the system are split into two groups. Figure~\ref{fig:applications-struct-live} (a) shows
the standard syntax-driven rules plus subcoeffecting. In (\emph{var}), the context contains just the
single accessed variable, which is annotated as live. Unused variables can be introduced using weakening.
A constant (\emph{const}) is accessed in an empty context, which also carries no annotations. The
subcoeffecting rule (\emph{sub}) uses a pointwise extension of the $\cclrd{\sqsubseteq}$ relation over two
vectors as defined in Section~\ref{sec:applications-strucutre-vec}.

% --------------------------------------------------------------------------------------------------

\begin{figure}[t]
a.) Ordinary, syntax-driven rules along with subcoeffecting

\begin{equation*}
\tyrule{var}{}
  {\coctx{x \!:\! \tau}{\alift{\cclrd{\ident L}}} \vdash x : \tau}
\end{equation*}
\begin{equation*}
\tyrule{const}
  {c:\tau \in \Delta}
  {\coctx{()}{\alift{}} \vdash c : \tau}
\end{equation*}
\begin{equation*}
\tyrule{abs}
  {\coctx{\Gamma, x \!:\! \tau_1}{\aclrd{\textbf{r}} \cons \alift{\cclrd{s}}} \vdash e : \tau_2}
  {\coctx{\Gamma}{\aclrd{\textbf{r}}} \vdash \lambda x . e : \tau_1 \xrightarrow{\cclrd{s}} \tau_2}
\end{equation*}
\begin{equation*}
\tyrule{app}
  { \coctx{\Gamma_1}{\aclrd{\textbf{r}}} \vdash e_1 : \tau_1 \xrightarrow{\cclrd{t}} \tau_2 \quad
    \coctx{\Gamma_2}{\aclrd{\textbf{s}}} \vdash e_2 : \tau_1}
  { \coctx{\Gamma_1, \Gamma_2}{\aclrd{\textbf{r}} \cons (\cclrd{t} \;\cclrd{\sqcup}\; \aclrd{\textbf{s}})} \vdash e_1 \, e_2 : \tau_2}
\end{equation*}
\begin{equation*}
\tyrule{let}
  { \coctx{\Gamma_1, x \!:\! \tau_1}{\aclrd{\textbf{r}} \times \langle{\cclrd{t}}\rangle} \vdash e_1 : \tau_2 \quad
    \coctx{\Gamma_2}{\aclrd{\textbf{s}}} \vdash e_2 : \tau_1}
  {\coctx{\Gamma_1, \Gamma_2}{\aclrd{\textbf{r}} \times (\cclrd{t} \;\cclrd{\sqcup}\; \aclrd{\textbf{s}})} \vdash \kvd{let}~x=e_2~\kvd{in}~e_1 : \tau_2}
\end{equation*}
\begin{equation*}
\tyrule{sub}
  {\coctx{\Gamma}{\aclrd{\textbf{r}}} \vdash e : \tau}
  {\coctx{\Gamma}{\aclrd{\textbf{r'}}} \vdash e : \tau}~\aclrd{\textbf{r}} \;\cclrd{\sqsubseteq}\; \aclrd{\textbf{r'}}
\end{equation*}
\vspace{1em}

b.) Structural rules for context manipulation

\begin{equation*}
\tyrule{weak}
  {\coctx{\Gamma}{ \aclrd{\textbf{r}} } \vdash e : \sigma}
  {\coctx{\Gamma,x \!:\! \tau}{\aclrd{\textbf{r}} \cons \alift{ \cclrd{\cclrd{\ident{D}}} }} \vdash e : \sigma}
\end{equation*}
\begin{equation*}
\tyrule{exch}
  {\coctx{\Gamma_1,x\!:\!\tau',y\!:\!\tau,\Gamma_2}{\aclrd{\textbf{r}} \cons\alift{\cclrd{s},\cclrd{t}} \cons \aclrd{\textbf{q}}} \vdash e : \sigma}
  {\coctx{\Gamma_1,y\!:\!\tau,x\!:\!\tau',\Gamma_2}{\aclrd{\textbf{r}} \cons\alift{\cclrd{t},\cclrd{s}} \cons \aclrd{\textbf{q}}} \vdash e : \sigma}
\quad
\begin{array}{l}
\slen{\Gamma_1} = \slen{\aclrd{\textbf{r}}}\\[-0.25em]
\slen{\Gamma_2} = \slen{\aclrd{\textbf{s}}}
\end{array}
\end{equation*}
\begin{equation*}
\tyrule{contr}
  {\coctx{\Gamma_1,y\!:\!\tau,z\!:\!\tau,\Gamma_2}{\aclrd{\textbf{r}} \cons \alift{\cclrd{s},\cclrd{t}} \cons \aclrd{\textbf{q}}} \vdash e : \sigma}
  {\coctx{\Gamma_1,x\!:\!\tau,\Gamma_2}{\aclrd{\textbf{r}} \cons \alift{\cclrd{s}\cclrd{\sqcap}\cclrd{t}} \cons \aclrd{\textbf{q}}} \vdash \subst{\subst{e}{z}{x}}{y}{x} : \sigma}
~
\begin{array}{l}
\slen{\Gamma_1} = \slen{\aclrd{\textbf{r}}}\\[-0.25em]
\slen{\Gamma_2} = \slen{\aclrd{\textbf{s}}}
\end{array}
\end{equation*}

\figcaption{Structural coeffect liveness analysis}
\label{fig:applications-struct-live}
\vspace{-1em}
\end{figure}

% --------------------------------------------------------------------------------------------------

In the (\emph{abs}) rule, the variable context of the body $\Gamma, x\!:\!\tau_1$ is annotated with
a vector $\mathbf{\aclrd{r}}\cons\alift{\cclrd{s}}$, where the vector $\mathbf{\aclrd{r}}$ corresponds
to $\Gamma$ and the singleton annotation $\cclrd{s}$ corresponds to the variable $x$. Thus, the
function is annotated with $\cclrd{s}$. Note that the free-variable context is annotated with vectors,
but functions take only a single input and so are annotated with primitive annotations.

The (\emph{app}) rule is similar to function applications in flat systems, but there is an important
difference. In structural systems, the two sub-expressions have separate variable contexts
$\Gamma_1$ and $\Gamma_2$. Therefore, the composed expression just concatenates the variables
and their corresponding annotations. (We can still use the same variable in both sub-expressions
thanks to the structural contraction rule.)

The context $\Gamma_1$ is used to evaluate $e_1$ and is thus annotated with $\aclrd{\mathbf{r}}$.
The annotation for $\Gamma_2$ is more interesting. It is a result of sequential composition of two
semantic functions -- the first one takes the (multi-variable) context $\Gamma_2$ and evaluates
$e_2$; the second takes the result of type $\tau_1$ and passes it to the function $\tau_1 \xrightarrow{\cclrd{t}} \tau_2$.
The composition is defined as follows:
%
\begin{equation*}
g : \tau_1 \times \ldots \times \tau_n \xrightarrow{\aclrd{\mathbf{s}}} \sigma
\qquad
f : \sigma \xrightarrow{\cclrd{t}} \tau
\qquad
f \circ g : \tau_1 \times \ldots \times \tau_n \xrightarrow{\cclrd{t} \,\cclrd{\sqcup}\, \aclrd{\mathbf{s}}} \tau
\end{equation*}
%
This definition is only for illustration and is revised in Chapter~\ref{ch:structural}. The function
$g$ takes a product of multiple variables (and is annotated with a vector). The function $f$ takes
just a single value and is annotated with the scalar. As in the flat system, sequential composition
is modelled using $\cclrd{\sqcup}$, but here we use a scalar-vector extension of the operation. Finally,
the (\emph{let}) rule follows similar reasoning (and also corresponds to the typing of $(\lambda x.e_2)~e_1$).

\paragraph{Structural typing rules.}
The structural typing rules are shown in Figure~\ref{fig:applications-struct-live} (b). They mirror
the rules know from substructural type systems (Section~\ref{sec:path-logic}). Weakening (\emph{weak})
extends the context with a single unused variable $x$ and adds the $\cclrd{\ident{D}}$ annotation to the
vector of coeffects.

The variable is always added to the end as in the (\emph{abs}) rule. However, the exchange rule
(\emph{exch}) lets us arbitrarily reorder variables. It flips the variables $x$ and $x'$ and their
corresponding coeffect annotations in the vector. This is done by requiring that the lengths of the
remaining, unchanged, parts of the vectors match.

Finally, contraction (\emph{contr}) makes it possible to use a single variable multiple times.
Given a judgement that contains variables $y$ and $z$, we can derive a judgement for an expression
where both $z$ and $y$ are replaced by a single variable $x$. Their annotations $\cclrd{s}, \cclrd{t}$
are combined into $\cclrd{s} \;\cclrd{\sqcap}\; \cclrd{t}$, which means that $x$ is live if either $z$ or $y$
were live in the original expression.

\paragraph{Example.} To demonstrate how the system works, we consider the expression
$(\lambda x . v)~y$. This is similar to an example where flat liveness mistakenly marks
the entire context as live. Despite the fact that the variable $y$ is accessed (syntactically), it
is not live -- because the function that takes it as an argument always returns $v$.

The typing derivation for the body uses (\emph{var}) and (\emph{abs}). However, we also need (\emph{weak})
to add the unused variable $x$ to the context:
%
\begin{equation*}
\tyrule{weak}
{ \tyruler{var}
    {}
    { \coctx{v\!:\!\tau}{\alift{ \cclrd{\cclrd{\ident{L}}} }} \vdash v : \tau } }
{ \tyruler{abs}
    { \coctx{v\!:\!\tau, x\!:\!\tau}{\alift{ \cclrd{\cclrd{\ident{L}}},\cclrd{\cclrd{\ident{D}}} }} \vdash v : \tau }
    { \coctx{v\!:\!\tau}{\alift{ \cclrd{\cclrd{\ident{L}}} }} \vdash (\lambda x . v) : \tau \xrightarrow{\cclrd{\cclrd{\ident{D}}}} \tau }}
\end{equation*}
%
The interesting part is the use of the (\emph{app}) rule in the next step. Although the variable $y$ is live in the expression $y$,
it is marked as dead in the overall expression, because the function is annotated with $\ident{\cclrd{D}}$:
%
\begin{equation*}
\hspace{-2em}
\tyrule{app}
  {
    \begin{array}{l}
    \vspace{-1.2em}
    \coctx{v\!:\!\tau}{\alift{ \cclrd{\cclrd{\ident{L}}} }} \vdash (\lambda x . v) : \tau \xrightarrow{\cclrd{\cclrd{\ident{D}}}} \tau
    \end{array} &
    \tyruler{var}{}{\coctx{y\!:\!\tau}{\alift{ \cclrd{\cclrd{\ident{L}}} }} \vdash y : \tau}
  }
  {
  \inference
  	{ \coctx{v\!:\!\tau, y\!:\!\tau}{\alift{ \cclrd{\cclrd{\ident{L}}} } \cons\, ({ \cclrd{\cclrd{\ident{D}}} \,\cclrd{\sqcup}\, \alift{\cclrd{\cclrd{\ident{L}}}} })}
  	     \vdash (\lambda x . v)~y : \tau }
  	{ \coctx{v\!:\!\tau, y\!:\!\tau}{\alift{ \cclrd{\cclrd{\ident{L}}}, \cclrd{\cclrd{\ident{D}}} }} \vdash (\lambda x . v)~y : \tau }
  }
\end{equation*}
%
The application is written in two steps -- the first one directly applies the (\emph{app}) rule
and the second one simplifies the coeffect annotation. The key part is the use of the scalar-vector
operator $\cclrd{\cclrd{\ident{D}}} \;\cclrd{\sqcup}\; \alift{\cclrd{\cclrd{\ident{L}}}}$. Using the definition of the scalar-vector
extension, this equals $\alift{\cclrd{\cclrd{\ident{D}}} \;\cclrd{\sqcup}\; \cclrd{\cclrd{\ident{L}}}}$ which is
$\alift{\cclrd{\cclrd{\ident{D}}}}$.

% --------------------------------------------------------------------------------------------------

\paragraph{Semantics.}
When defining the semantics of flat liveness calculus, we used an indexed form of the option type
$1 + \tau$ (which is $1$ for dead contexts and $\tau$ for live contexts). In the semantics of
expressions, the type constructor was applied to the entire context, \ie~$1+(\tau_1 \times \ldots \times \tau_n)$.
In the structural version, the semantics applies the option type constructor to individual elements
of the free-variable context pair: $(1+\tau_1) \times \ldots \times (1+ \tau_n)$. For each variable,
the type is indexed by the corresponding annotation:

\begin{equation*}
\begin{array}{r}
\\[-0.5em]
\sem{\coctx{x_1\!:\!\tau_1, \ldots, x_n\!:\!\tau_n}{ \alift{\cclrd{r_1}, \ldots, \cclrd{r_n}} } \vdash e : \tau}
  ~:~ (\tau'_1 \times \ldots \times \tau'_n) \rightarrow \tau\\[0.5em]
\textnormal{where}~\tau'_i = \begin{cases}
\tau_i & (\cclrd{r_i} = \cclrd{\cclrd{\ident{L}}})\\
1      & (\cclrd{r_i} = \cclrd{\cclrd{\ident{D}}})
\end{cases}
\end{array}
\end{equation*}
\vspace{0.5em}

\noindent
Note that the product of the free variables is not an ordinary tuple of our language, but a special
construction (we return to this topic in Section~\ref{sec:struct-semantics}). This follows from the
asymmetry of $\lambda$-calculus, as discussed in Section~\ref{sec:applications-strucutre-vec}. Functions
take just a single input and so they are interpreted in the same way as in flat calculus:

\vspace{-0.5em}
\begin{equation*}
\sem{\tau_1 \xrightarrow{\aclrd{\cclrd{\ident{L}}}} \tau_2} = \tau_1 \rightarrow \tau_2 \qquad\qquad
\sem{\tau_1 \xrightarrow{\aclrd{\cclrd{\ident{D}}}} \tau_2} = 1 \rightarrow \tau_2
\end{equation*}
\vspace{-0.5em}

\noindent
The rules that define the semantics are shown in Figure~\ref{fig:applications-struct-livesem}.
To make the definition simpler, we are somewhat vague when working with products. We write
variables of product type such as $\mathbf{v}$ in bold-face and individual values like $x$ in
normal face. We freely re-associate products and so $(\mathbf{v}, x)$ should not be seen as a
nested product, but simply as a product containing all variables from the product $\mathbf{v}$
together with one additional variable $x$ at the end. We shall be more precise in
Chapter~\ref{ch:structural}.

In (\emph{var}), the context contains just a single variable and so we do not even need to apply
projection; (\emph{cosnt}) receives no variables and uses global constant lookup function $\delta$.
In (\emph{abs}), we obtain two parts of the context and combine them into $(\mathbf{v}, x)$. This
works the same way regardless of whether the variables are live or dead. For simplicity, we omit
subcoeffecting, which just turns some of the available values $v_i$ to unit values $()$.

As dictated by the semantics, the application again needs to ``implement'' dead code elimination
(otherwise the type system would be unsound). When the input parameter of the function $f_1$ is live
(\emph{app-1}), we first evaluate $e_2$ and then pass the result to $f_1$. When the parameter is
dead (\emph{app-2}), we do not need to evaluate $e_2$ and so all values in $\mathbf{v_2}$ can be
dead, \ie~$()$.

In the structural rules, (\emph{weak}) receives context containing a dead variable as the last one.
It drops the $()$ value and evaluates the expression in a context $\mathbf{v}$. Exchange (\emph{exch})
simply swaps two variables. In contraction, we duplicate the value (no matter whether it is dead
or live) and we use an auxiliary definition $\restr{x}{\cclrd{r}}$ to replace a live value with
$()$ when only one of the contracted variables is live.

% --------------------------------------------------------------------------------------------------

\begin{figure}[t]
a.) Semantics of ordinary expressions
\begin{equation*}
\hspace{-1em}
\begin{array}{ll}
\\[-2.5em]
\hspace{4.75em}\semdef
  {\coctx{x \!:\! \tau}{\alift{\cclrd{\ident L}}} \vdash x : \tau}
  {\lambda (x). x}
& (\emph{var})
\\[-0.5em]
\hspace{4.5em}\semdef
  {\coctx{()}{\alift{}} \vdash n : \ident{num}}
  {\lambda (). n}
& (\emph{num})
\\[1.5em]
\hspace{1.75em}\semdeff
  {\coctx{\Gamma, y \!:\! \tau_1}{\aclrd{\textbf{r}} \cons \alift{\cclrd{s}}} \vdash e : \tau_2}
  {\coctx{\Gamma}{\aclrd{\textbf{r}}} \vdash \lambda y . e : \tau_1 \xrightarrow{\cclrd{s}} \tau_2}
  {f}
  {\lambda \textbf{v} . \lambda y . f~(\textbf{v}, y)}
& (\emph{abs})
\\[1.5em]
\hspace{-0.75em}\semdefff
  {\coctx{\Gamma_1}{\aclrd{\textbf{r}}} \vdash e_1 : \tau_1 \xrightarrow{\cclrd{\cclrd{\ident{L}}}} \tau_2}
  {\coctx{\Gamma_2}{\aclrd{\textbf{s}}} \vdash e_2 : \tau_1}
  {\coctx{\Gamma_1, \Gamma_2}{\aclrd{\textbf{r}} \cons (\cclrd{\cclrd{\ident{L}}} \,\cclrd{\sqcup}\, \aclrd{\textbf{s}})} \vdash e_1 \, e_2 : \tau_2}
  {f_1}
  {f_2}
  {\lambda (\mathbf{v_1}, \mathbf{v_2}) . (f_1\;\mathbf{v_1})~(f_2\;\mathbf{v_2})}
& (\emph{app-1})
\\[2.5em]
\hspace{-0.75em}\semdefff
  {\coctx{\Gamma_1}{\aclrd{\textbf{r}}} \vdash e_1 : \tau_1 \xrightarrow{\cclrd{\cclrd{\ident{D}}}} \tau_2}
  {\coctx{\Gamma_2}{\aclrd{\textbf{s}}} \vdash e_2 : \tau_1}
  {\coctx{\Gamma_1, \Gamma_2}{\aclrd{\textbf{r}} \cons (\cclrd{\cclrd{\ident{D}}} \,\cclrd{\sqcup}\, \aclrd{\textbf{s}})} \vdash e_1 \, e_2 : \tau_2}
  {f_1}
  {\_}
  {\lambda (\mathbf{v_1}, \mathbf{v_2}) . (f_1~\mathbf{v_1})~()}
& (\emph{app-2})
\end{array}
\end{equation*}
\vspace{1em}

b.) Semantics of structural context manipulation\\
Using the auxiliary definition $\restr{x}{\cclrd{\ident{L}}} = x$ and $\restr{x}{\cclrd{\ident{D}}} = ()$:

\begin{equation*}
\begin{array}{ll}
\hspace{2em}\semdeff
  {\coctx{\Gamma}{ \aclrd{\textbf{r}} } \vdash e : \sigma}
  {\coctx{\Gamma,x \!:\! \tau}{\aclrd{\textbf{r}} \cons \alift{ \cclrd{\cclrd{\ident{D}}} }} \vdash e : \sigma}
  {f}
  {\lambda(\textbf{v}, ()) . f~\textbf{v} }
& (\emph{weak})
\\[1.5em]
\hspace{0em}\mdeff
  {\begin{array}{l}\sem{\coctx{\Gamma_1,x\!:\!\tau_1,y\!:\!\tau_2,\Gamma_2\\[-0.25em]\qquad}
     {\aclrd{\textbf{r}} \atimes \alift{\cclrd{s},\cclrd{t}} \atimes \aclrd{\textbf{q}}} \vdash e : \tau}\end{array}}
  {\begin{array}{l}\sem{\coctx{\Gamma_1,y\!:\!\tau_2,x\!:\!\tau_1,\Gamma_2\\[-0.25em]\qquad}{\aclrd{\textbf{r}}
          \atimes \alift{\cclrd{t},\cclrd{s}} \atimes \aclrd{\textbf{q}}} \vdash e : \tau}\end{array}}
  {\begin{array}{l}~\\[-0.25em]\hspace{-0.5em}f\end{array}}
  {\hspace{-0.5em}\begin{array}{l}
  \lambda(\mathbf{v_1}, y, x, \mathbf{v_2}) .\\[-0.25em]
    \quad f~(\mathbf{v_1}, x, y, \mathbf{v_2})\end{array} }
& (\emph{exch})
\\[1.5em]
\hspace{-0.5em}\mdeff
  {\begin{array}{l}\sem{\coctx{\Gamma_1,y\!:\!\tau_1,z\!:\!\tau_1,\Gamma_2\\[-0.25em]\qquad}
      {\aclrd{\textbf{r}} \atimes \alift{\cclrd{s},\cclrd{t}} \atimes \aclrd{\textbf{q}}} \vdash e : \tau}\end{array}}
  {\begin{array}{l}\sem{\coctx{\Gamma_1,x\!:\!\tau_1,\Gamma_2}{\aclrd{\textbf{r}}
          \atimes \alift{\cclrd{s} \,\cpar\, \cclrd{t}} \atimes \aclrd{\textbf{q}}} \\[-0.25em] \qquad\vdash \subst{e}{z,y}{x} : \tau}\end{array}}
  {\begin{array}{l}~\\[-0.25em]\hspace{-0.5em}f\end{array}}
  {\hspace{-0.5em}\begin{array}{l}\lambda(\mathbf{v_1}, x, \mathbf{v_2}) .\\[-0.25em]
  \quad f~(\mathbf{v_1}, \restr{x}{\cclrd{s}}, \restr{x}{\cclrd{t}}, \mathbf{v_2})\end{array}\hspace{-0.5em}~}
& (\emph{contr})
\end{array}
\end{equation*}

\vspace{-0.5em}
\figcaption{Semantics of structural liveness}
\label{fig:applications-struct-livesem}
\end{figure}
\vspace{-0.5em}

% --------------------------------------------------------------------------------------------------

\paragraph{Summary.}
The structural liveness calculus is a typical example of a system that tracks per-variable
annotations. In a number of ways, the system is simpler than the flat coeffect calculi. In
lambda abstraction, we simply annotate function with the annotation of a matching variable
(this rule is the same for all upcoming systems). In application, the \emph{pointwise} composition
is no longer needed, because the sub-expressions use separate contexts. On the other hand,
we had to add weakening, contraction and exchange rules to let us manipulate contexts.

The semantics of weakening demonstrates an important point about coeffects that may be quite
confusing. When we read the \emph{typing rule} from top to bottom, weakening adds a variable
to the context. When we read the \emph{semantic rule}, weakening drops a variable value from the
context! This duality is caused by the fact that coeffects talk about context -- they describe
how to build the context required by the sub-expressions and so the semantics implements
transformation from the context in the (typing) conclusion to the (typing) assumption. How
should coeffects be understood, in general, is discussed further in Section~\ref{sec:flat-calculus-understanding}.

The structural systems discussed in the upcoming sections are remarkably similar to the one
shown here. We discuss two more examples to explore the design space, but omit details shared with
the system in this section.

%---------------------------------------------------------------------------------------------------

\subsection{Bounded variable use}
\label{sec:applications-struct-bll}

Liveness analysis checks whether a variable is used or unused. With structural coeffects, we can go
further and track how many times is the variable accessed. Girard et al. \cite{logic-bounded} coined
this idea as \emph{bounded linear logic} and use it to restrict well-typed programs to
polynomial-time algorithms. We first introduce the system in our, coeffect, style and then
relate it with the original formulation.

% --------------------------------------------------------------------------------------------------

\begin{figure}[t]
a.) Ordinary, syntax-driven rules along with subcoeffecting
\begin{equation*}
\tyrule{var}{}
  {\coctx{x \!:\! \tau}{\alift{\cclrd{1}}} \vdash x : \tau}
\end{equation*}
\begin{equation*}
\tyrule{abs}
  {\coctx{\Gamma, x \!:\! \tau_1}{\aclrd{\textbf{r}} \cons \alift{\cclrd{s}}} \vdash e : \tau_2}
  {\coctx{\Gamma}{\aclrd{\textbf{r}}} \vdash \lambda x . e : \tau_1 \xrightarrow{\cclrd{s}} \tau_2}
\end{equation*}
\begin{equation*}
\tyrule{app}
  { \coctx{\Gamma_1}{\aclrd{\textbf{r}}} \vdash e_1 : \tau_1 \xrightarrow{\cclrd{t}} \tau_2 \quad
    \coctx{\Gamma_2}{\aclrd{\textbf{s}}} \vdash e_2 : \tau_1}
  { \coctx{\Gamma_1, \Gamma_2}{\aclrd{\textbf{r}} \cons (\cclrd{t} \ast \aclrd{\textbf{s}})} \vdash e_1 \, e_2 : \tau_2}
\end{equation*}
\begin{equation*}
\tyrule{let}
  { \coctx{\Gamma_1, x \!:\! \tau_1}{\aclrd{\textbf{r}} \times \langle{\cclrd{t}}\rangle} \vdash e_1 : \tau_2 \quad
    \coctx{\Gamma_2}{\aclrd{\textbf{s}}} \vdash e_2 : \tau_1}
  {\coctx{\Gamma_1, \Gamma_2}{\aclrd{\textbf{r}} \times (\cclrd{t} \ast \aclrd{\textbf{s}})} \vdash \kvd{let}~x=e_2~\kvd{in}~e_1 : \tau_2}
\end{equation*}
\begin{equation*}
\tyrule{sub}
  {\coctx{\Gamma}{\aclrd{\textbf{r}}} \vdash e : \tau}
  {\coctx{\Gamma}{\aclrd{\textbf{r'}}} \vdash e : \tau}~\aclrd{\textbf{r}} \leq \aclrd{\textbf{r'}}
\end{equation*}
\vspace{0.5em}

b.) Structural rules for context manipulation
\begin{equation*}
\tyrule{weak}
  {\coctx{\Gamma}{ \aclrd{\textbf{r}} } \vdash e : \sigma}
  {\coctx{\Gamma,x \!:\! \tau}{\aclrd{\textbf{r}} \cons \alift{ \cclrd{0} }} \vdash e : \sigma}
\end{equation*}
\begin{equation*}
\tyrule{exch}
  {\coctx{\Gamma_1,x\!:\!\tau',y\!:\!\tau,\Gamma_2}{\aclrd{\textbf{r}} \cons\alift{\cclrd{s},\cclrd{t}} \cons \aclrd{\textbf{q}}} \vdash e : \sigma}
  {\coctx{\Gamma_1,y\!:\!\tau,x\!:\!\tau',\Gamma_2}{\aclrd{\textbf{r}} \cons\alift{\cclrd{t},\cclrd{s}} \cons \aclrd{\textbf{q}}} \vdash e : \sigma}
\quad
\begin{array}{l}
\slen{\Gamma_1} = \slen{\aclrd{\textbf{r}}}\\[-0.25em]
\slen{\Gamma_2} = \slen{\aclrd{\textbf{s}}}
\end{array}
\end{equation*}
\begin{equation*}
\tyrule{contr}
  {\coctx{\Gamma_1,y\!:\!\tau,z\!:\!\tau,\Gamma_2}{\aclrd{\textbf{r}} \cons \alift{\cclrd{s},\cclrd{t}} \cons \aclrd{\textbf{q}}} \vdash e : \sigma}
  {\coctx{\Gamma_1,x\!:\!\tau,\Gamma_2}{\aclrd{\textbf{r}} \cons \alift{\cclrd{s}+\cclrd{t}} \cons \aclrd{\textbf{q}}} \vdash \subst{\subst{e}{z}{x}}{y}{x} : \sigma}
~
\begin{array}{l}
\slen{\Gamma_1} = \slen{\aclrd{\textbf{r}}}\\[-0.25em]
\slen{\Gamma_2} = \slen{\aclrd{\textbf{s}}}
\end{array}
\end{equation*}

\figcaption{Structural coeffect bounded reuse analysis}
\label{fig:applications-struct-bll}
\vspace{-1em}
\end{figure}

% --------------------------------------------------------------------------------------------------

\paragraph{Bounded variable use.}
The system discussed in this section tracks the number of times a variable is accessed in the
call-by-name evaluation. Although we look at an example that tracks \emph{variable usage}, the same
system could be used to track access to resources that are always passed as a reference (and behave
effectively as call-by-name) and so the system is relevant for call-by-value languages too.
To demonstrate the idea, consider the following term:
%
\begin{equation*}
(\lambda v.x + v + v)~(x+y)
\end{equation*}
%
When evaluated, the body of the function directly accesses $x$ once and then twice indirectly, via
the function argument. Similarly, $y$ is accessed twice indirectly. Thus, the overall expression uses
$x$ three times and $y$ twice.

As discussed in Chapter~\ref{ch:structural}, the system preserves type and coeffect annotations under
the $\beta$-reduction. Reducing the expression in this case gives $x + (x+y) + (x+y)$. This has the
same bounds as the original expression -- $x$ is used three times and $y$ twice.

% --------------------------------------------------------------------------------------------------

\paragraph{Type system.}
The type system in Figure~\ref{fig:applications-struct-bll} annotates contexts with vectors of integers.
The rules have the same structure as those of the system for liveness analysis. The only difference is
how annotations are combined. Here, we use integer multiplication ($\ast$) for sequential composition
and addition ($+$) for point-wise composition.

Variable access (\emph{var}) annotates a variable with $1$, meaning that it has been used once. An
unused variable (\emph{weak}) is annotated with $0$. Multiple occurrences of the same variable are
introduced by contraction (\emph{contr}), which adds the numbers of the two contracted variables.

As previously, application (\emph{app}) and let binding (\emph{let}) combine two separate contexts.
The second part applies a function that uses its parameter $\cclrd{t}$-times to an argument that uses
variables in $\Gamma_2$ at most $\aclrd{\mathbf{s}}$-times (here, $\aclrd{\mathbf{s}}$ is a vector of
integers with an annotations for each variable in $\Gamma_2$). The sequential composition (modelling
call-by-name) multiplies the uses, meaning that the total number of uses is $(\cclrd{t} \ast \aclrd{\mathbf{s}})$
(where $\ast$ is a point-wise multiplication of a vector by a scalar). This models the fact that
for each use of the function parameter, we replicate the variable uses in $e_2$.

Finally, the subcoeffecting rule (\emph{sub}) safely overapproximates the number of accesses using
the pointwise $\leq$ relation. We can view any variable as being used a greater
number of times than it actually is.

\paragraph{Example.} To type check the expression $(\lambda v.x+v+v)~(x+y)$ discussed earlier, we need
to use abstraction, application, but also the contraction rule. Assuming the type judgement for the body,
abstractions yields:
%
\begin{equation*}
\tyrule{abs}
 { \coctx{x\!:\!\mathbb{Z},v:\mathbb{Z}}{\alift{\cclrd{1},\cclrd{2}}} \vdash x+v+v : \mathbb{Z} }
 { \coctx{x\!:\!\mathbb{Z}}{\alift{\cclrd{1}}} \vdash (\lambda v.x+v+v) : \mathbb{Z} \xrightarrow{\cclrd{2}} \mathbb{Z} }
\end{equation*}
%
To type-check the application, the contexts of $e_1$ and $e_2$ need to contain disjoint variables.
For this reason, we $\alpha$-rename $x$ to $x'$ in the argument $(x+y)$ and later join $x$ and $x'$ using
the contraction rule. Assuming $(x'+y)$ is checked in a context that marks $x'$ and $y$ as used once, the
application rule yields a judgement that is simplified as follows:
%
\begin{equation*}
\inference
  { \coctx{x\!:\!\mathbb{Z},x'\!:\!\mathbb{Z},y\!:\!\mathbb{Z}}
          {\alift{\cclrd{1}} \cons (\cclrd{2} \ast \alift{\cclrd{1},\cclrd{1}}) } \vdash (\lambda v.x+v+v)~(x'+y) : \mathbb{Z} }
{\tyrule{contr}
  { \coctx{x\!:\!\mathbb{Z},x'\!:\!\mathbb{Z},y\!:\!\mathbb{Z}}{\alift{\cclrd{1}, \cclrd{2}, \cclrd{2}} } \vdash (\lambda v.x+v+v)~(x'+y) : \mathbb{Z} }
  { \coctx{x\!:\!\mathbb{Z},y\!:\!\mathbb{Z}}{\alift{\cclrd{3}, \cclrd{2}}} \vdash (\lambda v.x+v+v)~(x+y)  : \mathbb{Z}} }
\end{equation*}
%
The first step performs scalar multiplication, producing the vector
$\alift{\cclrd{1},\cclrd{2},\cclrd{2}}$. In the second step, we use contraction to join variables
$x$ and $x'$ from the function and argument terms respectively.

% --------------------------------------------------------------------------------------------------

\paragraph{Semantics.}
In the previous examples, we defined the semantics -- somewhat informally -- using a simple
$\lambda$-calculus language to encode the model. More formally, this could be a Cartesian-closed
category. In that model, we can reuse variables arbitrarily and so it is not
a good fit for modelling bounded reuse. Girard et al. \cite{logic-bounded} model their bounded
linear logic in an (ordinary) linear logic where variables can be used at most once.

Following the same approach, we could model a variable $\tau$, annotated with $\cclrd{r}$ as
a product containing $\cclrd{r}$ copies of $\tau$, that is $\tau^{\cclrd{r}}$:
%
\begin{equation*}
\begin{array}{r}
\sem{\coctx{x_1\!:\!\tau_1, \ldots, x_n\!:\!\tau_n}{ \alift{\cclrd{r_1}, \ldots, \cclrd{r_n}} } \vdash e : \tau}
  ~:~ (\tau^{\cclrd{r_1}}_1 \times \ldots \times \tau^{\cclrd{r_n}}_n) \rightarrow \tau\\[0.5em]
\textnormal{where}~\tau^{\cclrd{r_i}}_i = \underbrace{\tau_i \times \ldots \times \tau_i}_{\cclrd{r_i}-\text{times}}
\end{array}
\end{equation*}
%
The functions are interpreted similarly. A function $\tau_1 \xrightarrow{\cclrd{t}} \tau_2$ is modelled
as a function taking $\cclrd{t}$-element product of $\tau_1$ values: $\tau_1^{\cclrd{t}} \rightarrow \tau_2$.

The rules that define the semantics of bounded calculus are easy to adapt from the semantic rules
of liveness in Figure~\ref{fig:applications-struct-livesem}. The ones that differ are those that use
sequential composition (application and let binding) and the contraction rule, which represents
pointwise composition.

In the following, we use vector names $\mathbf{v}_i$ for contexts containing multiple variables
\ie~have a type $\tau_1^{\cclrd{r_1}}\times\ldots\times\tau_m^{\cclrd{r_m}}$. Each vector contains
multiple copies of each variable, to model the fact that variables are used in an affine way (at most
once). We do not explicitly write the sizes of these vectors (number of variables in a context; number
of instances of a variable) as these are clear from the coeffect annotations. We assume that $\Gamma_2$
contains $n$ variables and that $\aclrd{s}=\alift{\cclrd{s}_1, \ldots, \cclrd{s}_n}$. First, consider
the (\emph{contr}) rule:
%
\begin{equation*}
\hspace{-1em}
\begin{array}{l}
\mdeff
  {\begin{array}{l}\sem{\coctx{\Gamma_1,y\!:\!\tau_1,z\!:\!\tau_1,\Gamma_2\\[-0.25em]\qquad}
      {\aclrd{\textbf{r}} \atimes \alift{\cclrd{s},\cclrd{t}} \atimes \aclrd{\textbf{q}}} \vdash e : \tau}\end{array}}
  {\begin{array}{l}\sem{\coctx{\Gamma_1,x\!:\!\tau_1,\Gamma_2}{\aclrd{\textbf{r}}
          \atimes \alift{\cclrd{s} + \cclrd{t}} \atimes \aclrd{\textbf{q}}} \\[-0.25em] \qquad\vdash \subst{e}{z,y}{x} : \tau}\end{array}}
  {\begin{array}{l}~\\[-0.25em]\hspace{-0.5em}f\end{array}}
  {\hspace{-0.5em}\begin{array}{l}\lambda(\mathbf{v_1}, (x_1, \ldots, x_{\cclrd{s}+\cclrd{t}}), \mathbf{v_2}) .\\[-0.25em]
  \quad f~(\mathbf{v_1}, (x_1, \ldots, x_{\cclrd{s}}), (x_{\cclrd{s}+1}, \ldots, x_{\cclrd{s}+\cclrd{t}} ), \mathbf{v_2})\end{array}\hspace{-0.5em}~}
\end{array}
\end{equation*}
%
The semantic function is called with $\cclrd{s}+\cclrd{t}$ copies of
a value for the $x$ variable. The values are split between $\cclrd{s}$ and $\cclrd{t}$ separate
copies of variables $y$ and $z$, respectively. The (\emph{app}) rule is similar in that it needs to
split the input variable context. However, it needs to split values of multiple variabless:
%
\begin{equation*}
\hspace{-1em}
\begin{array}{l}
\emdefff
  {\coctx{\Gamma_1}{\aclrd{\textbf{r}}} \vdash e_1 : \tau_1 \xrightarrow{\cclrd{t}} \tau_2}
  {\coctx{\Gamma_2}{\aclrd{\textbf{s}}} \vdash e_2 : \tau_1}
  {\begin{array}{l}\sem{\coctx{\Gamma_1, \Gamma_2}{\aclrd{\textbf{r}} \cons (\cclrd{t} \ast \aclrd{\textbf{s}})}\\[-0.25em]
    \qquad\vdash e_1 \, e_2 : \tau_2}\\[-0.25em]~\\[-0.25em]~\\[-0.25em]~\\[-0.25em]~\end{array}\hspace{-0.5em}~}
  {f_1}
  {f_2}
  {\hspace{-0.5em}\begin{array}{l}
  \lambda (\mathbf{v_1}, ((x_{1,1}, \ldots, x_{1,\cclrd{t}\ast\cclrd{s}_1}), \ldots, (x_{n,1}, \ldots, x_{n,\cclrd{t}\ast\cclrd{s}_n}) ) .\\[-0.25em]
  \quad(f_1~\textbf{v}_1)\\[-0.25em]
  \qquad (~f_2~((x_{1,1}, \ldots, x_{1,\cclrd{s}_1}), \ldots, \\[-0.25em]
  \hspace{4.0em}                                         ~(x_{n,1}, \ldots, x_{1,\cclrd{s}_n}) ~),\ldots,\\[-0.25em]
  \qquad \;~f_2~((x_{1,(\cclrd{t}-1)\ast\cclrd{s}_1 + 1}, \ldots, x_{1,\cclrd{t}\ast\cclrd{s}_1}), ~\ldots~, \\[-0.25em]
  \hspace{4.0em}                                         ~(x_{n,(\cclrd{t}-1)\ast\cclrd{s}_n + 1}, \ldots, x_{1,\cclrd{t}\ast\cclrd{s}_n})  ~)~)\\[-0.25em]
  \end{array}}}
\end{array}
\end{equation*}
\vspace{0.5em}

\noindent
In $x_{i,j}$, the index $i$ stands for an index of the variable while $j$ is an index of one of
multiple copies of the value. In the semantic function, the second
part of the context consists of $n$ variables where the multiplicity of each value is specified
by the annotation $\cclrd{s}_i$ multiplied by $\cclrd{t}$. The rule needs to evaluate the argument
$e_2$ $\cclrd{t}$-times and each call requires $\cclrd{s}_i$ copies of the $i^\textnormal{th}$
variable. To do this, we create contexts $\mathbf{y}_1$ to $\mathbf{y}_{\cclrd{t}}$, each containing
$\cclrd{s}_i$ copies of the variable (and so we require $\cclrd{s}_i\ast\cclrd{t}$ copies of each
variable). Note that the contexts are created such that each value is used exactly once.

It is worth noting that the (\emph{var}) rule requires exactly one copy of a variable and so
the system tracks precisely the number of uses. However, the (\emph{sub}) rule lets us
ignore additional copies of a value. Thus, permitting (\emph{sub}) rule is only possible if the
underlying model is \emph{affine} rather than \emph{linear}.

\paragraph{Bounded linear logic.}
The system presented in this section is based on the idea of bounded linear logic (BLL)
\cite{logic-bounded}, but it is adapted to follow the structure of other coeffect systems
discussed in this chapter. This elucidates the connection between BLL and coeffects.

The big difference, using the terminology from Section~\ref{sec:path-sem-contextdep}, is
that our system is written in \emph{language semantics} style, while BLL is written
in \emph{meta-language} style. We briefly consider the original BLL formulation.

The terms and types of our system are the terms and types of an ordinary $\lambda$-calculus,
with the only difference that functions carry coeffect annotations. In BLL, the language of
types is extended with a type constructor $!_k A$ (where $A$ is a proposition, corresponding
to a type $\tau$ in our system). The type denotes a value $A$ that can be used at most $k$ times.

As a result, BLL does not need to attach additional annotation to the variable context
as a whole. The requirements are attached to individual variables and so our context
$\coctx{\tau_1, ..., \tau_n}{\langle \cclrd{k_1}, ..., \cclrd{k_n}\rangle}$ corresponds
to a BLL assumption $!_{k_1} A_1, ..., !_{k_n} A_n$. Using the formulation of bounded logic
(and omitting the terms), the weakening and contraction rules are written as follows:

\[
\tyrule{weak}
  {\Gamma \vdash B}
  {\Gamma, !_0 A \vdash B}
\quad
\tyrule{contr}
  {\Gamma, !_n A, !_m A \vdash B}
  {\Gamma, !_{n+m} A \vdash B}
\]
%
The system captures the same idea as the structural coeffect system presented above.
Variable access in bounded linear logic is simply an operation that produces a value
$!_n A$ and so the system further introduces \emph{dereliction} rule which lets us
treat $!_1 A$ as a value $A$. We further explore difference between \emph{language
semantics} and \emph{meta-language} in Section~\ref{sec:further-meta}.

\paragraph{Summary.}
Comparing the structural coeffect calculus for tracking liveness and for bounded variable reuse
reveals which parts of the systems differ and which parts are shared. In particular, both systems
use the same vector operations ($\times$, $\alift{\cclrd{\textnormal{--}}}$) and also share the
lambda abstraction rule (\emph{abs}). They differ in the primitive values used to annotate used
and unused variables (\cclrd{L}, \cclrd{D} and $1$, $0$, respectively) and in the operators used
for sequential composition and contraction ($\cclrd{\sqcup}$, $\cclrd{\sqcap}$ and $\ast$, $+$, respectively).
The algebraic structure capturing these operators is developed in Chapter~\ref{ch:structural}.

The brief overview of bounded linear logic shows an alternative approach to tracking properties
related to individual variables -- we could attach annotations to the variables themselves
rather than attaching a \emph{vector} of annotations to the entire context. One benefit
of our approach is that it lets us unify flat and structural systems (Chapter~\ref{ch:further-unified}).

% --------------------------------------------------------------------------------------------------

\subsection{Dataflow languages revisited}
\label{sec:applications-structural-dataflow}

When discussing dataflow languages in an earlier section, we said that the context provides
past values of variables. In Section~\ref{sec:applications-flat-dataflow}, we tracked this as
a \emph{flat} property, which gives us a system that keeps the same number of past values for
all variables. However, dataflow can also be adapted to a structural system which keeps the number
of required past values individually for each variable. Consider the
following example:
%
\begin{equation*}
\kvd{let}~\ident{offsetAdd} = \ident{left} + \kvd{prev}~\ident{right}
\end{equation*}
%
The value \ident{offsetAdd} adds values of \ident{left} with previous values of \ident{right}.
To evaluate a current value of the stream, we need the current value of \ident{left} and one past
value of \ident{right}. Flat system is not able to capture this level-of-detail and simply
requires $1$ past values of both streams in the variable context.

Turning a flat dataflow system to a structural dataflow system is a change similar to the one
between flat ans structural liveness. In case of liveness analysis, we included the flat system
only as an illustration (it is not practically useful). For dataflow, the flat system is less
precise, but still practically useful (simplicity may outweigh precision).

% --------------------------------------------------------------------------------------------------

\begin{figure}[t]
\begin{equation*}
\tyrule{var}{}
  {\coctx{x \!:\! \tau}{\alift{\cclrd{0}}} \vdash x : \tau}
\end{equation*}
\begin{equation*}
\tyrule{prev}
  {\coctx{\Gamma}{\aclrd{\textbf{r}}} \vdash e : \tau}
  {\coctx{\Gamma}{1 + \aclrd{\textbf{r}}} \vdash \kvd{prev}~e : \tau}
\end{equation*}
\begin{equation*}
\tyrule{app}
  { \coctx{\Gamma_1}{\aclrd{\textbf{r}}} \vdash e_1 : \tau_1 \xrightarrow{\cclrd{t}} \tau_2 \quad
    \coctx{\Gamma_2}{\aclrd{\textbf{s}}} \vdash e_2 : \tau_1}
  { \coctx{\Gamma_1, \Gamma_2}{\aclrd{\textbf{r}} \cons (\cclrd{t} \,\cclrd{+}\, \aclrd{\textbf{s}})} \vdash e_1 \, e_2 : \tau_2}
\end{equation*}
\begin{equation*}
\tyrule{weak}
  {\coctx{\Gamma}{ \aclrd{\textbf{r}} } \vdash e : \sigma}
  {\coctx{\Gamma,x \!:\! \tau}{\aclrd{\textbf{r}} \cons \alift{ \cclrd{0} }} \vdash e : \sigma}
\end{equation*}
\begin{equation*}
\tyrule{contr}
  {\coctx{\Gamma_1,y\!:\!\tau,z\!:\!\tau,\Gamma_2}{\aclrd{\textbf{r}} \cons \alift{\cclrd{s},\cclrd{t}} \cons \aclrd{\textbf{q}}} \vdash e : \sigma}
  {\coctx{\Gamma_1,x\!:\!\tau,\Gamma_2}{\aclrd{\textbf{r}} \cons \alift{\cclrd{\textnormal{max}}(\cclrd{s},\cclrd{t})} \cons \aclrd{\textbf{q}}} \vdash \subst{\subst{e}{z}{x}}{y}{x} : \sigma}
\end{equation*}

\figcaption{Structural coeffect bounded reuse analysis}
\label{fig:applications-struct-df}
\vspace{-1em}
\end{figure}

% --------------------------------------------------------------------------------------------------

\paragraph{Type system.}
The type system in Figure~\ref{fig:applications-struct-df} annotates the variable context with a
vector of integers. This is similar as in the bounded reuse system, but the integers \emph{mean} a
different thing. Consequently, they are also calculated differently. We omit rules that are the
same for all structural coeffect systems (exchange, lambda abstraction).

In dataflow, we annotate both used variables (\emph{var}) and unused variables (\emph{weak}) with
$0$, meaning that no past values are required. This is the same as in flat dataflow, but different
from bounded reuse and liveness (where unused variables have a different coeffect). Primitive
requirements are introduced by the (\emph{prev}) rule, which increments the annotations of
all variables.

In flat dataflow, we identified sequential composition and pointwise composition as two primitive
operations that were used in the (flat) application. In the structural system, these are used in
(\emph{app}) and (\emph{contr}). Thus application combines coeffect annotations using $+$ and
contraction using \emph{max}. This contrasts with bounded reuse, which uses $\ast$ and $+$,
respectively.

\paragraph{Example.} As an example, consider a function $\lambda x.\kvd{prev}~(y+x)$ applied to an argument
$\kvd{prev}~(\kvd{prev}~y)$. The body of the function accesses the past value of two variables, one free
and one bound. The (\emph{abs}) rule splits the annotations between the declaration site and call site
of the function:
%
\begin{equation*}
\tyrule{abs}
  {\coctx{y\!:\!\mathbb{Z}, x\!:\!\mathbb{Z}}{\alift{1, 1}} \vdash \kvd{prev}~(y+x) : \mathbb{Z} }
  {\coctx{y\!:\!\mathbb{Z}}{\alift{1}} \vdash \lambda x . \kvd{prev}~(y+x) : \mathbb{Z} \xrightarrow{\cclrd{1}} \mathbb{Z} }
\end{equation*}
%
The expression always requires the previous value of $y$ and adds it to a previous value of the
parameter $x$. Evaluating the value of the argument $\kvd{prev}~(\kvd{prev}~y)$ requires two past
values of $y$ and so the overall requirement for the (free) variable $y$ is $3$ past values. In
order to use the contraction rule, we rename $y$ to $y'$ in the argument:
%
\begin{equation*}
\inference
  { \coctx{y\!:\!\mathbb{Z}}{\alift{1} } \vdash \lambda x.~(\ldots) : \mathbb{Z} \xrightarrow{\cclrd{1}} \mathbb{Z} &
    \coctx{x\!:\!\mathbb{Z}}{\alift{2}} \vdash (\kvd{prev}~(\kvd{prev}~y') : \mathbb{Z} }
{\inference
  { \coctx{y\!:\!\mathbb{Z}, y'\!:\!\mathbb{Z}}{\alift{1,3}} \vdash (\lambda x.\kvd{prev}~(y+x))~(\kvd{prev}~(\kvd{prev}~y')) : \mathbb{Z} }
  { \coctx{y\!:\!\mathbb{Z}}{\alift{3}} \vdash (\lambda x.\kvd{prev}~(y+x))~(\kvd{prev}~(\kvd{prev}~y)) : \mathbb{Z} } }
\end{equation*}
%
The derivation uses (\emph{app}) to get requirements $\alift{1,3}$ and then (\emph{contr}) to take
the maximum, showing three past values are sufficient.

Note that we get the same requirements when we perform $\beta$ reduction of the expression.
Substituting the argument for $x$ yields the expression $\kvd{prev}~(y+(\kvd{prev}~(\kvd{prev}~y)))$.
Semantically, this performs stream lookups $y[1]$ and $y[3]$ where the indices are the
number of enclosing $\kvd{prev}$ constructs.

\paragraph{Semantics.}
To define the semantics of our structural dataflow language, we can use the same approach as when
adapting flat liveness to structural liveness. Rather than wrapping the whole context in a type
constructor (list or option), we now wrap the individual components of the product representing
the variables in the context.

The result is similar as the structure used for bounded reuse. The only difference is that, given
a variable annotated with $\cclrd{r}$, we need $1+\cclrd{r}$ values. That is, we need the current
value, followed by $\cclrd{r}$ past values:
%
\begin{equation*}
\begin{array}{l}
\sem{\coctx{x_1\!:\!\tau_1, \ldots, x_n\!:\!\tau_n}{ \alift{\cclrd{r_1}, \ldots, \cclrd{r_n}} } \vdash e : \tau}
  ~:~ (\tau^{(\cclrd{r_1}+1)}_1 \times \ldots \times \tau^{(\cclrd{r_n}+1)}_n) \rightarrow \tau\\
\sem{\tau_1 \xrightarrow{\cclrd{s}} \tau_2} ~=~ \tau_1^{(\cclrd{s}+1)} \rightarrow \tau_2
\end{array}
\end{equation*}
%
Despite the similarity with the semantics for bounded reuse, the values here \emph{represent}
different things. Rather than providing multiple copies of a value (out of which each can be
used just once), the pair provides past values (that can be reused and freely accessed).
To illustrate the behaviour consider first the semantics of the \kvd{prev} expression:
%
\begin{equation*}
\hspace{-1em}
\begin{array}{l}
\emdeff
  {\coctx{\Gamma}{\alift{\cclrd{s}_1, \ldots, \cclrd{s}_n}} \vdash e : \tau}
  {\begin{array}{l}\sem{\coctx{\Gamma}{\alift{(\cclrd{s}_1 \hspace{-0.2em}+\hspace{-0.2em} 1), \ldots, (\cclrd{s}_n \hspace{-0.2em}+\hspace{-0.2em} 1)}}\\[-0.25em]
    \qquad\vdash \kvd{prev}~e : \tau}\end{array}\hspace{-0.5em}~}
  {f}
  {\hspace{-0.5em}\begin{array}{l}\lambda((x_{1,0}, \ldots, x_{1,{\cclrd{s}_1 + 1}}), \ldots, (x_{n,0}, \ldots, x_{n,{\cclrd{s}_n + 1}})).\\[-0.25em]
  \quad f~((x_{1,0}, \ldots, x_{1,\cclrd{s}_1}), \ldots, (x_{n,0}, \ldots, x_{n,\cclrd{s}_n}))\end{array}\hspace{-0.5em}~}
\end{array}
\end{equation*}
%
Here, the semantic function is called with an argument that stores values of $n$
variables, such that a variable $x_i$ has values ranging from $x_{i,0}$ to $x_{i,\cclrd{s}_i + 1}$.
Thus, there is one current value, followed by $\cclrd{s_i} + 1$ past values. The expression $e$
nested under $\kvd{prev}$ requires only $\cclrd{s_i}$ past values and so the semantics
drops the last value. The following shows the semantics of contraction:
%
\begin{equation*}
\hspace{-1em}
\begin{array}{l}
\mdeff
  {\begin{array}{l}\sem{\coctx{\Gamma_1,y\!:\!\tau_1,z\!:\!\tau_1,\Gamma_2\\[-0.25em]\qquad}
      {\aclrd{\textbf{r}} \atimes \alift{\cclrd{s},\cclrd{t}} \atimes \aclrd{\textbf{q}}} \vdash e : \tau}\end{array}}
  {\begin{array}{l}\sem{\coctx{\Gamma_1,x\!:\!\tau_1,\Gamma_2}{\aclrd{\textbf{r}}
          \atimes \alift{\cclrd{\textnormal{max}}(\cclrd{s}, \cclrd{t})} \atimes \aclrd{\textbf{q}}} \\[-0.25em] \qquad\vdash \subst{e}{z,y}{x} : \tau}\end{array}}
  {\begin{array}{l}~\\[-0.25em]\hspace{-0.5em}f\end{array}}
  {\hspace{-0.5em}\begin{array}{l}\lambda(\mathbf{v_1}, (x_0, x_1, \ldots, x_{ \textnormal{max}(\cclrd{s},\cclrd{t}) }), \mathbf{v_2}) .\\[-0.25em]
  \quad f~((\mathbf{v_1}, (x_0, \ldots, x_{\cclrd{s}}), (x_0, \ldots, x_{\cclrd{t}} ), \mathbf{v_2}))\end{array}\hspace{-0.5em}~}
\end{array}
\end{equation*}
%
The semantic function receives $\emph{max}(\cclrd{s}, \cclrd{t})$ values
of a specific variable $x$. It needs to produce values for two separate variables, $y$ and $z$ that
require $\cclrd{s}$ and $\cclrd{t}$ past values. Both of these numbers are certainly smaller than
(or equal to) the number of values available. Thus we simply take the first values. Unlike in the
contraction for BLL, the values are duplicated and the same values are used for both variables.

\paragraph{Summary.}
Two of the structural examples shown so far (liveness and dataflow) extend an earlier flat
version of a similar system. We discuss this relation in general later. However, a flat system
can generally be turned into a structural one -- although this only gives a useful system when
the flat version captures statically scoped properties, \ie~related to variables.

The dataflow example demonstrates that the a flat system can also be turned into structural
system. In general, this only works for systems where lambda abstraction duplicates context
requirements (as in Figure~\ref{fig:applications-flat-liveness}).

% --------------------------------------------------------------------------------------------------

\subsection{Security, tainting and provenance}
Tainting is a mechanism where variables coming from potentially untrusted sources are marked
(\emph{tainted}) and the use of such variables is disallowed in contexts where untrusted input
can cause security issues or other problems. Tainting can be done dynamically using a runtime mark
(\eg~in the Perl language) or using a static type system. Tainting can be viewed as a special
case of \emph{provenance tracking}, known from database systems \cite{app-provenance-db}, where
values are annotated with more detailed information about their source.

Static typed systems based on tainting have been use to prevent cross-site scripting attacks
\cite{app-tainting-xss} and SQL injection attacks \cite{app-tainting-sql,app-tainting-wasp}.
In the latter case, we want to check that SQL commands cannot be directly constructed from
unchecked inputs provided by the user. Consider the type checking of the following
expression in a context containing variables \ident{id} and \ident{msg}:
%
\begin{equation*}
\begin{array}{l}
\kvd{let}~\ident{name} = \ident{query}(\str{SELECT Name WHERE Id = \%1}, \ident{id})\\[-0.25em]
\ident{msg}~+~\ident{name}
\end{array}
\end{equation*}
%
In this example, \ident{id} must not come directly from a user input, because \ident{query} requires
untainted string. Otherwise, the attacker could specify values such as \str{1; DROP TABLE Users}.
The variable \ident{msg} may or may not be tainted, because it is not used in protected context
(i.e.~to construct an SQL query).

In runtime checking, all (string) values need to be wrapped in an object with a Boolean
flag (for tainting) or more complex data (for provenance). In static checking, the information
need to be associated with the variables in the variable context.

% --------------------------------------------------------------------------------------------------

\paragraph{Core dependency calculus.}
Taint checking is a special case of checking of the \emph{non-interference} property
in \emph{secure information flow}. There, the aim is to guarantee that sensitive information (such
as credit card number) cannot be leaked to contexts with low secrecy (\eg~sent via an unsecured
network channel). Volpano et al. \cite{app-secure-flow} provide the first (provably) sound type
system that guarantees non-inference and Sabelfeld et al. \cite{app-secure-information-flow} surveys
more recent work. Information flow checking has been also integrated (as a single-purpose
extension) in the FlowCaml \cite{app-security-flowcaml} language. Finally, Russo et al. and
Swamy et al. \cite{monad-secure-flow,monads-lightweight-ml} show that such properties can be checked
using a monadic library.

Systems for secure information flow typically define a lattice of security classes $(\mathcal{S}, \leq)$
where $\mathcal{S}$ is a finite set of classes and an ordering. For example a set $\{\cclrd{\ident{L}}, \ident{H}\}$
represents low and high secrecy, respectively with $\cclrd{\ident{L}} \leq \ident{H}$ meaning that low security
values can be treated as high security (but not the other way round).

\paragraph{Implicit flows.}
An important aspect of secure information flow is called \emph{implicit flows}. Consider the following
example which returns either $y$ or zero, depending on the value of $x$:
%
\begin{equation*}
\kvd{let}~z = \kvd{if}~x>0~\kvd{then}~y~\kvd{else}~0
\end{equation*}
%
If the value of $y$ is high-secure, then $z$ becomes high-secure after the assignment
(this is an \emph{explicit} flow). However, if $x$ is high-secure, then the value of
$z$ becomes high-secure, regardless of the security level of $y$, because the fact whether an
assignment is performed or not performed leaks information in its own (this is an
\emph{implicit} flow).

Although we do not describe a coeffect calculus for information flow checking, it is worth noting
that Abadi et al. \cite{app-dcc} realized that there is a number of analyses similar to secure information
flow and unified them using a single model called Dependency Core Calculus (DCC). This would be
a useful basis for coeffect-based information flow checking.

The DCC captures other cases where some information about expression relies on properties of variables
in the context where it executes.  This includes, for example, \emph{binding time analysis}
\cite{app-binding-time-analysis}, which detects which parts of programs can be partially evaluated
(do not depend on user input) and \emph{program slicing} \cite{app-slicing-survey} that identifies
parts of programs that contribute to the output of an expression.

\paragraph{Coeffect systems.}
The work outlined in this section is another area where coeffect systems could be applied.
We do not develop coeffect systems for taint tracking, security and provenance in detail,
but briefly mention some examples in the upcoming chapters.

The systems work in the same way as the examples discussed already. For example, consider the
tainting example with the \ident{query} function calling an SQL database. To capture such
tainting, we annotate variables with $\cclrd{\ident{T}}$ for \emph{tainted} and with
$\cclrd{\ident{U}}$ for \emph{untainted}. Accessing a variable marks it as untainted,
but using an expression that depends on some variable in certain dangerous contexts -- such
as in arguments of \ident{query} -- does introduce a taint on all the variables contributing to
the expression. This is captured using the standard application rule (\emph{app}):
%
\begin{equation*}
\hspace{-2em}
\tyrule{app}
  { \coctx{\Gamma}{\aclrd{r}} \vdash \ident{query} : \ident{string} \xrightarrow{\cclrd{\ident{T}}} \ident{Table} \qquad
    \coctx{\ident{id}:\ident{string}}{ \alift{\cclrd{\ident{U}}}} \vdash \ident{id} : \ident{string} }
  { \coctx{\Gamma, \ident{id}:\ident{string}}{ \aclrd{r} \cons \alift {\cclrd{\ident{T}}} \vdash \ident{query}(\str{...}, \ident{id}) : \ident{Table} } }
\end{equation*}
%
The derivation assumes that \ident{query} is a standard function that requires the parameters
to be tainted (it does not have to be a built-in language construct). The argument is a
variable and so it is not tainted in the assumptions.

In the conclusion, we need to derive an annotation for the variable \ident{id}. To do this, we
combine \cclrd{\ident{T}} (from the function) and \cclrd{\ident{U}} (from the argument). In case
of tainting, the variable is tainted whenever it is already tainted \emph{or} the function marks
it as tainted. For different kinds of annotations, the composition would work differently -- for
example, for provenance, we could union the \emph{set} of possible data sources, or even combine
\emph{probability distributions} modelling the influence of different sources on the value.
However, expanding such ideas is beyond the scope of this thesis.

% ==================================================================================================

\section{Beyond passive contexts}

In both flat and structural systems discussed so far, the context provides additional data (resources,
implicit parameters, historical values) or meta-data (security, provenance). However, \emph{within}
the language, it is impossible to write a function that modifies the context. We use the term
\emph{passive} context for such applications.

There is a number of systems that also capture contextual properties, but that make it possible to
\emph{change} the context -- not just by evaluating certain code block in a locally modified context
(\eg by wrapping it in $\ident{prev}$ in dataflow), but also by calling a function that, for example,
acquires new capabilities and returns those to the caller. Such actions appear to be closer to
effects than to coeffects. While this thesis focuses on systems with passive contexts, we briefly
consider the most important examples of the \emph{active} variant.

% --------------------------------------------------------------------------------------------------

\paragraph{Calculus of capabilities.}
\label{sec:applications-active-ccc}

Crary et al. \cite{app-capabilities} introduced the Calculus of Capabilities to provide
a sound system with region-based memory management for low-level code that can be easily
compiled to assembly language. They build on the work of Tofte and Talpin \cite{app-region-memory}
who developed an effect system (as discussed in Section~\ref{sec:path-sem-effects}) that uses
lexically scoped \emph{memory regions} to provide an efficient and controlled memory management.

In the work of Tofte and Talpin, the context is \emph{passive}. They extend a simple functional language
with the \kvd{letrgn} construct that defines a new memory region, evaluates an expression (possibly)
using memory in that region and then deallocates the memory of the region:

\begin{equation*}
\begin{array}{l}
\kvd{let}~\ident{calculate} = \lambda \ident{input} \rightarrow\\[-0.25em]
\quad \kvd{letrgn}~\rho~\kvd{in}\\[-0.25em]
\quad \kvd{let}~\ident{x} = \kvd{ref}_\rho ~\ident{input}~\kvd{in}\\[-0.25em]
\quad \ident{x} :=\, !\ident{x} + 1;~ !\ident{x}
\end{array}
\end{equation*}
%
The memory region $\rho$ is a part of the context, but only in the scope of the body of
\kvd{letrgn}. It is only available to the last two lines which allocate a memory cell in the region,
increment a value in the region and then read it. The region is de-allocated when the execution
leaves its lexical scope -- there is no way to allocate a region inside a function and pass it back
to the caller.

The calculus of capabilities differs in two ways. First, it allows explicit allocation and deallocation
of memory regions (and so region lifetimes do not necessarily follow strict LIFO ordering). Second,
it uses continuation-passing style. We ignore the latter aspect. The following example is almost
identical to the previous one:
%
\begin{equation*}
\begin{array}{l}
\kvd{let}~\ident{calculate} = \lambda \ident{input} \rightarrow\\[-0.25em]
\quad \kvd{letrgn}~\rho~\kvd{in}\\[-0.25em]
\quad \kvd{let}~\ident{x} = \kvd{ref}_\rho ~\ident{input}~\kvd{in}\\[-0.25em]
\quad \ident{x} :=\, !\ident{x} + 1;~ \ident{x}
\end{array}
\end{equation*}
%
The difference is that the example does not return the \emph{value} of a reference using
$!\ident{x}$, but returns the reference $\ident{x}$ itself. The reference is allocated in a newly
created region $\rho$. Together with the value, the function returns a \emph{capability} to access
the region $\rho$.

This is where systems with active context differ from systems with passive context. To type check
such programs, we do not only need to know what context is required to call \ident{calculate}
(\ie~context on the left-hand side of $\vdash$). We also need to know what effects an expression
has on the context when it evaluates and the current context meeds to be updated after a function
call. This is an effectful property that would appear on the right-hand side of $\vdash$.

\paragraph{Active contexts.}
In a systems with passive contexts, we only need an annotation that specifies the required
context. In semantics, this is reflected by having some structure (data type) $\C$ over the
\emph{input} of the function. Without giving any details, the semantics generally has the
following structure (with comonad to model coeffects on the left):
%
\begin{equation*}
\sem{\tau_1 \xrightarrow{\cclrd{r}} \tau_2} = \C^{\cclrd{r}} \tau_1 \rightarrow \tau_2
\end{equation*}
%
Systems with active contexts require two annotations -- one that specifies the context required
before the call is performed and one that specifies how the context changes after the call (this
could be either a \emph{new} context or \emph{update} to the original context). Thus the structure
of the semantics would look as follows (with comonad to model coeffects on the left and monad to
model effects on the right):
%
\begin{equation*}
\sem{\tau_1 \xrightarrow{\cclrd{r}, \cclrd{s}} \tau_2} = \C^{\cclrd{r}} \tau_1 \rightarrow \M^{\cclrd{s}} \tau_2
\end{equation*}
%
In case of Calculus of Capabilities, both of the structures could be the same and they could
carry a set of available memory regions. In this thesis, we focus only on passive contexts.
However, capturing active contexts is an interesting future work.

% --------------------------------------------------------------------------------------------------

~

\paragraph{Software updating.}
Another example of a system that uses contextual information actively is dynamic software updating
(DSU) \cite{app-dsu-programs,app-dsu}. The DSU systems have the ability to update programs at
runtime without stopping them. For example, Proteus developed by Stoyle et al. \cite{app-dsu-mutatis}
investigates what language support is needed to enable safe dynamic software updating in C-like
languages. The work is based on capabilities and follows a structure similar to the Calculus
of Capabilities \cite{app-capabilities}.

The system distinguishes between \emph{concrete} uses and \emph{abstract} uses of a value. When
a value is used concretely, the program examines its representation (and so it is not safe to
change the representation during an update). An abstract use of a value does not examine
the representation and so updating the value does not break the program.

The Proteus system uses capabilities to restrict what types may be used concretely after any point
in the program. All other types, not listed in the capability, can be dynamically updated as this
will not change concrete representation of types accessed later in the evaluation.

Similarly to Capability Calculus, capabilities in DSU can be changed by a function call. For
example, calling a function that may update certain types makes it impossible to use those types
concretely following the function call. This means that DSU uses the context \emph{actively}
and not just \emph{passively}.


%===================================================================================================
%
%     #####
%    #     # #    # #    # #    #   ##   #####  #   #
%    #       #    # ##  ## ##  ##  #  #  #    #  # #
%     #####  #    # # ## # # ## # #    # #    #   #
%          # #    # #    # #    # ###### #####    #
%    #     # #    # #    # #    # #    # #   #    #
%     #####   ####  #    # #    # #    # #    #   #
%
%===================================================================================================

\section{Summary}

This chapter served two purposes. The first aim was to present existing work on programming
languages and systems that include some notion of \emph{context}. Because there was no well-known
abstraction capturing contextual properties, the languages use a wide range of formalisms -- including
principled approaches based on comonads and modal S4, ad-hoc type system extensions and static analyses
as well as approaches based on monads. We looked at a number of applications including Haskell's implicit
parameters and type classes, dataflow languages such as Lucid, liveness analysis and also a number of
security properties.

The second aim of this chapter was to re-formulate the existing work in a more uniform style and thus
reveal that all \emph{context-dependent} languages share a common structure. In the upcoming three
chapters, we identify the common structure more precisely and develop three calculi to capture it. We
will then be able to re-create many of the examples discussed in this chapter just by instantiating our
unified calculi.

This chapter was divided into two major sections. First, we looked at \emph{flat} systems, which
track whole-context properties. Next, we look a \emph{structural} systems, which track per-variable
properties. Both of the variants are useful and important -- for example, implicit parameters can
only be expressed as \emph{flat} system, but liveness analysis is only useful as \emph{structural}.
For this reason, we explore both of these variants in this thesis (Chapter~\ref{ch:flat} and
Chapter~\ref{ch:structural}, respectively). We can, however, unify the two variants into a single
system discussed in Section~\ref{sec:further-unified}.


% ==================================================================================================

\ctparttext{In this part, we capture the similarities between the concrete context-aware langauges
  presented in the previous chapter. We also develop the key novel technical contributions of
  the thesis. We define a \emph{flat coeffect type system} (Chapter~\ref{ch:flat}) that is
  parameterized by a \emph{coeffect algebra} and a mechanism for choosing unique typing derivation.
  We instantiate a coeffect type system with a concrete coeffect algebra and procedure for choosing
  unique typing derivation for three languages to capture dataflow, implicit parameters and liveness.

  The type system is complemented with a translational semantics for coeffect-based context-aware
  programming languages (Chapter~\ref{ch:semantics}). The semantics is inspired by a categorical
  model based on \emph{indexed comonads} and it translates source context-aware program into a
  target program in a simple functional language with comonadically-inspired primitives. We give
  concrete definition of the primitives for dataflow, implicit parameters and liveness and
  present a syntactic safety proof for these three languages.

  The following page provides a detailed overview of the content of Chapters~\ref{ch:flat} and
  Chapters~\ref{ch:semantics}, highlighting the split between general definitions and properties
  (about the coeffect calculus) and concrete definitions and properties (about concrete
  context-awre language). The Chapter~\ref{ch:structural} mirrors the same development for
  \emph{structural coeffect systems}.
}
\setcounter{part}{1}
\part{Coeffect calculi}
\label{part:coeffect-calculi}

\begin{center}
\begin{tabular}{ | l | l | l |}
\hline
\multicolumn{3}{ | c | }{\spacedlowsmallcaps{Chapter 4}} \\ \hline \hline
\hspace{7.1em} & \textsc{Coeffect calculus} \hspace{2.5em} & \textsc{Language-specific} \hspace{6.9em} \\ \hline
\textsc{Syntax}
  & \hspace{-0.5em}\begin{tabular}{l} Coeffect $\lambda$-calculus \\[-0.3em] (Section~\ref{sec:flat-calculus}) \end{tabular}
  & \hspace{-0.5em}\begin{tabular}{l} Extensions such as \\[-0.3em] \ident{?param} and \kvd{prev} \\[-0.3em] (Section~\ref{sec:flat-calculus-examples})  \end{tabular} \\ \hline
\textsc{Type system}
  & \hspace{-0.5em}\begin{tabular}{l} Abstract coeffect \\[-0.3em] algebra (Section~\ref{sec:flat-calculus-algebra})  \end{tabular}
  & \hspace{-0.5em}\begin{tabular}{l} Concrete instances \\[-0.3em] of the coeffect algebra \\[-0.3em] (Section~\ref{sec:flat-calculus-examples})  \end{tabular} \\
\hline
  & \hspace{-0.5em}\begin{tabular}{l} Coeffect type system \\[-0.3em] parameterized by \\[-0.3em] the coeffect algebra \\[-0.3em] (Section~\ref{sec:flat-calculus-types})  \end{tabular}
  & \hspace{-0.5em}\begin{tabular}{l} Typing for language- \\[-0.3em] specific extensions \\[-0.3em] (Section~\ref{sec:flat-calculus-examples})  \end{tabular} \\
\hline
  &
  & \hspace{-0.5em}\begin{tabular}{l} Procedure for determining \\[-0.3em] a unique typing derivation \\[-0.3em]  (Section~\ref{sec:flat-unique})  \end{tabular} \\ \hline
\textsc{Properties}
  & \hspace{-0.5em}\begin{tabular}{l} Syntactic properties \\[-0.3em] of coeffect $\lambda$-calculus \\[-0.3em] (Section~\ref{sec:flat-syntax})  \end{tabular}
  & \hspace{-0.5em}\begin{tabular}{l} Uniqueness of the above \\[-0.3em]  (Section~\ref{sec:flat-unique})  \end{tabular} \\ \hline
\end{tabular}
\end{center}

~

\begin{center}
\begin{tabular}{ | l | l | l |}
\hline
\multicolumn{3}{ | c | }{\spacedlowsmallcaps{Chapter 5}} \\ \hline \hline
& \textsc{Coeffect calculus} & \textsc{Language-specific} \\ \hline
\textsc{Categorical}
  & \hspace{-0.5em}\begin{tabular}{l} Indexed comonads \\[-0.3em] (Sectiuon~\ref{sec:semantics-flat-idx}) \end{tabular}
  & \hspace{-0.5em}\begin{tabular}{l} Examples including indexed \\[-0.3em] product, list and maybe \\[-0.3em] comonads (Section~\ref{sec:semantics-flat-monoidal}) \end{tabular} \\
\hline
  & \hspace{-0.5em}\begin{tabular}{l} Categorical semantics \\[-0.4em] of coeffect $\lambda$-calculus \\[-0.3em] (Section~\ref{sec:semantics-flat-calculus}) \end{tabular}
  & \\ \hline
\textsc{Translational}
  & \hspace{-0.5em}\begin{tabular}{l} Functional target \\[-0.3em] language (Section~\ref{sec:semantics-translation-target}) \end{tabular}
  &  \\
\hline
  & \hspace{-0.5em}\begin{tabular}{l} Translation from coeffect \\[-0.3em] $\lambda$-calculus to target \\[-0.3em]  language (Section~\ref{sec:semantics-translation-transl}) \end{tabular}
  & \hspace{-0.5em}\begin{tabular}{l} Translation for language-\\[-0.3em] specific extensions (\kvd{prev}, \ident{?p}) \\[-0.3em] (Sections~\ref{sec:semantics-proofs-df} and \ref{sec:semantics-proofs-impl}) \end{tabular} \\ \hline
\textsc{Operational}
  & \hspace{-0.5em}\begin{tabular}{l} Abstract comonadically- \\[-0.3em] inspired primitives \\[-0.3em] (Section~\ref{sec:semantics-translation-transl}) \end{tabular}
  & \hspace{-0.5em}\begin{tabular}{l} Concrete reduction rules for \\[-0.3em] comonadically-inspired primitives \\[-0.3em] (Sections~\ref{sec:semantics-proofs-df} and \ref{sec:semantics-proofs-impl}) \end{tabular} \\
\hline
  &
  & \hspace{-0.5em}\begin{tabular}{l} Reduction rules for language-\\[-0.3em] specific extensions (\kvd{prev}, \ident{?p}) \\[-0.3em] (Sections~\ref{sec:semantics-proofs-df} and \ref{sec:semantics-proofs-impl}) \end{tabular} \\
\hline
  & \hspace{-0.5em}\begin{tabular}{l} Sketch of generalized \\[-0.3em] syntactic soundness \\[-0.3em] (Section~\ref{sec:semantics-generalising}) \end{tabular}
  & \hspace{-0.5em}\begin{tabular}{l} Syntactic soundness \\[-0.3em] (Sections~\ref{sec:semantics-proofs-df} and \ref{sec:semantics-proofs-impl}) \end{tabular} \\ \hline
\end{tabular}
\end{center}

% ==================================================================================================

%\begin{flushright}{\slshape    
%Even the most puritanical rationalist will be forced to stop arguing and \\
%use propaganda [...] because the psychological conditions have disappeared \\
%that allowed effective argument and therefore influence over the others.} \\ \medskip
%--- Paul Feyerabend, \emph{Against Method} \cite{philosophy-feyerabend}
%\end{flushright}
%\vspace{2em}

% ==================================================================================================

\chapter{Flat coeffect calculus} 
\label{ch:flat} 

Successful programming language abstractions need to generalize a wide range of recurring
problems while capturing the key commonalities. These two aims are typically in opposition -- 
more general abstractions are less powerful, while less general abstractions cannot be
used as often.

In the previous chapter, we outlined a number of systems that capture how computations
access the environment in which they are executed. We identified two kinds of systems --
\emph{flat} capturing whole-context properties and \emph{structural} capturing per-variable
properties. As we show in Chapter~\ref{ch:unified}, the systems can be unified using a single 
abstraction. This is useful when implementing and composing the systems, but such abstraction 
is  \emph{less powerful} -- \ie~its generality hides useful properties that we can see 
when we consider the systems separately. For this reason, this and the next chapter discusses 
\emph{flat} and \emph{structural} systems separately.

% ==================================================================================================

\section{Introduction}
\label{sec:flat-intro}

In the previous chapter, we looked at three important examples of systems that track whole-context 
properties. The type systems for whole-context liveness (Section~\ref{sec:applications-flat-live}) 
and whole-context data-flow (Section~\ref{sec:applications-flat-dataflow}) have a very similar 
structure -- their lambda abstraction duplicates the requirements and their application arises
from the combination of \emph{sequential} and \emph{point-wise} composition.

The system for tracking of implicit parameters (Section~\ref{sec:applications-flat-impl}), and
similar systems for rebindable resources, differ in two ways. In lambda abstraction, they split
the context requirements between the declaration-site and the call-site and they use only a single
operator on the indices, typically $\cup$.

%---------------------------------------------------------------------------------------------------

\subsection{Contributions}

All of the examples are practically useful and important and so we want to be able to capture all
of them. Despite the differences, the systems can fit the same framework. The contributions of this
chapter are as follows:

\begin{itemize}
\item We present a \emph{flat coeffect calculus} with a type system that is parameterized by a 
  \emph{flat coeffect algebra} and can be instantiated to obtain all of the three examples
  discussed (Section~\ref{sec:flat-calculus}).
  
\item We give the equational theory of the calculus and discuss type-preservation for call-by-name
  and call-by-value reduction (Section~\ref{sec:flat-syntax}). We also extend the calculus
  with pairs and recursion (Section~\ref{sec:flat-exts}).
  
\item We present the semantics of the calculus in terms of \emph{indexed comonads}, which is a
  generalization of comonads, a category-theoretical dual of monads (Section~\ref{sec:flat-semantics}).
  The semantics provides deeper insight into how (and why) the calculus works.

\item We develop an alternative presentation of the system in terms of a simple structure 
  (semi-lattice) and use it to develop a type inference algorithm for flat coeffect calculus.
\end{itemize}

%---------------------------------------------------------------------------------------------------

\subsection{Related work}

The development in this chapter can be seen as a counterpart to the well-known development of 
\emph{effect systems} \cite{effects-gifford} and the use of \emph{monads} \cite{monad-notions}
in programming languages. The syntax and type system of the flat coeffect calculus follows 
similar style as effect systems \cite{effects-polymorphic,effects-talpin-et-al}, but differs
in the structure, as explained in the previous chapter, most importantly in lambda abstraction.

Wadler and Thiemann famously show a correspondence between effect systems to monads 
\cite{monads-effects-marriage}, relating effectful functions $\tau_1 \xrightarrow{\sigma} \tau_2$ 
to monadic computations $M^\sigma \tau_1 \rightarrow \tau_2$. In this chapter, we show a similar
correspondence between \emph{coeffect systems} and \emph{comonads}. However, due to the asymmetry 
of $\lambda$-calculus, this is not a simple mechanical dualization.

The main purpose of the comonadic semantics presented in this chapter is to provide a semantic
motivation for the flat coeffect calculus. The semantics is inspired by the work of Uustalu and
Vene \cite{comonads-notions} who present the semantics of contextual computations (mainly for
data-flow) in terms of comonadic functions $C \tau_1 \rightarrow \tau_2$. Our \emph{indexed 
comonads} annotate the structure with information about the required context, \ie~$C^\sigma \tau_1 \rightarrow \tau_2$.
This is similar to the recent work on \emph{parameterized monads} by Katsumata \cite{monads-parametric}.

% ==================================================================================================

\section{Flat coeffect calculus}
\label{sec:flat-calculus}

The flat coeffect calculus is defined in terms of \emph{flat coeffect algebra}, which defines
the structure of context annotations, such as $\cclrd{r}, \cclrd{s}, \cclrd{t}$. These can be
sets of implicit parameters, integers or other values. The expressions of the calculus are those
of the $\lambda$-calculus with \emph{let} binding; assuming $T$ ranges over base types, the 
types of the calculus are defined as follows:
%
\begin{equation*}
\begin{array}{rcl}
e &::=& x \sep \lambda x.e \sep e_1~e_2 \sep \kvd{let}~x = e_1~\kvd{in}~e_2\\
\tau &::=& T \sep \tau_1 \xrightarrow{\cclrd{r}} \tau_2
\end{array}
\end{equation*}
%
We discuss pairs and recursion in Section~\ref{sec:flat-exts}. The type $\tau_1 \xrightarrow{\cclrd{r}} \tau_2$
represents a function from $\tau_1$ to $\tau_2$ that requires additional context $\cclrd{r}$.
It can be viewed as a pure function that takes $\tau_1$ \emph{with} or \emph{wrapped in} a 
context $r$. 

In the categorical semantics, the function $\tau_1 \xrightarrow{\cclrd{r}} \tau_2$ is modelled
by a morphism $C^{\cclrd{r}} \tau_1 \rightarrow \tau_2$. However, the object $C^{\cclrd{r}}$
does not exist as a syntactical value. This is because we use comonads to define the 
\emph{semantics} rather than \emph{embedding} them into the language as in the meta-language
approaches (the distinction between the two approaches has been discussed in detail in 
Section~\ref{sec:path-sem-langs}). The annotations $\cclrd{r}$ are formed by an algebraic
structure discussed next.

%---------------------------------------------------------------------------------------------------

\subsection{Reconciling lambda abstraction}
\label{sec:flat-calculus-lambda}

Recall the lambda abstraction rules for the implicit parameters system (annotating the context
with sets of required parameters) and the data-flow system (annotating the context with the
number of past required values):
%
\begin{equation*}
\tyrule{abs-imp}
  {\coctx{\Gamma, x:\tau_1}{\aclrd{r} \cup \aclrd{s}} \vdash e : \tau_2}
  {\coctx{\Gamma}{\aclrd{r}} \vdash \lambda x.e : \tau_1 \xrightarrow{\aclrd{s}} \tau_2 }
\;
\tyrule{abs-df}
  {\coctx{\Gamma, x:\tau_1}{\aclrd{n}} \vdash e : \tau_2}
  {\coctx{\Gamma}{\aclrd{n}} \vdash \lambda x.e : \tau_1 \xrightarrow{\aclrd{n}} \tau_2 }
\end{equation*}

~

In order to capture both systems using a single calculus, we need a way of unifying the two
systems. For the data-flow system, this can be achieved by over-approximating the number of 
required past elements:
%
\begin{equation*}
\tyrule{abs-min}
  {\coctx{\Gamma, x:\tau_1}{\textnormal{min}(\aclrd{n}, \aclrd{m})} \vdash e : \tau_2}
  {\coctx{\Gamma}{\aclrd{n}} \vdash \lambda x.e : \tau_1 \xrightarrow{\aclrd{m}} \tau_2 }
\end{equation*}
%
The rule (\emph{abs-df}) is admissible in a system that includes the (\emph{abs-min}) rule. 
If we include sub-typing rule (on annotations of functions) and sub-coeffecting rule (on 
annotations of contexts), then the reverse is also true -- because 
$\textit{min}(\aclrd{n}, \aclrd{m}) \leq \aclrd{m}$ and $\textit{min}(\aclrd{n}, \aclrd{m}) \leq \aclrd{n}$.

%---------------------------------------------------------------------------------------------------

\subsection{Flat coeffect algebra}
To make the flat coeffect system general enough, the algebra consists of three operations.
Two of them, $\cseq$ and $\cpar$, represent the \emph{sequential} and \emph{point-wise} composition, 
respectively and the third one, $\czip$ represents context \emph{merging}. The term merging should be 
understood semantically -- the operation models what happens when the semantics of lambda abstraction 
combines context available at the declaration-site and the call-site.

In addition to the three operations, we also require two special values used to annotate
variable access and constant access and a relation that defines the ordering.

\begin{definition}
A \emph{\cclrd{flat coeffect algebra}} $(\C, \cseq, \cpar, \czip, \cunit, \czero, \cleq)$ is a set 
$\C$ together with elements $\cunit, \czero \in \C$, relation $\cleq$ and binary operations 
$\cseq, \cpar, \czip$ such that $(\C, \cseq, \cunit)$ and $(\C, \cpar, \czero)$ are monoids,
$(\C, \cleq)$ is a pre-order and $(\C, \czip)$ is a band (idempotent semigroup). That is, 
for all $r,s,t\in \C$:
%
\begin{equation*}
\begin{array}{ccr}
r \;\cseq\; (s \;\cseq\; t) = (r \;\cseq\; s) \;\cseq\; t  &
\cunit \;\cseq\; r = r = r \;\cseq\; \cunit &
\textnormal{(monoid)}   
\\
r \;\cpar\; (s \;\cpar\; t) = (r \;\cpar\; s) \;\cpar\; t &
\czero \;\cpar\; r = r = r \;\cpar\; \czero &
\textnormal{(monoid)}   
\\
r \;\czip\; (s \;\czip\; t) = (r \;\czip\; s) \;\czip\; t &
r\; \czip\; r = r &
\textnormal{(band)}   
\\
\textnormal{if}~~r\; \cleq\; s ~~\textnormal{and}~~s\; \cleq\; t~~\textnormal{then}~~r\; \cleq\; t&
t\; \cleq\; t &
\textnormal{(pre-order)}   
\end{array}
\end{equation*}
%
In addition, the following distributivity axioms hold:
\begin{align*}
\quad (\cclrd{r}\, \cpar\, \cclrd{s}) \;\cseq\; \cclrd{t} & = (\cclrd{r} \,\cseq\, \cclrd{t}) \;\cpar\; (\cclrd{s}\, \cseq\, \cclrd{t}) \\
\quad \cclrd{t} \;\cseq\; (\cclrd{r}\, \cpar\, \cclrd{s}) & = (\cclrd{t} \,\cseq\, \cclrd{r}) \;\cpar\; (\cclrd{t}\, \cseq\, \cclrd{s})
\end{align*}
\end{definition}

\noindent
In two of the three systems, some of the operators of the flat coeffect algebra coincide,
but the data-flow system requires all three. Similarly, the two special elements also 
coincide in some, but not all systems. The required laws are motivated by the aim to capture
common properties of the three examples, without unnecessarily restricting the system:

\begin{itemize}
\item The monoid $(\C, \cseq, \cunit)$ represents \emph{sequential} composition of (semantic)
functions. The laws of a monoid are required in order to form a category structure in the 
categorical model (Section~\ref{sec:flat-semantics}).

\item The monoid $(\C, \cpar, \czero)$ represents \emph{point-wise} composition, \ie~the case when the
same context is passed to multiple (independent) computations. The monoid laws guarantee 
that usual syntactic transformations on tuples and the unit value (Section~\ref{sec:flat-exts})
preserve the coeffect. 

\item For the $\czip$ operation, we require associativity and idempotence. The idempotence
requirement makes it possible to duplicate the coeffects and place the same requirement on both
call-site and declaration-site, \ie~it makes the (\emph{abs-df}) rule admissible. In some cases, 
the operator forms a monoid with the unit being the greatest element of the set. This alternative is 
discussed when we consider recursion (Section~\ref{sec:flat-exts}).
\end{itemize}

It is worth noting that the operators $\cpar$ and $\czip$ are dual in some of the systems. For 
example, in data-flow computations, they are \emph{max} and \emph{min} respectively. However, this
duality does not hold for implicit parameters. Using the syntactic reading, they represent 
\emph{merging} and \emph{splitting} of context requirements -- in the (\emph{abs}) rule, 
$\czip$ appears in the assumption and the combined context requirements of the body are split 
between two positions in the conclusions; in the (\emph{app}) rule, $\cpar$ appears in the 
conclusion and combines two context requirements from the assumptions.

\paragraph{Ordering.}

The flat coeffect algebra requires a pre-order relation $\cleq$, which is used to define 
sub-coeffecting rule of the type system. When the monoid $(\C, \cpar, \czero)$ is idempotent
and commutative monoid (semi-lattice), the $\cleq$ relation can be defined in terms of $\cpar$ as:
%
\begin{equation*}
r \;\cleq\; s \;\Longleftrightarrow\; r \;\cpar\; s \;=\; s
\end{equation*}
%
This definition is consistent with all three examples that motivate flat coeffect calculus, but
it cannot be used with the structural coeffects (where it fails for the bounded reuse 
calculus) and so we choose not to use it.

Furthermore, the $\cunit$ coeffect is often the top (greatest) or the bottom (smallest) 
element of the semi-lattice, but not in general. When this is the case, we are able
to prove certain properties of the calculus (Section~\ref{sec:syntax}).

%---------------------------------------------------------------------------------------------------

\subsection{Understanding flat coeffects}

Before looking at the type system in Figure~\ref{fig:flat-types}, let us clarify how the rules
should be understood. The coeffect calculus provides both analysis of context dependence (type 
system) and semantics for context (how it is propagated). These two aspects provide different
ways of reading the judgements $\coctx{\Gamma}{\cclrd{r}} \vdash e : \tau$ and the typing rules
used to define it.

\begin{itemize}
\item \textbf{Analysis of context dependence.}
Syntactically, coeffect annotations $\cclrd{r}$ model \emph{context requirements}. This means
we can over-approximate them and require more than is actually needed at runtime. 

Syntactically, the typing rules should be read top-down. In (\emph{app}), the context requirements 
of multiple assumptions are \emph{merged}; in (\emph{abs}), they are split between the declaration-site
and the call-site.

\item \textbf{Semantics of context passing.}
Semantically, coeffect annotations $\cclrd{r}$ mo\-del \emph{contextual capabilities}. This means
that we can throw away capabilities, if a sub-expression requires fewer than we 
currently have.

Semantically, the typing rules should be read bottom-up. In application, the capabilities 
provided to the term $e_1~e_2$ are \emph{split} between the two sub-expressions; in abstraction,
the capabilities provided by the call-site and declaration-site are \emph{merged} and passed
to the body.
\end{itemize}

The reason for this asymmetry follows from the fact that the context appears in a \emph{negative
position} in the semantic model (Section~\ref{sec:flat-semantics}). It means that we need to be
careful about using the words \emph{split} and \emph{merge}, because they can be read as meaning
opposite things. To disambiguate, we always use the term \emph{context requirements} when using
the syntactic view and \emph{context capabilities} or just \emph{available context} when using 
the semantic view.

%---------------------------------------------------------------------------------------------------

\begin{figure}[t]
\begin{equation*}
\tyrule{var}
  {x : \tau \in \Gamma}
  {\coctx{\Gamma}{\cunit} \vdash x : \tau }
\end{equation*}
\begin{equation*}
\tyrule{const}
  {c : \tau \in \Delta}
  {\coctx{\Gamma}{\czero} \vdash c : \tau }
\end{equation*}
\begin{equation*}
\tyrule{sub}
  {\coctx{\Gamma}{\cclrd{r'}} \vdash e : \tau }
  {\coctx{\Gamma}{\cclrd{r}} \vdash e : \tau }\quad\quad(\cclrd{r'} \cleq \cclrd{r})
\end{equation*}
\begin{equation*}
\tyrule{app}
  {\coctx{\Gamma}{\cclrd{r}} \vdash e_1 : \tau_1 \xrightarrow{\cclrd{t}} \tau_2 &
   \coctx{\Gamma}{\cclrd{s}} \vdash e_2 : \tau_1 }
  {\coctx{\Gamma}{\cclrd{r} \;\cpar\; (\cclrd{s} \,\cseq\, \cclrd{t})} \vdash e_1~e_2 : \tau_2}
\end{equation*}
\begin{equation*}
\tyrule{abs}
  {\coctx{\Gamma, x:\tau_1}{\cclrd{r}\;\czip\;\cclrd{s}} \vdash e : \tau_2}
  {\coctx{\Gamma}{\cclrd{r}} \vdash \lambda x.e : \tau_1 \xrightarrow{\cclrd{s}} \tau_2 }
\end{equation*}
\begin{equation*}
\tyrule{let}
  { \coctx{\Gamma}{\cclrd{r}} \vdash e_1 : \tau_1 &
    \coctx{\Gamma, x:\tau_1}{\cclrd{s}} \vdash e_2 : \tau_2}
  {\coctx{\Gamma}{\cclrd{s} \;\cpar\; (\cclrd{s} \,\cseq\, \cclrd{r})} \vdash \kvd{let}~x=e_1~\kvd{in}~e_2 : \tau_2 }
\end{equation*}

\caption{Type system for the flat coeffect calculus}
\label{fig:flat-types}
\end{figure}

%---------------------------------------------------------------------------------------------------

\subsection{Flat coeffect types}
\label{sec:flat-calculus-types}

The type system for flat coeffect calculus is shown in Figure~\ref{fig:flat-types}. Variables 
(\emph{var}) and constants (\emph{const}) are annotated with special values provided by the 
coeffect algebra. Following the top-down syntactic reading, the (\emph{sub}) rule allows us to 
treat an expression with fewer context requirements as an expression with more context requirements. 

The (\emph{abs}) rule is defined as discussed in Section~\ref{sec:flat-calculus-lambda}. The
body is annotated with context requirements $\cclrd{r} \,\czip\, \cclrd{s}$, which are then split
between the context-requirements on the declaration-site $\cclrd{r}$ and context-requirements on
the call-site $\cclrd{s}$. Examples of the $\czip$ operator are discussed in the next section.

In function application (\emph{app}), context requirements of both expressions and the 
function are combined as discussed in Chapter~\ref{ch:applications}. The pointwise composition
$\cpar$ is used to combine the context requirements of the expression representing a function 
$\cclrd{r}$ and the context requirements of the argument, sequentially composed with the 
context-requirements of the function $\cclrd{s}\, \cseq \,\cclrd{t}$.

The type system also includes a rule for let-binding. The rule is \emph{not} equivalent to the
derivation for $(\lambda x.e_2)~e_1$, but it represents one admissible typing derivation. We
return to let-binding after looking a number of examples. Additional constructs such 
as recursion and tuples are covered in Section~\ref{sec:flat-exts}.

%---------------------------------------------------------------------------------------------------

\subsection{Examples of flat coeffects}

The flat coeffect calculus generalizes the flat systems discussed in 
Section~\ref{sec:applications-flat} of the previous chapter. We can instantiate it to a specific
use just by providing a flat coeffect algebra. The following summary defines the systems for implicit 
parameters, liveness and data-flow. For the latter two, we obtain more general (but compatible) rule 
for lambda abstraction.

\begin{example}[Implicit parameters]
Assuming \ident{Id} is a set of implicit parameter names, the flat coeffect algebra 
is formed by $(\mathcal{P}(\ident{Id}), \cup, \cup, \cup, \emptyset, \emptyset, \subseteq)$.
\end{example}

\noindent
For simplicity, we assume that all parameters have the same type $\rho$ and so the annotations only
track sets of names. The definition uses set union for all three operations. Both variables and
constants are are annotated with $\emptyset$ and the ordering is defined by $\subseteq$. The 
definition satisfies the flat coeffect algebra laws because $(S, \cup, \emptyset)$ is an idempotent, 
commutative monoid. The system has a single additional typing rule for accessing the value of a
parameter:
%
\begin{equation*}
\tyrule{param}
  { \ident{?p} \in \cclrd{c} }
  { \coctx{\Gamma}{\cclrd{c}} \vdash \ident{?p} : \rho }
\end{equation*}
%
The rule specifies that the accessed parameter $\ident{?p}$ needs to be in the set of required
parameters $\cclrd{c}$. As discussed earlier, we use the same type $\rho$ for all parameters, but
it is easy to define an extension tracking set of parameters with type annotations.

\begin{example}[Liveness]
Let $L=\{ \ident{L}, \ident{D} \}$ be a two-point lattice such that $\ident{D} \sqsubseteq \ident{L}$
with a join $\sqcup$ and meet $\sqcap$. The flat coeffect algebra for liveness is then formed by
$(L, \sqcap, \sqcup, \sqcap, \ident{L}, \ident{D}, \sqsubseteq)$.
\end{example}

\noindent
As in Section~\ref{sec:applications-flat-live}, sequential composition $\cseq$ is modelled by 
the meet operation $\sqcap$ and point-wise composition $\cpar$ is modelled by join $\sqcup$. 
Two-point lattice is a commutative, idempotent monoid. The distributivity 
$(r \sqcup s) \sqcap t = (r \sqcap t) \sqcup (s \sqcap t)$ does not hold for \emph{every} 
lattice, but it trivially holds for a two-point lattice used here.

The definition uses join $\sqcup$ for the $\czip$ operator that is used by lambda abstraction.
This means that, when the body is live $\ident{L}$, both declaration-site and call-site are 
marked as live $\ident{L}$. When the body is dead $\ident{D}$, the declaration-site and call-site
can be marked as dead $\ident{D}$, or as live $\ident{L}$, which is less precise, but permissible
over-approximation, which could otherwise be achieved via sub-typing.

\begin{example}[Data-flow]
In data-flow, context is annotated with natural numbers and the flat coeffect algebra is formed 
by $(\mathbb{N}, +, \mathit{max}, \mathit{min}, 0, 0, \leq)$.
\end{example}

\noindent
As discussed earlier, sequential composition $\cseq$ is represented by $+$ and point-wise 
composition $\cpar$ uses $\emph{max}$. For data-flow, we need a third separate operator for
lambda abstraction. Annotating the body with $\emph{min}(\cclrd{r}, \cclrd{s})$ ensures that
both call-site and declaration-site annotations are equal or greater than the annotation 
of the body. As with liveness, this allows over-approximation. 

As required by the laws, $(\mathbb{N}, +, 0)$ and $(\mathbb{N}, \mathit{max}, 0)$ form monoids
and $(\mathbb{N}, \mathit{min})$ forms a band. Note that data-flow is our first example where 
$+$ is not idempotent. The distributivity laws require the following to be the case:
$\mathit{max}(r,s) + t = \mathit{max}(r+t, s+t)$, which is easy to see. Finally, a simple 
data-flow language includes an additional rule for $\kvd{prev}$:
%
\begin{equation*}
\tyrule{prev}
  { \coctx{\Gamma}{\cclrd{c}} \vdash e : \tau }
  { \coctx{\Gamma}{\cclrd{c}+1} \vdash \kvd{prev}~e : \tau }
\end{equation*}
%
As a further example that was not covered earlier, it is also possible to combine liveness analysis
and data-flow. In the above calculus, $0$ denotes that we require current value, but no previous
values. However, for constants, we do not even need the current value.

\begin{example}[Optimized data-flow]
In optimized data-flow, context is annotated with natural numbers extended with the $\bot$ element,
that is $\mathbb{N}_{\bot} = \mathbb{N} \cup \{\bot \}$ such that $\forall n \in \mathbb{N}. \bot \leq n$.
The flat coeffect algebra is $(\mathbb{N}_{\bot}, +, \mathit{max}, \mathit{min}, 0, \bot, \leq)$
where $m + n$ is $\bot$ whenever $m=\bot$ or $n=\bot$ and \emph{min}, \emph{max} treat $\bot$ as the
least element.
\end{example}

\noindent
Note that $(\mathbb{N}_{\bot}, +, 0)$ is a monoid for the extended definition of $+$,
$(\mathbb{N}, \emph{max}, \bot)$ is also a monoid and $(\mathbb{N}, \emph{min})$ is a band.
The required distributivity laws also holds for this algebra.

%---------------------------------------------------------------------------------------------------

\subsection{Typing of let binding}

Recall the (\emph{let}) rule in Figure~\ref{fig:flat-types}. It annotates the expression 
$\kvd{let}~x=e_1~\kvd{in}~e_2$ with context requirements $\cclrd{s}\;\cpar\;(\cclrd{s}\,\cseq\,\cclrd{r})$.
This is a special case of typing of an expression $(\lambda x.e_2)~e_1$, using the idempotence
of $\czip$ as follows:
%
\begin{equation*}
\tyrule{app}
  {\begin{array}{l}
   \vspace{-1.5em}
   \coctx{\Gamma}{\cclrd{r}} \vdash e_1 : \tau_1
   \end{array} &
   \tyruler{abs}
       { \coctx{\Gamma, x:\tau_1}{\cclrd{s}} \vdash e_2 : \tau_2 }
       { \coctx{\Gamma}{\cclrd{s}} \vdash \lambda x.e_2 : \tau_1 \xrightarrow{\cclrd{s}} \tau_2 } }
  { \coctx{\Gamma}{\cclrd{s}\;\cpar\;(\cclrd{s}\,\cseq\,\cclrd{r})} \vdash (\lambda x.e_2)~e_1 : \tau_2 }    
\end{equation*}
%
This design decision is similar to ML value restriction, but it works the other way round. Our
\emph{let} binding is more restrictive rather than more general. The choice is motivated by the 
fact that the typing obtained using the special rule for let-binding is more precise (with respect 
to sub-coeffecting) for all the examples considered in this chapter. Table~\ref{tab:flat-simplelet}
shows how the coeffect annotations are simplified for our examples.

\begin{table}[!h]
\begin{center}
\begin{tabular}{ | l | c | c |}
\hline
& \textbf{\footnotesize Definition\hspace{1em}} & \textbf{\footnotesize Simplified\hspace{1em}} \\ \hline
\hspace{-1em}{\footnotesize Implicit parameters} & $\cclrd{s} \cup (\cclrd{s} \cup \cclrd{r})$ & $\cclrd{s} \cup \cclrd{r}$ \\ \hline
\hspace{-1em}{\footnotesize Liveness} & $\cclrd{s} \sqcap (\cclrd{s} \sqcup \cclrd{r})$ & $\cclrd{s}$ \\ \hline
\hspace{-1em}{\footnotesize Data-flow} & $\mathit{max}(\cclrd{s}, \cclrd{s} + \cclrd{r})$ & $\cclrd{s} + \cclrd{r}$ \\ \hline
\end{tabular}
\end{center}

\vspace{-0.5em}
\caption{Simplified annotation for let binding in sample flat calculi}
\label{tab:flat-simplelet}
\end{table}

\noindent
The simplified annotations directly follow from the definitions of particular flat coeffect 
algebras. It is perhaps somewhat unexpected that the annotation can be simplified in different
ways for different examples. 

To see that the simplified annotations are \emph{better}, assume that we used arbitrary 
splitting $\cclrd{s} = \cclrd{s_1}\,\czip\,\cclrd{s_2}$ rather than idempotence. The
``Definition'' column would use $\cclrd{s_1}$ and $\cclrd{s_2}$ for the first and second 
$\cclrd{s}$, respectively. The corresponding simplified annotation (using idempotence) would
have $\cclrd{s_1}\,\czip\,\cclrd{s_2}$ instead of $\cclrd{s}$. For all our systems, the 
simplified annotation (on the right) is more precise than the original (on the left):
%
\begin{equation*}
\begin{array}{rclll}
\cclrd{s_1} \cup (\cclrd{s_2} \cup \cclrd{r}) &\supseteq& (\cclrd{s_1} \cup \cclrd{s_2}) \cup \cclrd{r} 
  && \textnormal{(implicit parameters)}\\
\cclrd{s_1} \sqcap (\cclrd{s_2} \sqcup \cclrd{r}) &\sqsupseteq&  (\cclrd{s_1} \sqcap \cclrd{s_2}) 
  && \textnormal{(liveness)} \\
\mathit{max}(\cclrd{s_1}, \cclrd{s_2} + \cclrd{r}) &\geq& \mathit{min}(\cclrd{s_1}, \cclrd{s_2}) + \cclrd{r} 
  &\quad& \textnormal{(data-flow)} \\
\end{array}
\end{equation*}
%
The inequality cannot be proved from other properties of the flat coeffect algebra. To make
the flat coeffect system as general as possible, we do not \emph{in general} require it as
an additional axiom, although the above examples provide reasonable basis for requiring 
that the specialized annotation for let binding is the least possible annotation for the 
expression $(\lambda x.e_2)~e_1$.

% ==================================================================================================

\section{Categorical motivation}
\label{sec:flat-semantics}

The type system of flat coeffect calculus arises as a generalization of the examples discussed in 
Chapter~\ref{ch:applications}, but we can also obtain it by looking at the categorical semantics
of context-dependent computations. This is a direction that we explore in this section. Although
the development presented here is interesting in its own, our main focus is \emph{using} categorical
semantics to motivate and explain the design of flat coeffect calculus.

% --------------------------------------------------------------------------------------------------

\subsection{Categorical semantics}

As discussed in Section~\ref{sec:path-sem}, categorical semantics interprets terms as morphisms
in some category. For typed calculi, the semantics defined by $\sem{-}$ usually interprets typing 
judgements $x_1 \!:\! \tau_1 \ldots x_n \!:\! \tau_n \vdash e: \tau$ as morphisms 
$\sem{\tau_1 \times \ldots \times \tau_n} \rightarrow \sem{\tau}$.

As a best known example, Moggi \cite{monad-notions} showed that the semantics of various effectful 
computations can be captured uniformly using the (\emph{strong}) \emph{monad} structure. In that 
approach, computations are interpreted as $\tau_1 \times \ldots \times \tau_n \rightarrow \mtyp{}{\tau}$
for some monad $\mtyp{}{}$. For example, $\mtyp{}{\alpha} = \alpha \cup \{ \bot \}$ models 
partiality (maybe monad), $\mtyp{}{\alpha} = \mathcal{P}(\alpha)$ models non-determinism (list 
monad) and $\mtyp{}{\alpha} = (\alpha \times S)^S$ models side-effects (state monad). Here, the 
structure of a strong monad provides necessary ``plumbing'' for composing monadic computations.

Following similar approach to Moggi, Uustalu and Vene \cite{comonads-notions} showed that 
(\emph{monoidal}) \emph{comonads} uniformly capture the semantics of various kinds of context-dependent 
computations~\cite{comonads-notions}. For example, data-flow computations over non-empty lists
$\ident{NEList}\,\alpha = \alpha + (\alpha \times \ident{NEList}\,\alpha)$ are modelled using
the non-empty list comonad.

The monadic and comonadic model outlined here represents at most a binary analysis of effects or 
context-dependence. A function $\tau_1 \rightarrow \tau_2$ performs \emph{no} effects (requires no 
context) whereas $\tau_1 \rightarrow \mtyp{}{\tau_2}$ performs \emph{some} effects and
$\ctyp{}{\tau_1} \rightarrow \tau_2$ requires \emph{some} context. In the next section, we introduce
\emph{indexed comonads}, which provide a more precise analysis and let us model computations with
context requirements $\cclrd{r}$ as functions $\ctyp{\cclrd{r}}{\tau_1} \rightarrow \tau_2$ using
an \emph{indexed comonad} $\ctyp{\cclrd{r}}{}$.

% --------------------------------------------------------------------------------------------------

\subsection{Introducing comonads}

In category theory, \emph{comonad} is a dual of \emph{monad}. Informally, we get a comonad by 
taking a monad and ``reversing the arrows''. More formally, one of the equivalent definitions of
comonad looks as follows:

\begin{definition}
A \emph{comonad} over a category $\catc$ is a triple $(C, \ident{counit}, \ident{cobind})$ where:
\begin{compactitem}
\item $C$ is a mapping on objects (types) $C : \catc \rightarrow \catc$
\item $\ident{counit}$ is a mapping $\ctyp{}{\alpha} \rightarrow \alpha$ 
\item $\ident{cobind}$ is a mapping $(\ctyp{}{\alpha} \rightarrow \beta) 
  \rightarrow (\ctyp{}{\alpha} \rightarrow \ctyp{}{\beta})$
\end{compactitem}
such that, for all $f:\ctyp{}{\alpha} \rightarrow \beta$ and $g:\ctyp{}{\beta} \rightarrow \gamma$:
\begin{align}
\tag{\emph{left identity}}
  \ident{cobind}~\ident{counit} &= \idf{}
  \\
\tag{\emph{right identity}}
  \ident{counit} \circ \ident{cobind}~f &= f
  \\
\tag{\emph{associativity}}
  \ident{cobind}~(g \circ \ident{cobind}~f) &= (\ident{cobind}~g) \circ (\ident{cobind}~f)
\end{align}
\end{definition}

\noindent
From the functional programming perspective, we can see $\ctyp{}{}$ as a parametric data type such as
\ident{NEList}. The $\ident{counit}$ operations extracts a value $\alpha$ from a value that carries 
additional context $\ctyp{}{\alpha}$. The $\ident{cobind}$ operation turns a context-dependent function 
$\ctyp{}{\alpha} \rightarrow \beta$ into a function that takes a value with context, applies
the context-dependent function to value(s) in the context and then propagates the context.

As mentioned earlier, Uustalu and Vene \cite{comonads-notions} use comonads to model data-flow
computations. They describe infinite (coinductive) streams and non-empty lists as example comonads.

\begin{example}[Non-empty list]
A non-empty list is a recursive data-type defined as $\ident{NEList}\,\alpha = \alpha + (\alpha \times \ident{NEList}\,\alpha)$.
We write \kvd{inl} and \kvd{inr} for constructors of the left and right cases, respectively. The 
type \ident{NEList} forms a comonad together with the following \ident{counit} and \ident{cobind} mappings:
%
\begin{equation*}
\begin{array}{rclll}
\ident{counit}~l &\narrow{=}& h &\quad&\textnormal{when}~l=\kvd{inl}~h\\[-0.25em]
\ident{counit}~l &\narrow{=}& h &&\textnormal{when}~l=\kvd{inr}~(h, t)\\[0.5em]
\ident{cobind}\,f~l &\narrow{=}& \kvd{inl}~(f\,l) &&\textnormal{when}~l=\kvd{inl}~h\\[-0.25em]
\ident{cobind}\,f~l &\narrow{=}& \kvd{inr}~(f\,l,\;\ident{cobind}~f~t) &&\textnormal{when}~l=\kvd{inr}~(h, t)
\end{array}
\end{equation*}
\end{example}

\noindent
The \ident{counit} operation returns the head of the non-empty list. Note that it is crucial that
the list is \emph{non-empty}, because we always need to be able to obtain a value. The \ident{cobind}
defined here returns a list of the same length as the original where, for each element, the 
function $f$ is applied on a \emph{suffix} list starting from the element. Using a simplified
notation for list, the result of applying \ident{cobind} to a function that sums elements of a
list gives the following behaviour:
%
\begin{equation*}
\ident{cobind}~\ident{sum}~(7,6,5,4,3,2,1,0) = (28,21,15,10,6,3,1,0)
\end{equation*}
%
The fact that the function $f$ is applied to a \emph{suffix} is important in order to satisfy the
\emph{left identity} law, which requires that $\ident{cobind}~\ident{counit}~l = l$.

It is also interesting to examine some data types that do \emph{not} form a comonad. As already
mentioned, list $\ident{List}~\alpha = 1 + (\alpha \times \ident{List}~\alpha)$ is not a comonad,
because the \ident{counit} operation is not defined for the value $\kvd{inl}~()$. Similarly,
the \ident{Maybe} data type defined as $1 + \alpha$ is not a comonad for the same reason.
However, if we consider flat coeffect calculus for liveness, it appears natural to model computations
as function $\ident{Maybe}~\tau_1 \rightarrow \tau_2$. To use such model, we first need to 
generalise comonads to \emph{indexed comonads}.

% --------------------------------------------------------------------------------------------------

\subsection{Generalising to indexed comonads}
\label{sec:flat-semantics-idx}

The flat coeffect algebra includes a monoid $(\C, \cseq, \cunit)$, which defines the behaviour of
sequential composition, where the annotation $\cunit$ represents a variable access. An indexed 
comonad is formed by a data type (object mapping) $\ctyp{\cclrd{r}}{\alpha}$ where the annotation 
$\cclrd{r}$ determines what context is required. 

\begin{definition}
Given a monoid $(\C, \cseq, \cunit)$ with binary operator $\cseq$ and unit $\cunit$, an 
\emph{indexed comonad} over a category $\catc$ is a triple 
$(\ctyp{\cclrd{r}{}}, \ident{counit}_{\cunit}, \ident{cobind}_{\cclrd{r}, \cclrd{s}})$ where:

\begin{compactitem}
\item $\ctyp{\cclrd{r}}{}$ for all $\cclrd{r} \in \C$ is a family of object mappings 
\item $\ident{counit}_{\cunit}$ is a mapping $\ctyp{\cunit}{\alpha} \rightarrow \alpha$ 
\item $\ident{cobind}_{\cclrd{r}, \cclrd{s}}$ is a mapping $(\ctyp{\cclrd{r}}{\alpha} \rightarrow \beta) 
  \rightarrow (\ctyp{\cclrd{r}\cseq\cclrd{s}}{\alpha} \rightarrow \ctyp{\cclrd{s}}{\beta})$
\end{compactitem}
such that, for all $f:\ctyp{\cclrd{r}}{\alpha} \rightarrow \beta$ and $g:\ctyp{\cclrd{s}}{\beta} \rightarrow \gamma$
and the identity $\idf{\cclrd{s}} : \ctyp{\cclrd{s}}\alpha \rightarrow \ctyp{\cclrd{s}}\alpha$:
\begin{align}
\tag{\emph{left identity}}
  \ident{cobind}_{\cunit, \cclrd{s}}~\ident{counit}_{\cunit} &= \idf{}
  \\
\tag{\emph{right identity}}
  \ident{counit}_{\cunit} \circ \ident{cobind}_{\cclrd{r}, \cunit}~f &= f
  \\
\tag{\emph{associativity}}
\hspace{-10em}
  \ident{cobind}_{\cclrd{r}\cseq\cclrd{s},\cclrd{t}}~(g \circ \ident{cobind}_{\cclrd{r}, \cclrd{s}}~f) &= 
    (\ident{cobind}_{\cclrd{s}, \cclrd{t}}~g) \circ (\ident{cobind}_{\cclrd{r}, \cclrd{s}\cseq\cclrd{t}}~f)
\end{align}
\end{definition}

\noindent
Rather than defining a single mapping $\ctyp{}{}$, we are now defining a family of mappings 
$\ctyp{\cclrd{r}}{}$ indexed by a monoid structure. Similarly, the operation $\ident{cobind}_{\cclrd{r}, \cclrd{s}}$ 
operation is now also formed by a \emph{family} of mappings for different pairs of indices 
$\cclrd{r}, \cclrd{s}$. To be fully precise, $\ident{cobind}$ is a family of natural transformations 
and we should include $\alpha, \beta$ as indices, writing $\ident{cobind}_{\cclrd{r},\cclrd{s}}^{\alpha, \beta}$.
For the purpose of this thesis, it is sufficient to treat $\ident{cobind}$ as a family of
mappings or, when it does not  introduce ambiguity, view it as a single mapping.

The $\ident{counit}$ operation is not defined for all $\cclrd{r} \in \C$, but only for 
the unit $\cunit$. We still include the unit as an index writing $\ident{counit}_{\cunit}$, 
but this is merely for symmetry. Crucially, this means that the operation is defined only
for some special contexts.

If we look at the indices in the laws, we can see that the left and right identity 
require $\cunit$ to be the unit of $\cseq$. Similarly, the associativity law implies the 
associativity of the $\cseq$ operator. 

The category that models sequential composition is formed by the unit arrow $\ident{counit}$ 
together with the (associative) composition operation that composes computations with 
contextual requirements as follows:
%
\begin{equation*}
\begin{array}{ccl}
\textnormal{--}\, \hat{\circ} \,\textnormal{--}&\narrow{:}& (\ctyp{\cclrd{r}}{\tau_1} \rightarrow \tau_2) 
  \rightarrow (\ctyp{\cclrd{s}}{\tau_2} \rightarrow \tau_3) 
  \rightarrow (\ctyp{\cclrd{r} \cseq \cclrd{s}}{\tau_1} \rightarrow \tau_3) \\
g \, \hat{\circ} \, f &\narrow{=}& g \circ (\ident{cobind}_{\cclrd{r}, \cclrd{s}} f)
\end{array}
\end{equation*}
%
The composition $\hat{\circ}$ best expresses the intention of indexed comonads. Given two functions
with contextual requirements $\cclrd{r}$ and $\cclrd{s}$, their composition is a function that 
requires $\cclrd{r}\,\cseq,\cclrd{s}$. The contextual requirements propagate \emph{backwards} and
are attached to the input of the composed function.

% --------------------------------------------------------------------------------------------------

\paragraph{Examples.}

Any comonad can be turned into an indexed comonad using a trivial monoid. However, indexed comonads
are more general and can be used with other data types, including indexed \ident{Maybe}. 

\begin{example}[Comonads]
Any comonad $\ctyp{}{}$ is an indexed comonad with an index provided by a trivial monoid $(\{1\},\ast,1)$
where $1\ast 1 = 1$ and $\ctyp{1}{}$ is the underlying mapping $\ctyp{}{}$ of the original comonad. The
operations $\ident{counit}_1$ and $\ident{cobind}_{1,1}$ are defined by the operations $\ident{counit}$
and $\ident{cobind}$ of the comonad.
\end{example}

\begin{example}[Indexed option]
The indexed option comonad is defined over a monoid $(\{ \ident{L},\ident{D} \}, \sqcup,\ident{L})$ 
where $\sqcup$ is defined as earlier, \ie~$\ident{L} = \cclrd{r} \sqcup \cclrd{s} \Longleftrightarrow \cclrd{r}=\cclrd{s}=\ident{L}$.
Assuming $1$ is the unit type inhabited by $()$, the mappings are defined as follows:
%
\begin{equation*}
\begin{array}{l}
\ctyp{\ident{L}}{\alpha} = \alpha\\
\ctyp{\ident{D}}{\alpha} = 1\\
\\
\ident{counit}_{\ident{L}} : \ctyp{\ident{L}}{\alpha} \rightarrow \alpha \\
\ident{counit}_{\ident{L}}~v = v\\
\end{array}
\qquad
\begin{array}{lcl}
\multicolumn{3}{l}{
  \ident{cobind}_{\cclrd{r},\cclrd{s}} ~:~ (\ctyp{\cclrd{r}}{\alpha} \rightarrow \beta) 
    \rightarrow (\ctyp{\cclrd{r}\sqcup\cclrd{s}}{\alpha} \rightarrow \ctyp{\cclrd{s}}{\beta}) }\\
\ident{cobind}_{\ident{L},\ident{L}}~f~x &\narrow{=}& f~x\\
\ident{cobind}_{\ident{L},\ident{D}}~f~() &\narrow{=}& ()\\
\ident{cobind}_{\ident{D},\ident{L}}~f~() &\narrow{=}& f~()\\
\ident{cobind}_{\ident{D},\ident{D}}~f~() &\narrow{=}& ()\\
\end{array}
\end{equation*}
\end{example}

\noindent
The indexed option comonad models the semantics of the liveness coeffect system discussed in 
\ref{sec:applications-flat-live}, where $\ctyp{\ident{L}}{\alpha} = \alpha$ models a live context 
and $\ctyp{\ident{D}}{\alpha}=1$ models a dead context which does not contain a value. The \ident{counit}
operation extracts a value from a live context; \ident{cobind} can be seen as an implementation of 
dead code elimination. The definition only evaluates $f$ when the result is marked as live and is thus
required, and it only accesses $x$ if the function $f$ requires its input.

The indexed family $\ctyp{\cclrd{r}}{}$ in the above example is analogous to the \ident{Maybe}
(or option) data type $\ident{Maybe}\,\alpha = 1 + \alpha$. As mentioned earlier, this type does not 
permit (non-indexed) comonad structure, because $\ident{counit}~()$ is not defined. This is not a 
problem with indexed comonads, because \ident{counit} only needs to be defined on live context.

\begin{example}[Indexed product]
The semantics of implicit parameters is modelled by an indexed product comonad. We use a monoid
$(\mathcal{P}(\ident{Id}), \cup, \emptyset)$ where \ident{Id} is the set of (implicit parameter) names.
As previously, all parameters have the type $\rho$. The data type $\ctyp{\cclrd{r}}{\alpha}
= \alpha \times (\cclrd{r} \rightarrow R)$ represents a value $\alpha$ together with a function that
associates a parameter value $\rho$ with every implicit parameter name in $\cclrd{r} \subseteq \ident{Id}$.
The cobind and counit operations are defined as:
%
\begin{equation*}
\begin{array}{l}
\ident{counit}_{\emptyset} : \ctyp{\emptyset}{\alpha} \rightarrow \alpha \\[-0.25em]
\ident{counit}_{\emptyset}~(a, g) = a\\[0.5em]
\end{array}
\quad
\begin{array}{l}
\ident{cobind}_{\cclrd{r},\cclrd{s}} ~:~ (\ctyp{\cclrd{r}}{\alpha} \rightarrow \beta) 
    \rightarrow (\ctyp{\cclrd{r}\cup\cclrd{s}}{\alpha} \rightarrow \ctyp{\cclrd{s}}{\beta})\\[-0.25em]
\ident{cobind}_{\cclrd{r},\cclrd{s}}~f~(a,g) = (f(a,\restr{g}{\cclrd{r}}), \restr{g}{\cclrd{s}})\\[0.5em]
\end{array}
\end{equation*}
\end{example}

\noindent
The definition of \ident{counit} simply ignores the function and returns the value in the context.
The \ident{cobind} operation uses the restriction operation $\restr{f}{\cclrd{r}}$, which we 
already defined when discussing semantics of implicit parameters in Section~\ref{sec:applications-flat-impl}
(indeed, \ident{cobind} here captures an essential part of the semantics).

The function $g$ in \ident{cobind} is defined on the union of the implicit parameters, 
\ie~$\cclrd{r}\cup\cclrd{s}\rightarrow\rho$. When passing it to $f$, we restrict it to
just $\cclrd{r}$ and when returning it as a result, we restrict it to $\cclrd{s}$.

% --------------------------------------------------------------------------------------------------

\subsection{Properties and related notions}

We discuss additional examples in Section~\ref{sec:flat-semantics-monoidal}, after we look at the
remaining structure that is needed to define the semantics of flat coeffect calculus. Before doing 
so, we discuss additional properties and categorical structures that have been proposed mainly in 
the context of monads and effects and are related to indexed comonads.

\paragraph{Shape preservation.}
Ordinary comonads have the \emph{shape preservation} property \cite{comonads-codo}. Intuitively, 
this means that the shape of the additional context does not change during the computation. For
example, in the \ident{NEList} comonad, the length of the list stays the same after applying 
\ident{cobind}.

Indexed comonads are not restricted by this property of comonads. For example, given the indexed
product monad, in the computation $\ident{cobind}_{\cclrd{r}, \cclrd{s}} f$ above, the shape of
the context changes from containing implicit parameters $\cclrd{r} \cup \cclrd{s}$  to containing
just implicit parameters $\cclrd{s}$.

\paragraph{Families of monads.}
When linking effect systems and monads, Wadler and Thiemann \cite{monad-notions} propose a
\emph{family of monads} as the categorical structure. The dual structure, \emph{family of
comonads}, is defined as follows.

\begin{definition}
\label{def:flat-family}
A \emph{family of comonads} is formed by triples $(\ctyp{\cclrd{r}}{}, \ident{cobind}_{\cclrd{r}}, 
  \ident{counit}_{\cclrd{r}})$ for all $\cclrd{r}$ such that each triple forms a comonad. Given 
$\cclrd{r}, \cclrd{r'}$ such that $\cclrd{r} \leq \cclrd{r'}$, there is also a mapping 
$\iota_{\cclrd{r'}, \cclrd{r}} : \ctyp{\cclrd{r'}}{} \rightarrow \ctyp{\cclrd{r}}{}$ satisfying
certain coherence conditions.
\end{definition}

Family of comonads is more restrictive than indexed comonad, because each of the data types needs
to form a comonad separately. For example, our indexed option does not form a family of comonads
(again, because \ident{counit} is not defined on $\ctyp{\ident{D}}{\alpha}=1$). However, given a 
family of comonads and indices such that $\cclrd{r} \leq \cclrd{r}\cseq\cclrd{s}$, we can define 
an indexed comonad. Briefly, to define $\ident{cobind}_{\cclrd{r},\cclrd{s}}$ of an indexed comonad, 
we use $\ident{cobind}_{\cclrd{r}\cseq\cclrd{s}}$ from the family, together with two lifting operations:
$\iota_{\cclrd{r}\cseq\cclrd{s}, \cclrd{r}}$ and $\iota_{\cclrd{r}\cseq\cclrd{s}, \cclrd{s}}$.

\paragraph{Parameteric effect monads.}
Parametric effect monads introduced by Katsumata \cite{monads-parametric} (independently to our 
indexed comonads) are closely related to our definition.  Although presented in a more general 
categorical framework (and using monads), the model defines \ident{unit} operation only on the 
unit of a monoid and \ident{bind} operation composes effect annotations using the provided monoidal 
structure.


% --------------------------------------------------------------------------------------------------

\subsection{Flat indexed comonads}
\label{sec:flat-semantics-monoidal}

Indexed comonads model the semantics of sequential composition, but additional structure is needed
to model the semantics of the flat coeffect calculus. This is where the duality between monads and 
comonads can no longer help us, because context is propagated differently than effects in lambda 
abstraction and application.

Whereas Moggi~\cite{monad-notions} requires \emph{strong} monad to model effectful $\lambda$-calculus, 
Uustalu and Vene~\cite{comonads-notions} require \emph{lax semi-monoidal} comonad to model 
$\lambda$-calculus with contextual properties. The structure requires a monoidal operation:
%
\begin{equation*}
\ident{m} : \ctyp{}{\alpha} \times \ctyp{}{\beta} \rightarrow \ctyp{}{(\alpha \times \beta)}
\end{equation*}
%
The \ident{m} operation is needed in the semantics of lambda abstraction. It represents merging of 
contexts and is used to merge the context of the declaration-site (containing free variables)
and the call-site (containing bound variable). For example, for implicit parameters, this combines
the additional parameters defined in the two contexts.

The semantics of flat coeffect calculus requires operations for \emph{merging}, but also for
\emph{splitting} of contexts. These are provided by \emph{lax} and \emph{oplax} monoidal 
structures. In addition, we also need a lifting operation (similar to $\iota$ from 
Definition~\ref{def:flat-family}) to model sub-coeffecting.

\begin{definition}
Given a flat coeffect algebra $(\C, \cseq, \cpar, \czip, \cunit, \czero, \cleq)$,
an \emph{flat indexed comonad} is an indexed comonad over the monoid $(\C, \cseq, \cunit)$
equipped with families of operations $\ident{merge}_{\cclrd{r},\cclrd{s}}$, $\ident{split}_{\cclrd{r},\cclrd{s}}$ 
and $\ident{lift}_{\cclrd{r'},\cclrd{r}}$ where:
%
\begin{compactitem}
\item $\ident{merge}_{\cclrd{r},\cclrd{s}}$ is a family of mappings
  $\ctyp{\cclrd{r}}{\alpha} \times \ctyp{\cclrd{s}}{\beta} \rightarrow \ctyp{\cclrd{r}\czip\cclrd{s}}{(\alpha \times \beta)}$
\item $\ident{split}_{\cclrd{r},\cclrd{s}}$ is a family of mappings
  $\ctyp{\cclrd{r}\cpar\cclrd{s}}{(\alpha \times \beta)} \rightarrow \ctyp{\cclrd{r}}{\alpha} \times \ctyp{\cclrd{s}}{\beta}$
\item $\ident{lift}_{\cclrd{r'},\cclrd{r}}$ is a family of mappings
  $\ctyp{\cclrd{r'}}{\alpha} \rightarrow \ctyp{\cclrd{r}}{\alpha}$~ for all $\cclrd{r'}, \cclrd{r}$ such that $\cclrd{r}\,\cleq\,\cclrd{r'}$
\end{compactitem}
%
% We should require lift to be comonad morphism
%
% \tag{\emph{comonad morphism}}
%   \ident{cobind}_{\cclrd{r}, \cclrd{s}}~f \circ \ident{lift}_{\cclrd{r'}\cseq\cclrd{s'}, \cclrd{r}\cseq\cclrd{s}} &=
%   \ident{lift}_{\cclrd{s'}, \cclrd{s}} \circ \ident{cobind}_{\cclrd{r'}, \cclrd{s'}}~(f \circ \ident{lift}_{\cclrd{r'}, \cclrd{r}} )
%
% ... but adapting the coherence conditions for counit/cobind is ugly!
%
\end{definition}

\noindent
The $\ident{merge}_{\cclrd{r},\cclrd{s}}$ operation is the most interesting one. Given two comonadic
values with additional contexts specified by $\cclrd{r}$ and $\cclrd{s}$, it combineds them into a 
single value with additional context $\cclrd{r}\czip\cclrd{s}$. The $\czip$ operation often represents
\emph{greatest lower bound}\footnote{The $\czip$ and $\cpar$ operations are the greatest and least upper 
bounds for the liveness and data-flow examples, but not for implicit parameters. However, they are useful 
as an informal analogy.}, elucidating the fact that merging may result in the loss of some parts of 
the contexts $\cclrd{r}$ and $\cclrd{s}$. We look at examples of this operation in the next section.

The $\ident{split}_{\cclrd{r},\cclrd{s}}$ operation splits a single comonadic value (containing a tuple)
into two separate values. Note that this does not simply duplicate the value, because the additional
context is also split. To obtain coeffects $\cclrd{r}$ and $\cclrd{s}$, the input needs to provide 
\emph{at least} $\cclrd{r}$ and $\cclrd{s}$, so the tags are combined using the $\cpar$, which is often 
the \emph{least upper-bound}\footnotemark[1].

Finally, $\ident{lift}_{\cclrd{r'}, \cclrd{r}}$ is a family of operations that ``forget'' some part of
a context. This models the sub-coeffecting operation and lets us, for example, forget some of the
available implicit parameters, or turn a live context (containing a value) into a dead context (empty).

\paragraph{Alternative definition.}
Although we do not require this as a general law, in all our systems, it is the case that
$\cclrd{r} \;\cleq\; \cclrd{r}\,\cpar\,\cclrd{s}$ and $\cclrd{s} \;\cleq\; \cclrd{r}\,\cpar\,\cclrd{s}$.
This allows a simpler definition of \emph{indexed flat comonad} by expressing the \ident{split} operation
in terms of the lifting (sub-coeffecting) as follows:
%
\begin{equation*}
\begin{array}{rcl}
\ident{map}_{\cclrd{r}}~f &\narrow{=}& \ident{cobind}_{\cclrd{r}, \cclrd{r}}~(f\circ\ident{counit}_{\cunit}) \\
\ident{split}_{\cclrd{r}, \cclrd{s}}~c &\narrow{=}&
  ( \ident{map}_{\cclrd{r}}~\ident{fst}~(\ident{lift}_{\cclrd{r}\cpar\cclrd{s}, \cclrd{r}}~c), 
    \ident{map}_{\cclrd{s}}~\ident{snd}~(\ident{lift}_{\cclrd{r}\cpar\cclrd{s}, \cclrd{s}}~c) )
\end{array}
\end{equation*}
%
The $\ident{map}_{\cclrd{r}}$ operation is the mapping on functions that corresponds to the object 
mapping $\ctyp{\cclrd{r}}{}$. The definition is dual to the standard definition of \ident{map} 
for monads in terms of \ident{bind} and \ident{unit}. The functions \ident{fst} and \ident{snd}
are first and second projections from a two-element pair. To define the 
$\ident{split}_{\cclrd{r}, \cclrd{s}}$ operation, we duplicate the argument $c$, then use 
lifting to throw away additional parts of the context and then transform the values in the 
context.

This alternative is valid for our examples, but we do not use it for two reasons. Firstly, it 
requires duplication of the value $c$, which is not required elsewhere in our model. So, using 
explicit \ident{split}, our model could be embedded in a linear or affine model. Secondly, it
is similar to the definition that is needed for structural coeffects in Chapter~\ref{ch:structural}
and so it makes the connection between the two system easier to see.

\paragraph{Examples.}
All examples of \emph{indexed comonads} discussed in Section~\ref{sec:flat-semantics-idx} can
be extended into \emph{flat indexed comonads}. 

\begin{example}[Monoidal comonads]
Just like indexed comonads generalise co\-monads, the additional structure of
flat indexed comonads generalises symmetric semimonoidal comonads of Uustalu 
and Vene \cite{comonads-notions}. The flat coeffect algebra is defined as $(\{1\}, \ast, \ast, \ast, 1, 1, =)$
where $1\ast1=1$ and $1=1$. The additional operation $\ident{merge}_{1,1}$ is provided by the 
monoidal operation called \ident{m} by Uustalu and Vene. The $\ident{split}_{1,1}$ operation 
is defined by duplication and $\ident{lift}_{1,1}$ is the identity function.
\end{example}

\begin{example}[Indexed option]
Flat coeffect algebra for liveness defines $\cpar$ and $\czip$ as $\sqcup$ and $\sqcap$, respectively 
and specifies that $\ident{D} \sqsubseteq \ident{L}$. Recall also that the object mapping is defined 
as $\ctyp{\ident{L}}{\alpha} = \alpha$ and $\ctyp{\ident{D}}{\alpha} = 1$. The additional operations 
of a flat indexed comonad are defined as follows:
%
\begin{equation*}
\begin{array}{rcl}
\ident{merge}_{\ident{L}, \ident{L}}~(a, b) &\narrow{=}& (a, b)\\
\ident{merge}_{\ident{L}, \ident{D}}~(a, ()) &\narrow{=}& ()\\
\ident{merge}_{\ident{D}, \ident{L}}~((), b) &\narrow{=}& ()\\
\ident{merge}_{\ident{D}, \ident{D}}~((), ()) &\narrow{=}& ()\\
\end{array}
\quad
\begin{array}{rcl}
\ident{split}_{\ident{L}, \ident{L}}~(a, b) &\narrow{=}& (a, b)\\
\ident{split}_{\ident{L}, \ident{D}}~(a, b) &\narrow{=}& (a, ())\\
\ident{split}_{\ident{D}, \ident{L}}~(a, b) &\narrow{=}& ((), b)\\
\ident{split}_{\ident{D}, \ident{D}}~() &\narrow{=}& (a, b)\\
\end{array}
\quad
\begin{array}{rcl}
\ident{lift}_{\ident{L}, \ident{D}}~v &\narrow{=}& ()\\
\ident{lift}_{\ident{L}, \ident{L}}~v &\narrow{=}& v\\
\ident{lift}_{\ident{D}, \ident{D}}~() &\narrow{=}& ()\\
\end{array}
\end{equation*}
\end{example}

\noindent
Without the indexing, the \ident{merge} operations implements \emph{zip} on option values,
returning an option only when both values are present. The behaviour of the \ident{split} 
operation is partly determined by the indices. When the input is \emph{dead}, both values have 
to be dead (this is also the only solution of $\ident{D}=\cclrd{r}\sqcap\cclrd{D}$), but when
the input is \emph{live}, the operation can perform implicit sub-coeffecting and drop one of
the values.

Explicit sub-coeffecting using the (\emph{sub}) rule is modelled by the \ident{lift} operation.
This can turn a \emph{live} value $v$ into a dead value $()$, or it can behave as identity.
The behaviour is, again, determined by the index.

\begin{example}[Indexed product]
For implicit parameters, both $\czip$ and $\cpar$ are the $\cup$ operation and the relation
$\cleq$ is formed by the subset relation $\subseteq$. Recall that the data type $\ctyp{\cclrd{r}}{\alpha}$
is $\alpha \times (\cclrd{r} \rightarrow R)$ where $R$ is some representation of a parameter value.
The additional operations are defined as:
%
\begin{equation*}
\begin{array}{rcl}
\ident{split}_{\cclrd{r}, \cclrd{s}}~((a,b), g) &\narrow{=}& ((a, \restr{g}{\cclrd{r}}), (b, \restr{g}{\cclrd{s}}))\\
\ident{merge}_{\cclrd{r}, \cclrd{s}}~((a, f), (b, g)) &\narrow{=}& ((a, b), f \uplus g)\\
\ident{lift}_{\cclrd{r'}, \cclrd{r}}~(a, g) &\narrow{=}& (a, \restr{g}{\cclrd{r}})\\
\end{array}
\quad
\begin{array}{l}
\textnormal{where}~f \uplus g = \\[-0.25em]
\quad\restr{f}{\, \textit{dom}(f) \setminus \textit{dom}(g)} \cup g 
\end{array}
\end{equation*}
\end{example}

\noindent
The \ident{split} operation splits the tuple and restricts the function (representing available
implicit parameters) to the required sub-sets. This corresponds to the definition in terms
of \ident{lift}, which performs just the restriction. The \ident{merge} operation is more 
interesting. It uses $\uplus$ operation that we defined when introducing implicit parameters
in Section~\ref{sec:applications-flat-impl}. It merges the values, preferring the definitions from
the right-hand side (call-site) over left-hand side (declaration-site). Thus the operation is not
symmetric.

\begin{example}[Indexed list]
Our last example provides the semantics of data-flow computations. The flat coeffect algebra 
is formed by $(\mathbb{N}, +, \mathit{max}, \mathit{min}, 0, 0, \leq)$. In a 
non-indexed version, the semantics is provided by a non-empty list. In the indexed semantics,
the index represents the length of the storing past values. The data type is then a pair of 
the current value, followed by $n$ past values. The mappings that form the flat indexed comonad 
are defined as follows:

\begin{equation*}
\begin{array}{l}
\ident{counit}_{0}\langle a_0 \rangle = a_0
\\[0.45em]
\ident{cobind}_{m, n}~f \langle a_0, \ldots a_{m+n} \rangle = \\[-0.25em]
\quad \langle f \langle a_0, \ldots, a_m \rangle, \ldots \langle a_{n}, \ldots, a_{m+n} \rangle \rangle
\\[0.45em]
\ident{merge}_{m, n} (\langle a_0, \ldots, a_m \rangle, \langle b_0, \ldots, b_n\rangle) = \\[-0.25em]
\quad \langle (a_0, b_0), \ldots, (a_{\mathit{min}(m,n)}, b_{\mathit{min}(m,n)}) \rangle
\\[0.45em]
\ident{split}_{m, n} \langle (a_0, b_0), \ldots, (a_{\mathit{max}(m,n)}, b_{\mathit{max}(m,n)}) \rangle = \\[-0.25em]
\quad (\langle a_0, \ldots, a_m \rangle, \langle b_0, \ldots, b_n\rangle)
\\[0.45em]
\ident{lift}_{n', n} \langle a_0, \ldots, a_n' \rangle = \qquad(\hspace{-0.95em}\textnormal{\footnotesize when}~n\leq n') \\[-0.25em]
\quad \langle a_0, \ldots, a_n \rangle
\end{array}
\begin{array}{l}
\hspace{-2em}\ctyp{n}{\alpha} = \underbrace{\alpha \times \ldots \times \alpha}_{(n+1)-\textnormal{times}}\\[11em]
~\\
\end{array}
\end{equation*}
\end{example}

\noindent
The reader is invited to check that the number of required past elements in each of the mappings
matches the number specified by the indices. The index specifies the number of \emph{past} elements
and so the list always contains at least one value. Thus \ident{counit} returns the element of a
singleton list.

The $\ident{cobind}_{m,n}$ operation requires $m + n$ elements in order to generate $n$ past results 
of the $f$ function, which itself requires $m$ past values. When combining two lists, 
$\ident{merge}_{m,n}$ behaves as \emph{zip} and produces a list that has the length of the shorter 
argument. When splitting a list, $\ident{split}_{m, n}$ needs the maximum of the required lengths. 
Finally, the lifting operation just drops some number of elements from a list.

% --------------------------------------------------------------------------------------------------

\subsection{Semantics of flat calculus}

In Section~\ref{sec:applications-flat}, we defined the semantics of concrete (flat) context-dependent
computations including implicit parameters, liveness and data-flow. Using the \emph{flat indexed 
comonad} structure, we can now define a single uniform semantics that is capable capturing of all 
our examples, as well as other computations that can be modelled by the structure.

\paragraph{Contexts and functions.}
The modelling of contexts and functions generalizes the earlier concrete examples. We use the 
family of mappings $\ctyp{\cclrd{r}}{}$ as an (indexed) data-type that wraps the product of 
free variables of the context and the arguments of functions:
%
\begin{equation*}
\begin{array}{rcl}
\sem{\coctx{x_1\!:\!\tau_1, \ldots, x_n\!:\!\tau_n}{ \cclrd{r} } \vdash e : \tau} 
  &:& \ctyp{\cclrd{r}}{(\tau_1 \times \ldots \times \tau_n)} \rightarrow \tau\\
\sem{\tau_1 \xrightarrow{\cclrd{r}} \tau_2} &=& \ctyp{\cclrd{r}}{\tau_1} \rightarrow \tau_2
\end{array}
\end{equation*}

% --------------------------------------------------------------------------------------------------

\newcommand{\ctx}{\textit{ctx}}
\begin{figure*}[t]

\begin{equation*}
\begin{array}{ll}
\sem{\coctx{\Gamma}{\cunit} \vdash x_i : \tau_i }~\ctx =
  \pi_i~(\ident{counit}_{\cunit}~\ctx) & (\emph{var})
\\[0.5em]
\sem{\coctx{\Gamma}{\czero} \vdash c_i : \tau }~\ctx =
  \delta~(c_i) & (\emph{const})
\\[0.5em]
\sem{\coctx{\Gamma}{\cclrd{r}} \vdash e : \tau }~\ctx = & (\emph{sub})\\[-0.25em]
  \hspace{3em}\sem{\coctx{\Gamma}{\cclrd{r'}} \vdash e : \tau }~(\ident{lift}_{\cclrd{r}, \cclrd{r'}}~\ctx)~
    \hspace{7.5em}(\hspace{-0.7em}\textnormal{\footnotesize when}~\cclrd{r'} \leq \cclrd{r})
\\[0.5em]
\sem{\coctx{\Gamma}{\cclrd{r}} \vdash \lambda x.e : \tau_1 \xrightarrow{\cclrd{s}} \tau_2 }~\ctx = \lambda v.& (\emph{abs})\\[-0.25em]
  \hspace{3em}\sem{\coctx{\Gamma, x:\tau_1}{\cclrd{r}\;\czip\;\cclrd{s}} \vdash e : \tau_2}~(\ident{merge}_{\cclrd{r}, \cclrd{s}}~(\ctx, v))
\\[0.5em]
\sem{\coctx{\Gamma}{\cclrd{r} \;\cpar\; (\cclrd{s} \,\cseq\, \cclrd{t})} \vdash e_1~e_2 : \tau_2}~\ctx = & (\emph{app})\\[-0.25em]
  \hspace{2.5em}
  \begin{array}{l}  
  \kvd{let}~(\ctx_1, \ctx_2) = \ident{split}_{\cclrd{r}, \cclrd{s} \,\cseq\, \cclrd{t}}~
    (\ident{map}_{\cclrd{r} \;\cpar\; (\cclrd{s} \,\cseq\, \cclrd{t})}~(\lambda x.(x,x))~\ctx) \\[-0.25em]
  \kvd{in}~\sem{\coctx{\Gamma}{\cclrd{r}} \vdash e_1 : \tau_1 \xrightarrow{\cclrd{t}} \tau_2}~\ctx_1~
      (\ident{cobind}_{\cclrd{s}, \cclrd{t}}~\sem{\coctx{\Gamma}{\cclrd{s}} \vdash e_2 : \tau_1 }~\ctx_2)
  \end{array}    
\end{array}
\end{equation*}

\caption{Categorical semantics of the flat coeffect calculus}
\label{fig:flat-semantics}
\end{figure*}

% --------------------------------------------------------------------------------------------------


\paragraph{Expressions.}
The definition of the semantics is shown in Figure~\ref{fig:flat-semantics}. For readability, we 
write the definitions in a simple programming language notation as opposed to the point-free 
categorical style. However, it can be equally written using just the operations of flat indexed 
comonad together with $i^{th}$ projection from a tuple represented by $\pi_i$, \emph{curry} and 
\emph{uncurry}, function composition, value duplication ($\Delta : A \rightarrow A \times A$) and 
function pairing (given $f:A\rightarrow B$ and $g:C\rightarrow D$ then $f\times g : A\times C \rightarrow B \times D$). 
These operations can be provided by \eg~a Cartesian Closed Category.

The semantics of variable access and abstraction are the same as in the semantics of Uustalu and 
Vene \cite{comonads-notions}, modulo the indexing. The semantics of variable access (\emph{var}) uses 
$\ident{counit}_{\cunit}$ to extract product of free-variables from the context and then projection
$\pi_i$ to obtain the variable value. Abstraction (\emph{abs}) takes the context $\ctx$ and function argument 
$v$ and merges their additional contexts using $\ident{merge}_{\cclrd{r}, \cclrd{s}}$. Assuming
the context $\Gamma$ contains variables of types $\sigma_1, \ldots, \sigma_n$, this gives us a 
value $\ctyp{\cclrd{r}\czip\cclrd{s}}((\sigma_1 \times \ldots \times \sigma_n) \times \tau_1)$.
Assuming that $n$-element tuples are associated to the left, the wrapped context is equivalent to
$\sigma_1 \times \ldots \times \sigma_n \times \tau_1$, which can then be passed to the body of the
function.

The semantics of application is more complex. It first duplicates the free-variable product inside the 
context (using $\ident{map}_{\cclrd{r}}$ and duplication). Then it splits this context using 
$\ident{split}_{\cclrd{r}, \cclrd{s} \cpar \cclrd{t}}$. The two contexts contain the same variables
(as required by sub-expressions $e_1$ and $e_2$), but different coeffect annotations. The first
context (with index $\cclrd{r}$) is used to evaluate $e_1$, resulting in a function 
$\ctyp{\cclrd{t}}{\tau_1} \rightarrow \tau_2$. To obtain the result, we compose this with a function
created by applying $\ident{cobind}_{\cclrd{s}, \cclrd{t}}$ on the semantics of sub-expression
$e_2$, which is of type $\ctyp{\cclrd{s}\cseq\cclrd{t}}{\sigma_1 \times \ldots \times \sigma_n}
\rightarrow \ctyp{\cclrd{t}}{\tau_1}$.

Finally, constants (\emph{const}) are modelled by a global dictionary $\delta$ and sub-coeffecting
is interpreted by dropping additional context from the provided context $\ctx$ using 
$\ident{lift}_{\cclrd{r}, \cclrd{r'}}$ and providing it to the semantics of the assumption.

\paragraph{Properties.}

The categorical semantics can be used to embed context-dependent computations in functional 
programming languages, similarly to how monads provide a way of embedding effectful computations.
More importantly, it also provides validation for the design of the type system developed in 
Section~\ref{sec:flat-calculus-types}. As stated in the following theorem, the annotations in 
the type system match those of the semantic functions.

\vspace{2em}

\begin{remark}[Correspondence]
In all of the typing rules of the flat coeffect system, the context annotations $\cclrd{r}$ of typing 
judgements $\coctx{\Gamma}{\cclrd{r}} \vdash e : \tau$ and function types $\tau_1 \xrightarrow{\cclrd{r}} \tau_2$ 
correspond to the indices of mappings $\ctyp{\cclrd{r}}{}$ in the corresponding semantic function defined 
by $\sem{\coctx{\Gamma}{\cclrd{r}} \vdash e : \tau}$.
\end{remark}
\begin{proof}
By analysis of the semantic rules in Figure~\ref{fig:flat-semantics}.
\end{proof}

\noindent
Thanks to the indexing, the statement of the remark is significantly stronger than for a 
non-indexed system, because it provides the justification for our choice of indices in the typing
rules. In particular, we can see that the annotations follow from the annotations on primitive 
functions that define the semantics. Also, each function defining the semantics uses a distinct 
operation of the coeffect algebra and so the type system is the most general possible definition
(within the comonadic framework we use).

% ==================================================================================================


\newcommand{\ccat}[0]{\mathcal{C}}
\newcommand{\cobind}[2]{#1^\dagger_{#2}}
\newcommand{\cmerge}[0]{ \ident{m} }
\newcommand{\csplit}[0]{ \ident{n} }
\newcommand{\counit}[0]{ \varepsilon }
\newcommand{\llangle}{\langle\hspace{-0.25em}\langle}
\newcommand{\rrangle}{\rangle\hspace{-0.25em}\rangle}

\section{Equational theory}
\label{sec:flat-syntax}

\subsection{Call-by-value evaluation}
\subsection{Call-by-name evaluation}

% ==================================================================================================

\section{Syntactic extensions}
\label{sec:flat-exts}

\subsection{Lambda abstraction}
\subsection{Constants and pairs}
\subsection{Recursion}

% ==================================================================================================

\section{Type inference}
\label{sec:flat-inference}

\subsection{Semi-lattice formulation}
\subsection{Type inference algorithm}

% ==================================================================================================

\section{Related work}

\subsection{What can be monad}
\label{sec:flat-related-monads}

% ==================================================================================================

\section{Summary}

\section{----------------JUNK!}




% ==================================================================================================

~
\newpage
~

% ==================================================================================================

\section{Syntax-based equational theory}
\label{sec:syntactic}

Operational semantics of every context-dependent language differs as the notion of context 
is always different. However, for coeffect calculi satisfying certain conditions we can 
define a universal equational theory. This suggests a pathway to an operational semantics for 
two out of our three examples (the notion of context for data-flow is more complex).

In a pure $\lambda$-calculus, $\beta$ and $\eta$ equality for
functions (also called \emph{local soundness} and \emph{completeness}
respectively~\cite{logic-modal-reconstruction})
describe how pairs of abstraction and application can be eliminated:
$(\lambda x . e_2) e_1 \equiv_\beta \subst{e_1}{x}{e_2}$ and $(\lambda x . e \, x) 
\equiv_\eta e$. The $\beta$ equality rule, using the usual Barendregt convention of 
syntactic substitution, implies a \emph{reduction},
giving part of an operational semantics for the calculus.

The call-by-name evaluation strategy modelled by $\beta$-reduction is
not suitable for impure calculi therefore a restricted $\beta$ rule,
corresponding to call-by-value, is used, \ie~$(\lambda x . e_2) v
\equiv \subst{e_2}{x}{v}$. Such reduction can be encoded by a
\emph{let}-binding term, $\kvd{let}~x=e_1~\kvd{in}~e_2$, which
corresponds to sequential composition of two computations, where the
resulting pure value of $e_1$ is substituted into
$e_2$~\cite{monads-inaction,monad-notions}.

%
% We consider here both a notion of \emph{let}-binding for the coeffect
% calculus, useful for a CBV evaluation, and a notion of substitution 
% for a CBV evaluation.
% 

For an equational theory of coeffects, consider first a notion 
of \emph{let}-binding equivalent to $(\lambda x . e_2)~e_1$, which
has the following type and coeffect rule:
%
\begin{equation}
\inference
  {\ctyp{s}{\Gamma} \vdash e_1 : \tau_1 &
   \ctyp{r_1 \wedge r_2}{(\Gamma, x : \tau_1)} \vdash e_2 : \tau_2}
  {\ctyp{r_1 \vee (r_2 \cseq s)}{\Gamma} \vdash \kvd{let}~x = e_1~\kvd{in}~e_2
: \tau_2 }
\label{eq:let1}
\end{equation}

\noindent
For our examples, $\wedge$ is idempotent (\ie{}, $r \wedge r = r$)
implying a simpler rule:

% For the three examples we consider, a simpler rule gives a more
% precise coeffect. Because $r \wedge r = r$ for all our examples, we
% can also a coeffect $r \vee (r \cseq s)$. Moreover,
%for our examples (but not necessarily for \emph{all} coeffect
%systems), $r \vee (r \cseq s) \leq r_1 \vee (r_2 \cseq s)$ meaning
%that the following gives more precise coeffects:
%
\begin{equation}
\inference
  {\ctyp{s}{\Gamma} \vdash e_1 : \tau_1 &
   \ctyp{r}{(\Gamma, x : \tau_1)} \vdash e_2 : \tau_2}
  {\ctyp{r \vee (r \cseq s)}{\Gamma} \vdash \kvd{let}~x = e_1~\kvd{in}~e_2 : \tau_2 }
\label{eq:let2}
\end{equation}
%%
For our examples (but not necessarily \emph{all} coeffect
systems), this defines a more ``precise'' coeffect with respect to $\leq$
where $r \vee (r \cseq s) \leq r_1 \vee (r_2 \cseq s)$.

This rule removes the non-principality of the first rule
(\ie~multiple possible typings).  However, using idempotency 
to split coeffects in abstraction would remove additional
flexibility needed by the implicit parameters example.

The coeffect $r \vee (r \cseq s)$ can
also be simplified for all our examples, leading to more intuitive
rules -- for implicit parameters $r \cup (r \cup s) = r \cup s$; for
liveness we get that $r \sqcup (r \sqcap s) = r$ and for dataflow we
obtain $\textit{max}(r, r+s) = r + s$.

Our calculus can be extended with \emph{let}-binding and \eqref{eq:let2}.
However, we also consider the cases when a
syntactic substitution $e_2[x \leftarrow e_1]$ has the coeffects
specified by the above rule \eqref{eq:let2} 
and prove \emph{subject reduction} theorem
for certain coeffect calculi.  We consider two common special cases
when the coeffect of variables $\cunit$ is the greatest ($\top$) or
least ($\bot$) element of the semi-lattice $(S, \vee)$ and derive
additional conditions that have to hold about the coeffect algebra:

\begin{lemma}[Substitution]
\label{thm:subst}
Given $C^r (\Gamma, x : \tau_2) \vdash e_1 : \tau_1$ and $C^s \Gamma \vdash e_2 : \tau_2$
then $C^{r \vee (r \oplus s)} \Gamma \vdash \subst{e_2}{x}{e_1} : \tau_1$ if 
the coeffect algebra satisfies the conditions that 
$\cunit$ is either the greatest or least element of the semi-lattice,
$\oplus = \wedge$, and $\oplus$ distributes over $\vee$,
\ie{}, $X \oplus (Y \vee Z) = (X \oplus Y) \vee (X \oplus Z)$.
\end{lemma}

\begin{proof}
By induction over $\vdash$, using the laws (\S\ref{sec:calculus}) and additional assumptions.
\end{proof}

Assuming $\rightarrow_\beta$ is the usual call-by-name reduction, the
following theorem models the evaluation of coeffect calculi with
coeffect algebra that satisfies the above requirements. We do not
consider \emph{call-by-value}, because our calculus does not have a
notion of \emph{value}, unless explicitly provided by
\emph{let}-binding (even a function ``value'' $\lambda x.e$ may have
immediate contextual requirements).

\begin{theorem}[Subject reduction]
\label{thm:reduction}
For a coeffect calculus, satisfying the conditions of Lemma~\ref{thm:subst}, if
$\ctyp{r}{\Gamma} \vdash e : \tau$ and $e \rightarrow_\beta e'$ then 
$\ctyp{r}{\Gamma} \vdash e' : \tau$.
\end{theorem}
\begin{proof}
A direct consequence of Lemma~\ref{thm:subst}. 
\end{proof}

The above theorem holds for both the liveness and resources examples,
but not for dataflow.  In the case of liveness, $\cunit$ is the
greatest element ($r \vee \cunit = \cunit$); in the case of
resources, $\cunit$ is the \emph{least} element ($r \vee \cunit =
r$) and the proof relies on the fact that additional
context-requirements can be placed at the context $\ctyp{r}{\Gamma}$
(without affecting the type of function when substituted under
$\lambda$ abstraction).

However, the coeffect calculus also captures context-dependence in
languages with more complex evaluation strategies than
\emph{call-by-name} reduction based on syntactic substitution.  In
particular, syntactic substitution does not provide a suitable evaluation
for dataflow (because a substituted expression needs to capture the
context of the original scope).

Nevertheless, the above results show that -- unlike effects --
context-dependent properties can be integrated with
\emph{call-by-name} languages. Our work also provides a model of
existing work, namely Haskell implicit parameters
\cite{app-implicit-parameters}.

% ==================================================================================================

\section{Related and further work}
\label{sec:related}

This paper follows the approaches of effect systems \cite{effects-gifford,effects-talpin-et-al,monads-effects-marriage}
and categorical semantics based on monads and comonads \cite{monad-notions,comonads-notions}. Syntactically,
\emph{coeffects} differ from \emph{effects} in that they model systems where $\lambda$-abstraction 
may split contextual requirements between the declaration-site and call-site.

Our \emph{indexed (monoidal) comonads} (\S\ref{sec:comonads}) fill the gap between (non-indexed)
\emph{(monoidal) comonads} of Uustalu and Vene \cite{comonads-notions}
and indexed monads of Atkey~\cite{monads-parameterised-notions}, Wadler and Thiemann
\cite{monads-effects-marriage}. Interestingly, \emph{indexed} comonads are \emph{more
general} than comonads, capturing more notions of context-dependence (\S\ref{sec:motivation}).

% --------------------------------------------------------------------------------------------------

\vspace{-1em}
\paragraph{Comonads and modal logics.}

Bierman and de Paiva \cite{logic-intuitionistic-modal} model the
$\square$ modality of an intuitionistic S4 modal logic using monoidal
comonads, which links our calculus to modal logics.  This link can be
materialized in two ways.

Pfenning et al. and Nanevski et al.  derive term languages using the Curry-Howard
correspondence~\cite{logic-modal-reconstruction,logic-intuitionistic-modal,logic-cmtt},
building a \emph{metalanguage} (akin to Moggi's monadic metalanguage
\cite{monad-notions}) that includes $\square$ as a type
constructor. For example, in \cite{logic-modal-reconstruction}, the
modal type $\Box \tau$ represents closed terms.
In contrast, the \emph{semantic} approach uses monads or comonads
\emph{only} as a semantics.  This has been employed by Uustalu and
Vene and (again) Moggi \cite{monad-notions,comonads-notions}.  We
follow the semantic approach.

Nanevski et al. extend an S4 term language to a \emph{contextual}
modal type theory (CMTT)~\cite{logic-cmtt}.
The \emph{context} is a set of variables required by a computation, which
makes CMTT useful for meta-programming and staged computations. Our contextual types are
indexed by a coeffect algebra, which is more general and can capture
variable contexts, but also integers, two-point lattices, \emph{etc.}.

The work on CMTT suggests two extensions to coeffects. The first is
developing the logical foundations. We briefly considered special cases
of our system that permits local soundness in \S\ref{sec:syntactic} and
local completeness can be treated similarly. The second problem is 
developing the coeffects \emph{metalanguage}. The use of coeffect algebras
would provide an additional flexibility over CMTT, allowing a wider range 
of applications.

% --------------------------------------------------------------------------------------------------

\vspace{-1em}
\paragraph{Relating effects and coeffects.} 
The difference between effects and coeffects is mainly in the (\emph{abs}) rule. While the 
semantic model (monads vs. comonads) is very different, we can consider extending the two to 
obtain equivalent syntactic rules. To allow splitting of implicit parameters in lambda abstraction, 
the reader monad needs an operation that eagerly performs some effects of a function: 
$(\tau_1 \rightarrow \mtyp{r \oplus s}{\tau_2}) \rightarrow \mtyp{r}{(\tau_1 \rightarrow \mtyp{s}{\tau_2})}$.
To obtain a pure lambda abstraction
for coeffects, we need to restrict the $\cmerge_{r, s}$ 
operation of indexed comonads, so that the first parameter is annotated with $\cunit$ (meaning
no effects): $\ccat^{\cunit} A \times \ccat^r B \rightarrow \ccat^{r}(A \times B)$.

\newcommand{\cprd}{\times}
\newcommand{\cvop}{\oplus}

\vspace{-1em}
\paragraph{Structural coeffects.} To make the liveness analysis practical, we need to associate
information with individual variables (rather than the entire context). We can generalize the 
calculus from this paper by adding a product operation $\times$ to the coeffect algebra.
A variable context $x:\tau_1, y:\tau_2, z:\tau_3$ is then annotated with
$r\times s \times t$ where each component of the tag corresponds to a single variable. The system
then needs to be extended with structural rules such as:
%
\begin{equation*}
\inference[(\emph{abs})]
  {\ctyp{r \cprd s}{(\Gamma, x:\tau_1)} \vdash e : \tau_2}
  {\ctyp{r}{\Gamma} \vdash \lambda x.e : \ctyp{s}{\tau_1} \rightarrow \tau_2 }
\quad
\inference[(\emph{contr})]
  {\ctyp{r \cprd s}{(x:\tau_1, y:\tau_1)} \vdash e : \tau_2}
  {\ctyp{r \cvop s }{(z:\tau_1)} \vdash \subst{\subst{e}{x}{z}}{y}{z} : \tau_2 }
\end{equation*}
%
The context-requirements associated with function are exactly those linked to the specific
variable of the lambda abstraction. Rules such as contraction manipulate variables and perform
a corresponding operation on the indices.

The structural coeffect system is related to bunched typing \cite{types-bunched} (but generalizes
it by adding indices). We are currently investigating how to use structural coeffects to capture 
fine-grained context-dependence properties such as secure information flow \cite{app-secure-flow}
or, more generally, those captured by dependency core calculus \cite{types-dcc}.

% ==================================================================================================

\section{Conclusions}

We examined three simple calculi with associated
static analyses (liveness analysis, implicit parameters, and dataflow
analysis). These were unified in the \emph{coeffect calculus},
providing a general coeffect system parameterised by an
algebraic structure describing the propagation of context
requirements throughout a program.

We model the semantics of coeffect calculus using \emph{indexed comonad} -- a novel structure, which
is more powerful than (monoidal) comonads. Indices of the indexed comonad operations manifest the 
semantic propagation of context such that the propagation of information in the general coeffect
type system corresponds exactly to the semantic propagation of context in our categorical model.

We consider the analysis of context to be essential, not least for the examples here but 
also given increasingly rich and diverse distributed systems.

\chapter{Coeffect meta-language} 
\label{ch:coeffect-metalanguage} 

%---------------------------------------------------------------------------------------------------

Both flat coeffect calculus and structural coeffect calculus (presented in the past two chapters)
use indexed comonads to define the semantics of the langauge. In this section, we follow the 
meta-language style and embed indexed comonads into the language -- the type constructor
$\ctyp{r}{\alpha}$ becomes a first-class value and we add language constructs corresponding to
primitive operations of the indexed comonad.

%===================================================================================================

\section{Introduction}
\label{sec:metalanguage-intro}

\section{Type system}
\ExecuteMetaData[rules/rules.tex]{cml-types}

\section{Operational properties}

\section{Categorical semantics}

\section{Applications}

\subsection{Meta-programming}

\subsection{Mobile computations}

\section{Related work}
This chapter is closely related to Contextual Modal Type Theory (CMTT) of Nanevski et al. However
they develop their language using model logic as a basis, while we use categorical foundations 
as the basis - leading to a different system.

\section{Summary}

%!TEX root = ../main.tex

\chapter{Unified coeffect language} 
\label{ch:unified} 

%===================================================================================================

\section{Introduction}

The main goal of this thesis is to provide a \emph{unified} calculus for tracking context 
dependence. We have not achieved this goal yet. In Chapter~\ref{ch:applications}, we identified 
two kinds of contextual properties that we further covered separately -- flat coeffects in 
Chapter~\ref{ch:flat} track whole-context properties and structural coeffects, covered in 
Chapter~\ref{ch:structural}, track per-variable properties. In this chapter, we unify
the two notions. We introduce a \emph{unified coeffect} calculus that generalizes the two systems
and can be instantiated to track both flat and structural properties (Section~\ref{sec:unified-unified}). 

Although the results presented in this thesis are mainly of a theoretical nature, we indeed 
believe that coeffects should be integrated in main-stream programming languages. In the second
part of this chapter, we discuss an alternative approach to defining coeffect 
systems which highlights the relationship between our work and related work arising from modal logics 
(Section~\ref{sec:unified-meta}). Finally, we outline one possible approach for practical
implementations of coeffects (Section~\ref{sec:unified-impl}).

% ==================================================================================================

\section{The unified coeffect calculus}
\label{sec:unified-unified}

The flat coeffect calculus (Figure~\ref{fig:flat-types}) and the structural coeffect calculus 
differ in a number of ways (Figure~\ref{fig:struct-types}). Understanding the differences is the 
key to reconciling the two systems:

\begin{itemize}
\item Structural coeffect calculus contains explicit rules for context manipulation
  (weakening, contraction, exchange). In the flat coeffect calculus, these rules are not defined
  explicitly, but are admissible.

\item In the structural coeffect calculus, the variable context is treated as a vector
  and is annotated with a vector of (scalar) coeffects. In the flat coeffect calculus,
  the variable context is a set and is annotated with a single (scalar) coeffect.

\item In the flat coeffect calculus, we distinguish between splitting of the context requirements
  and merging of context requirements ($\cpar$ and $\czip$, respectively). In the structural
  coeffect calculus, the operations (which model splitting and appending vectors) are invertible 
  and so the structural coeffect algebra requires just $\atimes$.
\end{itemize}

\noindent
In the unified calculus presented in this section, we address the three differences as follows.
We use calculus with explicit rules for context manipulation. In systems that arise from flat
calculi, the rules can be applied freely without changing the coeffects. We generalize the 
structure of coeffect annotations using the notion of a ``container'' which can be specialized 
to obtain a single annotation or a vector of annotations. Finally, we distinguish between 
splitting and merging of context requirements (using the notation $\apar$ and $\azip$, 
respectively). For structural coeffect calculi, the two operators coincide, but for flat
coeffect calculi, they provide the needed flexibility.

\subsection{Shapes and containers}

Our notion of a \emph{coeffect container} is based on the idea of a container introduced by 
Abbott et al. \cite{types-containers}. Interestingly, the work on containers has later been linked
to comonads by Ahman et al. \cite{comonads-containers}. Intuitively a container describes data
types such as lists, trees or streams. A container is formed by shapes (\eg~lengths of lists).
For every shape, we can obtain a set of positions in the container (\eg~offsets in a list of a
specified length). More formally:

\begin{definition}
A \emph{container} $S \triangleleft P$ is given by a set $S$ of shapes and a shape-indexed family
$P : S \rightarrow \ident{Set}$ of positions.
\end{definition}

\noindent
Well-known examples of containers include lists, non-empty lists, (un\-boun\-ded) streams and 
singleton data type (which contains exactly one element). Two containers relevant to our work
are lists and singleton data types:

\begin{itemize}
\item The container representing lists is given by $S \triangleleft P$ where 
  shapes are integers $S = \ident{Nat}$ (lengths of a list). The set of positions for a 
  given length $n$ is a set of indices $P n = \{ 1 \ldots n \}$.

\item The contianer representing singleton data type is given by $S \triangleleft P$ where
  shapes are given by a singleton set $S = \{ \ast \}$ and the set of positions for the
  shape $\ast$ contains exactly one position $P \ast = \{ 0 \}$.
\end{itemize}

\noindent
In the unified coeffect calculus, the structure of coeffect annotations is defined by a 
container with additional operations (discussed later) that links it with the free-variable 
context $\Gamma$. 

\subsection{Structure of coeffects}
In the structural coeffect calculus, the structure annotation was formed by a vector of coeffect
sclars. The unified coeffect calculus is similar, but a \emph{vector} is replaced with a 
\emph{container}. The primitive coeffect annotations in the unified calculus are formed by
a \emph{coeffect scalar}, which remains the same as in structural coeffect calculus
(Definition~\ref{def:structural-scalar}). In this section, we refer to it as
\emph{\cclrd{unified coeffect scalar}} (and we repeat the definition below). Then we
define \emph{\sclrd{unified coeffect containers}} which determines how coeffect scalar values 
are attached to the free-variable context. Finally, we define the \emph{\aclrd{unified coeffect 
algebra}} which consists of shape-indexed coeffect scalar values.

As in the structural coeffect calculus, the contexts in the unified calculus are annotated with 
shape-indexed coeffects, written as $\coctx{\Gamma}{\aclrd{\textbf{r}}} \vdash e : \tau$; 
functions take just a single input parameter and so are annotated with scalar coeffect values 
$\sigma \xrightarrow{\cclrd{r}} \tau$.

\paragraph{Coeffect scalar.}
The following definition of the coeffect scalar structure repeats the Definition~\ref{def:structural-scalar}
from the previous chapter.

\begin{definition}
A \emph{\cclrd{unified coeffect scalar}} $(\C, \cseq, \cpar, \cunit, \czero, \cleq)$ is a set 
$\C$ together with elements $\cunit, \czero \in \C$, relation $\cleq$ and binary operations 
$\cseq, \cpar$ such that $(\C, \cseq, \cunit)$ and $(\C, \cpar, \czero)$ are monoids and
$(\C, \cleq)$ is a pre-order. That is, for all $r,s,t\in \C$:
%
\begin{equation*}
\begin{array}{ccr}
r \;\cseq\; (s \;\cseq\; t) = (r \;\cseq\; s) \;\cseq\; t  &
\cunit \;\cseq\; r = r = r \;\cseq\; \cunit &
\textnormal{(monoid)}   
\\
r \;\cpar\; (s \;\cpar\; t) = (r \;\cpar\; s) \;\cpar\; t &
\czero \;\cpar\; r = r = r \;\cpar\; \czero &
\textnormal{(monoid)}   
\\
\textnormal{if}~~r\; \cleq\; s ~~\textnormal{and}~~s\; \cleq\; t~~\textnormal{then}~~r\; \cleq\; t&
t\; \cleq\; t &
\textnormal{(pre-order)}   
\end{array}
\end{equation*}
\end{definition}

\noindent
As previously, the monoid $(\C,\cseq,\cunit)$ models sequential composition; the laws guarantee
an underlying category structre; $\cunit$ and $\czero$ represent an accessed and unused variable, 
respectively.

The $\cpar$ operation models combining of context requirements arising from multiple parts of a 
program. The meaning depends on the coeffect container. The operation can either combine requirements
of individual variables (structural) or requirements attached to the whole context of multiple
sub-exp\-ressions (flat).

\paragraph{Coeffect containers.}
The coeffect container is a container that determines how are scalar coeffect annotations attached 
to free-variable contexts. In addition to a container $\SHP$ with shapes and shape-indexed positions, the
coeffect container also provides a mapping that returns the shape of a free-variable context. The
mapping between the shape of the variable context and the shape of the coeffect annotation
is not necessarily bijective. For example, coeffect annotations in flat systems have just a single 
shape $\SH = \{ \ast \}$.

In the coeffect judgment $\coctx{\Gamma}{\aclrd{\textbf{r}}} \vdash e : \tau$, the coeffect annotation
$\aclrd{\textbf{r}}$ is drawn from the set of coeffect scalars $\C$ indexed by the shape of $\Gamma$.
We write $\sclrd{s} = \slen{\Gamma}$ for the shape corresponding to $\Gamma$. The operation $\SP\sclrd{s}$
returns a \emph{set} of positions and so we can write $\aclrd{\textbf{r}} \in \SP\sclrd{s} \rightarrow \C$ 
as a mapping from positions (defined by the shape) to scalar coeffects. We usually write this as 
the exponent $\aclrd{\textbf{r}} \in \C^{\SP\sclrd{s}}$.

The coeffect container is also equipped with an operation that appends shapes (when we concatenate
variable contexts) and two special shapes in $\SH$ representing empty context and singleton context. 

\begin{definition}
A \emph{\sclrd{coeffect container}} structure $(\SHP, \stimes, \sempty, \sunit, \slen{-})$ 
comprises a container $\SHP$ with a binary operation $\stimes$ on $\SH$ for appending shapes, a 
mapping from free-variable contexts to shapes $\slen{\Gamma} \in \SH$, and elements $\sempty,\sunit \in
\SH$ such that $(\SH, \stimes, \sempty)$ is a monoid.

The elements $\sempty$ and $\sunit$ represent the shapes of empty and singleton free-variable contexts 
respectively. The $\stimes$ operation corresponds to concatentation of free-variable contexts. Given
$\Gamma_1$ and $\Gamma_2$ such that $\sclrd{s_1}=\slen{\Gamma_1}, \sclrd{s_2}=\slen{\Gamma_2}$, 
we require that $\sclrd{s_1}\,\stimes\,\sclrd{s_2}=\slen{\Gamma_1, \Gamma_2}$. 
\end{definition}

\noindent
As said earlier, we use two kinds of coeffect containers that describe the structure of vectors 
(for structural coeffects) and the shape of trivial singleton container (for flat coeffects): 

\begin{example}
\label{ex:unified-struct-shape}
Structural coeffect container is defined as $\sclrd{(\SHP, {|}\textnormal{--}{|}, +, 0, 1)}$
where $\SH=\mathbb{N}$ and $\SP \sclrd{n}=\{ 1 \ldots n \}$. The shape mapping ${|}\Gamma{|}$ returns the 
number of variables in $\Gamma$. Empty and singleton contexts are annotated with $0$ and $1$, 
respectively, and shapes of combined contexts are added so that ${|}\Gamma_1, \Gamma_2{|} = 
{|}\Gamma_1{|} + {|}\Gamma_2{|}$. 

Therefore, a coeffect annotation is a \emph{vector} 
$\aclrd{\textbf{r}} \in \C^{\SP\sclrd{n}}$ and assigns a coeffect scalar $\aclrd{\textbf{r}}(i) \in \C$ 
for each position (corresponding to a variable $x_i$ in the context).
\end{example}

\begin{example}
\label{ex:unified-flat-shape}
Flat coeffect container is defined as $\sclrd{(\SHP, {|}-{|}, \diamond, \sflat, \sflat)}$.
The container is defined as a singleton data type $\SH = \{\sflat \}$ and $\SP \sflat = \{ 0 \}$
with a constant function ${|}\Gamma{|}=\sflat$ and a trivial operation $\sflat\,\diamond\,\sflat = \sflat$. 

That is, there is a single shape
$\sflat$ with a single position and all free-variable contexts have the same singleton shape.
Therefore, a coeffect annotation is drawn from $\C^\{\sflat\}$ which is isomorphic to $\C$
and so a coeffect scalar $\cclrd{r}\in \C$ is associated with every free-variable context.
\end{example}

\newpage
\begin{example}
Similarly to the previous example, we can also define a coeffect container with \emph{no} positions,
\ie~$\sclrd{(\SHP, {|}-{|}, \diamond, \sflat, \sflat)}$ where $\SH = \{\sflat \}$, $\SP \sflat = \emptyset$,
${|}\Gamma{|}=\sflat$ and $\sflat\,\diamond\,\sflat = \sflat$. 

This reduces our system to the simply-typed $\lambda$-calculus with no context annotations, because
$\SP\sflat=\emptyset$ and so coeffect annotations would be from the set $\C^\emptyset$.
\end{example}


\paragraph{Unified coeffect algebra.}
The coeffect calculus annotates judgments with shape-indexed coeffect annotations.
The \emph{\aclrd{unified coeffect algebra}} combines a coeffect scalar and coeffect container
to define shape-indexed coeffects and operations for manipulating these. 

The definition here reconciles the third point discussed in Section~\ref{sec:unified-unified} --
the fact that flat coeffects use separate operations for splitting and merging ($\cpar$ and $\czip$)
while structural coeffects use tensor $\atimes$. In the unified calculus, we use two operators
that can, however, be coincide.

\begin{definition}
Given a \cclrd{unified coeffect scalar} $(\C, \cseq, \cpar, \cunit, \czero, \cleq)$ and a 
\sclrd{coeffect container} $(\SHP, \slen{-}, \stimes, \sempty, \sunit)$ a \emph{\aclrd{unified 
coeffect algebra}} extends the two structures with $(\azip, \apar, \azero)$ where $\azero \in \C^{\SP \sempty}$
is a coeffect annotation for the empty context and $\azip, \apar$ are families of operations that 
combine coeffect annotations indexed by shapes. That is $\forall \sclrd{n},\sclrd{m} \in \SH$:
%
\begin{equation*}
\apar_{\sclrd{m},\sclrd{n}}, \azip_{\sclrd{m},\sclrd{n}} ~:~ \C^{\SP \sclrd{m}} \times \C^{\SP \sclrd{n}} \rightarrow \C^{\SP (\sclrd{m}\,\stimes\,\sclrd{n})}
\end{equation*}
\end{definition}

\noindent
A coeffect algebra induces the following three additional operations:
%
\begin{equation*}
\begin{array}{rclcrcl}
 \alift{-} &\narrow{:}& \C \rightarrow \C^{\SP \sunit}  \\
 \alift{\cclrd{x}} &\narrow{=}& \lambda \_ . \cclrd{x}   \\
\\[-0.5em]
 \aseq_{\sclrd{m}} &\narrow{:}&  \C \times \C^{\SP \sclrd{m}} \rightarrow \C^{\SP \sclrd{m}}   \\
 \cclrd{r}\, \aseq\, \aclrd{\textbf{s}} &\narrow{=}& \lambda \sclrd{s} . \cclrd{r}\, \cseq\, \aclrd{\textbf{s}}(\sclrd{s}) \\
\\[-0.5em]
 \alen{-} &\narrow{:}& \C^{\SP \sclrd{m}} \rightarrow \sclrd{m}\\
 \alen{\aclrd{\textbf{r}}} &\narrow{=}& \sclrd{m}
\end{array}
\end{equation*}
%
The $\alift{-}$ operation lifts a scalar coeffect to a shape-indexed coeffect that is indexed by 
the singleton context shape. The $\aseq_{\sclrd{m}}$ operation is a left multiplication of a 
vector by scalar. As we always use bold font for vectors and ordinary font for scalars (as well 
as distinct colour), using the same symbol is not ambiguous. We also tend to omit the subscript
$\sclrd{m}$ and write $\aseq$. 

Finally, we define $\alen{}$ as an operation that returns the shape of a given shape-indexed 
coeffect. The only purpose is to simplify notation, as we often need to specify that shapes of 
variable context and coeffect match, \eg~$\alen{\aclrd{\textbf{r}}} = \slen{\Gamma}$.

\paragraph{Splitting and merging coeffects.}
The operators $\apar$ and $\azip$ combine shape-indexed coeffects associated with two contexts. 
For example, assume we have $\Gamma_1$ and $\Gamma_2$ with coeffects $\aclrd{\textbf{r}} \in 
\C^{\SP\sclrd{m}}$ and $\aclrd{\textbf{s}} \in \C^{\SP\sclrd{n}}$. In the structural system, the 
context shapes $\sclrd{m}, \sclrd{n}$ denote the number of variables in the two contexts. The 
combined context $\Gamma_1, \Gamma_2$ has a shape $\sclrd{m}\,\stimes\,\sclrd{n}$ and the combined 
coeffects $\aclrd{\textbf{r}} \,\azip\, \aclrd{\textbf{s}}, \aclrd{\textbf{r}} \,\apar\, \aclrd{\textbf{s}} 
\in \C^{\SP(\sclrd{m} \stimes \sclrd{n})}$ are indexed by that shape.

For structural coeffect systems such as bounded reuse, both $\apar$ and $\azip$ are
just the tensor product $\times$ of vectors. For flat coeffect systems, the operations
can be defined independently, letting $\azip=\czip$ and $\apar=\cpar$.

The difference between $\azip$ and $\apar$ is clarified by the semantics (Sec\-tion~\ref{sec:unified-semantics}), 
where $\aclrd{\textbf{r}}\,\azip\,\aclrd{\textbf{s}}$ is an annotation of the \emph{codomain} of 
a morphism that merges the capabilities provided by two contexts (in the syntactic 
reading, splits the context requirements), while $\aclrd{\textbf{r}}\,\apar\,\aclrd{\textbf{s}}$ 
is an annotation of the \emph{domain} of a morphism that splits the capabilities of a single 
context into two parts (in the syntactic reading, merges their context requirements). 
Syntactically, this means that we always use $\azip$ in the rule \emph{assumptions} and 
$\apar$ in \emph{conclusions}. 

% --------------------------------------------------------------------------------------------------

\newcommand{\tystmt}[2]{ (\text{\footnotesize{#1}})~~{#2} }
\newcommand{\ctxtrans}[3]{ #2 \rightsquigarrow #1, #3 }
\newcommand{\ctxtransnl}[3]{ \begin{array}{l} #2 \rightsquigarrow\\ #1, #3 \end{array}}

\begin{figure}[t]

{\small a.) Syntax-driven typing rules:}
\begin{equation*}
\tyrule{var}
  { }
  {\coctx{x\!:\!\tau}{\alift{\cunit}} \vdash x : \tau }
\end{equation*}
\begin{equation*}
\tyrule{const}
  {c \!:\! \tau \in \Delta}
  {\coctx{()}{\azero} \vdash c : \tau }
\end{equation*}
\begin{equation*}
\tyrule{app}
  {\coctx{\Gamma_1}{\aclrd{\textbf{r}}} \vdash e_1 : \tau_1 \xrightarrow{\cclrd{t}} \tau_2 &
   \coctx{\Gamma_2}{\aclrd{\textbf{s}}} \vdash e_2 : \tau_1 }
  {\coctx{\Gamma_1, \Gamma_2}{\aclrd{\textbf{r}\;\apar\;(\cclrd{t} \,\aseq\, \aclrd{\textbf{s}})}} \vdash e_1~e_2 : \tau_2}
\end{equation*}
\begin{equation*}
\tyrule{abs}
  {\coctx{\Gamma, x\!:\!\tau_1}{\aclrd{\textbf{r}}\,\azip\,\alift{\cclrd{s}}} \vdash e : \tau_2}
  {\coctx{\Gamma}{\aclrd{\textbf{r}}} \vdash \lambda x.e : \tau_1 \xrightarrow{\cclrd{s}} \tau_2 }
\end{equation*}
\begin{equation*}
\tyrule{let}
  { \coctx{\Gamma_1}{\aclrd{\textbf{r}}} \vdash e_1 : \tau_1 &
    \coctx{\Gamma_2, x\!:\!\tau_1}{\aclrd{\textbf{s}}\;\azip\;\alift{\cclrd{t}}} \vdash e_2 : \tau_2}
  {\coctx{\Gamma_1, \Gamma_2}{(\cclrd{t} \,\aseq\, \aclrd{\textbf{r}}) \;\apar\; \aclrd{\textbf{s}}} \vdash \kvd{let}~x=e_1~\kvd{in}~e_2 : \tau_2 }
\end{equation*}

\vspace{1.5em}
{\small b.) Structural rules for context manipulation:}
\begin{equation*}
\tyrule{sub}
  {\coctx{\Gamma_1,x\!:\!\tau_1,\Gamma_2}{\aclrd{\textbf{r}} \azip \alift{\cclrd{s'}} \azip \aclrd{\textbf{q}}} \vdash e : \tau}
  {\coctx{\Gamma_1,x\!:\!\tau_1,\Gamma_2}{\aclrd{\textbf{r}} \apar \alift{\cclrd{s}} \apar \aclrd{\textbf{q}}} \vdash e : \tau}
\quad
(\cclrd{s'} \cleq \cclrd{s})
\end{equation*}
\begin{equation*}
\tyrule{weak}
  {\coctx{\Gamma}{ \aclrd{\textbf{r}} } \vdash e : \tau}
  {\coctx{\Gamma,x \!:\! \tau_1}{\aclrd{\textbf{r}} \apar \alift{ \czero }} \vdash e : \tau} 
\end{equation*}
\begin{equation*}
\tyrule{exch}
  {\coctx{\Gamma_1,x\!:\!\tau_1,y\!:\!\tau_2,\Gamma_2}{\aclrd{\textbf{r}} \azip \alift{\cclrd{s}} \azip \alift{\cclrd{t}} \azip \aclrd{\textbf{q}}} \vdash e : \tau}
  {\coctx{\Gamma_1,y\!:\!\tau_2,x\!:\!\tau_1,\Gamma_2}{\aclrd{\textbf{r}} \apar \alift{\cclrd{t}} \apar \alift{\cclrd{s}} \apar \aclrd{\textbf{q}}} \vdash e : \tau}
\quad
\begin{array}{l}
 \slen{\Gamma_1} = \alen{\aclrd{\textbf{r}}}\\[-0.25em]
 \slen{\Gamma_2} = \alen{\aclrd{\textbf{s}}}
\end{array}
\end{equation*}
\begin{equation*}
\hspace{2.5em} 
\tyrule{contr}
  {\coctx{\Gamma_1,y\!:\!\tau_1,z\!:\!\tau_1,\Gamma_2}{\aclrd{\textbf{r}} \azip \alift{\cclrd{s}} \azip \alift{\cclrd{t}} \azip \aclrd{\textbf{q}}} \vdash e : \tau}
  {\coctx{\Gamma_1,x\!:\!\tau_1,\Gamma_2}{\aclrd{\textbf{r}} \apar \alift{\cclrd{s} \,\cpar\, \cclrd{t}} \apar \aclrd{\textbf{q}}} \vdash \subst{e}{z,y}{x} : \tau}
~
\begin{array}{l}
 \slen{\Gamma_1} = \alen{\aclrd{\textbf{r}}}\\[-0.25em]
 \slen{\Gamma_2} = \alen{\aclrd{\textbf{s}}}
\end{array}
\end{equation*}

\vspace{0.5em}
\caption{Type system for the unified coeffect calculus}
\label{fig:unif-types}
\end{figure}

% --------------------------------------------------------------------------------------------------

\subsection{Unified coeffect type system}
\label{sec:unified-types}
The unified coeffect system in Figure~\ref{fig:unif-types} resembles the structural type system
shown in Figure~\ref{fig:struct-types}. Rather than explaining the rules one-by-one, we focus on
the key differences between the two.

The type system for the unified coeffect calculus is parameterized by a coeffect scalar 
$(\C, \cseq, \cpar, \cunit, \czero, \cleq)$ together with a coeffect algebra $(\azip, \apar, \azero)$ 
and the derived constructs $\alift{-}$, $\alen{-}$ and $\aseq$. 
As in the structural system, free-variable contexts $\Gamma$ are treated as vectors modulo duplicate
use of variables -- the associativity is built-int. The order of variables matters, but can be changed 
using the structural rules. The context annotations $\aclrd{\textbf{r}}, \aclrd{\textbf{s}}, \aclrd{\textbf{t}} \in \C^\SH$ 
are shape-indexed coeffects (rather than simple vectors as before). As before, functions are
annotated with coeffects scalars. % $\cclrd{r}, \cclrd{s}, \cclrd{t} \in \C$.

\paragraph{Syntax-driven rules.} 
The (\emph{var}) rule is syntactically the same as in the structural system, but it should be read
differently. The $\alift{-}$ operation does not create a \emph{vector}, but a \emph{shape-indexed
coeffect} that returns $\cunit$ for all positions in the singleton shape $\sunit$. 
The (\emph{const}) rule annotates empty context with a special annotation of the unit shape.

In a structural system, the two annotations correspond to a singleton and empty vector, respectively.
However, for a singleton shape with one position, the annotations are equivalent to annotating
variables with a scalar $\cunit$ and constants with a scalar $\czero$.

In the (\emph{app}), (\emph{abs}) and (\emph{let}) rules, the only change from the structural
system is that vector concatentation $\atimes$ is now replaced with context splitting/merging 
of the unified coeffect algebra. In particular, we use splitting of context requirements $\azip$
in rule \emph{assumptions} and merging of context requirements $\apar$ in rule \emph{conclusions}.

Note that we use the terms \emph{merging} and \emph{splitting} in the syntactic (top-down) sense. As 
discussed in Chapter~\ref{ch:flat}, Section~\ref{sec:flat-calculus-undestanding}, we can also read 
the rules in semantic (bottom-up) sense, in which case assumptions merge available contextual 
information and conclusions split available contextual information.

\paragraph{Structural rules.}
The merging/splitting operations in the structural rules are changed in the same way as in the
rules above. It is also worth noting that structural coeffect system used vectors of multiple 
elements. For example, $\alift{\cclrd{s},\cclrd{t}}$ and $\alift{\cclrd{t},\cclrd{s}}$ in the (\emph{exch}) 
rule denoted a two-element vector. In the unified system, this is replaced with merging/splitting
of two lifted scalars: $\alift{\cclrd{s}} \azip \alift{\cclrd{t}}$ and $\alift{\cclrd{t}} \apar \alift{\cclrd{s}}$.

In structural systems, this means the same thing -- we are simply concatenating or splitting two 
singleton vectors. However, the generalization allows us to capture flat coeffect systems as well.
The lifting operation in flat systems simply returns the lifted scalar and operators $\azip$ and
$\apar$ correspond to operations on coeffect scalars. And, as discussed in Section~\ref{sec:unified-flat},
thanks to the properties of coeffect scalars, contraction, weakening and exchange that do not 
affect the coeffect annotation are admissible for all flat systems.

% --------------------------------------------------------------------------------------------------

\subsection{Structural coeffects}
\label{sec:unified-structural}

The unified coeffect system uses a general notion of context shape, but it has been designed with 
structural and flat systems in mind. In this and the next section, we show how it captures 
the two kinds of systems. 

The unified calculus is modelled after the structural system and so using it to model structural
systems is easy -- given a coeffect scalar, we use the coeffect container that describes a \emph{vector} 
of annotations (Example~\ref{ex:unified-struct-shape}) and define a coeffect algebra formed by
a vector (free monoid) of scalars.

\begin{definition}
\label{def:unified-struct}
Given a coeffect scalar $(\C, \cseq, \cpar, \cunit, \czero, \cleq)$ a \emph{structural coeffect system} 
is defined by: 

\begin{itemize}{}
\item[--] Coeffect container $\sclrd{(\SHP, {|}\textnormal{--}{|}, +, 0, 1)}$ where $\SH=\mathbb{N}$ and
  $\SP \sclrd{n}=\{ 1 \ldots n \}$ and  ${|}x_1\!:\!\tau_1,\ldots,x_n\!:\!\tau_n{|}=n$.

\item[--] Coeffect algebra $\aclrd{(\times, \times, \epsilon)}$ where $\times$ and $\epsilon$ are
  shape-indexed versions of the binary operation and the unit of a free monoid over $\C$.
  That is $\times : \C^{\SP\sclrd{n}} \times \C^{\SP\sclrd{m}} \rightarrow \C^{\SP(\sclrd{n + m})}$ 
  appends vectors and $\epsilon : \C^{\SP\sclrd{0}}$ represents empty vectors.
\end{itemize}
\end{definition}

\noindent
The definition is valid since the shape operations form a monoid
$\sclrd{(\mathbb{N}, +, 0)}$ and $\slen{\textnormal{--}}$ (calculating the length of
a list) is a monoid homomorphism from the free monoid to the monoid of
shapes.

\paragraph{Properties.}
An important property of the unified system is that, when used in a structural way as discussed
above, it gives calculi with the same properties as the structural system described in 
Chapter~\ref{ch:structural}. This can be easily seen by comparing the Figure~\ref{fig:unif-types}
with the Figure~\ref{fig:struct-types} and using the free monoid interpretation of the unified
coeffect algebra.

\begin{remark}
The system described in Definition~\ref{def:unified-struct} is equivalent to the structural coeffect
system described in Figure~\ref{fig:struct-types}. That is, a typing derivation using a structural
coeffect embedded in the unified system is valid if and only if the corresponding derivation is
valid in the structural system.
\end{remark}

\noindent
Using the above definition, our unified coeffect system can capture per-variable coeffect properties
discussed in Chapter~\ref{ch:applications}, Section~\ref{sec:applications-structural}. This includes
the system for bounded reuse (which is only used in the structural form) and precise tracking 
of per-variable dataflow and liveness. 

% --------------------------------------------------------------------------------------------------

\subsection{Flat coeffects}
\label{sec:unified-flat}

The same unified coeffect system can be used to capture systems that track whole-context (flat) 
coeffects such as implicit parameters. This is achieved using a singleton-shaped container for
coeffect annotations. The resulting system is has explicit structural rules (and is syntactically
different from the standard flat coeffect system), but we show that they are equivalent.

Flat coeffect systems are characterised by a singleton set of shapes (Example~\ref{ex:unified-flat-shape})
In this setting, the context annotations $\C^{ \SP \sflat } = \{0\} \rightarrow \C$ are equivalent 
to coeffect scalars $\C$. In addition to the coeffect scalar structure, we also need to define $\apar$ 
and $\azip$. Our examples of flat coeffects use $\cpar$ (merging of scalar coeffects) for $\apar$ 
(merging of shaped coeffect annotations). The $\azip$ operation (corresponding to $\czip$
in flat coeffect calculus) needs to be provided explicitly.

\begin{definition}
\label{def:unified-flat}
Given a coeffect scalar $(\C, \cseq, \cpar, \cunit, \czero, \cleq)$ and a binary operation 
$\cclrd{\wedge} : \C \times \C \rightarrow \C$ such that $(\cclrd{r \wedge s}) \,\cleq\, (\cclrd{r}\,\cpar\,\cclrd{s})$, 
the unified coeffect algebra modelling flat coeffect systems consists of:
%
\begin{itemize}{}
\item[--] Flat coeffect container $\sclrd{(\SHP, {|}-{|}, \diamond, \sflat, \sflat)}$ as defined in 
  Example~\ref{ex:unified-flat-shape}.

\item[--] Flat coeffect algebra $(\czip, \cpar, \czero)$, \ie~$\apar = \cpar$ 
  and $\azero = \czero$ with an additional binary operation $\azip = \czip$. 
\end{itemize}
\end{definition}

\noindent
Intuitively, the requirement $(\cclrd{r}\,\czip\,\cclrd{s}) \;\cleq\; (\cclrd{r}\,\cpar\,\cclrd{s})$,
which could be also written as $(\aclrd{r}\,\azip\,\aclrd{s}) \;\aclrd{\leq}\; (\aclrd{r}\,\apar\,\aclrd{s})$,
denotes that splitting context requirements and then re-combining them preserves all the requirements
from the assumptions. The system may be imprecise and conclusions can overapproximate assumptions, but
it cannot lose requirements. This is fundamental for showing that exchange and contraction are admissible
in the unified system. 

\paragraph{Properties.}
To show that the typing of flat properties in the unified system is equivalent to the typing in the
flat system, we show that a valid typing judgement in the first system is a valid typing judgement in 
the second system and vice versa.

In one direction, we show that the unified system (capturing flat properties) permits weakening, 
contraction and exchange rules that do not change the coeffect annotations. This guarantees that
a valid judgement in flat system is also valid in the unified system.

\begin{lemma}
\label{thm:unified-weak}
A unified coeffect calculus capturing flat properties admits weakening that does not change the
coeffect annotation.
\end{lemma}
\begin{proof}
The rule is admissible using the following derivation:
\begin{equation*}
\inference
  { \inference
      {\coctx{\Gamma}{ \cclrd{r} } \vdash e : \tau}
      {\coctx{\Gamma,x \!:\! \tau_1}{\cclrd{r} \;\apar\; \alift{ \czero }} \vdash e : \tau} }
  { \inference
      {\coctx{\Gamma,x \!:\! \tau_1}{\cclrd{r} \,\cpar\, \czero} \vdash e : \tau}      
      {\coctx{\Gamma,x \!:\! \tau_1}{\cclrd{r}} \vdash e : \tau} }
\end{equation*}

\noindent
We write $\cclrd{r}$ rather than $\aclrd{\textbf{r}}$, because we are tracking flat properties.
The first step is an application of the (\emph{weak}) rule. Next, we use the fact that
$\apar=\cpar$ and $\alift{\czero}=\czero$, which is the unit of the monoid $(\C, \cpar, \czero)$.
\end{proof}

\begin{lemma}
A unified coeffect calculus capturing flat properties admits exchange that does not change the
coeffect annotation.
\end{lemma}
\begin{proof}
We use the idempotence of $\czip$ and $\cpar$ together with the (\emph{exch}) rule:
\begin{equation*}
\inference
  { \inference
      { \inference
          {\coctx{\Gamma_1,x\!:\!\tau_1,y\!:\!\tau_2,\Gamma_2}{\cclrd{r}} \vdash e : \tau}
          {\coctx{\Gamma_1,x\!:\!\tau_1,y\!:\!\tau_2,\Gamma_2}{\cclrd{r} \,\czip\, \cclrd{r} \,\czip\, \cclrd{r} \,\czip\, \cclrd{r}} \vdash e : \tau} }
      { \coctx{\Gamma_1,x\!:\!\tau_1,y\!:\!\tau_2,\Gamma_2}{\cclrd{r} \azip \alift{\cclrd{r}} \azip \alift{\cclrd{r}} \azip \cclrd{r}} \vdash e : \tau } }
  { \inference 
      {\coctx{\Gamma_1,y\!:\!\tau_2,x\!:\!\tau_1,\Gamma_2}{\cclrd{r} \apar \alift{\cclrd{r}} \apar \alift{\cclrd{r}} \apar \cclrd{r}} \vdash e : \tau}
      {\inference
         {\coctx{\Gamma_1,y\!:\!\tau_2,x\!:\!\tau_1,\Gamma_2}{\cclrd{r}\,\cpar\,\cclrd{r}\,\cpar\,\cclrd{r}\,\cpar\,\cclrd{r}} \vdash e : \tau}
         {\coctx{\Gamma_1,y\!:\!\tau_2,x\!:\!\tau_1,\Gamma_2}{\cclrd{r}} \vdash e : \tau} } }
\end{equation*}

\noindent
Using idempotence, we first duplicate the annotation $\cclrd{r}$ to get a coeffect in the form
$\cclrd{r} \azip \alift{\cclrd{r}} \azip \alift{\cclrd{r}} \azip \cclrd{r}$ as required by the
assumption of the (\emph{exch}) rule. Note that $\alift{-}$ can be added freely as
$\alift{\cclrd{r}}$ is equivalent to $\cclrd{r}$. After applying (\emph{exch}), we use 
idempotence of $\apar$.
\end{proof}

\begin{lemma}
A unified coeffect calculus capturing flat properties admits contraction that does not change the
coeffect annotation.
\end{lemma}
\begin{proof}
Similarly to exchange, the proof uses idempotence of $\czip$ and $\cpar$:
\begin{equation*}
\inference
  { \inference
      {\coctx{\Gamma_1,y\!:\!\tau_1,z\!:\!\tau_1,\Gamma_2}{\cclrd{r} \czip\, \cclrd{r} \czip\, \cclrd{r} \czip\, \cclrd{r}} \vdash e : \tau}
      {\coctx{\Gamma_1,y\!:\!\tau_1,z\!:\!\tau_1,\Gamma_2}{\cclrd{r} \azip \alift{\cclrd{r}} \azip \alift{\cclrd{r}} \azip \cclrd{r}} \vdash e : \tau} }
  { \inference
      {\coctx{\Gamma_1,x\!:\!\tau_1,\Gamma_2}{\cclrd{r} \apar \alift{\cclrd{r} \,\cpar\, \cclrd{r}} \apar \cclrd{r}} \vdash \subst{e}{z,y}{x} : \tau}
      {\coctx{\Gamma_1,x\!:\!\tau_1,\Gamma_2}{\cclrd{r} \,\cpar\, \cclrd{r} \,\cpar\, \cclrd{r} \,\cpar\, \cclrd{r}} \vdash \subst{e}{z,y}{x} : \tau} }
\begin{array}{r} \\[2.5em] \qedhere \end{array}
\end{equation*}
\end{proof}

\noindent
In the last two cases, we need to turn the coeffect into a form that is required by the exchange
and contraction rules. Aside from idempotence, we could use the unit property and obtain 
\eg~$\czero\,\czip\,\cclrd{r}\,\czip\,\czero\,\czip\,\czero$. However, this appraoch does not 
work because $\cpar$ and $\czip$ may have different unit elements (in fact, we do not ever require
the existence of unit of $\czip$).

In the other direction, we need to show that any valid judgement in the unified system (tracking
flat properties) is also valid in the flat system. The syntax-directed rules are the same in both
systems, but we need to show that any use of (explicit) weakening, contraction and exchange can be
derived in the flat system. 

\begin{lemma}
Weakening, as defined in a unified coeffect calculus capturing flat properties, is admissible in 
the flat coeffect calculus.
\end{lemma}
\begin{proof}
Similar to the proof in Lemma~\ref{thm:unified-weak}. The property follows from the fact that 
$\bot=\alift{\czero}$ is the unit of $\cpar$.
\end{proof}

\begin{lemma}
\label{thm:unified-rev-contr}
Contraction, as defined in a unified coeffect calculus capturing flat properties, is admissible in 
the flat coeffect calculus.
\end{lemma}
\begin{proof}
The application of (\emph{contr}) rule has the following form:
\begin{equation*}
\inference
 { \inference
    {\coctx{\Gamma_1,y\!:\!\tau_1,z\!:\!\tau_1,\Gamma_2}{\cclrd{r} \,\czip\, \cclrd{s} \,\czip\, \cclrd{t} \,\czip\, \cclrd{q}} \vdash e : \tau}
    {\coctx{\Gamma_1,y\!:\!\tau_1,z\!:\!\tau_1,\Gamma_2}{\cclrd{r} \,\azip\, \alift{\cclrd{s}} \,\azip\, \alift{\cclrd{t}} \,\azip\, \cclrd{q}} \vdash e : \tau} }
 { \inference
    {\coctx{\Gamma_1,x\!:\!\tau_1,\Gamma_2}{\cclrd{r} \,\apar\, \alift{\cclrd{s} \,\cpar\, \cclrd{t}} \,\apar\, \cclrd{q}} \vdash \subst{e}{z,y}{x} : \tau}
    {\coctx{\Gamma_1,x\!:\!\tau_1,\Gamma_2}{\cclrd{r} \,\cpar\, \cclrd{s} \,\cpar\, \cclrd{t} \,\cpar\, \cclrd{q}} \vdash \subst{e}{z,y}{x} : \tau} } 
\end{equation*}
%
From Definition~\ref{def:unified-flat}, we know that $(\cclrd{r} \,\czip\, \cclrd{s} \,\czip\, \cclrd{t} \,\czip\, \cclrd{q}) 
\;\cleq\; (\cclrd{r} \,\cpar\, \cclrd{s} \,\cpar\, \cclrd{t} \,\cpar\, \cclrd{q})$. Thus, the judgement
 can be derived using the (\emph{sub}) rule of flat coeffect calculus.
\end{proof}


\begin{lemma}
\label{thm:unified-rev-exch}
Exchange, as defined in a unified coeffect calculus capturing flat properties, is admissible in 
the flat coeffect calculus.
\end{lemma}
\begin{proof}
The application of (\emph{exch}) rule has the following form:
\begin{equation*}
\inference
 {\inference
    {\coctx{\Gamma_1,x\!:\!\tau_1,y\!:\!\tau_2,\Gamma_2}{\cclrd{r} \,\czip\, \cclrd{s} \,\czip\, \cclrd{t} \,\czip\, \cclrd{q}} \vdash e : \tau}
    {\coctx{\Gamma_1,x\!:\!\tau_1,y\!:\!\tau_2,\Gamma_2}{\cclrd{r} \,\azip\, \alift{\cclrd{s}} \,\azip\, \alift{\cclrd{t}} \,\azip\, \cclrd{q}} \vdash e : \tau} }
 {\inference
    {\coctx{\Gamma_1,y\!:\!\tau_2,x\!:\!\tau_1,\Gamma_2}{\cclrd{r} \,\apar\, \alift{\cclrd{t}} \,\apar\, \alift{\cclrd{s}} \,\apar\, \cclrd{q}} \vdash e : \tau}
    {\coctx{\Gamma_1,y\!:\!\tau_2,x\!:\!\tau_1,\Gamma_2}{\cclrd{r} \,\cpar\, \cclrd{t} \,\cpar\, \cclrd{s} \,\cpar\, \cclrd{q}} \vdash e : \tau} }
\end{equation*}
%
From idempotence of $\czip$, we get that $\cclrd{r} \,\czip\, \cclrd{s} \,\czip\, \cclrd{t} \,\czip\, \cclrd{q} =
\cclrd{r} \,\czip\, \cclrd{t} \,\czip\, \cclrd{s} \,\czip\, \cclrd{q}$. Thus the judgement can be derived using
(\emph{sub}) as in contraction.
\end{proof}

\noindent
A consequence of the equivalence discussed above is that the unified coeffect system can capture 
all properties that can be captured by the flat coeffect system -- including implicit parameters,
rebindable resources, Haskell type classes (discussed by Orchard~\cite{comonads-dom-thesis}),
data-flow and variable liveness.

%--------------------------------------------------------------------------------------------------

\subsection{Semanitcs of unified calculus}
\label{sec:unified-semantics}

The semantics of unified calculus can be obtained by generalizing the semantics of the structural
calculus in the same way as we generalized the type system. However, it is worth exploring the type 
signatures of the operations of the underlying (comonad-based) structure.

The semantics of the unified coeffect calculus follows the same pattern as the semantics of the
structural calculus (Section~\ref{sec:struct-semantics}) with a number of differences. The 
index $\aclrd{\textbf{r}}$ of the object mapping $\ctyp{\aclrd{\textbf{r}}}$ is an element of 
the structure $\C^{\SP \sclrd{n}}$ for a specified container $\SH \triangleleft \SP$ and a shape 
$\sclrd{n}$ determined as $\sclrd{n} = \slen{\Gamma}$. To match the type system discussed in 
Section~\ref{sec:unified-types}, we need to modify the domains and codomains of the \ident{merge}, 
\ident{split} and \ident{dup} operations as follows.
%
\begin{equation*}
\begin{array}{lcl}
 \ident{merge}_{\aclrd{\textbf{r}},\aclrd{\textbf{s}}} &\narrow{:}& 
  \ctyp{\aclrd{\textbf{r}}}{\alpha} \times \ctyp{\aclrd{\textbf{s}}}{\beta} \rightarrow \ctyp{\aclrd{\textbf{r}}\azip\aclrd{\textbf{s}}}{(\alpha \xtimes \beta)} \\
 \ident{split}_{\aclrd{\textbf{r}},\aclrd{\textbf{s}}} &\narrow{:}& 
  \ctyp{\aclrd{\textbf{r}}\apar\aclrd{\textbf{s}}}{(\alpha \xtimes \beta)} \rightarrow \ctyp{\aclrd{\textbf{r}}}{\alpha} \times \ctyp{\aclrd{\textbf{s}}}{\beta} \\
 \ident{dup}_{\cclrd{r}, \cclrd{s}} &\narrow{:}& 
  \ctyp{\alift{\cclrd{r} \cpar \cclrd{s}}}{\alpha} \rightarrow \ctyp{\alift{\cclrd{r}}\,\azip\,\alift{\cclrd{s}}}{(\alpha \xtimes \alpha)} \\
\end{array}
\end{equation*}
%
The key difference is that the vector concatenation $\atimes$ is replaced with $\apar$ in the
\emph{domain} of the operations (in \ident{split}) and with $\azip$ in the codomain
(in \ident{merge}, but also in \ident{dup} which models contraction).
This means that the splitting and merging the context can change the coeffect annotation:
%
\begin{equation*}
\ident{merge}_{\aclrd{\textbf{r}},\aclrd{\textbf{s}}} \circ \ident{split}_{\aclrd{\textbf{r}},\aclrd{\textbf{s}}} ~:~
  \ctyp{\aclrd{\textbf{r}}\apar\aclrd{\textbf{s}}}{(\alpha \xtimes \beta)}
  \rightarrow \ctyp{\aclrd{\textbf{r}}\azip\aclrd{\textbf{s}}}{(\alpha \xtimes \beta)} 
\end{equation*}
%
For flat coeffect calculi where $(\aclrd{r}\,\azip\,\aclrd{s}) \;\aclrd{\leq}\; (\aclrd{r}\,\apar\,\aclrd{s})$,
splitting and then merging a context may lose some of the avilable contextual capabilities
(and increase context requirements). This corresponds to the fact that our proofs of 
Lemma~\ref{thm:unified-rev-contr} and Lemma~\ref{thm:unified-rev-exch} relied on sub-coeffecting
in the flat system.

\paragraph{Related and future work.}
The semantics presented here serves mainly to inform the design of the type system. For this 
reason, we keep the semantics simple and close to the concrete notions used in the type system
such as flat, structural and unified coeffect algebras. In particular, the indexed comonad 
structure consists of object mapping $\ctyp{r}{}$ that is indexed by a corresponding algebra:

\begin{compactitem}
\item[-] In the flat coeffect calculus, $\ctyp{\cclrd{r}}{}$ is indexed by scalar coeffects $\cclrd{r} \in \C$
\item[-] In the structural coeffect calculus, the object mapping $\ctyp{\aclrd{\textbf{r}}}{}$ is 
  indexed by vectors of scalars $\aclrd{\textbf{r}} \in \bigcup_{\sclrd{m}\in \sclrd{\mathbb{N}}} \C^{\sclrd{m}}$
\item[-] In unified coeffect calculus with a coeffect container $\SH\,\sclrd{\triangleleft}\,\SP$,
  the mapping $\ctyp{\aclrd{\textbf{r}}}{}$ is indexed by shape-indexed scalars 
  $\aclrd{\textbf{r}} \in \bigcup_{\sclrd{s}\in \SH} \C^{\SP\sclrd{m}}$
\end{compactitem}

\noindent
In this treatment, the indices are formed by objects that are modelled outside of category theory.
A fully categorical semantics can be found in a joint paper with Orchard \cite{coeffects-icfp14}. 
There, the index (coeffect scalars and shape-indexed scalars) are modelled as categories too.
The definition requires the following notation:

\begin{compactitem}
\item[-] $[\mathbb{C}, \mathbb{D}]$ is a category of functors between categories $\mathbb{C}$ and $\mathbb{D}$
\item[-] $B^A$ is an exponential object $A \Rightarrow B$ in cartesian closed category
\end{compactitem}

\noindent
In the categorical semantics of coeffect calculi, coeffect scalars are modelled as a
category of scalars $\cclrd{\mathbb{I}}$ such that $\textit{obj}(\cclrd{\mathbb{I}}) = \C$.
The structure of the context (indexed comonad) over the input then becomes a functor
indexed by the free-variable context shape $\slen{\Gamma}$ and coeffect annotations
$\aclrd{\textbf{r}}$, \ie~$\ctyp{\slen{\Gamma}}_{\aclrd{\textbf{r}}}$:
%
\begin{equation*}
\sem{\coctx{\Gamma}{\aclrd{R}} \vdash e : \tau} : \ctyp{\slen{\Gamma}}_{\aclrd{R}}
\sem{\Gamma} \rightarrow \sem{\tau}
\end{equation*}
%
The structure $\ctyp{}$ can be thought of as a dependent product of functors $\ctyp{\sclrd{n}}$ 
over possible shapes $\sclrd{n} \in \SH$ where:
%
\begin{equation*}
\ctyp{\sclrd{n}} : \cclrd{\mathbb{I}}^{\SP \sclrd{n}} \rightarrow [\mathbb{C}^{\SP \sclrd{n}}, \mathbb{C}]
\end{equation*}
%
For a fixed context shape $\sclrd{n}$ the functor 
$\ctyp{\sclrd{n}} : \cclrd{\mathbb{I}}^{\SP \sclrd{n}} \rightarrow [\mathbb{C}^{\SP \sclrd{n}}, \mathbb{C}]$ 
maps a coeffect indexed by positions $\SP \sclrd{n}$ to a functor from a context 
$\mathbb{C}^{\sclrd{n}}$ to an object in $\mathbb{C}$. That is, given a coeffect annotation 
(matching the shape of the context), we get a functor $\in [\mathbb{C}^{\sclrd{n}},\mathbb{C}]$. 
From a programming perspective, this functor defines a data structure that models the additional
context provided to the program. The shape of this data structure depends on the coeffect
annotation $\cclrd{\mathbb{I}}^{\sclrd{n}}$. For example, in bounded reuse, the annotation defines
the number of values needed for each variable and the functor will be 
formed by lists of length matching the required number. 

In the model above, shapes and positions are still treated as ordinary sets. Indeed, the paper
\cite{coeffects-icfp14} treats shapes as just sets of sets. In this chapter, we instead define
\emph{coeffect container} as an extension of containers, which have a well-defined 
categorical structure \cite{semantics-containers}. A future work is adapting the semantics of
the unified coeffect calculus so that it is based on the semantics of containers. However, note 
that context is \emph{not} a (simple) container, because containers are used as indices (coeffect
annotations) of the object that wrap the free-variable context.



% =================================================================================================
%                                                                                             
%    #   #          #                    ##                                                    
%    #   #          #                     #                                                    
%    ## ##   ###   ####    ###            #     ###   # ##    ## #  #   #   ###    ## #   ###  
%    # # #  #   #   #         #  #####    #        #  ##  #  #  #   #   #      #  #  #   #   # 
%    #   #  #####   #      ####           #     ####  #   #   ##    #   #   ####   ##    ##### 
%    #   #  #       #  #  #   #           #    #   #  #   #  #      #  ##  #   #  #      #     
%    #   #   ###     ##    ####          ###    ####  #   #   ###    ## #   ####   ###    ###  
%                                                            #   #                #   #        
%                                                             ###                  ###         
% =================================================================================================

\section{Coeffect meta-language}
\label{sec:unified-meta} 

In Section~\ref{sec:path-sem-langs}, we discussed two ways of using monads in programming
language semantics introduced by Moggi \cite{monad-notions}. The first approach is to use
monads in the \emph{semantics} of an effectful language. The second approach is to extend the
language with (additional) monadic constructs that can then be used for writing effectful
monads explicitly.

In this thesis, we focused on the first approach. In both flat and structural coeffect calculi,
the term language is that of simply-typed $\lambda$-calculus, and we used (flat or structural)
indexed comonads to give the semantics for the language and to derive type system for it.

In this section, we briefly discuss the alternative approach. That is, we embed indexed
comonads into a $\lambda$-calculus as additional constructs. To do that, we introduce the
type constructor $\ctyp{\cclrd{r}}{\tau}$ which represents a value $\tau$ wrapped in additional
context (semantically, this corresponds to an indexed comonad) and we add language constructs
that correspond to the operations of indexed comonads.

This section provides a brief sketch of coeffect meta-language to highlight the relationship
between coeffects and important related work on contextual modal type theory (CMTT) \cite{logic-cmtt}.
Developing the system further is an interesting future research direction.

% -------------------------------------------------------------------------------------------------

\subsection{Coeffects and contextual modal type theory}

As discussed in Section~\ref{sec:path-sem-contextdep}, context-aware computations are related
to modal logics -- comonads have been used to model the $\square$ modality and as a basis for
meta-languages that include $\square$ as a type constructor
\cite{logic-intuitionistic-modal,logic-modal-reconstruction,logic-intuitionistic-modal,logic-cmtt}.
Nanevski et al. \cite{logic-cmtt} extend an S4 term language to a contextual modal type theory (CMTT).
From the perspective of this thesis, CMTT can be viewed as a \emph{meta-language} version of our
coeffect calculus.

\paragraph{Context in CMTT and coeffects.}
Aside from the fact that coeffect calculi use comonads for \emph{semantics} and CMTT embeds
comonads (the $\square$ modality) into the meta-language, there are two important differences.

Firstly, the \emph{context} in CMTT is a set of variables required by a computation, which
makes CMTT useful for meta-programming and staged computations. In coeffect calculi, the context
requirements are formed by a coeffect algebra, which is more general and can capture
variable contexts, but also integers, two-point lattices, \emph{etc.}

Secondly, CMTT uses different intuitive understanding of the comonad (type constructor) and
the associated operations. In categorical semantics of coeffect calculi, the $\ctyp{\cclrd{r}}{\tau}$
constructor refers to a value of type $\tau$ \emph{together} with additional context specified
by $\cclrd{r}$ (\eg~list of past values or additional resources). In contrast in 
CMTT\footnote{To avoid using different notations, we write $\ctyp{\Psi}{\tau}$ instead of the 
original $[\Psi]{\tau}$} the type $\ctyp{\Psi}{\tau}$ models a value that \emph{requires} the 
context $\Psi$ in order to produce value $\tau$. This also changes the interpretation of the 
two operations of a comonad:
\begin{equation*}
\begin{array}{rcl}
 \ident{counit} &\narrow{:}& \ctyp{\cunit}{\alpha} \rightarrow \alpha\\
 \ident{cobind} &\narrow{:}& (\ctyp{\cclrd{r}}{\alpha} \rightarrow \beta) \rightarrow 
    \ctyp{\cclrd{r}\cseq\cclrd{s}}{\alpha} \rightarrow \ctyp{\cclrd{s}}{\beta}
\end{array}
\end{equation*}
Both readings are possible, but they give quite different meanings to the operations:

\begin{itemize}
\item \emph{Coeffect interpretation.} The \ident{counit} operation extracts a value 
  and does not require any additional context; the \ident{cobind} operation requires
  context $\cclrd{r}\,\cseq\,\cclrd{s}$, uses the part specified by $\cclrd{r}$ to 
  evaluate the function, ending with a value $\beta$ together with remaining context
  $\cclrd{s}$.
\item \emph{CMTT interpretation.} The \ident{counit} operation evaluates a computation 
  that requires no additional context to obtain a $\alpha$ value; given a function that
  turns a computation requiring context $\cclrd{r}$ into a value $\beta$, the \ident{cobind}
  operation can turn a computation that requires context $\cclrd{r}\,\cseq\,\cclrd{s}$ 
  into a computation that requires just $\cclrd{s}$ and contains $\beta$ (a part of the
  context requirements is eliminated by the function).
\end{itemize}

\noindent
Although the different reading does not affect formal properties of the systems, it is important
to understand the difference when discussing the two systems. 

The sketch of a coeffect meta-language in the following section attempts to bridge the gap
between coeffects and CMTT. Just like CMTT, it embeds comonads as language constructs, but
it annotates them with a (flat) coeffect algebra, thus it generalizes CMTT which tracks only
sets of variables. Future work on the coeffect meta-language would thus be an interesting 
development for both coeffect systems and CMTT.

% -------------------------------------------------------------------------------------------------

\subsection{Coeffect meta-language}

The coeffect meta-langauge could be designed using both flat and structural indexed comonads.
For simplicity, this section only discusses the flat variant. The syntax of types and terms
of the langauge includes the type constructor $\ctyp{\cclrd{r}}{\tau}$ and four expression forms:
%
\begin{equation*}
\begin{array}{rcl}
 \tau &\narrow{::=}& \alpha \sep \tau_1 \rightarrow \tau_2 \sep \ctyp{\cclrd{r}}{\tau}\\
 e &\narrow{::=}& v \sep \lambda x.e \sep e_1~e_2 \sep !e    \\[-0.25em]
  &        \sep& \kvd{let\;box}~x=e_1~\kvd{in}~e_2           \\[-0.25em]
  &        \sep& \kvd{split}~e_1~\kvd{into}~x,y~\kvd{in}~e_2 \\[-0.25em]
  &        \sep& \kvd{merge}~e_1, e_2~\kvd{into}~x~\kvd{in}~e_2 
\end{array}
\end{equation*}
%
The $!e$ and $\kvd{let\;box}$ constructs correspond to the \ident{counit} and \ident{cobind}
operation of the comonad. To define meta-langauge for flat indexed comonads, we also include 
\kvd{split} and \kvd{merge} that embed the corresponding operations. 

\paragraph{Types for coeffect meta-language.}
The type system for the language is shown in Figure~\ref{fig:conclusions-cml}. The first part
shows the usual typing rules for simply-typed $\lambda$-calculus. For simplicity, we omit typing 
rules for pairs, but those need to be present as the \kvd{merge} operation works on tuples.

The second part of the typing rules is more interesting. The (\emph{counit}) operation extracts
a value from a comonadic context and corresponds to variable access in coeffect calculi. The
\kvd{let\;box} construct (\emph{cobind}) takes an input $e_1$ with context $\cclrd{r}\,\cseq\,\cclrd{s}$
and a computation that turns a variable $x$ with a context $\cclrd{r}$ into a value $\tau_2$.
The result is a computation that produces a $\tau_2$ value with the remaining context specified
by $\cclrd{s}$. Note that the expression $e_2$ and $e_1$ corresponds to the first and second
arguments of the \ident{cobind} operation. The keyword \kvd{let\;box} is chosen following 
CMTT\footnote{The rule is similar to the \ident{letbox} rule for ICML \cite[p. 14]{logic-cmtt},
although it differs because of our generalization of comonads where \ident{bind} composes
coeffect annotations rather than requiring the same annotation everywhere.}.

The \kvd{split} and \kvd{merge} constructs follow a similar pattern. They both apply some 
transformation on one or two values in a context and then add the new value as a fresh variable
to the variable context. For simplicity, we omit sub-coeffecting, but it could be easily added 
following the pattern used elsewhere.

% -------------------------------------------------------------------------------------------------

\begin{figure}[t]

{\small a.) Typing rules for the simply typed $\lambda$-calculus}

\begin{equation*}
\tyrule{var}
  {x : \tau \in \Gamma}
  {\Gamma \vdash x : \tau }
\end{equation*}
\begin{equation*}
\tyrule{app}
  {\Gamma \vdash e_1 : \tau_2 \rightarrow \tau_1 &
   \Gamma \vdash e_2 : \tau_2 }
  {\Gamma \vdash e_1~e_2 : \tau_1 }
\end{equation*}
\begin{equation*}
\tyrule{abs}
  {\Gamma, x:\tau_1 \vdash e : \tau_2}
  {\Gamma \vdash \lambda x.e : \tau_1 \rightarrow \tau_2 }
\end{equation*}

\vspace{1em}
{\small b.) Additional typing rules arising from \emph{flat indexed comonads}}

\begin{equation*}
\tyrule{cobind}
  {\Gamma \vdash e_1 : \ctyp{\cclrd{r}\,\cseq\,\cclrd{s}}{\tau_1} & \Gamma, x : \ctyp{\cclrd{r}}{\tau_1} \vdash e_2 : \tau_2 }
  {\Gamma \vdash \kvd{let\;box}~x=e_1~\kvd{in}~e_2 : \ctyp{\cclrd{s}}{\tau_2}}
\end{equation*}
\begin{equation*}
\tyrule{counit}
  {\Gamma \vdash e : \ctyp{\cunit}{\tau}}
  {\Gamma \vdash !e : \tau}
\end{equation*}
\begin{equation*}
\tyrule{split}
  {\Gamma \vdash e_1 : \ctyp{\cclrd{r}\,\cpar\,\cclrd{s}}{\tau_1} &
   \Gamma, x:\ctyp{\cclrd{r}}{\tau_1}, y:\ctyp{\cclrd{s}}{\tau_1} \vdash e_2 : \tau_2 }
  {\Gamma \vdash \kvd{split}~e_1~\kvd{into}~x,y~\kvd{in}~e_2 : \tau_2 }
\end{equation*}
\begin{equation*}
\tyrule{merge}
  {\Gamma \vdash e_1 : \ctyp{\cclrd{r}}{\tau_1} &
   \Gamma \vdash e_2 : \ctyp{\cclrd{s}}{\tau_2} \\
   \Gamma, x:\ctyp{\cclrd{r}\,\czip\,\cclrd{s}}{(\tau_1 \times \tau_2)} \vdash e_2 : \tau_3 }
  {\Gamma \vdash \kvd{merge}~e_1, e_2~\kvd{into}~x~\kvd{in}~e_2 : \tau_3 }
\end{equation*}

\caption{Type system for the (flat) coeffect meta-language}
\label{fig:conclusions-cml}
\end{figure}

% -------------------------------------------------------------------------------------------------

\subsection{Embedding flat coeffect calculus}

The \emph{meta-language} approach of embedding comonads in a language is more general than the
\emph{semantics} approach. This thesis focuses on a narrower use that better guides the design of
a type system for context-aware programming languages. 

However, it is worth noting that 
(flat) coeffect calculus can be embedded in the meta-language described above. This may be 
desirable, \eg~when using the meta-language for reasoning about context-aware computations. 
We briefly consider the embedding as it illuminates the relationship between coeffects and
CMTT (although it is not possible to embed coeffect calculi in CMTT because of the more general
annotations structure).

Given a typing judgement $\coctx{\Gamma}{\cclrd{r}} \vdash e : \tau$ in a flat coeffect calculus,
we define $\sem{\coctx{\Gamma}{\cclrd{r}} \vdash e : \tau}_v$ as its emebdding in the coeffect
meta-language. Note that the translation is indexed by $v$, which is a name of variable used 
to represent the entire variable context of the source language. The translation is defined in
Figure~\ref{fig:conclusions-embed}. The embedding resembles the semantics discussed in 
Section~\ref{sec:flat-semantics}. This is not surprising as the meta-language directly mirrors
the operations of a monad.

% -------------------------------------------------------------------------------------------------

\begin{figure}[t]
\begin{equation*}
\begin{array}{l}
 \sem{ \coctx{\Gamma}{\cunit}\vdash x_i : \tau_i }_v = \pi_i (!v) \\[0.25em]
 \sem{ \coctx{\Gamma}{\cclrd{r}\;\cpar\;(\cclrd{s}\,\cseq\,\cclrd{t})} \vdash e_1~e_2 : \tau_2 }_v =\\[-0.25em]
   \qquad \kvd{split}~v~\kvd{into}~v_s, v_{rt}~\kvd{in} \\[-0.25em]
   \qquad \sem{ \coctx{\Gamma}{\cclrd{s}} \vdash e_1 : \tau_1 \xrightarrow{\cclrd{t}} \tau_2 }_{v_s}
      ~(\kvd{let\;box}~v_r = v_{rt}~\kvd{in}~\sem{ \coctx{\Gamma}{\cclrd{r}} \vdash e_2 : \tau_1}_{v_r}) \\[0.25em]
 \sem{ \coctx{\Gamma}{\cclrd{r}} \vdash \lambda x.e:\tau_1 \xrightarrow{\cclrd{s}} \tau_2 }_v = \lambda x. \\[-0.25em]
   \qquad \kvd{merge}~v,x~\kvd{into}~v_{rs}~\kvd{in}~\sem{ \coctx{(\Gamma,x:\tau_1)}{\cclrd{r}\,\czip\,\cclrd{s}} \vdash e : \tau_2 }_{v_{rs}}
\end{array}
\end{equation*}
\caption{Embedding flat coeffect calculus in coeffect meta-language }
\label{fig:conclusions-embed}
\end{figure}



% =================================================================================================
%                                                                   
%      ####                         #       #                   ##   
%      #   #                        #                            #   
%      #   #  # ##    ###    ###   ####    ##     ###    ###     #   
%      ####   ##  #      #  #   #   #       #    #   #      #    #   
%      #      #       ####  #       #       #    #       ####    #   
%      #      #      #   #  #   #   #  #    #    #   #  #   #    #   
%      #      #       ####   ###     ##    ###    ###    ####   ###  
%                                                                
% =================================================================================================

\section{Towards practical coeffects}
\label{sec:unified-impl}

As discussed earlier, the main focus of this thesis is the development of the much needed 
\emph{theory of context-aware computations} and so discussing the details of a practical 
implementation of coeffect tracking is beyond the scope of the thesis. However, this section
briefly outlines one possible pathway towards this goal.

Many of the examples of contextual computation that we discussed earlier have been implemented as 
a single-purpose programming language feature (\eg~implicit parameters \cite{app-implicit-parameters} 
or distributed computations \cite{app-distributed-ml5,app-distributed-links}). However, 
the main contribution of this thesis is that it captures
\emph{multiple} different notions of context-aware computations using just a \emph{single}
common abstraction. For this reason, we advocate that future practical implementations of
coeffects should not be single-purpose language features, but instead reusable abstractions that
can be instantiated with a concrete \emph{coeffect algebra} specified by the user.

In order to do this, programming languages need to provide two features; one that allows 
embedding of context-aware computations themselves in programs akin to the ``do'' notation
in Haskell (Section~\ref{sec:unified-impl-embed}) and one that allows tracking of the 
contextual information in the type system (Section~\ref{sec:unified-impl-types}). To make
constraint resolution in type inference easier, we also discuss how the coeffect algebra 
can be simplified (Section~\ref{unified-impl-semilattice})

% -------------------------------------------------------------------------------------------------

\subsection{Embedding contextual computations}
\label{sec:unified-impl-embed}

The embedding of \emph{contextual} computations in programming languages can learn from better 
explored emedding of \emph{effectful} computations. In purely functional programming languages
such as Haskell, effectful computations are embedded by \emph{implementing} them and inserting the
necessary (monadic) plumbing. This is made easier by the ``do'' notation \cite{other-haskell98} 
that inserts the monadic operations automatically.

The recently proposed ``codo'' notation \cite{comonads-codo} provides a similar plumbing for
context-aware computations based on comonads. The notation is close to the semantics of our 
flat coeffect calculus (Chapter~\ref{ch:flat}). Extending the ``codo'' notation to support calculi based 
on the structural coeffect calculus (Chapter~\ref{ch:structural}) is an interesting futrue work -- 
this requires explicitly manipulating individual context variables and application of structural
rules, which is not needed in flat coeffects.

In ML-like languages, effects (and many coeffects) are built-in into the language semantics, 
but they can still use a special notation for explicitly marking effectful (coeffectful) 
blocks of code. In F\#, this is done using \emph{computation expressions}  \cite{app-computation-zoo}
that differ from the ``do'' notation in two ways. First, they support wider range of 
constructs, making it possible to wrap existing F\# code in a block without other changes.
Second, they support other abstractions including monads, applicative functors and monad
transformers. It would be interesting to see if computation expressions can be extended to
handle computations based on flat/structural indexed comonads.

More lightweight syntax for effectful computation can be obtained using
techniques that automatically insert the necessary monadic plumbing (wihtout a
special syntax). This has been done in the context of effecful computations \cite{monads-lightweight-ml}
and similar approach would be worth exploring for coeffects. 

% -------------------------------------------------------------------------------------------------

\subsection{Coeffect annotations as types}
\label{sec:unified-impl-types}

The other aspect of practical implementation of coeffects is tracking the context requirements
(coeffect annotations) in the type system. To achieve this (without resorting to a single-purpose
language feature) the type system needs to be able to capture various kinds of coeffect algebras. 
The structures used in this thesis include sets (with union or interesection), natural numbers (with 
addition, maximum, minimum and multiplication), two-point lattice (for liveness) and free monoids 
(vectors of annotations).

The work on embedding effect systems in Haskell \cite{effects-embedding} shows that the
recent additions to the Haskell type system provide enough poer to implement structures such
as sets at the type level. Using these to embed coeffect systems in Haskell is one fruitful 
future direction for applied coeffects.  

In dependently-typed programming languages such as Agda \cite{other-agda} or Idris \cite{other-idris},
the embedding of coeffects can be implemented more directly. However, we believe that coeffect
tracking does not require full dependent types and can be made accessible in more main-stream
languages. Dependent ML \cite{types-dependent-ml} provides an interesting example of a language
with some dependent typing which is still close to its non-dependently-typed predecessor ML.

Another approach for embedding computations into the type system has been pioneered by F\#
\emph{type providers} \cite{app-inforich}. Technically, type providers are compiler extensions 
that generate types based on external data sources or other information in order to provide easy 
access to data or services. A similar approach could be used for embedding \emph{algberas} such
as coeffect algebras into the type system. A \emph{algebra provider} would be a library that 
specifies the objects of the algebra, equational laws and generalization rules for type inference.
This could provide an easy to use way of embedding coeffect tracking in pragmatic languages
such as F\#. It is worth noting that the mechanism could also subsume F\# units of measure
\cite{types-units-of-measure}, which could be provided via one such \emph{algebra provider}.

% -------------------------------------------------------------------------------------------------

\subsection{Alternative formulation using coeffect lattice}
\label{sec:unified-impl-semilattice}

The fact that coeffect scalar structure consists of four operations ($\cseq,\czip,\cpar$ 
and a relation $\cleq$) makes working with coeffect annotations difficult -- it compilcates tasks
such as type generalization and constraint solving in type inference. Here, we look at one possible 
simplification of the structure.

\begin{itemize}
\item As mentioned in Section~\ref{sec:flat-calculus-algebra}, when the monoid $(\C,\cpar,\czero)$ is a
semi-lattice, the $\cleq$ relation can often be expressed in terms of the $\cpar$ operation 
($\cclrd{r} \;\cleq\; \cclrd{s} \;\Longleftrightarrow\; \cclrd{r} \;\cpar\; \cclrd{s} \;=\; \cclrd{s}$).
This does not give a definition equivalent to our, but it is a simplification that is consistent 
with all examples in this thesis.

\item Another approach that works for most (but not all) systems we discussed is to require the structure
$(\C,\cpar,\czip)$ to form a lattice. This is consistent with all our structural coeffect systems as
well as flat systems for liveness and data-flow, but not with the system for tracking of implicit
parameters (and other sets).
\end{itemize}

\noindent
The system for tracking implicit parameters does not fit the lattice-based model, because it uses the 
same operation $\cup$ for both $\cpar$ and $\czip$. This also means that the system for implicit parameters 
cannot be used with a system where lambda abstraction duplicates the context requirements 
(Section~\ref{sec:flat-exts-lambda}). 

\paragraph{Lattice-based coeffects.} Putting implicit parameters aside, there are many other coeffect
systems for which we could use the following definition of coeffect scalars (for simplicity, we only 
consider flat systems):

\begin{definition}
A \emph{\cclrd{flat coeffect lattice}} $(\C, \cseq, \cleq, \cunit, \czero)$ is a set 
$\C$ together with elements $\cunit, \czero \in \C$ and an ordering $\cleq$ such 
that $(\C, \cseq, \cunit)$ is a monoid and $(\C, \cleq)$ is a lattice with the least 
element $\czero$. 

Assuming $\cpar$ and $\czip$ are the \emph{least upper bound} and \emph{greatest lower bound}
of the lattice, we require the following distributivity axioms:
%
\begin{align*}
\quad (\cclrd{r}\, \cpar\, \cclrd{s}) \;\cseq\; \cclrd{t} & = (\cclrd{r} \,\cseq\, \cclrd{t}) \;\cpar\; (\cclrd{s}\, \cseq\, \cclrd{t}) \\
\quad \cclrd{t} \;\cseq\; (\cclrd{r}\, \cpar\, \cclrd{s}) & = (\cclrd{t} \,\cseq\, \cclrd{r}) \;\cpar\; (\cclrd{t}\, \cseq\, \cclrd{s})
\end{align*}
\end{definition}

\noindent
The definition provides the same structure as flat coeffect algebra, but it defines a more rigid 
structure. In particular, $\czip$ and $\cpar$ are defined in terms of $\cleq$. This simplifies the 
system as the operations of the underlying indexed comonad structure can be expressed using just
$\cseq$ and $\cleq$:
%
\begin{equation*}
\begin{array}{lcll}
 \ident{counit}_{\cunit} &\narrow{:}& 
    \ctyp{\cunit}{\alpha} \rightarrow \alpha \\
 \ident{cobind}_{\cclrd{r}, \cclrd{s}} &\narrow{:}& 
    (\ctyp{\cclrd{r}}{\alpha} \rightarrow \beta) \rightarrow (\ctyp{\cclrd{r}\cseq\cclrd{s}}{\alpha} \rightarrow \ctyp{\cclrd{s}}{\beta}) \\
\ident{merge}_{\cclrd{r},\cclrd{s},\cclrd{t}} &\narrow{:}& 
    \ctyp{\cclrd{r}}{\alpha} \times \ctyp{\cclrd{s}}{\beta} \rightarrow \ctyp{\cclrd{t}}{(\alpha \times \beta)} &
    (\cclrd{t}\,\cleq\,\cclrd{r}, \cclrd{t}\,\cleq\,\cclrd{s}) \\
\ident{split}_{\cclrd{r},\cclrd{s},\cclrd{t}} &\narrow{:}& 
    \ctyp{\cclrd{t}}{(\alpha \times \beta)} \rightarrow \ctyp{\cclrd{r}}{\alpha} \times \ctyp{\cclrd{s}}{\beta} &
    (\cclrd{r}\,\cleq\,\cclrd{t}, \cclrd{s}\,\cleq\,\cclrd{t}) \\
\ident{lift}_{\cclrd{r},\cclrd{s}} &\narrow{:}& 
    \ctyp{\cclrd{r}}{\alpha} \rightarrow \ctyp{\cclrd{s}}{\alpha} & 
    (\cclrd{s}\,\cleq\,\cclrd{r})
\end{array}
\end{equation*}
%
Note that we still need two \emph{distinct} operators in the system. The one used for
sequential composition $\cseq$ is not necessarilly related in any way to the ordering on
the lattice $\cleq$. 

However, the operations $\ident{merge}$, $\ident{split}$ and $\ident{lift}$ can now all
use ordering. For example, the \ident{merge} operation previously returned
a result with greatest lower bound $\cclrd{r}\;\czip\;\cclrd{s}$. Now, we require any
element that is smaller than $\cclrd{r}$ and $\cclrd{s}$. This adds an implicit sub-coeffecting
to the operation -- for any $\cclrd{t}$ such that $\cclrd{t}\;\cleq\;\cclrd{r}$ and 
$\cclrd{t}\;\cleq\;\cclrd{s}$, it is also the case that $\cclrd{t}\;\cleq\;(\cclrd{r}\,\czip\,\cclrd{s})$.
This means that we can compose the \ident{merge} operation which returns greatest lower
bound with $\ident{lift}$ to get the (less precise) operation required here.

% -------------------------------------------------------------------------------------------------

\begin{figure}[t]
{\small a.) Littice-based flat systems (selected rules):}
\begin{equation*}
\tyrule{abs}
  {\coctx{\Gamma, x:\tau_1}{\cclrd{t}} \vdash e : \tau_2}
  {\coctx{\Gamma}{\cclrd{r}} \vdash \lambda x.e : \tau_1 \xrightarrow{\cclrd{s}} \tau_2 }
~~(\cclrd{t}\,\cleq\,\cclrd{r}, \cclrd{t}\,\cleq\,\cclrd{s})
\end{equation*}
\begin{equation*}
\tyrule{app}
  {\coctx{\Gamma}{\cclrd{r}} \vdash e_1 : \tau_1 \xrightarrow{\cclrd{t}} \tau_2 &
   \coctx{\Gamma}{\cclrd{s}} \vdash e_2 : \tau_1 }
  {\coctx{\Gamma}{\cclrd{u}} \vdash e_1~e_2 : \tau_2}
~~(\cclrd{r}\,\cleq\,\cclrd{u}, (\cclrd{s} \,\cseq\, \cclrd{t})\,\cleq\,\cclrd{u})
\end{equation*}

\vspace{1em}
{\small b.) Littice-based structural systems (selected rules):}
\begin{equation*}
\tyrule{contr}
  {\coctx{\Gamma_1,y\!:\!\tau_1,z\!:\!\tau_1,\Gamma_2}{\aclrd{\textbf{r}} \atimes \alift{\cclrd{s},\cclrd{t}} \atimes \aclrd{\textbf{q}}} \vdash e : \tau}
  {\coctx{\Gamma_1,x\!:\!\tau_1,\Gamma_2}{\aclrd{\textbf{r}} \atimes \alift{\cclrd{u}} \atimes \aclrd{\textbf{q}}} \vdash \subst{e}{z,y}{x} : \tau}
~~(\cclrd{s}\,\cleq\,\cclrd{u}, \cclrd{t}\,\cleq\,\cclrd{u})
\end{equation*}

\caption{Flat and structural coeffects using lattice-based formulation}
\label{fig:unified-lattice-types}
\end{figure}

% -------------------------------------------------------------------------------------------------

\paragraph{Lattice-based type system.}

Using lattice-based coeffect structure leads to the typing rules shown in Figure~\ref{fig:unified-lattice-types}.
In the flat coeffect system, this affects both application and abstraction -- in abstraction,
the context-requirements of the body should be smaller than the context available on the 
call-site and declaration-site. In application, we require more than what is required by
the sub-expressions.

Structural coeffect systems only require least upper bound $\cpar$ in the (\emph{contr}) rule.
This can be rewritten similarly to flat application --  when contracting two variables, the
context requirements of the resulting variable are greater than the requirements associated with
each of the original variables.

In smmary, the lattice-based coeffect system has some advantages over the design used in this
thesis. In particular, it uses better-known structures and is simpler. However, we do not use
it as it rules out one of the important motivating examples (implicit parameters) and, even with
the simplification, solving constraints with $\cseq$ and $\cleq$ is still an open question.


% ==================================================================================================
%                                                                             
%        ###                         ##                    #                        
%       #   #                         #                                             
%       #       ###   # ##    ###     #    #   #   ###    ##     ###   # ##    ###  
%       #      #   #  ##  #  #   #    #    #   #  #        #    #   #  ##  #  #     
%       #      #   #  #   #  #        #    #   #   ###     #    #   #  #   #   ###  
%       #   #  #   #  #   #  #   #    #    #  ##      #    #    #   #  #   #      # 
%        ###    ###   #   #   ###    ###    ## #  ####    ###    ###   #   #  ####  
%                                                                             
% ==================================================================================================

\section{Conclusions}

In this chapter, we completed our goal of presenting a \emph{unified} system for tracking contextual
properties and then we discussed two important topics of future and related work that require 
in-depth discussion.

The key technical contribution of this chapter is the \emph{unified coeffect system} 
(Section~\ref{sec:unified-unified}), which unifies the flat and structural system presented
in the previous two chapters. To achieve this, we introduced \emph{coeffect container}, which
determines how are coeffect annotations attached to variable contexts. We then discussed two
instances of the structure that capture flat and structural properties.

In the rest of the chapter, we discussed two topis of future and related work. First
(Section~\ref{sec:unified-impl}), we considered pathways to practical implementations of
coeffect systems, including a discussion of a novel lattice-based coeffect system, which is simpler,
but cannot capture all our motivating examples. Finally, we discussed how our work relates to 
meta-language based on comonads (Section~\ref{sec:unified-meta}). We present a \emph{coeffect 
meta-language} that follows similar style to CMTT, but is based on indexed comonads. Next, we 
discussed how to embed flat coeffect systems in this coeffect meta-language, which elucidates the 
relationship between our work and CMTT. 
%%!TEX root = ../main.tex

\chapter{Conclusions} 
\label{ch:conclusions} 

%---------------------------------------------------------------------------------------------------

How to make this practical?

codo notation / computation expressions

embedding algebraic things


%---------------------------------------------------------------------------------------------------

Both flat coeffect calculus and structural coeffect calculus (presented in the past two chapters)
use indexed comonads to define the semantics of the langauge. In this section, we follow the 
meta-language style and embed indexed comonads into the language -- the type constructor
$\ctyp{r}{\alpha}$ becomes a first-class value and we add language constructs corresponding to
primitive operations of the indexed comonad.

%===================================================================================================

\newcommand{\munit}{\mathsf{e}}
\newcommand{\mseq}[0]{ \oplus }


\section{Introduction}

\section{Meta-language}
\label{sec:conclusions-meta} 

\begin{figure}
\begin{equation*}
\tyrule{var}
  {x : \alpha \in \Gamma}
  {\Gamma \vdash x : \alpha }
\end{equation*}
\begin{equation*}
\tyrule{app}
  {\Gamma \vdash e_1 : \alpha \rightarrow \beta &
   \Gamma \vdash e_2 : \alpha }
  {\Gamma \vdash e_1~e_2 : \beta}
\end{equation*}
\begin{equation*}
\tyrule{abs}
  {\Gamma, x:\alpha \vdash e : \beta}
  {\Gamma \vdash \lambda x.e : \alpha \rightarrow \beta }
\end{equation*}
\begin{equation*}
\tyrule{letbox}
  {\Gamma \vdash e_1 : \ctyp{r \mseq s}{\alpha} & \Gamma, x : \ctyp{r}{\alpha} \vdash e_2 : \beta }
  {\Gamma \vdash \kvd{let~box}~x=e_1~\kvd{in}~e_2 : \ctyp{s}{\beta}}
\end{equation*}
\begin{equation*}
\tyrule{eval}
  {\Gamma \vdash e : \ctyp{\munit}{\alpha}}
  {\Gamma \vdash !e : \alpha}
\end{equation*}
\begin{equation*}
\tyrule{sub}
  {\Gamma \vdash e : \ctyp{s}{\alpha} }
  {\Gamma \vdash e : \ctyp{r}{\alpha} }
\;(s\leq r)  
\end{equation*}

\caption{Type system for the coeffect meta-language cml}
\label{fig:cml-types}
\end{figure}

% ==================================================================================================

\section{Related work}
This chapter is closely related to Contextual Modal Type Theory (CMTT) of Nanevski et al. However
they develop their language using model logic as a basis, while we use categorical foundations 
as the basis - leading to a different system.

\section{Also related work}

This paper follows the approaches of effect systems \cite{effects-gifford,effects-talpin-et-al,monads-effects-marriage}
and categorical semantics based on monads and comonads \cite{monad-notions,comonads-notions}. Syntactically,
\emph{coeffects} differ from \emph{effects} in that they model systems where $\lambda$-abstraction 
may split contextual requirements between the declaration-site and call-site.

Our \emph{indexed (monoidal) comonads} (\S\ref{sec:comonads}) fill the gap between (non-indexed)
\emph{(monoidal) comonads} of Uustalu and Vene \cite{comonads-notions}
and indexed monads of Atkey~\cite{monads-parameterised-notions}, Wadler and Thiemann
\cite{monads-effects-marriage}. Interestingly, \emph{indexed} comonads are \emph{more
general} than comonads, capturing more notions of context-dependence (\S\ref{sec:motivation}).

% --------------------------------------------------------------------------------------------------

\vspace{-1em}
\paragraph{Comonads and modal logics.}

Bierman and de Paiva \cite{logic-intuitionistic-modal} model the
$\square$ modality of an intuitionistic S4 modal logic using monoidal
comonads, which links our calculus to modal logics.  This link can be
materialized in two ways.

Pfenning et al. and Nanevski et al.  derive term languages using the Curry-Howard
correspondence~\cite{logic-modal-reconstruction,logic-intuitionistic-modal,logic-cmtt},
building a \emph{metalanguage} (akin to Moggi's monadic metalanguage
\cite{monad-notions}) that includes $\square$ as a type
constructor. For example, in \cite{logic-modal-reconstruction}, the
modal type $\Box \tau$ represents closed terms.
In contrast, the \emph{semantic} approach uses monads or comonads
\emph{only} as a semantics.  This has been employed by Uustalu and
Vene and (again) Moggi \cite{monad-notions,comonads-notions}.  We
follow the semantic approach.

Nanevski et al. extend an S4 term language to a \emph{contextual}
modal type theory (CMTT)~\cite{logic-cmtt}.
The \emph{context} is a set of variables required by a computation, which
makes CMTT useful for meta-programming and staged computations. Our contextual types are
indexed by a coeffect algebra, which is more general and can capture
variable contexts, but also integers, two-point lattices, \emph{etc.}.

The work on CMTT suggests two extensions to coeffects. The first is
developing the logical foundations. We briefly considered special cases
of our system that permits local soundness in \S\ref{sec:syntactic} and
local completeness can be treated similarly. The second problem is 
developing the coeffects \emph{metalanguage}. The use of coeffect algebras
would provide an additional flexibility over CMTT, allowing a wider range 
of applications.

\section{Summary}
%% ==================================================================================================

\chapter{Implementation}
\label{ch:impl}

In the previous three chapters, we presented the theory of coeffects consisting of type system
and comonadically inspired semantics of two parameterized coeffect calculi. The theory provides
a framework that simplifies the implementation of safe context-aware programming languages. To
support this claim, this chapter presents a prototype implementation of three coeffect languages --
language with implicit parameters and both flat and structural versions of a data-flow language.

The implementation directly follows the thoery presented in the previous three chapters. It
consists of a common framework that provides type checking and translation to a simple functional
target language with comonadically-inspired primitives. Each concrete context-aware language then
adds a domain-specific rule for choosing unique typing derivation (as discussed in
Section~\ref{sec:flat-unique}) together with a domain-specific definition of the
comonadically-inspired primitives that define the runtime semantics (see Section~\ref{sec:semantics-proofs}).

The main goal of the implementation is to show that the theory is practically useful and to present it
in a more practical way. However, we do not intend to build a complete real-world programming language.
For this reason, the implementation is available primarilly as an interactive web-based essay, though
it can be also downloaded and run locally.

\paragraph{Chapter structure and contributions}

\begin{itemize}
\item We discuss how the implementation follows from the theory presented earlier (Section~\ref{sec:impl-theory}).
  This applies to the implementation of the \emph{type checker} and the implementation of the \emph{translation}
  to a simple target langauge that is then iterpereted. We also discuss how the common framework makes it
  easy to implement additional context-aware languages (Section~\ref{sec:impl-theory-ext}).

\item We consider a number of case studies (Section~\ref{sec:impl-case}) that illustrate ointeresting
  aspects of the theories discussed earlier. This includes the typing of lambda abstraction and the
  difference between flat and structural systems (Section~\ref{sec:impl-case-typing}) and the
  comonadically-inspired translation (Section~\ref{sec:impl-case-transl}).

\item We briefly outline some of the interesting technical details of the implementation
  (Section~\ref{sec:impl-tech}). To make the implementation easily accessible, we make it
  available via web and we also implement a limited support for type inference.

\item The implementation is available not just as downloadable code, but also in the format of interactive essay
  (Section~\ref{sec:impl-essay}), which aims to make coeffects accessible to a broader audience.
  We conclude this chapter by discussing the most interesting aspects of the presentation.
\end{itemize}


% ==================================================================================================
%
%    #######
%       #    #    # ######  ####  #####  #   #
%       #    #    # #      #    # #    #  # #
%       #    ###### #####  #    # #    #   #
%       #    #    # #      #    # #####    #
%       #    #    # #      #    # #   #    #
%       #    #    # ######  ####  #    #   #
%
% ==================================================================================================

\section{From theory to implementation}
\label{sec:impl-theory}

The theory discussed so far provides the two key components of the implementation. In
Chapter~\ref{ch:flat}, we discussed the type checking of context-aware programs and
Chapter~\ref{ch:semantics} models the execution of context-aware programs (in terms of
translation and operational semantics). For structural coeffects, the same components are discussed
in Chapter~\ref{ch:structural}. In this section, we discuss how those provide foundation for the
implementation.

\subsection{Type checking}
\label{sec:impl-theory-typing}


(Section~\ref{sec:flat-unique})

\paragraph{Implementation.}
From the presentation in this section, it might appear that resolving the ambiguity related to
lambda abstraction for implicit parameters requires a type system that is quite different from
the core flat coeffect type system shown earlier in Figure~\ref{fig:flat-types}. This is not the
case. As discussed in Chapter~\ref{ch:impl}, the required changes in the implementation are
simpler.

Briefly, the implementation collects constraints on the coeffects and then finds the smallest sets
of implicit parameters to satisfy the constraints. We still need to track implicit parameters in
scope $\cclrd{\Delta}$, but the rest of the (\emph{abs}) rule from the implementation is close
to the one from Figure~\ref{fig:flat-types}:
%
\begin{equation*}
\tyrule{abs}
  {\coctx{\Gamma, x\!:\!\tau_1;\cclrd{\Delta}}{\cclrd{t}} \vdash e : \tau_2 ~|~ C}
  {\coctx{\Gamma;\cclrd{\Delta}}{\cclrd{r}} \vdash \lambda x\!:\!\tau_1.e : \tau_1 \xrightarrow{\cclrd{s}} \tau_2 ~|~
    C\cup\{\cclrd{t}=\cclrd{r} \,\czip\, \cclrd{s}, \cclrd{r}=\cclrd{\Delta} \}}
\end{equation*}
%
Given a typing derivation for the body that produced constraints $C$, we generate an additional
constraint that restricts $\cclrd{r}$ (declaration-site demands) to those available in the
current static scope $\cclrd{\Delta}$. The constraint satisfaction algorithm then finds
the minimal set $\cclrd{s}$ which is $\cclrd{t}\cclrd{\setminus \Delta}$.

\paragraph{Implementation.}
As with implicit parameters, the implementation (discussed in Chapter~\ref{ch:impl}) does not
require changing the typing (\emph{abs}) rule of the flat coeffect system. The typing specified
by (\emph{idabs}) can be easily obtained by generating additional constraints:
%
\begin{equation*}
\tyrule{abs}
  {\coctx{\Gamma, x\!:\!\tau_1}{\cclrd{t}} \vdash e : \tau_2 ~|~ C}
  {\coctx{\Gamma}{\cclrd{s}} \vdash \lambda x\!:\!\tau_1.e : \tau_1 \xrightarrow{\cclrd{t}} \tau_2 ~|~
    C\cup\{\cclrd{t}=\cclrd{r} \,\czip\, \cclrd{s}, \cclrd{r}=\cclrd{t}, \cclrd{s}=\cclrd{t} \}}
\end{equation*}
%
Here, the two additional constraints restrict both $\cclrd{r}$ and $\cclrd{s}$ to be equal to the
coeffect of the body $\cclrd{t}$ and so the only possible resolution is the one specified by
(\emph{idabs}).


Type checking
 * DSL specific constraint resolution
 * easier for structural


\subsection{Translation}
\label{sec:impl-theory-transl}

Translation ~ Semantics

\subsection{Extensions}
\label{sec:impl-theory-ext}

It is really easy to add another DSL

\section{Case studies}
\label{sec:impl-case}

\subsection{Typing}
\label{sec:impl-case-typing}

let both =
  let ?fst = 100 in
  fun trd -> ?fst + ?snd + trd in
let ?fst = 200 in
both 1

num -{ ?snd:num }-> num


let struct x y = x + prev y in
struct

num -{ 1 }-> num -{ 1 }-> num

let flat x y = x + prev y in
flat

num -{ 0 }-> num -{ 1 }-> num

\subsection{Translation}
\label{sec:impl-case-transl}

let ?param = 10 in
fun x -> ?param + ?other

let (ctx2, ctx3) = split (duplicate finput) in
let ctx1 = letimpl ?param (ctx2, 10) in
fun x ->
  let ctx4 = merge (x, ctx1) in
  let (ctx5, ctx6) = split (duplicate ctx4) in
  lookup ?param ctx5 + lookup ?other ctx6


fun x -> (fun v -> prev v) (prev x)

fun x ->
  let ctx1 = merge (x, sinput) in
  (fun v ->
    let ctx3 = merge (v, choose_0 ctx1) in
    counit (prev ctx3)) (cobind (fun ctx2 ->
      counit (prev ctx2)) (choose_1 ctx1))

\section{Technical details}
\label{sec:impl-tech}

1. F# + JavaScript
2. Single AST and parsing
3. Type checking and constraint generation
4. Constraint resolution

\section{Interactive essay}
\label{sec:impl-essay}

%\chapter{Conclusions and further work} 
\label{ch:conclusions}

Some of the most fundamental academic work is not the one solving hard research problems, but the
one that changes how we understand the world. Some philosophers argue that \emph{language} is the
key for understanding how we think, while in science the dominant thinking is determined by
\emph{paradigms} \cite{philosophy-kuhn} or \emph{research programmes} \cite{philosophy-lakatos}.
In a way, programming languages play a similar role for computer science and software development.

This thesis aims to change how developers and programming language designers think about
\emph{context} or \emph{execution environments} of programs. Such context or execution environment
has numerous forms -- mobile applications access network, GPS location or user's personal data;
journalists obtain information published on the web or through open government data initiatives;
in data-flow programming, we have access to past values of expressions.
This thesis aims to change our understanding of \emph{context} so that the above examples are
viewed uniformly through a single programming language abstraction, which we call \emph{coeffects},
rather than as disjoint cases.

In this chapter, we give a brief summary of the technical development completed in this thesis
(Section~\ref{sec:conc-summary}). The thesis looks at two kinds of context-dependence, identifies
common patterns and captures those using three \emph{coeffect calculi}. Next, we give an overview
of further work (Section~\ref{sec:conc-further}), often referring to longer discussion in earlier
chapters. Finally, Section~\ref{sec:conc-conclusions} concludes the thesis.


% ==================================================================================================
%
%       ###
%      #   #
%      #      #   #  ## #   ## #    ###   # ##   #   #
%       ###   #   #  # # #  # # #      #  ##  #  #   #
%          #  #   #  # # #  # # #   ####  #      #  ##
%      #   #  #  ##  # # #  # # #  #   #  #       ## #
%       ###    ## #  #   #  #   #   ####  #          #
%                                                #   #
%                                                 ###
% =================================================================================================

\section{Summary}
\label{sec:conc-summary}

Modern computer programs run in rich and diverse environments or \emph{contexts}. The richness means
that environments provides additional resources and capabilities that may be accessed by the program.
The diversity means that programs often need to run in multiple different environments, such as mobile
phones, servers, web browsers or even on the GPU. In this thesis, we present the foundations for
building programming languages that simplify writing software for such rich and diverse environments.

% --------------------------------------------------------------------------------------------------

\paragraph{Notions of context.}

In $\lambda$-calculus, the term \emph{context} is usually refers to the free-variable context.
However, other programming language features are also related to context or program's
execution environment. In Chapter~\ref{ch:applications}, we revisit many of such features --
including resource tracking in distributed computing, cross-platform development, data-flow
programming and liveness analysis, but also Haskell's implicit parameters.

The main contribution of the chapter is that it presents the disjoint language features in a
unified way. We show type systems and semantics for many of the languages, illuminating
the fact that they are closely related.

Considering applications is one way of approaching the theory of coeffects introduced in this thesis.
Other pathways to coeffects are discussed in Chapter~\ref{ch:pathways}, which looks at theoretical
developments leading to coeffects, including the work on effect systems, comonadic semantics and
linear logics.

% --------------------------------------------------------------------------------------------------

\paragraph{Flat coeffect calculus.}

The applications discussed in Chapter~\ref{ch:applications} fall into two categories. In the first
category (Section~\ref{sec:applications-flat}), the additional contextual information are
\emph{whole-context} properties. They either capture properties of the execution environment or
affect the whole free-variable context.

In Chapter~\ref{ch:flat}, we develop a \emph{flat coeffect calculus} which gives us an unified way
of tracking \emph{whole-context} properties. The calculus is parameterized by a \emph{flat coeffect
algebra} that captures the algebraic properties of contextual information. Concrete
instances of flat coeffects include Haskell's implicit parameters, whole-context liveness and
whole-context data-flow.

Our focus is on the syntactic properties of the calculus -- we give a type system
(Section~\ref{sec:flat-calculus}) and discuss equational theory of the calculus (Section~\ref{sec:flat-syntax}).
In the flat coeffect calculus, $\beta$-reduction and $\eta$-expansion do not generally preserve
the type of an expression, but we identify two conditions when this is the case -- this gives us
a basis for operational semantics of our calculi for liveness and implicit parameters.

The design of the type system is further validated by categorical semantics
(Section~\ref{sec:flat-semantics}) which models flat coeffect calculus in terms of
\emph{flat indexed comonad} -- a structure based on a categorical dual of monads.

% --------------------------------------------------------------------------------------------------

\paragraph{Structural coeffect calculus.}

~ The second category of context-aware systems discussed in Section~\ref{sec:applications-structural}
captures \emph{per-variable} contextual properties. The systems discussed here resemble substructural
logics, but rather than \emph{restricting} variable use, they \emph{track} how the variables are used.

We unify systems with \emph{per-variable} contextual properties in Chapter~\ref{ch:structural},
which describes our \emph{structural coeffect calculus} (Section~\ref{sec:struct-calculus}).
Similarly to the flat variant, the calculus is parameterized by a \emph{structural coeffect algebra}.
Concrete instances of the calculus track bounded variable use (\ie~how many times is a variable
accessed), data-flow properties (how many past values are needed) and liveness (\ie~can variable
be accessed).

The structural coeffect calculus has desirable equational properties (Section~\ref{sec:struct-syntax}).
In particular, type preservation for $\beta$-reduction and $\eta$-expansion holds for all instances of the
structural coeffect calculus. This follows from the fact that structural coeffect
associates contextual requirements with individual variables and preserves the connection by
including explicit structural rules (weakening, exchange and contraction).

% --------------------------------------------------------------------------------------------------

\paragraph{Unified calculus.}

The Chapters~\ref{ch:flat} and \ref{ch:structural} present the main novel technical contributions of this
thesis. In Chapter~\ref{ch:unified}, we discuss the similarities and differences between the two.
Although the calculi are worth considering separately (as they have different equational properties),
they can be presented in a uniform way. In Section~\ref{sec:unified-unified}, we present the
\emph{unified coeffect calculus}, which can be instantiate to track both per-variable and
whole-context properties. The key trick is to parameterize the calculus by the \emph{shape} of
a context, which can be a vector (corresponding to the vector of free variables) or a singleton
container (attaching one annotation to the whole context).

% =================================================================================================
%
% 	 #####                 #     #                                                #
% 	 #                     #     #                                                #
% 	 #      #   #  # ##   ####   # ##    ###   # ##          #   #   ###   # ##   #   #
% 	 ####   #   #  ##  #   #     ##  #  #   #  ##  #         #   #  #   #  ##  #  #  #
% 	 #      #   #  #       #     #   #  #####  #             # # #  #   #  #      ###
% 	 #      #  ##  #       #  #  #   #  #      #             # # #  #   #  #      #  #
% 	 #       ## #  #        ##   #   #   ###   #              # #    ###   #      #   #
%
% =================================================================================================

\section{Further work}
\label{sec:conc-further}

Most of the further work has been discussed throughout the thesis, so this section
serves mainly as an overview. We follow the same structure as when discussing pathways to
coeffects (Chapter~\ref{ch:pathways}). We consider further work related to syntactic approach
to coeffect systems, categorical semantics of coeffects, and connections with substructural type
systems and bunched logics.

% -------------------------------------------------------------------------------------------------

\paragraph{Syntactic coeffect systems.}
The coeffect calculi presented in this thesis are based on a simple $\lambda$-calculus, comprising
variables, lambda abstraction, application and let-binding. This lets us focus on fundamental
properties of coeffect systems (and explain how coeffect system differ from better-known effect
systems), but it is hardly sufficient for a realistic programming language.

Further work is to extend the coeffect systems presented here to a full programming language. A
useful reference is the work of Nielson and Nielson~\cite{effects-nielson} who consider a
similar development for effect systems, adding conditionals, recursion and polymorphism.

Coeffect annotations in context-aware programming languages can contain significant amount
of information. Thus, the programming language also needs a form of type inference that can propagate
such information. This has been done for effect systems \cite{effects-polymorphic}. For
\emph{flat coeffects}, type inference is complicated by the non-determinism in the typing rule
for lambda abstraction, but the problem is simpler for \emph{structural coeffects}. In addition
to the \emph{algebra providers} (discussed in Section~\ref{sec:unified-impl-types}), it would also
be worth developing a bidirectional type system \cite{types-bidirectional} for coeffects, where
\emph{some} annotations are required, but most types and coeffects are reconstructed automatically.

% -------------------------------------------------------------------------------------------------

\paragraph{Language semantics.}

As discussed in Section~\ref{sec:path-sem}, comonads can be used to define the semantics of a
programming language in two ways. The first, ``language semantics'', approach is to use a
single comonad to describe the semantics of a specific context-aware property. The second,
``meta-language'' approach is to extend the language with explicit constructs representing the
operations of the underlying comonad.

The categorical semantics in this thesis used the ``language semantics'' style, but mainly as a
guide for the development of the type systems. Further work includes more precise treatment of
the categorical structure of indexed comonads. This has partly been done for flat coeffects by
Orchard~\cite{comonads-dom-thesis}. A joint work with Orchard~\cite{coeffects-icfp14} shows the
first steps for structural systems. As discussed in Section~\ref{sec:unified-semantics}, further
work is to develop a similar categorical model for the unified coeffect calculus, possibly using
the categorical notion of containers \cite{types-containers}.

We briefly outlined a calculus based on the ``meta-language'' approach in Section~\ref{sec:unified-meta}.
Developing this technique further could unify coeffect systems with the work on Contextual Modal
Type-Theory (CMTT) \cite{logic-cmtt} and allow other interesting applications of coeffects such
as meta-programming \cite{logic-cmtt} and distributed computing with explicit modalities \cite{app-distributed-ml5}.

% -------------------------------------------------------------------------------------------------

\paragraph{Substructural and bunched logics.}

In the \emph{flat} and \emph{structural} coeffect calculi, we attach annotations to the \emph{entire
context} and to \emph{individual variables}, respectively. We later unified the two systems
in Section~\ref{sec:unified-unified}.

An intriguing question is whether coeffects can generalize \emph{bunched typing} \cite{substruct-bunched},
which uses a tree-like structure of variable contexts (also discussed in Section~\ref{sec:path-logic}).
The definition of the \emph{unified coeffect calculus} is likely not expressive enough -- bunched
typing requires tree-like variable context, while we use a vector. Furthermore, bunched typing
annotates internal nodes of the tree (sub-contexts are combined using ``;'' or using ``,'') rather
than just variables. Finding a simple coeffect system that is capable of capturing bunched typing
is thus an interesting further work. Indeed, this could lead to numerous uses of coeffects as
bunched typing is the basis for widely-used separation logic \cite{substruct-separation-logic}.

% -------------------------------------------------------------------------------------------------

\paragraph{Applications.}
The last direction for further development is developing practical programming languages or
libraries based on the theory presented in this thesis. We believe that the most useful approach
is to extend programming languages with a reusable coeffect system and allow library developers
define their own coeffect algebras. This option has been discussed in Section~\ref{sec:unified-impl}.
For languages like Haskell, this can be done through a light-weight syntactic sugar akin to the
``do'' notation.

Another option is to use coeffects just as the underlying theory of a single-purpose programming
language feature. When discussing why context-aware programming matters (Section~\ref{sec:intro-whymatters}),
we covered a number of concrete problems that developers are facing today. Each of those would
presents an important further work. The most important practical challenges are cross-compilation
(often solved using complex $\prepk{\#if}$ pre-processor rules) and tracking of provenance
and security properties.



% =================================================================================================
%
% 	  ###                         ##                    #
% 	 #   #                         #
% 	 #       ###   # ##    ###     #    #   #   ###    ##     ###   # ##    ###
% 	 #      #   #  ##  #  #   #    #    #   #  #        #    #   #  ##  #  #
% 	 #      #   #  #   #  #        #    #   #   ###     #    #   #  #   #   ###
% 	 #   #  #   #  #   #  #   #    #    #  ##      #    #    #   #  #   #      #
% 	  ###    ###   #   #   ###    ###    ## #  ####    ###    ###   #   #  ####
%
% =================================================================================================

\section{Summary}
\label{sec:conc-conclusions}

We believe that understanding what programs \emph{require} from the world is equally important as
how programs \emph{affect} the world.

The latter has been uniformly captured by effect systems and monads. Those provide not just
technical tools for defining semantics and designing type systems, but they also \emph{shape}
our thinking -- they let us view seemingly unrelated programming language features (exceptions,
state, I/O) as instances of the same concept and thus reduce the number of distinct language
features that developers need to understand.

This thesis aims to provide a similar unifying theory and tools for capturing context-dependence
in programming languages. We showed that programming language features (liveness, data-flow,
implicit parameters, etc.) that were previously treated separately can be captured by a common
framework developed in this thesis. The main technical contribution of this thesis is that it
provides the necessary tools for programming language designers -- including parameterized type
systems, categorical semantics based on indexed comonads and equational theory.

If there is a one thing that the reader should remember from this thesis, it is the fact that
there is a unified notion of \emph{context}, capturing many common scenarios in programming.


%----------------------------------------------------------------------------------------
%	THESIS CONTENT - APPENDICES
%----------------------------------------------------------------------------------------


% \part{Appendix} % New part of the thesis for the appendix

% \include{Chapters/Chapter0B} % Appendix B - empty template

%----------------------------------------------------------------------------------------
%	POST-CONTENT THESIS PAGES
%----------------------------------------------------------------------------------------

\cleardoublepage\include{Bibliography} % Bibliography
% \cleardoublepage\include{FrontBackMatter/Colophon} % Colophon
% \cleardoublepage% Declaration

\refstepcounter{dummy}
\pdfbookmark[0]{Declaration}{declaration} % Bookmark name visible in a PDF viewer

\chapter*{Declaration} % Declaration section text

\thispagestyle{empty}

This dissertation is my own work and includes nothing which is the outcome of work done in collaboration
with others except where specifically indicated in the text.
This dissertation does not exceed the regulation length of 60,000 words, including tables and footnotes. % Declaration

%----------------------------------------------------------------------------------------

%\appendix
%%!TEX root = ../main.tex

\chapter{Appendix A} 
\label{ch:appendix} 

%---------------------------------------------------------------------------------------------------

\section{Internalized substitution}

\subsection{First transformation}

\begin{equation*}
(\kvd{glet}~x=e_1~\kvd{in}~e_2)~e_3 \leadsto \kvd{glet}~x=e_1~\kvd{in}~(e_2~e_3)
\end{equation*}

\begin{equation*}
\tyrule{app}
  { \tyrule{glet}
      { \coctx{\Gamma}{\cclrd{s}} \vdash e_1 : \tau_1 &
        \coctx{\Gamma, x\!:\!\tau_1}{\cclrd{r}} \vdash e_2 : \tau_3 \xrightarrow{\cclrd{t}} \tau_2 }
      { \coctx{\Gamma}{\cclrd{r}\;\cpar\;({\cclrd{s}\,\cseq\,\cclrd{r})}} \vdash \kvd{glet}~x=e_1~\kvd{in}~e_2 : \tau_3 \xrightarrow{\cclrd{t}} \tau_2 } &
    \coctx{\Gamma}{\cclrd{u}} \vdash e_3 : \tau_3  }
  { \coctx{\Gamma}{(\cclrd{r}\;\cpar\;(\cclrd{s}\,\cseq\,\cclrd{r})) \;\cpar\; (\cclrd{u}\,\cseq\,\cclrd{t}) } \vdash (\kvd{glet}~x=e_1~\kvd{in}~e_2)~e_3 : \tau_2 }
\end{equation*}

to

\begin{equation*}
\tyrule{glet}
  { \coctx{\Gamma}{\cclrd{s}} \vdash e_1 : \tau_1 &
    \tyrule{app}
      { \coctx{\Gamma, x\!:\!\tau_1}{\cclrd{r}} \vdash e_2 : \tau_3 \xrightarrow{\cclrd{t}} \tau_2 & 
        \coctx{\Gamma}{\cclrd{u}} \vdash e_3 : \tau_3 }
      { \coctx{\Gamma}{\cclrd{r}\;\cpar\;(\cclrd{u}\,\cseq\,\cclrd{t})} \vdash e_2~e_3 : \tau_2 } }
  { \coctx{\Gamma}{ (\cclrd{r}\;\cpar\;(\cclrd{u}\,\cseq\,\cclrd{t})) \;\cpar\; (\cclrd{s}\,\cseq\,(\cclrd{r}\;\cpar\;(\cclrd{u}\,\cseq\,\cclrd{t})))  }
      \vdash \kvd{glet}~x=e_1~\kvd{in}~(e_2~e_3) : \tau_2 }
\end{equation*}

meaning
\begin{equation*}
\begin{array}{l}
(\cclrd{r}\;\cpar\;(\cclrd{s}\,\cseq\,\cclrd{r})) \;\cpar\; (\cclrd{u}\,\cseq\,\cclrd{t}) =
\end{array}
\end{equation*}

\subsection{Second transformation}


Second transformation

\begin{equation*}
\tyrule{glet}
  { \coctx{\Gamma}{\cclrd{s}} \vdash e_s : \tau_s &
    \coctx{\Gamma, x\!:\!\tau_1}{\cclrd{r}} \vdash e_r : \tau_r &
    \coctx{\Gamma, x\!:\!\tau_1}{\cclrd{t}} \vdash e_t : \tau_t }
  { \coctx{\Gamma}{
      \cclrd{t}
      \;\cpar\;
      (  (\cclrd{r}\;\cpar\;(\cclrd{s}\,\cseq\,\cclrd{r}))
         \,\cseq\, 
         \cclrd{t} )
    } 
    \vdash \kvd{glet}~x_r=(\kvd{glet}~x_s=e_s~\kvd{in}~e_r)~\kvd{in}~e_t : \tau_t }
\end{equation*}

or

\begin{equation*}
\tyrule{glet}
  { \coctx{\Gamma}{\cclrd{s}} \vdash e_s : \tau_s &
    \coctx{\Gamma, x\!:\!\tau_1}{\cclrd{r}} \vdash e_r : \tau_r &
    \coctx{\Gamma, x\!:\!\tau_1}{\cclrd{t}} \vdash e_t : \tau_t }
  { \coctx{\Gamma}{
      (\cclrd{t}\;\cpar\;(\cclrd{r}\,\cseq\,\cclrd{t})) 
      \;\cpar\;
      (  \cclrd{s}
         \,\cseq\, 
         (\cclrd{t}\;\cpar\;(\cclrd{r}\,\cseq\,\cclrd{t})) )
    } 
    \vdash \kvd{glet}~x_s=e_s~\kvd{in}~(\kvd{glet}~x_r=e_r~\kvd{in}~e_t) : \tau_t }
\end{equation*}


\begin{equation*}
\begin{array}{l}
  \cclrd{t}
  \;\cpar\;
  (  (\cclrd{r}\;\cpar\;(\cclrd{s}\,\cseq\,\cclrd{r}))
     \,\cseq\, 
     \cclrd{t} ) = 
\\
  \cclrd{t}
  \;\cpar\;
  (\cclrd{r}\,\cseq\, \cclrd{t})
  \;\cpar\;
  (\cclrd{s}\,\cseq\,\cclrd{r}\,\cseq\, \cclrd{t}) = 
\\
  \cclrd{s}\,\cseq\,\cclrd{r}\,\cseq\, \cclrd{t} 
\end{array}
\end{equation*}

\begin{equation*}
\begin{array}{l}
  (\cclrd{t}\;\cpar\;(\cclrd{r}\,\cseq\,\cclrd{t})) 
  \;\cpar\;
  (  \cclrd{s}
     \,\cseq\, 
     (\cclrd{t}\;\cpar\;(\cclrd{r}\,\cseq\,\cclrd{t})) ) = 
\\     
  \cclrd{t}\;\cpar\;(\cclrd{r}\,\cseq\,\cclrd{t})
  \;\cpar\;
  (\cclrd{s} \,\cseq\, \cclrd{t})
  \;\cpar\;
  (\cclrd{s} \,\cseq\, \cclrd{r}\,\cseq\,\cclrd{t})  = 
\\     
  \cclrd{s} \,\cseq\, \cclrd{r}\,\cseq\,\cclrd{t}
\end{array}
\end{equation*}

require

\begin{equation*}
\cclrd{r}\;\cpar\;(\cclrd{r} \,\cseq\, \cclrd{s}) = \cclrd{r} \,\cseq\, \cclrd{s}
\end{equation*}

%%!TEX root = ../main.tex

\chapter{Appendix B} 
\label{ch:appendix} 

This appendinx provides additional details for some of the proofs for equational theory
of flat coeffect calculus from Chapter~\ref{ch:flat} and structural coeffect calculus
from Chapter~\ref{ch:structural}.



% =================================================================================================
%
%      #####   ##            #    
%      #        #            #    
%      #        #     ###   ####  
%      ####     #        #   #    
%      #        #     ####   #    
%      #        #    #   #   #  # 
%      #       ###    ####    ##  
%                               
% =================================================================================================

\section{Substitution for flat coeffects}
\label{sec:appendix-flat-cbn}
In Section~\ref{sec:flat-syntax-cbn}, we stated that, for a bottom-pointed flat coeffect
algebra (\ie~$\forall r \in \C \;.\; r \,\cgeq\, \cunit $), the call-by-name substitution 
preserves type if all operators of the flat coeffect algebra coincide (Lemma~\ref{thm:cbn-substitution-bot}).
This section provides the corresponding proof.

\begin{lemma*}[Bottom-pointed substitution]
In a bottom-pointed flat coeffect calculus with an algebra $(\C, \cseq, \cpar, \czip, \cunit, \czero, \cleq)$ 
where $\czip = \cseq = \cpar$ and the operation is also idempotent and commutative and
$\cclrd{r}\,\cleq\,\cclrd{r'} \Rightarrow \forall\cclrd{s}.\cclrd{r}\,\cseq\,\cclrd{s}\;\cleq\;\cclrd{r'}\,\cseq\,\cclrd{s}$ then:
%
\begin{equation*}
\begin{array}{l}
 \coctx{\Gamma}{\cclrd{S}} \vdash e_s : \tau_s \;\; \wedge \;\; 
 \coctx{\Gamma_1,  x : \tau_s, \Gamma_2}{ \cclrd{r}  } \vdash e_r : \tau_r\\
\quad \Rightarrow \;\; \coctx{\Gamma_1,\Gamma,\Gamma_2}{ \cclrd{r} \,\cseq\, \cclrd{S} } \vdash \subst{e_r}{x}{e_s} : \tau_r
\end{array}
\end{equation*}

\end{lemma*}
\begin{proof}
Assume that $\coctx{\Gamma}{\cclrd{S}} \vdash e_s : \tau_s$ and we are substituting a term $e_s$ for 
a variable $x$. Note that we use upper-case $\cclrd{S}$ to distinguish the coeffect of the expression
that is being substituted into an expression. Using structural induction over $\vdash$:

\paragraph{(var)} Given the following derivation using (\emph{var}):
\[
\inference
  { }
  {\coctx{\Gamma_1, y:\tau, \Gamma_2}{\cunit} \vdash y : \tau }
\]
There are two cases depending on whether $y$ is the variable $x$ or not:
\begin{itemize}
\item If $y=x$ then also $\tau = \tau_s$ and thus $\subst{y}{x}{e_s} = e_s$. Using the assumption,
implicit weakening and the fact that $\cunit$ is a unit of $\cseq$:
\[
\inference
 {\inference 
  {\coctx{\Gamma}{\cclrd{S}} \vdash \subst{y}{x}{e_s} : \tau_s }
  {\coctx{\Gamma_1, \Gamma, \Gamma_2}{\cclrd{S}} \vdash \subst{y}{x}{e_s} : \tau } }
 {\coctx{\Gamma_1, \Gamma, \Gamma_2}{\cunit\,\cseq\,\cclrd{S}} \vdash \subst{y}{x}{e_s} : \tau } 
\]
\item If $y\neq x$ then $\subst{y}{x}{e_s} = y$. Using the fact that $\cunit$ is the bottom element
and sub-coeffecting:
\[
\inference
  {\coctx{\Gamma_1, y:\tau, \Gamma_2}{\cunit} \vdash y : \tau }
  {\coctx{\Gamma_1, y:\tau, \Gamma_2}{\cunit\,\cseq\,\cclrd{S}} \vdash y : \tau }
\]
\end{itemize}

\paragraph{(const)} Similar to the (\emph{var}) case when the variable is not substituted.

\paragraph{(sub)} Given the following typing derivation using (\emph{sub}):
\[
\inference
  {\coctx{\Gamma_1,x:\tau_s,\Gamma_2}{\cclrd{r'}} \vdash e : \tau }
  {\coctx{\Gamma_1,x:\tau_s,\Gamma_2}{\cclrd{r}} \vdash e : \tau }\quad\quad(\cclrd{r'} \cleq \cclrd{r})
\]
From the induction hypothesis, we have that
$\coctx{\Gamma_1,\Gamma,\Gamma_2}{\cclrd{r'}\,\cseq\,\cclrd{S}} \vdash \subst{e}{x}{e_s} : \tau$.
The condition on $\cleq$ means that $\cclrd{r'}\,\cseq\,\cclrd{S}\;\cleq\;\cclrd{r}\,\cseq\,\cclrd{S}$
and so we can apply the (\emph{sub}) rule to obtain
$\coctx{\Gamma_1,\Gamma,\Gamma_2}{\cclrd{r}\,\cseq\,\cclrd{S}} \vdash \subst{e}{x}{e_s} : \tau$.

\paragraph{(abs)} Given the following typing derivation using (\emph{abs}):
\[
\inference
  {\coctx{\Gamma_1,x:\tau_s,\Gamma_2, y:\tau_1}{\cclrd{r}\;\czip\;\cclrd{s}} \vdash e : \tau_2}
  {\coctx{\Gamma_1,x:\tau_s,\Gamma_2}{\cclrd{r}} \vdash \lambda y.e : \tau_1 \xrightarrow{\cclrd{s}} \tau_2 }
\]
Assume w.l.o.g. that $x\neq y$. From the induction hypothesis, we have that:
\[
\coctx{\Gamma_1,\Gamma,\Gamma_2, y:\tau_1}{ \cclrd{r} \,\cseq\, \cclrd{s} } \vdash \subst{e}{x}{e_s} : \tau_2
\]
Now using the fact that $\czip\,=\,\cseq$, associativity and commutativity and (\emph{abs}):
\[
\inference
 {\inference
  {\coctx{\Gamma_1,\Gamma,\Gamma_2, y:\tau_1}{ (\cclrd{r} \,\czip\, \cclrd{s})\,\cseq\,\cclrd{S} } \vdash \subst{e}{x}{e_s} : \tau_2}
  {\coctx{\Gamma_1,\Gamma,\Gamma_2, y:\tau_1}{ (\cclrd{r} \,\cseq\, \cclrd{S})\,\czip\,\cclrd{s} } \vdash \subst{e}{x}{e_s} : \tau_2}}
 {\inference
  {\coctx{\Gamma_1,\Gamma,\Gamma_2}{\cclrd{r} \,\cseq\, \cclrd{S}} \vdash \lambda y.(\subst{e}{x}{e_s}) : \tau_1 \xrightarrow{\cclrd{s}} \tau_2}
  {\coctx{\Gamma_1,\Gamma,\Gamma_2}{\cclrd{r} \,\cseq\, \cclrd{S}} \vdash \subst{(\lambda y.e)}{x}{e_s} : \tau_1 \xrightarrow{\cclrd{s}} \tau_2} }
\]

\paragraph{(app)} Given the following typing derivation using (\emph{app}):
\[
\inference
  {\coctx{\Gamma_1,x:\tau_s,\Gamma_2}{\cclrd{r}} \vdash e_1 : \tau_1 \xrightarrow{\cclrd{t}} \tau_2 &
   \coctx{\Gamma_1,x:\tau_s,\Gamma_2}{\cclrd{s}} \vdash e_2 : \tau_1 }
  {\coctx{\Gamma_1,x:\tau_s,\Gamma_2}{\cclrd{r} \;\cpar\; (\cclrd{s} \,\cseq\, \cclrd{t})} \vdash e_1~e_2 : \tau_2}
\]
From the induction hypothesis, we have that:
\[
\begin{array}{l}
 \coctx{\Gamma_1, \Gamma, \Gamma_2}{\cclrd{r}\,\cseq\,\cclrd{S}} \vdash \subst{e_1}{x}{e_s} : \tau_1 \xrightarrow{\cclrd{t}} \tau_2 \\
 \coctx{\Gamma_1, \Gamma, \Gamma_2}{\cclrd{s}\,\cseq\,\cclrd{S}} \vdash \subst{e_2}{x}{e_s} : \tau_1
\end{array}\quad(*)
\]
Now using the (\emph{app}) rule and the fact that $\cpar\,=\,\cseq$, associativity, commutativity and idempotence
(note that all three properties are needed):
\[
\inference
 { (*) }
 {\inference
   { \coctx{\Gamma_1,\Gamma,\Gamma_2}{(\cclrd{r}\,\cseq\,\cclrd{S}) \;\cpar\; ((\cclrd{s}\,\cseq\,\cclrd{S}) \,\cseq\, \cclrd{t})} 
       \vdash \subst{e_1}{x}{e_s}~\subst{e_2}{x}{e_s} : \tau_2}
   { \coctx{\Gamma_1,\Gamma,\Gamma_2}{(\cclrd{r} \;\cpar\; (\cclrd{s} \,\cseq\, \cclrd{t}))\;\cseq\;\cclrd{S}} 
       \vdash \subst{(e_1~e_2)}{x}{e_s} : \tau_2} }
\]

\paragraph{(let)} Given the following typing derivation using (\emph{let}):
\[
\inference
  { \coctx{\Gamma_1,x:\tau_s,\Gamma_2}{\cclrd{r}} \vdash e_1 : \tau_1 &
    \coctx{\Gamma_1,x:\tau_s,\Gamma_2, y:\tau_1}{\cclrd{s}} \vdash e_2 : \tau_2}
  {\coctx{\Gamma_1,x:\tau_s,\Gamma_2}{\cclrd{s} \;\cpar\; (\cclrd{s} \,\cseq\, \cclrd{r})} \vdash \kvd{let}~y=e_1~\kvd{in}~e_2 : \tau_2 }
\]
From the induction hypothesis, we have that:
\[
\begin{array}{l}
 \coctx{\Gamma_1, \Gamma, \Gamma_2}{\cclrd{r}\,\cseq\,\cclrd{S}} \vdash \subst{e_1}{x}{e_s} : \tau_1 \\
 \coctx{\Gamma_1, \Gamma, \Gamma_2, y:\tau_1}{\cclrd{s}\,\cseq\,\cclrd{S}} \vdash \subst{e_2}{x}{e_s} : \tau_2
\end{array}\quad(\dagger)
\]
Now using the (\emph{let}) rule and similarly to the (\emph{app}) case:
\[
\hspace{-2em}
\inference
 { (\dagger) }
 { \inference
   { \coctx{\Gamma_1, \Gamma, \Gamma_2}{(\cclrd{s}\,\cseq\,\cclrd{S}) \;\cpar\; ((\cclrd{s}\,\cseq\,\cclrd{S}) \,\cseq\, (\cclrd{r}\,\cseq\,\cclrd{S}))} 
         \vdash \kvd{let}~y=\subst{e_1}{x}{e_s}~\kvd{in}~\subst{e_2}{x}{e_s} : \tau_2 } 
   { \coctx{\Gamma_1, \Gamma, \Gamma_2}{(\cclrd{s} \;\cpar\; (\cclrd{s} \,\cseq\, \cclrd{r})) \,\cseq\, \cclrd{S}} 
         \vdash \subst{(\kvd{let}~y=e_1~\kvd{in}~e_2)}{x}{e_s} : \tau_2 } }
\]
\end{proof}
\newpage



% =================================================================================================
%                                                                      
%     ###    #                           #                           ##   
%    #   #   #                           #                            #   
%    #      ####   # ##   #   #   ###   ####   #   #  # ##    ###     #   
%     ###    #     ##  #  #   #  #   #   #     #   #  ##  #      #    #   
%        #   #     #      #   #  #       #     #   #  #       ####    #   
%    #   #   #  #  #      #  ##  #   #   #  #  #  ##  #      #   #    #   
%     ###     ##   #       ## #   ###     ##    ## #  #       ####   ###  
%
% =================================================================================================


\section{Substitution for structural coeffects}
\label{sec:appendix-struct-cbn}

In order to prove that $\beta$-reduction and $\eta$-expansion preserve the type of an expression
in Section~\ref{sec:struct-syntactic-subst}, we required a multi-nary form of the substitution 
lemma (Lemma~\ref{thm:structural-substitution}). This section provides the corresponding proof.

\begin{lemma*}[Multi-nary substitution]
Given an expression with multiple holes filled by variables $x_{Ti}\!:\!\tau_{Ti}$ with coeffects $\cclrd{s_k}$:
%
\begin{equation*}
\coctx{\Gamma}{\aclrd{\textbf{r}}}~[\,\coctx{x_{T1}\!:\!\tau_{T1}}{\alift{\cclrd{s_1}}} \,|\, \ldots \,|\,
  \coctx{x_{Tk}\!:\!\tau_{Tk}}{\alift{\cclrd{s_k}}}\,] \vdash e_r : \tau_r
\end{equation*}
%
and a expressions $e_{Ti}$ with free-variable contexts $\Gamma_{Ti}$ annotated with $\aclrd{\mathbf{T_i}}$:
%
\begin{equation*}
\coctx{\Gamma_1}{\aclrd{\mathbf{T_1}}} \vdash e_{T1} : \tau_{T1}
\quad \ldots \quad
\coctx{\Gamma_k}{\aclrd{\mathbf{T_k}}} \vdash e_{Tk} : \tau_{Tk}
\end{equation*}
%
then substituting the expressions $e_{Ti}$ for variables $x_{Ti}$ results in an expression with a context
where the original holes are filled by contexts $\Gamma_{Ti}$ with coeffects $\cclrd{s_i} \aseq \aclrd{\mathbf{T_i}}$:
%
\begin{equation*}
\coctx{\Gamma}{\aclrd{\textbf{r}}}~[\,\coctx{\Gamma_{T1}}{\cclrd{s_1}\aseq\,\aclrd{\textbf{T}_1}} \,|\, \ldots \,|\, 
  \coctx{\Gamma_{Tk}}{\cclrd{s_k}\aseq\,\aclrd{\textbf{T}_k}}\,] \vdash \subst{\subst{e_r}{x_{T1}}{e_{T1}}\ldots}{x_{Tk}}{e_{Tk}} : \tau_r
\end{equation*}
\end{lemma*}
\begin{proof}
Assume that $\coctx{\Gamma}{\aclrd{\mathbf{T_i}}} \vdash e_{Ti} : \tau_{Ti}$ and we are substituting terms $e_{Ti}$ for 
variables $x_{Ti}$. Furthermore, we assume that are variables that are being substituted for actually appear in the original 
term (which means that $k$ is at most the number of variables). Note that we use upper-case $\aclrd{\textbf{T}}$ to distinguish 
the coeffect of the expression that is being substituted into an expression. Using structural induction over $\vdash$:


% -------------------------------------------------------------------------------------------------

\paragraph{Syntax-driven typing rules}
\paragraph{(var)} Given the following derivation using (\emph{var}):
\[
\inference
  { }
  {\coctx{y:\tau}{\alift{\cunit}} \vdash y : \tau }
\]
Here, the context contains exactly one variable and so $k=1$. There are two cases depending on whether $y$ is 
the (only substituted) variables $x_{T1}$ or not:
\begin{itemize}
\item If $y=x_{T1}$ then also $\tau = \tau_{T1}$ and thus $\subst{y}{x_{T1}}{e_{T_1}} = e_{T1}$. 
In this case, the context contains only a single hole $\coctx{\Gamma}{\aclrd{\mathbf{r}}} = \coctx{-}{-}$. Using the
assumption and the fact that $\cunit$ is a unit of $\cseq$:
\[
\inference
 { \inference 
   {\coctx{\Gamma_{T1}}{\aclrd{\mathbf{T_i}}} \vdash \subst{y}{x_{T1}}{e_{T1}} : \tau_r }
   {\coctx{\Gamma}{\aclrd{\mathbf{r}}}~[\,\coctx{\Gamma_{T1}}{\aclrd{\mathbf{T_i}}}\,] \vdash \subst{y}{x_{T1}}{e_{T1}} : \tau_r } }
 { \coctx{\Gamma}{\aclrd{\mathbf{r}}}~[\,\coctx{\Gamma_{T1}}{\cunit\,\aseq\,\aclrd{\mathbf{T_i}}}\,] \vdash \subst{y}{x_{T1}}{e_{T1}} : \tau_r }   
\]
\item If $y\neq x_{T1}$ then there is no substitution that could be performed ($y$ does not 
appear in the context) and so (trivially):
\[
\coctx{\Gamma}{\aclrd{\mathbf{r}}}~[\,] \vdash y : \tau_r
\]
\end{itemize}

\paragraph{(const)} Similar to the (\emph{var}) case when the variable is not substituted.

\paragraph{(app)} In the (\emph{app}) rule, the context $\Gamma$ is obtained as a tensor product $\Gamma_1, \Gamma_2$.
  Given $\Gamma$ of length $k$, we assume that $\Gamma_1$ has length $l$ and $\Gamma_2$ has length $k-l$. 
  Now, given the following typing derivation using (\emph{app}):
\[
\inference
  {\coctx{\Gamma_1}{\aclrd{\mathbf{s_1}}} 
   [\,\coctx{x_{T1}}{\alift{\cclrd{s_1}}} \,|\, \ldots \,|\, \coctx{x_{Tl}}{\alift{\cclrd{s_l}}}\,] 
      \vdash e_1 : \tau_1 \xrightarrow{\cclrd{t}} \tau_2 \\
   \coctx{\Gamma_2}{\aclrd{\mathbf{s_2}}} 
   [\,\coctx{x_{Tl+1}}{\alift{\cclrd{s_{l+1}}}} \,|\, \ldots \,|\, \coctx{x_{Tk}}{\alift{\cclrd{s_k}}}\,]
      \vdash e_2 : \tau_1 }
  { \begin{array}{l}
    \coctx{\Gamma_1, \Gamma_2}{\aclrd{\mathbf{s_1}\;\atimes\;(\cclrd{t} \,\aseq\, \aclrd{\mathbf{s_2}})}}~
     [\,\coctx{x_{T1}}{\alift{\cclrd{s_1}}} \,|\, \ldots \,|\, \coctx{x_{Tl}}{\alift{\cclrd{s_l}}}, \\[-0.35em] \hspace{9.3em}
       \coctx{x_{Tl+1}}{\alift{\cclrd{t} \,\cseq\, \cclrd{s_{l+1}}}} \,|\, \ldots \,|\, 
          \coctx{x_{Tk}}{\alift{\cclrd{t} \,\cseq\, \cclrd{s_k}}}\,] \vdash e_1~e_2 : \tau_2 
    \end{array} }
\]
Here, we use $\cclrd{t}\;\aseq\;\aclrd{\mathbf{s_2}}$ as a pointwise extension of $\czip$ that is additionally
applied to holes such that $\cclrd{t}\;\czip\; - = -$. That is, a hole in the context remains a hole and so
the second part of the context is filled with $\cclrd{t} \,\cseq\, \cclrd{s_i}$ for all $i \in \{ l+1 \ldots k \}$.
Next, from the induction hypothesis, we have that:
\[
\begin{array}{l}
 \coctx{\Gamma_1}{\aclrd{\mathbf{s_1}}} 
   [\,\coctx{\Gamma_{T1}}{\cclrd{s_1}\,\aseq\,\aclrd{\mathbf{T_1}}} \,|\, \ldots \,|\, \coctx{\Gamma_{Tl}}{\cclrd{s_l}\aseq\,\aclrd{\mathbf{T_l}}}\,] 
   \vdash e_1~[\,\ldots\,] : \tau_1 \xrightarrow{\cclrd{t}} \tau_2 \\
 \coctx{\Gamma_2}{\aclrd{\mathbf{s_{2}}}} 
   [\,\coctx{\Gamma_{Tl+1}}{\cclrd{s_{l+1}}\aseq\,\aclrd{\mathbf{T}_{l+1}}} \,|\, \ldots \,|\, \coctx{\Gamma_{Tk}}{\cclrd{s_k}\aseq\,\aclrd{\mathbf{T_k}}}\,]
   \vdash e_2~[\,\ldots\,] : \tau_1
\end{array}\quad(*)
\]
Now using the (\emph{app}) rule in the first step and associativity of $\cseq$ on the second part of the context in the second step:
\[
\hspace{-2em}
\inference 
 { \inference
   { (*) }
   { \begin{array}{l}
      \coctx{\Gamma_1, \Gamma_2}{\aclrd{\mathbf{s_1}\;\atimes\;(\cclrd{t} \,\aseq\, \aclrd{\mathbf{s_2}})}} ~~
       [\,\coctx{\Gamma_{T1}}{ \cclrd{s_1}\,\aseq\,\aclrd{\mathbf{T_1}} } \,|\, \ldots \,|\, 
          \coctx{\Gamma_{Tl}}{ \cclrd{s_l}\,\aseq\,\aclrd{\mathbf{T_l}} }, 
         \\[-0.35em] \qquad
         \coctx{\Gamma_{Tl+1}}{\cclrd{t} \,\aseq\, (\cclrd{s_{l+1}}\,\aseq\,\aclrd{\mathbf{T_{l+1}}}) } \,|\, \ldots \,|\,
         \coctx{\Gamma_{Tk}}{\cclrd{t} \,\aseq\, (\cclrd{s_k}\,\aseq\,\aclrd{\mathbf{T_k}})}\,] \vdash (e_1~e_2)~[\,\ldots\,] : \tau_2 
      \end{array} } }
 { \begin{array}{l}
    \coctx{\Gamma_1, \Gamma_2}{\aclrd{\mathbf{s_1}\;\atimes\;(\cclrd{t} \,\aseq\, \aclrd{\mathbf{s_2}})}} ~~
     [\,\coctx{\Gamma_{T1}}{ \cclrd{s_1}\,\aseq\,\aclrd{\mathbf{T_1}} } \,|\, \ldots \,|\, 
        \coctx{\Gamma_{Tl}}{ \cclrd{s_l}\,\aseq\,\aclrd{\mathbf{T_l}} }, 
       \\[-0.35em] \qquad
       \coctx{\Gamma_{Tl+1}}{(\cclrd{t} \,\cseq\, \cclrd{s_{l+1}})\,\aseq\,\aclrd{\mathbf{T_{l+1}}} } \,|\, \ldots \,|\,
       \coctx{\Gamma_{Tk}}{(\cclrd{t} \,\cseq\, \cclrd{s_k})\,\aseq\,\aclrd{\mathbf{T_k}}}\,] \vdash (e_1~e_2)~[\,\ldots\,] : \tau_2 
    \end{array} }
\]

\paragraph{(abs)} Without the loss of generality, we can assume that the bound variable is 
not one of the variables being substituted. Thus, the last variable (and the corresponding coeffect)
are not holes. The typing derivation using (\emph{abs}) then looks as follows:
\[
\inference
  {\coctx{\Gamma, x\!:\!\tau_1}{\aclrd{\textbf{r}}\,\atimes\,\alift{\cclrd{s}}} 
   ~[\,\coctx{x_{T1}\!:\!\tau_{T1}}{\alift{\cclrd{s_1}}} \,|\, \ldots \,|\, \coctx{x_{Tk}\!:\!\tau_{Tk}}{\alift{\cclrd{s_k}}}\,]
    \vdash e : \tau_2}
  {\coctx{\Gamma}{\aclrd{\textbf{r}}} 
   ~[\,\coctx{x_{T1}\!:\!\tau_{T1}}{\alift{\cclrd{s_1}}} \,|\, \ldots \,|\, \coctx{x_{Tk}\!:\!\tau_{Tk}}{\alift{\cclrd{s_k}}}\,]
    \vdash \lambda x.e : \tau_1 \xrightarrow{\cclrd{s}} \tau_2 }
\]
From the induction hypothesis, we have that:
\[
\coctx{\Gamma, x\!:\!\tau_1}{\aclrd{\textbf{r}}\,\atimes\,\alift{\cclrd{s}}}~
 [\,\coctx{\Gamma_{T1}}{\cclrd{s_1}\aseq\,\aclrd{\textbf{T}_1}} \,|\, \ldots \,|\, 
    \coctx{\Gamma_{Tk}}{\cclrd{s_k}\aseq\,\aclrd{\textbf{T}_k}}\,] \vdash e~[\,\ldots\,] : \tau_2
\]
Because the last position in the vector of variables is an actual variable rather than a hole, we
just need to apply the (\emph{abs}) rule:
\[
\inference 
 { \coctx{\Gamma, x\!:\!\tau_1}{\aclrd{\textbf{r}}\,\atimes\,\alift{\cclrd{s}}}~
   [\,\coctx{\Gamma_{T1}}{\cclrd{s_1}\aseq\,\aclrd{\textbf{T}_1}} \,|\, \ldots \,|\, 
      \coctx{\Gamma_{Tk}}{\cclrd{s_k}\aseq\,\aclrd{\textbf{T}_k}}\,] \vdash e~[\,\ldots\,] : \tau_2 }
 { \coctx{\Gamma}{\aclrd{\textbf{r}}}~
   [\,\coctx{\Gamma_{T1}}{\cclrd{s_1}\aseq\,\aclrd{\textbf{T}_1}} \,|\, \ldots \,|\, 
      \coctx{\Gamma_{Tk}}{\cclrd{s_k}\aseq\,\aclrd{\textbf{T}_k}}\,] \vdash \lambda x.e~[\,\ldots\,] : \tau_1 \xrightarrow{\cclrd{s}} \tau_2 }
\]

\paragraph{(let)} In the structural coeffect calculus let-binding can be viewed as a syntactic
sugar for abstraction/application and so the case follows from (\emph{abs}) and (\emph{app}).

\vspace{0.75em}
\noindent
Compared to the similar proof for the flat coeffect calculus, the proof for the structural system
requires fewer properties of the coeffect algebra. In particular, we only needed associativity of 
$\cseq$ in the (\emph{app}) rule the fact that $\cunit$ is a unit of $\cseq$ in (\emph{var}).


% -------------------------------------------------------------------------------------------------

\paragraph{Structural rules}

\paragraph{(contr)} In case of contraction, we can assume that the two variables to be contracted
(in the assumption) are not the variables that are being substituted. However, the resulting
variable  (in the conclusion) can be one of the variables being substituted for.

\begin{itemize}
\item Assuming that $x \neq x_{Ti}$ for all $i$, the original derivation is:
\[
\inference
  {{ \begin{array}{l}
     \coctx{\Gamma_1,y\!:\!\tau_1,z\!:\!\tau_1,\Gamma_2}{\aclrd{\textbf{r}} \atimes \alift{\cclrd{s},\cclrd{t}} \atimes \aclrd{\textbf{q}}} 
     \\[-0.35em] \hspace{4em}
     [\,\coctx{x_{T1}\!:\!\tau_{T1}}{\alift{\cclrd{s_1}}} \,|\, \ldots \,|\, \coctx{x_{Tk}\!:\!\tau_{Tk}}{\alift{\cclrd{s_k}}}\,]
    \vdash e : \tau
    \end{array} }}
  { \begin{array}{l}
    \coctx{\Gamma_1,x\!:\!\tau_1,\Gamma_2}{\aclrd{\textbf{r}} \atimes \alift{\cclrd{s} \,\cpar\, \cclrd{t}} \atimes \aclrd{\textbf{q}}} 
    \\[-0.35em] \quad
    [\,\coctx{x_{T1}\!:\!\tau_{T1}}{\alift{\cclrd{s_1}}} \,|\, \ldots \,|\, \coctx{x_{Tk}\!:\!\tau_{Tk}}{\alift{\cclrd{s_k}}}\,]
    \vdash \subst{e}{z,y}{x} : \tau
    \end{array} }
\]
Applying (\emph{contr}) to the induction hypothesis gives the required result:
\[
\inference
  {{ \begin{array}{l}
     \coctx{\Gamma_1,y\!:\!\tau_1,z\!:\!\tau_1,\Gamma_2}{\aclrd{\textbf{r}} \atimes \alift{\cclrd{s},\cclrd{t}} \atimes \aclrd{\textbf{q}}} 
     \\[-0.35em] \hspace{4em}
     [\,\coctx{\Gamma_{T1}}{\cclrd{s_1}\aseq\,\aclrd{\textbf{T}_1}} \,|\, \ldots \,|\, \coctx{\Gamma_{Tk}}{\cclrd{s_k}\aseq\,\aclrd{\textbf{T}_k}}\,]
    \vdash e~[\,\ldots\,] : \tau
    \end{array} }}
  { \begin{array}{l}
    \coctx{\Gamma_1,x\!:\!\tau_1,\Gamma_2}{\aclrd{\textbf{r}} \atimes \alift{\cclrd{s} \,\cpar\, \cclrd{t}} \atimes \aclrd{\textbf{q}}} 
    \\[-0.35em] \quad
    [\,\coctx{\Gamma_{T1}}{\cclrd{s_1}\aseq\,\aclrd{\textbf{T}_1}} \,|\, \ldots \,|\, \coctx{\Gamma_{Tk}}{\cclrd{s_k}\aseq\,\aclrd{\textbf{T}_k}}\,]
    \vdash \subst{e}{z,y}{x}[\,\ldots\,] : \tau
    \end{array} }
\]

\item In the other case, $x=x_{Ti}$ for some $i$. The original typing is:
\[
\inference
  {{ \begin{array}{l}
     \coctx{\Gamma_1,-,-,\Gamma_2}{\aclrd{\textbf{r}} \atimes \alift{-,-} \atimes \aclrd{\textbf{q}}}~
         [\,\coctx{x_{T1}\!:\!\tau_{T1}}{\alift{\cclrd{s_1}}} \,|\, \ldots \,|\,
     \\[-0.35em] \hspace{4em}
     \coctx{y\!:\!\tau_1}{\alift{\cclrd{s}}} \,|\,
     \coctx{z\!:\!\tau_1}{\alift{\cclrd{t}}}
     \,|\, \ldots \,|\, \coctx{x_{Tk}\!:\!\tau_{Tk}}{\alift{\cclrd{s_k}}}\,]
    \vdash e : \tau
    \end{array} }}
  { \begin{array}{l}
    \coctx{\Gamma_1,-,\Gamma_2}{\aclrd{\textbf{r}} \atimes \alift{-} \atimes \aclrd{\textbf{q}}}~
        [\,\coctx{x_{T1}\!:\!\tau_{T1}}{\alift{\cclrd{s_1}}} \,|\, \ldots \,|\,
     \\[-0.35em] \quad
     \coctx{x\!:\!\tau_1}{ \alift{\cclrd{s} \,\cpar\, \cclrd{t}} }
     \,|\, \ldots \,|\, \coctx{x_{Tk}\!:\!\tau_{Tk}}{\alift{\cclrd{s_k}}}\,]
    \vdash \subst{e}{z,y}{x} : \tau
    \end{array} }
\]
Applying (\emph{contr}) to the induction hypothesis gives the following result:
\[
\inference
  {{ \begin{array}{l}
     \coctx{\Gamma_1,-,-,\Gamma_2}{\aclrd{\textbf{r}} \atimes \alift{-,-} \atimes \aclrd{\textbf{q}}}~
         [\,\coctx{\Gamma_{T1}}{\cclrd{s_1}\aseq\,\aclrd{\textbf{T}_1}} \,|\, \ldots \,|\,
     \\[-0.35em] \quad
     \coctx{\Gamma_{Ti}}{\cclrd{s}\,\aseq\,\aclrd{\mathbf{T_k}} } \,|\,
     \coctx{\Gamma_{Ti}}{\cclrd{t}\,\aseq\,\aclrd{\mathbf{T_k}} }
     \,|\, \ldots \,|\, \coctx{\Gamma_{Tk}}{\cclrd{s_k}\aseq\,\aclrd{\textbf{T}_k}}\,]
    \vdash e~[\,\ldots\,] : \tau
    \end{array} }}
  {{ \begin{array}{l}
     \coctx{\Gamma_1,-,\Gamma_2}{\aclrd{\textbf{r}} \atimes \alift{-} \atimes \aclrd{\textbf{q}}}~
         [\,\coctx{\Gamma_{T1}}{\cclrd{s_1}\aseq\,\aclrd{\textbf{T}_1}} \,|\, \ldots \,|\,
     \\[-0.35em] \quad
     \coctx{\Gamma_{Ti}}{ (\cclrd{s}\,\aseq\,\aclrd{\mathbf{T_k}}) \aparstr (\cclrd{t}\,\aseq\,\aclrd{\mathbf{T_k}})}
     \,|\, \ldots \,|\, \coctx{\Gamma_{Tk}}{\cclrd{s_k}\aseq\,\aclrd{\textbf{T}_k}}\,]
    \vdash e~[\,\ldots\,] : \tau
    \end{array} }}
\]
Here the $\aparstr$ operation represents a pointwise extension of $\cpar$. Thus for $i^{\textit{th}}$ 
substituted coeffect, we have $(\cclrd{s}\,\cseq\,\cclrd{T_i})\;\cpar\;(\cclrd{t}\,\cseq\,\cclrd{T_i})$.
Using the distributivity law of structural coeffect algebra, we obtain the required structure:
$(\cclrd{s}\,\cpar\,\cclrd{t})\;\cseq\;\cclrd{T_i}$.

\paragraph{(sub)} As in the (\emph{contr}) case, in the (\emph{sub}) case we distinguish two situations.
If the sub-coeffecting is applied to variable that is \emph{not} being substituted for, then the case is
easy (sub-coeffecting does not interact with substitution), so we only consider the case when $x$ is one
of the variables being substituted for:
\[
\inference
  {{\begin{array}{l}
   \coctx{\Gamma_1,-,\Gamma_2}{\aclrd{\textbf{r}} \atimes \alift{-} \atimes \aclrd{\textbf{q}}} ~
   [\,\coctx{x_{T1}\!:\!\tau_{T1}}{\alift{\cclrd{s_1}}} \,|\, \ldots \,|\,
     \\[-0.35em] \qquad
     \coctx{x\!:\!\tau_1}{\alift{\cclrd{s'}}}
     \,|\, \ldots \,|\, \coctx{x_{Tk}\!:\!\tau_{Tk}}{\alift{\cclrd{s_k}}}\,]
   \vdash e : \tau
   \end{array}}}
  {{\begin{array}{l}
   \coctx{\Gamma_1,-,\Gamma_2}{\aclrd{\textbf{r}} \atimes \alift{-} \atimes \aclrd{\textbf{q}}} ~
   [\,\coctx{x_{T1}\!:\!\tau_{T1}}{\alift{\cclrd{s_1}}} \,|\, \ldots \,|\,
     \\[-0.35em] \qquad
     \coctx{x\!:\!\tau_1}{\alift{\cclrd{s}}}
     \,|\, \ldots \,|\, \coctx{x_{Tk}\!:\!\tau_{Tk}}{\alift{\cclrd{s_k}}}\,]
   \vdash e : \tau
   \end{array}}}
\quad(\cclrd{s'}\;\cleq\;\cclrd{s})   
\]
From the induction hypothesis, we have the following:
\[
  {{\begin{array}{l}
   \coctx{\Gamma_1,-,\Gamma_2}{\aclrd{\textbf{r}} \atimes \alift{-} \atimes \aclrd{\textbf{q}}} ~
   [\,\coctx{\Gamma_{T1}}{\cclrd{s_1}\aseq\,\aclrd{\textbf{T}_1}} \,|\, \ldots \,|\,
     \\[-0.35em] \qquad
     \coctx{\Gamma_{Ti}}{\cclrd{s'}\aseq\,\aclrd{\mathbf{T_k}} }
     \,|\, \ldots \,|\, \coctx{\Gamma_{Tk}}{\cclrd{s_k}\aseq\,\aclrd{\textbf{T}_k}}\,]
   \vdash e : \tau
   \end{array}}}
\]
To complete the case, we need to apply (\emph{sub}) repeatedly on each of the variables in 
$\Gamma_{Ti}$. For $i^{\textit{th}}$ variable $x_i$, the coeffect annotation is $\cclrd{s'}\cseq\,\cclrd{T_i}$.
Using the fact that combining coeffects with $\cseq$ preserves the ordering, we get that 
$(\cclrd{s'}\cseq\,\cclrd{T_i}) \cleq (\cclrd{s}\,\cseq\,\cclrd{T_i})$ and so the conditions
of (\emph{sub}) are satisfied.

\paragraph{(weak)} We again need to consider whether the removed variable is 
one of the variables that are being substituted for. If this is not the case,
the proof is easy, so we only look at the other case:
\[
\inference
  {\coctx{\Gamma}{ \aclrd{\textbf{r}} }~
   [\,\coctx{x_{T1}\!:\!\tau_{T1}}{\alift{\cclrd{s_1}}} \,|\, \ldots \,|\, \coctx{x_{Tk}\!:\!\tau_{Tk}}{\alift{\cclrd{s_k}}}\,]
   \vdash e : \tau}
  {\coctx{\Gamma, -}{\aclrd{\textbf{r}} \atimes \alift{ - }}~
   [\,\coctx{x_{T1}\!:\!\tau_{T1}}{\alift{\cclrd{s_1}}} \,|\, \ldots \,|\, \coctx{x_{Tk}\!:\!\tau_{Tk}}{\alift{\cclrd{s_k}}}\,|\,\coctx{x\!:\!\tau_1}{\czero}\,]
   \vdash e : \tau} 
\]
Now, we use the induction hypothesis, apply the (\emph{weak}) rule and use properties of the
structural coeffect algebra:
\[
\inference
  {{ \begin{array}{l}
    \coctx{\Gamma}{ \aclrd{\textbf{r}} }~
    [\,\coctx{\Gamma_{T1}}{\cclrd{s_1}\aseq\,\aclrd{\textbf{T}_1}} \,|\, \ldots \,|\, 
     \\[-0.35em] \qquad
     \coctx{\Gamma_{Tk-1}}{\cclrd{s_{k-1}}\aseq\,\aclrd{\textbf{T}_{k-1}}}\,]
   \vdash e~[\,\ldots\,] : \tau 
   \end{array} }}
{\tyrule{sub}
 {{\begin{array}{l}
   \coctx{\Gamma, -}{\aclrd{\textbf{r}} \atimes \alift{ - }}~
   [\,\coctx{\Gamma_{T1}}{\cclrd{s_1}\aseq\,\aclrd{\textbf{T}_1}} \,|\, \ldots \,|\, 
    \\[-0.35em] \qquad
    \coctx{\Gamma_{Tk-1}}{\cclrd{s_{k-1}}\aseq\,\aclrd{\textbf{T}_{k-1}}}
      \,|\,\coctx{x\!:\!\tau_1}{\czero}\,]
   \vdash e~[\,\ldots\,] : \tau
   \end{array} }} 
 {{\begin{array}{l}
   \coctx{\Gamma, -}{\aclrd{\textbf{r}} \atimes \alift{ - }}~
   [\,\coctx{\Gamma_{T1}}{\cclrd{s_1}\aseq\,\aclrd{\textbf{T}_1}} \,|\, \ldots \,|\, 
    \\[-0.35em] \qquad
    \coctx{\Gamma_{Tk-1}}{\cclrd{s_{k-1}}\aseq\,\aclrd{\textbf{T}_{k-1}}}
      \,|\,\coctx{x\!:\!\tau_1}{\czero \aseq\,\aclrd{\textbf{T}_k} }\,]
   \vdash e~[\,\ldots\,] : \tau
   \end{array} }}  }
\]
The derivation first applies the standard (\emph{weak}) rule and then uses sub-coeffecting
rule and the property $\czero\;\cleq\;(\czero\,\cseq\,\cclrd{r})$ to obtain conclusion of the required form.

\paragraph{(exch)} In the (\emph{exch}) case, the property follows directly from the 
induction hypothesis. The required conclusion is obtained by applying (\emph{exch}) 
repeatedly (as we now need to exchange not just two individual variables, but two 
contexts, possibly containing multiple variables).

\end{itemize}
\end{proof}




\end{document}

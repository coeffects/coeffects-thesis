%!TEX root = ../main.tex

\chapter{Structural coeffect language} 
\label{ch:structural}

As already discussed, the aim of this thesis is to identify abstractions for context-aware
programming languages. We attempt to find abstractions that are general enough to capture a wide
range of useful programming language features, but specific enough to let us identify
interesting properties of the languages.

In Chapter~\ref{ch:applications}, we identified two notions of context. We generalized the 
class of flat calculi that capture whole-context properties in Chapter~\ref{ch:flat}. 
In this chapter, we turn our attention to \emph{structural} coeffect calculi that capture
per-variable properties.

The flat coeffect system captures interesting use-cases (implicit parameters, liveness and
data-flow), but provides relatively weak properties. We can define its categorical semantics,
but the equational theory proofs had numerous additional requirements. For this reason, it
is worthwhile to consider structural systems in a separate chapter. We will see that 
structural coeffects have a number of desirable properties that hold for all instances of the
calculus.



% ==================================================================================================
%                                                                                      
%     ###           #                       #                 #       #                 
%      #            #                       #                 #                         
%      #    # ##   ####   # ##    ###    ## #  #   #   ###   ####    ##     ###   # ##  
%      #    ##  #   #     ##  #  #   #  #  ##  #   #  #   #   #       #    #   #  ##  # 
%      #    #   #   #     #      #   #  #   #  #   #  #       #       #    #   #  #   # 
%      #    #   #   #  #  #      #   #  #  ##  #  ##  #   #   #  #    #    #   #  #   # 
%     ###   #   #    ##   #       ###    ## #   ## #   ###     ##    ###    ###   #   # 
%                                                                                     
% ==================================================================================================

\section{Introduction}
\label{sec:flat-intro}

Two examples of flat systems from the previous chapter were liveness and data-flow. As discussed
in \ref{sec:applications-structural}, these are interesting for theoretical reasons. However, tracking
liveness of the whole context is not practically useful. Structural versions of liveness and 
data-flow let us track more fine-grained properties. Moreover, the equational theory of flat 
coeffect calculus did not reveal many useful properties for flat liveness and data-flow. 
As we show in this chapter, this is not the case with structural versions.

In this chapter, we focus on three example applications. We look at structural liveness and
data-flow and we also consider calculus for bounded reuse, which checks how many times a variable
is accessed and generalizes linear logics (that restrict variables to be used exactly once).

%---------------------------------------------------------------------------------------------------

\subsection{Contributions}

Compared to the previous chapter, the structural coeffect calculi we consider are more 
homogeneous and so finding the common pattern is in some ways easier. However, the systems
are somewhat more complicated as they need to keep annotations attached to individual variables. 
The contributions of this chapter are as follows:

\begin{itemize}
\item We present a \emph{structural coeffect calculus} with a type system that is parameterized by a 
  \emph{structural coeffect algebra} and can be instantiated to obtain all of the three examples
  discussed (Section~\ref{sec:struct-calculus}).
  
\item We give the equational theory of the calculus. We prove the type-preservation property 
  for all structural calculi for both call-by-name and call-by-value (Section~\ref{sec:struct-syntax}). 
  
\item We show how to extend indexed comonads introduced in the previous section to 
  \emph{structural indexed comonads} and use them to give the semantics of structural 
  coeffect calculus (Section~\ref{sec:struct-semantics}). As with the flat version, the
  categorical semantics provides a motivation for the design of the calculus.
\end{itemize}

%---------------------------------------------------------------------------------------------------

\subsection{Related work}

In the previous chapter, we discussed the correspondence between coeffects and effects (and 
between comonads and monads). As noted earlier, the $\lambda$-calculus is assymetric in that
an expression has multiple inputs (variables in the context), but just a single result 
(the resulting value) and so monads and effects have no notion directly corresponding to 
structural coeffect systems.

The work in this chapter is more closely related to sub-structural type systems \cite{substruct-attpl-intro}.
While sub-structural systems remove some or all of \emph{weakening}, \emph{contraction} and
\emph{exchange} rules, our systems keep them, but use them to manipulate both the context
and its annotations.

Our work follows the language semantics style in that we provide a structural semantics to
the terms of ordinary $\lambda$-calculus. The most closely related work has been done in the
meta-langauge style, which extends the terms and types with constructs working with the context
explicitly. This includes Contextual Modal Type Theory (CMTT) \cite{logic-cmtt}, where variables
may be of a type $A[\Psi]$ denoting a value of type $A$ that requires context $\Psi$. In CMTT, 
$A[\Psi]$ is a first-class type, while structural coeffect systems do not expose coeffect 
annotations as stand-alone types.

Structural coeffect systems annotate the whole variable context with a \emph{vector} of annotations.
For example, a context with variables $x$ and $y$ annotated with $\cclrd{s}$ and $\cclrd{t}$, 
respectively is written as $\coctx{x\!:\!\tau_1,y\!:\!\tau_2}{\alift{\cclrd{s},\cclrd{t}}}$.
This means that the typing judgements have the same structure as those of the flat coeffect
calculus. As discussed in Chapter~\ref{ch:unified}, this makes it possible to unify the two 
systems and compose tracking of flat and structural properties.



% =================================================================================================
%                                                            
%       ###           ##                   ##                 
%      #   #           #                    #                 
%      #       ###     #     ###   #   #    #    #   #   ###  
%      #          #    #    #   #  #   #    #    #   #  #     
%      #       ####    #    #      #   #    #    #   #   ###  
%      #   #  #   #    #    #   #  #  ##    #    #  ##      # 
%       ###    ####   ###    ###    ## #   ###    ## #  ####  
%                                                           
% =================================================================================================

\section{Structural coeffect calculus}
\label{sec:struct-calculus}

In the structural coeffect calculus, a vector of variables in the free-variable context is 
annotated with a vector of primitive (scalar) coeffect annotations. These annotations differ
for different coeffect calculi and their properties are captured by the \emph{structural 
coeffect scalar} definition below. The scalar annotations can be integers (how many past values we 
need) or annotations specifying whether a variable is live or not.

Scalar annotations are written as $\cclrd{r}, \cclrd{s}, \cclrd{t}$ (following the style used in 
the previous chapter). Functions always have exactly one input variable and so they are annotated
with a coeffect scalar. Thus the expressions and types of structural coeffect calculi are the same
as in the previous chapter (except that annotation on function type is now a structural coeffect 
scalar):
%
\begin{equation*}
\begin{array}{rcl}
e &::=& x \sep \lambda x.e \sep e_1~e_2 \sep \kvd{let}~x = e_1~\kvd{in}~e_2\\
\tau &::=& T \sep \tau_1 \xrightarrow{\cclrd{r}} \tau_2
\end{array}
\end{equation*}
%
In the previous chapter, the free variable context $\Gamma$ has been treated as a set. In the 
structural coeffect calculus, the order of variables matters. Thus we treat free variable context
as a vector with a uniqueness condition. We also write $\xlift{-}$ for the length
of the vector:
%
\begin{equation*}
\begin{array}{rcl}
 \Gamma &\narrow{=}& \langle x_1\!:\tau_1, \ldots, x_n\!:\!\tau_n \rangle \qquad \textnormal{such that}~~\forall i, j ~.~ i \neq j \implies x_i \neq x_j\\
 \xlift{\Gamma} &\narrow{=}& n
\end{array}
\end{equation*}
%
For readability, we use the usual notation $x_1\!:\!\tau_1, \ldots, x_1\!:\!\tau_1 \vdash e : \tau$ for
typing judgements, but the free variable context should be understood as a vector.
Furthermore, the usual notation $\Gamma_1, \Gamma_2$ stands for the tensor product.
Given $\Gamma_1 = \langle x_1\!:\tau_1, \ldots, x_n\!:\!\tau_n \rangle $ and
$\Gamma_2 = \langle x_{n+1}\!:\tau_{n+1}, \ldots, x_m\!:\!\tau_m \rangle $ then
$\Gamma_1, \Gamma_2 = \Gamma_1 \times \Gamma_2 = \langle x_1\!:\tau_1, \ldots, x_m\!:\!\tau_m \rangle$.

The free variable contexts are annotated with vectors of structural coeffect scalars. In what follows,
we write the vectors of coeffects as $\alift{\cclrd{r_1}, \ldots, \cclrd{r_n}}$. Meta-variables ranging 
over vectors are written as $\aclrd{\textbf{r}}, \aclrd{\textbf{s}}, \aclrd{\textbf{t}}$ (using bold face 
and colour to distinguish them from scalar meta-variables) and the length of a coeffect vector
is written as $\slift{\aclrd{\textbf{r}}}$. The structure for working with vectors
of coeffects is provided by the \emph{structural coeffect algebra} definition below. 

% -------------------------------------------------------------------------------------------------

\subsection{Structural coeffect algebra}

The structural coeffect scalar structure is similar to \emph{flat coeffect algebra} with the 
exception that it drops the $\czip$ operation. It only provides a monoid $(\C, \cseq, \cunit)$
modelling sequential composition of computations and a monoid $(\C, \cpar, \czero)$ representing
point-wise composition, as well as a relation $\cleq$ that defines sub-coeffecting.

\begin{definition}
A \emph{\cclrd{structural coeffect scalar}} $(\C, \cseq, \cpar, \cunit, \czero, \cleq)$ is a set 
$\C$ together with elements $\cunit, \czero \in \C$, relation $\cleq$ and binary operations 
$\cseq, \cpar$ such that $(\C, \cseq, \cunit)$ and $(\C, \cpar, \czero)$ are monoids and
$(\C, \cleq)$ is a pre-order. That is, for all $r,s,t\in \C$:
%
\begin{equation*}
\begin{array}{ccr}
r \;\cseq\; (s \;\cseq\; t) = (r \;\cseq\; s) \;\cseq\; t  &
\cunit \;\cseq\; r = r = r \;\cseq\; \cunit &
\textnormal{(monoid)}   
\\
r \;\cpar\; (s \;\cpar\; t) = (r \;\cpar\; s) \;\cpar\; t &
\czero \;\cpar\; r = r = r \;\cpar\; \czero &
\textnormal{(monoid)}   
\\
\textnormal{if}~~r\; \cleq\; s ~~\textnormal{and}~~s\; \cleq\; t~~\textnormal{then}~~r\; \cleq\; t&
t\; \cleq\; t &
\textnormal{(pre-order)}   
\end{array}
\end{equation*}
%
In addition, the following distributivity axioms hold:
\begin{align*}
\quad (\cclrd{r}\, \cpar\, \cclrd{s}) \;\cseq\; \cclrd{t} & = (\cclrd{r} \,\cseq\, \cclrd{t}) \;\cpar\; (\cclrd{s}\, \cseq\, \cclrd{t}) \\
\quad \cclrd{t} \;\cseq\; (\cclrd{r}\, \cpar\, \cclrd{s}) & = (\cclrd{t} \,\cseq\, \cclrd{r}) \;\cpar\; (\cclrd{t}\, \cseq\, \cclrd{s})
\end{align*}
\end{definition}

\noindent
In the flat coeffect calculus, we used the $\czip$ operation to merge the annotations of contexts
available from the declaration-site and the call-site or, in the syntactic reading, to split the 
context requirements.

In the structural coeffect calculus, we use a vector instead -- combining and splitting of 
coeffects becomes just vector a concatenation or splitting, respectively, which is provided
by the tensor product. The operations on vectors are indexed by integers representing the lengths
of the vectors. The additional structure required by the type system for structural coeffect 
calculi is given by the following definition.

\begin{definition}
A \emph{\aclrd{structural coeffect algebra}} is formed by a structural coeffect scalar 
$(\C, \cseq, \cpar, \cunit, \czero, \cleq)$ equipped with the following additional structures:

\begin{itemize}
\item Coeffect vectors $\aclrd{\textbf{r}}, \aclrd{\textbf{s}}, \aclrd{\textbf{t}}$, ranging over
  structural coeffect scalars indexed by vector lengths $\sclrd{m}, \sclrd{n} \in \sclrd{\mathbb{N}}$.

\item An operation that constructs a vector from scalars indexed by the vector length
  $\alift{-}_{\sclrd{n}} : \C \times \ldots \times \C \rightarrow \C^{\sclrd{n}}$
  and an operation that returns the vector length such that 
  $\slift{\aclrd{\textbf{r}}} = \sclrd{n}$ for $\aclrd{\textbf{r}} : \C^{\sclrd{n}}$

\item A point-wise extension of the $\cseq$ operator written as $\cclrd{t} \,\aseq\, \aclrd{\textbf{s}}$
  such that\\ $\cclrd{t} \,\aseq\, \alift{\cclrd{r_1}, \ldots, \cclrd{r_n}} = 
    \alift{\cclrd{t} \,\cseq\,\cclrd{r_1}, \ldots, \cclrd{t} \,\cseq\,\cclrd{r_n}}$.

\item An indexed tensor product $\atimes_{\sclrd{n},\sclrd{m}} : \C^{\sclrd{n}} \times \C^{\sclrd{m}} \rightarrow \C^{\sclrd{n}+\sclrd{m}}$
  that is used in both directions -- for vector concatenation and for splitting -- 
  which is defined as $\alift{\cclrd{r_1}, \ldots, \cclrd{r_n}} 
     \atimes_{\sclrd{n}, \sclrd{m}} \alift{\cclrd{s_1}, \ldots, \cclrd{s_m}}
   = \alift{\cclrd{r_1}, \ldots, \cclrd{r_n}, \cclrd{s_1}, \ldots, \cclrd{s_m}}$
\end{itemize}
\end{definition}

\noindent
The fact that the tensor product $\atimes_{\sclrd{n},\sclrd{m}}$ is indexed by the lengths
of the two vectors means that we can use it unambiguously for both concatenation of vectors
and for splitting of vectors, provided that the lengths of the resulting vectors are known.
In the following text, we usually omit the indices and write just 
$\aclrd{\textbf{r}} \atimes \aclrd{\textbf{s}}$, because the lengths of the coeffect vectors 
can be determined from the lengths of the matching free variable context vectors.

More generally, we could see the the coeffect annotations as a \emph{container} 
\cite{types-containers} that supports certain operations. This approach is used in 
Chapter~\ref{ch:unified} as a way of unifying the flat and structural systems. 

% -------------------------------------------------------------------------------------------------

\subsection{Structural coeffect types}
\label{sec:struct-calculus-types}

The type system for structural coeffect calculus is similar to sub-structural type systems
in how it handles free variable contexts. The \emph{syntax-driven} rules do not implicitly allow
weakening, exchange or contraction -- this is done by checking the types of sub-expressions
in disjoint parts of the free variable context. Unlike in sub-structural logics, our system
allows weakening, exchange and contraction, but using explicit \emph{structural} rules that
perform corresponding transformation on the coeffect annoation.

\paragraph{Syntax-driven rules.}
The variable access rule (\emph{var}) annotates the corresponding variable as being used
using $\cunit$. Note that, as in sub-structural systems, the free variable context contains 
\emph{only} the accessed variable. Other variables can be introduced using explicit weakening.
Constants (\emph{const}) are type checked in an empty variable context, which is annotated with
an empty vector of coeffect annotations.

The (\emph{abs}) rule assumes that the free variable context of the body can be split into
a potentially empty \emph{declaration site} and a singleton context containing the bound
variable. The corresponding splitting is performed on the coeffect vector, uniquely associating
the annotation $\cclrd{s}$ with the bound variable $x$. This means that the typing rule removes
non-determinism present in flat coeffect systems.

In (\emph{app}), the sub-expressions $e_1$ and $e_2$ use free variable contexts $\Gamma_1, \Gamma_2$
with coeffect vectors $\aclrd{\textbf{r}}, \aclrd{\textbf{s}}$, respectively. The function
value is annotated with a coeffect scalar $\cclrd{t}$.
The coeffect annotation of the composed expression is obtained by combining the annotations
associated with variables in $\Gamma_1$ and $\Gamma_2$. Variables in $\Gamma_1$ are only
used to obtain the function value, resulting in coeffects $\aclrd{\textbf{r}}$. The 
variables in $\Gamma_2$ are used to obtain the argument value, which is then sequentially
composed with the function, resulting in $\cclrd{t}\,\aseq\,\aclrd{\textbf{s}}$.

\paragraph{Structural rules.}
The remaining rules, shown in Figure~\ref{fig:struct-types} (b), are not syntax-directed.
They allow different transformation of the free variable context. We include sub-coeffecting
(\emph{sub}) as one of the rules, allowing sub-coeffecting on coeffect scalars belonging
to individual variables. The remaining rules capture \emph{weakening}, \emph{exchange} and
\emph{contraction} known from sub-structural systems.

The (\emph{weak}) allows adding a variable to the context, extending the coeffect
vector with $\czero$ to mark it as unused, (\emph{exch}) provides a way to rearrange variables
in the context, performing the same reordering on the coeffect vector. Finally recall that 
variables in the free variable context are required to be \emph{unique}. The (\emph{contr}) rule 
allows re-using a variable as we can type check sub-expressions using two separate varaibles and 
then unify them using substitution. The resulting variable is annotated with $\cpar$ and it
is the only place in the structural coeffect system where context requiremens are combined,
or semantically, where the same context is shared.

%---------------------------------------------------------------------------------------------------

\begin{figure}[t]
{\small a.) Syntax-driven typing rules:}
\begin{equation*}
\tyrule{var}
  { }
  {\coctx{x\!:\!\tau}{\alift{\cunit}} \vdash x : \tau }
\end{equation*}
\begin{equation*}
\tyrule{const}
  {c \!:\! \tau \in \Delta}
  {\coctx{()}{\alift{}} \vdash c : \tau }
\end{equation*}
\begin{equation*}
\tyrule{app}
  {\coctx{\Gamma_1}{\aclrd{\textbf{r}}} \vdash e_1 : \tau_1 \xrightarrow{\cclrd{t}} \tau_2 &
   \coctx{\Gamma_2}{\aclrd{\textbf{s}}} \vdash e_2 : \tau_1 }
  {\coctx{\Gamma_1, \Gamma_2}{\aclrd{\textbf{r}\;\atimes\;(\cclrd{t} \,\aseq\, \aclrd{\textbf{s}})}} \vdash e_1~e_2 : \tau_2}
\end{equation*}
\begin{equation*}
\tyrule{abs}
  {\coctx{\Gamma, x\!:\!\tau_1}{\aclrd{\textbf{r}}\,\atimes\,\alift{\cclrd{s}}} \vdash e : \tau_2}
  {\coctx{\Gamma}{\aclrd{\textbf{r}}} \vdash \lambda x.e : \tau_1 \xrightarrow{\cclrd{s}} \tau_2 }
\end{equation*}
\begin{equation*}
\tyrule{let}
  { \coctx{\Gamma_1}{\aclrd{\textbf{r}}} \vdash e_1 : \tau_1 &
    \coctx{\Gamma_2, x\!:\!\tau_1}{\aclrd{\textbf{s}}\;\atimes\;\alift{\cclrd{t}}} \vdash e_2 : \tau_2}
  {\coctx{\Gamma_1, \Gamma_2}{(\cclrd{t} \,\aseq\, \aclrd{\textbf{r}}) \;\atimes\; \aclrd{\textbf{s}}} \vdash \kvd{let}~x=e_1~\kvd{in}~e_2 : \tau_2 }
\end{equation*}

\vspace{1em}
{\small b.) Structural rules for context manipulation:}
\begin{equation*}
\tyrule{sub}
  {\coctx{\Gamma_1,x\!:\!\tau_1,\Gamma_2}{\aclrd{\textbf{r}} \atimes \alift{\cclrd{s'}} \atimes \aclrd{\textbf{q}}} \vdash e : \tau}
  {\coctx{\Gamma_1,x\!:\!\tau_1,\Gamma_2}{\aclrd{\textbf{r}} \atimes \alift{\cclrd{s}} \atimes \aclrd{\textbf{q}}} \vdash e : \tau}
\quad
(\cclrd{s'} \cleq \cclrd{s})
\end{equation*}
\begin{equation*}
\tyrule{weak}
  {\coctx{\Gamma}{ \aclrd{\textbf{r}} } \vdash e : \tau}
  {\coctx{\Gamma,x \!:\! \tau_1}{\aclrd{\textbf{r}} \cons \alift{ \czero }} \vdash e : \tau} 
\end{equation*}
\begin{equation*}
\tyrule{exch}
  {\coctx{\Gamma_1,x\!:\!\tau_1,y\!:\!\tau_2,\Gamma_2}{\aclrd{\textbf{r}} \atimes \alift{\cclrd{s},\cclrd{t}} \atimes \aclrd{\textbf{q}}} \vdash e : \tau}
  {\coctx{\Gamma_1,y\!:\!\tau_2,x\!:\!\tau_1,\Gamma_2}{\aclrd{\textbf{r}} \atimes \alift{\cclrd{t},\cclrd{s}} \atimes \aclrd{\textbf{q}}} \vdash e : \tau}
\quad
\begin{array}{l}
 \xlift{\Gamma_1} = \slift{\aclrd{\textbf{r}}}\\[-0.25em]
 \xlift{\Gamma_2} = \slift{\aclrd{\textbf{s}}}
\end{array}
\end{equation*}
\begin{equation*}
\hspace{2.5em} 
\tyrule{contr}
  {\coctx{\Gamma_1,y\!:\!\tau_1,z\!:\!\tau_1,\Gamma_2}{\aclrd{\textbf{r}} \atimes \alift{\cclrd{s},\cclrd{t}} \atimes \aclrd{\textbf{q}}} \vdash e : \tau}
  {\coctx{\Gamma_1,x\!:\!\tau_1,\Gamma_2}{\aclrd{\textbf{r}} \atimes \alift{\cclrd{s} \,\cpar\, \cclrd{t}} \atimes \aclrd{\textbf{q}}} \vdash \subst{e}{z,y}{x} : \tau}
~
\begin{array}{l}
 \xlift{\Gamma_1} = \slift{\aclrd{\textbf{r}}}\\[-0.25em]
 \xlift{\Gamma_2} = \slift{\aclrd{\textbf{s}}}
\end{array}
\end{equation*}

\caption{Type system for the structural coeffect calculus}
\label{fig:struct-types}
\end{figure}

%---------------------------------------------------------------------------------------------------

\subsection{Understanding structural coeffects}

The type system for structural coeffects appears more complicated when compared to the flat
version, but it is in many ways simpler -- it removes the ambiguity arising from the use
of $\czip$ in lambda abstraction and, as discussed in Section~\ref{sec:struct-syntax}, has
a cleaner equational theory.

\paragraph{Flat and structural context.} In flat systems, lambda abstraction splits context
requirements using $\czip$ and application combines them using $\cpar$. In the structural
version, both of these are replaced with $\atimes$. The $\czip$ operation is not needed, but
$\cpar$ is still used in the (\emph{contr}) rule.

This suggests that $\czip$ and $\cpar$ serve two roles in flat coeffects. First, they are used
as over- and under-approximations of $\atimes$. This is demonstrated by the (\emph{approximation})
requirement introduced in Section~\ref{sec:flat-syntax-cbv}, which requires that
$\cclrd{r}\,\czip\,\cclrd{t}\;\cleq\;\cclrd{r}\,\cpar\,\cclrd{t}$. Semantically, flat abstraction
combines available context, potentially discarding parts of it (under-approximation), while flat
application splits available context, potentially duplicating parts of it (over-approxi\-mation)\footnote{
Because of this duality, earlier version of coeffects in \cite{coeffects-icalp13}
used $\wedge$ and $\vee$.}.

Second, the operator $\cpar$ is used when the semantics passes the same context to multiple
sub-expressions. In flat systems, this happens in (\emph{app}) and (\emph{pair}), because the 
sub-expressions may share variables. In structural systems, this is separated into an 
explicit contraction rule.

\paragraph{Let binding.}
The other aspect that makes structural systems simpler is that they remove the need for 
separate let binding. As discussed in Section~\ref{sec:flat-calculus-let}, flat calculi include
let binding that gives a \emph{more precise} typing than combination of abstraction and 
application. This is not the case for structural coeffects.

\begin{remark}[Let binding]
In a structural coeffect calculus, the typing of $(\lambda x.e_2)~e_1$ is equivalent
to the typing of $\kvd{let}~x=e_1~\kvd{in}~e_2$.
\end{remark}
\begin{proof}
Consider the following typing derivation for $(\lambda x.e_2)~e_1$. Note that 
in the last step, we apply (\emph{exch}) repeatedly to swap $\Gamma_1$ and $\Gamma_2$. 
%
\begin{equation*}
\inference
  { \begin{array}{l}\coctx{\Gamma_1}{\aclrd{\textbf{r}}} \vdash e_1 : \tau_1\vspace{-1.6em}\end{array} &
    \inference
      { \coctx{\Gamma_2, x\!:\!\tau_1}{\aclrd{\textbf{s}}\;\atimes\;\alift{\cclrd{t}}} \vdash e_2 : \tau_2 }
      { \coctx{\Gamma_2}{\aclrd{\textbf{s}}} \vdash \lambda x.e_2 : \tau_1 \xrightarrow{\cclrd{t}} \tau_2} \vspace{0.2em} }
  { \inference
      { \coctx{\Gamma_2, \Gamma_1}{\aclrd{\textbf{s}} \;\atimes\; (\cclrd{t} \,\aseq\, \aclrd{\textbf{r}})} \vdash (\lambda x.e_2)~e_1 : \tau_2 }
      { \coctx{\Gamma_1, \Gamma_2}{(\cclrd{t} \,\aseq\, \aclrd{\textbf{r}}) \;\atimes\;  \aclrd{\textbf{s}} } \vdash (\lambda x.e_2)~e_1 : \tau_2 } }
\end{equation*}

\noindent
The assumptions and conclusions match those of the (\emph{let}) rule. 
\end{proof}


%---------------------------------------------------------------------------------------------------

\subsection{Examples of structural coeffects}

The structural coeffect calculus can be instantiated to obtain the structural coeffect calculi
presented in Section~\ref{sec:applications-structural}. Two of them -- structural data-flow and
structural liveness provide a more precise tracking of properties that can be tracked using 
flat systems. Formally, a flat coeffect algebra can be turned into a structural coeffect algebra
(by dropping the $\czip$ operator), but this does not always give us a meaningful system -- for
example, it is not clear why one would associate implicit parameters with individual variables.

On the other hand, some of the structural systems do not have a flat equivalent, typically 
because there is no appropriate $\czip$ operator that could be added to form the flat coeffect
algebra. This is the case, for example, for the bounded variable use.

\begin{example}[Structural liveness]
The structural coeffect algebra for liveness is formed by
$(L, \sqcap, \sqcup, \ident{L}, \ident{D}, \sqsubseteq)$, where $L=\{ \ident{L}, \ident{D} \}$ is 
the same two-point lattice as in the flat version, that is $\ident{D} \sqsubseteq \ident{L}$
with a join $\sqcup$ and meet $\sqcap$. 
\end{example}

\begin{example}[Structural data-flow]
In data-flow, context is annotated with natural numbers and the flat coeffect algebra is formed 
by $(\mathbb{N}, +, \mathit{max}, 0, 0, \leq)$.
\end{example}

\noindent
For the two examples that have both flat and structural version, obtaining the strucutral coeffect
algebra is easy. As shown by the examples above, we simply omit the $\czip$ operation. The laws
required by a structural coeffect algebra are the same as those required by the flat version
and so the above definitions are both valid. Similar construction can be used for the
\emph{optimized data-flow} example from Section~\ref{sec:flat-calculus-examples}.

It is important to note that this gives us a systems with \emph{different} properties. The 
information are now tracked per-variable rather than for the enitre context. For data-flow,
we also need to adapt the typing rule for the \kvd{prev} construct. Here, we write $\aclrd{+}$ 
for a point-wise extension of the $+$ operator, such that
$\alift{\cclrd{r_1}, \ldots, \cclrd{r_n}} + k = \alift{\cclrd{r_1} + k, \ldots, \cclrd{r_n} + k}$.
%
\begin{equation*}
\tyrule{prev}
  { \coctx{\Gamma}{\aclrd{\textbf{r}}} \vdash e : \tau }
  { \coctx{\Gamma}{\aclrd{\textbf{r}} \,\aclrd{+}\, 1} \vdash \kvd{prev}~e : \tau }
\end{equation*}
%
The rule appears similar to the flat one, but there is an important difference. Because of the
structural nature of the type system, it only increments the required number of values for 
variables that are used in the expression $e$. Annotations of other variables can be left 
unchanged. 

Before looking at the semantics and equational properties of structural coeffect systems, 
we consider bounded variable use, which is an example of structural system that does not
have a flat counterpart.

\begin{example}[Bounded variable reuse]
The structural coeffect algebra for tracking bounded variable use is given by
$(\mathbb{N}, \ast, +, 1, 0, \leq)$
\end{example}

\noindent
Similarly to the structural calculus for data-flow, the calculus for bounded variable reuse
annotates each variable with an integer. However, the integer denotes how many times is the variable 
\emph{accessed} rather than how many \emph{past values} are needed. The resulting type system is 
the one shown in Figure~\ref{fig:applications-struct-live} in Chapter~\ref{ch:applications}.

% ==================================================================================================
%                                              
%      #####  #                                 
%        #    #                                 
%        #    # ##    ###    ###   # ##   #   # 
%        #    ##  #  #   #  #   #  ##  #  #   # 
%        #    #   #  #####  #   #  #      #  ## 
%        #    #   #  #      #   #  #       ## # 
%        #    #   #   ###    ###   #          # 
%                                         #   # 
%                                          ###                                                                         
% ==================================================================================================

\section{Categorical motivation}
\label{sec:struct-semantics}

When introducing structural coeffect systems in Section~\ref{sec:applications-structural}, we 
included a concrete semantics of structural liveness and bounded variable reuse. In this section,
we generalize the examples using the notion of \emph{structural indexed comonad}, which is an 
extension of \emph{indexed comonad} structure. As in the previous chapter, the main aim of this 
section is to motivate and explain the design of the structural coeffect calculus shown in 
Section~\ref{sec:struct-calculus}. The semantics highlights the similarities and differences 
between the two systems.

Most of the differences between flat and structural systems arise from the fact that contexts
in structural coeffect systems are treated as \emph{vectors} rather than sets modelled using 
categorical products, so we start by discussing our treatment of vectors.

% --------------------------------------------------------------------------------------------------

\subsection{Semantics of vectors}

In the flat coeffect calculus, the context is interpreted as a product and so a typing judgement
$\coctx{x_1\!:\tau_1, \ldots, x_n:\tau_n}{\cclrd{r}} \vdash e : \tau$ is interpreted as a morphism 
$\ctyp{\cclrd{r}}{(\tau_1 \times \ldots \times \tau_n)} \rightarrow \tau$. In this model, we can 
freely transform the value contained in the context modelled using an indexed comonad
$\ctyp{\cclrd{r}}$. For example, the function $\ident{map}_{\cclrd{r}}~\pi_i$ transforms
a context $\ctyp{\cclrd{r}}{(\tau_1 \times \ldots \times \tau_n)}$ into a value 
$\ctyp{\cclrd{r}}{\tau_i}$. This changes the carried value without affecting the coeffect $\cclrd{r}$.

The ability to freely transform the variable structure is not desirable in the model of 
structural coeffect systems. Our aim is to guarantee (by construction) that the structure of
the coeffect annotations matches the structure of variables. To achieve this, we model vectors
using a structure distinct from ordinary products which we denote $-\xtimes-$. 
For example, the judgement $\coctx{x_1\!:\tau_1, \ldots, x_n:\tau_n}{\alift{\cclrd{r_1},\ldots,\cclrd{r_n}}} \vdash e : \tau$
is modelled as a morphism $\ctyp{\alift{\cclrd{r_1},\ldots,\cclrd{r_n}}}{(\tau_1 \xtimes \ldots \xtimes \tau_n)} \rightarrow \tau$.

The operator is a bifunctor, but it is \emph{not} a product in the categorical sense. In particular, 
there is no way to turn $\tau_1 \xtimes \ldots \xtimes \tau_n$ into $\tau_i$ (the structure does 
not have projections) and so there is also no way of turning
$\ctyp{\alift{\cclrd{r_1},\ldots,\cclrd{r_n}}}{(\tau_1 \xtimes \ldots \xtimes \tau_n)}$ into
$\ctyp{\alift{\cclrd{r_1},\ldots,\cclrd{r_n}}}{\tau_i}$, which would break the correspondence 
between coeffect annotations and variable structure.

The structure created using $-\xtimes-$ can be manipulated only using operations provided by
the \emph{strucutral indexed comonad}, which operate over variable contexts contained in an 
indexed comonad $\ctyp{\aclrd{\textbf{r}}}$.

In what follows, we model (finite) vectors of length $n$ as $\tau_1 \xtimes \ldots \xtimes \tau_n$.
We assume that the use of the operator can be freely re-associated. If an operation requires an 
input in the form $(\tau_1 \xtimes \ldots \xtimes \tau_i) \xtimes (\tau_{i+1} \xtimes \ldots \xtimes \tau_n)$,
we call it with $(\tau_1 \xtimes \ldots \xtimes \tau_n)$ as an argument and assume that the 
appropriate transformation is inserted.

% --------------------------------------------------------------------------------------------------

\subsection{Indexed comonads, revisited}

The semantics of structural coeffect calculus reuses the definition of \emph{indexed comonad} almost
without a change. The additional structure that is required for context manipulation (merging and 
splitting) is different and is provided by the \emph{structural indexed comonad} structure that we
introduce in this section.

Recall the definition from Section~\ref{sec:flat-semantics-idx}, which defines an indexed comonad 
over a monoid $(\C, \cseq, \cunit)$ as a triple $(\ctyp{\cclrd{r}{}}, \ident{counit}_{\cunit}, 
\ident{cobind}_{\cclrd{r}, \cclrd{s}})$. The triple consists of a family of object mappings $\ctyp{\cclrd{r}{}}$,
and two mappings that involve context-dependent morphisms of the form $\ctyp{\cclrd{r}} \tau \rightarrow \tau'$.

In the structural coeffect calculus, we work with morphisms of the form $\ctyp{\cclrd{r}} \tau \rightarrow \tau'$
representing function values (appearing in the language), but also of the form
$\ctyp{\alift{\cclrd{r_1},\ldots,\cclrd{r_n}}}{(\tau_1 \xtimes \ldots \xtimes \tau_n)} \rightarrow \tau$, 
modelling expressions in a context. To capture this, we need to generalize some of the indices from
\emph{coeffect scalars} $\cclrd{r}, \cclrd{s}, \cclrd{t}$ to \emph{coeffect vectors} $\aclrd{r}, \aclrd{s}, \aclrd{t}$.

\begin{definition}
Given a monoid $(\C, \cseq, \cunit)$ with a point-wise extension of the $\cseq$ operator to a vector 
(written as $\cclrd{t}\,\aseq\,\aclrd{\textbf{s}}$) and an operation lifting scalars to vectors 
$\alift{-}$, an \emph{indexed comonad} over a category $\catc$ is a triple 
$(\ctyp{\cclrd{r}{}}, \ident{counit}_{\cunit}, \ident{cobind}_{\cclrd{s}, \aclrd{\textbf{r}}})$:

\begin{compactitem}
\item $\ctyp{\aclrd{\textbf{r}}}{}$ for all $\aclrd{\textbf{r}} \in \bigcup_{\sclrd{m}\in \sclrd{\mathbb{N}}} \C^{\sclrd{m}}$ is a family of object mappings 
\item $\ident{counit}_{\cunit}$ is a mapping $\ctyp{\alift{\cunit}}{\alpha} \rightarrow \alpha$ 
\item $\ident{cobind}_{\cclrd{s}, \aclrd{\textbf{r}}}$ is a mapping $(\ctyp{\aclrd{\textbf{r}}}{\alpha} \rightarrow \beta) 
  \rightarrow (\ctyp{\cclrd{s}\aseq\aclrd{\textbf{r}}}{\alpha} \rightarrow \ctyp{\alift{\cclrd{s}}}{\beta})$
\end{compactitem}
\end{definition}

\noindent
The object mapping $\ctyp{\aclrd{\textbf{r}}}$ is now indexed by a vector rather than by a scalar 
$\ctyp{\cclrd{r}}$ as in the previous chapter. This new definition supersedes the old one, because a 
flat coeffect annotation can be seen as singleton vectors.

The operation $\ident{counit}_{\cunit}$ operates on a singleton-vector. This means that it will always
return a single variable value rather than a vector created using $-\xtimes-$. The $\ident{cobind}_{\cclrd{s}, \aclrd{\textbf{r}}}$
operation is, perhaps surprisingly, indexed by a coeffect vector and a coeffect scalar. This assymmetry
is explained by the fact that the input function ($\ctyp{\aclrd{\textbf{r}}}{\alpha} \rightarrow \beta$)
takes a vector of variables, but always produces just a single value. Thus the resulting function 
also takes a vector of variables, but always returns a context with singleton variable vector.
In other words, $\alpha$ may contain $\xtimes$, but $\beta$ may not, because the coeffect calculus has
no way of constructing values containing $\xtimes$.

% --------------------------------------------------------------------------------------------------

\subsection{Structural indexed comonads}

The flat indexed comonad structure extends indexed comonads with operations 
$\ident{merge}_{\cclrd{r},\cclrd{s}}$ and $\ident{split}_{\cclrd{r},\cclrd{s}}$ 
that combine or split the additional (flat) context and are annotated with the flat 
coeffect operations $\czip$ and $\cpar$, respectively. In the structural version, we use
corresponding operations that operate on variable vectors represented using $\xtimes$ and
are annotated with a tensor $\atimes$ which mirrors the variable structure.

The following definition also includes $\ident{lift}_{\cclrd{r'},\cclrd{r}}$, which is similar as
before and models sub-coeffecting and also $\ident{dup}_{\cclrd{r}, \cclrd{s}}$ which models
duplication of a variable in a context needed for the semantics of contraction:

\begin{definition}
Given a structural coeffect algebra formed by $(\C, \cseq, \cpar, \cunit, \czero, \cleq)$
with operations $\alift{-}$ and $\aseq$, a \emph{structural indexed comonad} is an indexed comonad over 
the monoid $(\C, \cseq, \cunit)$ equipped with families of operations $\ident{merge}_{\aclrd{\textbf{r}},\aclrd{\textbf{s}}}$, 
$\ident{split}_{\aclrd{\textbf{r}},\aclrd{\textbf{s}}}$,
$\ident{dup}_{\cclrd{r}, \cclrd{s}}$ and $\ident{lift}_{\cclrd{r'},\cclrd{r}}$ where:
%
\begin{compactitem}
\item $\ident{merge}_{\aclrd{\textbf{r}},\aclrd{\textbf{s}}}$ is a family of mappings
  $\ctyp{\aclrd{\textbf{r}}}{\alpha} \times \ctyp{\aclrd{\textbf{s}}}{\beta} \rightarrow \ctyp{\aclrd{\textbf{r}}\atimes\aclrd{\textbf{s}}}{(\alpha \xtimes \beta)}$
\item $\ident{split}_{\aclrd{\textbf{r}},\aclrd{\textbf{s}}}$ is a family of mappings
  $\ctyp{\aclrd{\textbf{r}}\atimes\aclrd{\textbf{s}}}{(\alpha \xtimes \beta)} \rightarrow \ctyp{\aclrd{\textbf{r}}}{\alpha} \times \ctyp{\aclrd{\textbf{s}}}{\beta}$
\item $\ident{dup}_{\cclrd{r}, \cclrd{s}}$ is a family of mappings
  $\ctyp{\alift{\cclrd{r} \cpar \cclrd{s}}}{\alpha} \rightarrow \ctyp{\alift{\cclrd{r}, \cclrd{s}}}{(\alpha \xtimes \alpha)}$
\item $\ident{lift}_{\cclrd{r'},\cclrd{r}}$ is a family of mappings
  $\ctyp{\alift{\cclrd{r'}}}{\alpha} \rightarrow \ctyp{\alift{\cclrd{r}}}{\alpha}$~ for all $\cclrd{r'}, \cclrd{r}$ such that $\cclrd{r}\,\cleq\,\cclrd{r'}$
\end{compactitem}
%
Such that the following equalities hold:
\begin{equation*}
\begin{array}{l}
  \ident{merge}_{\aclrd{\textbf{r}},\aclrd{\textbf{s}}} \circ \ident{split}_{\aclrd{\textbf{r}},\aclrd{\textbf{s}}} \equiv \ident{id} \\
  \ident{split}_{\aclrd{\textbf{r}},\aclrd{\textbf{s}}} \circ \ident{merge}_{\aclrd{\textbf{r}},\aclrd{\textbf{s}}} \equiv \ident{id} 
\end{array}
\end{equation*}

\end{definition}

\noindent
The operations differ from those of the flat indexed comonad in that the \ident{merge} and 
\ident{split} operations are required to be inverse functions and to preserve the additional 
information about the context. This was not required for the flat system where the operations
could under- or over-approximate. Note that the operations use $\xtimes$ to combine or
split the contained values. This means that they operate on free-variable vectors rather than 
on ordinary products.

The \ident{dup} mapping is a new operation that was not required for a flat calculus. It 
takes a variable context with a single variable annotated with $\cclrd{r}\cpar\cclrd{s}$,
duplicates the value of the variable $\alpha$ and splits the additional context between the two
new variables. In flat calculus, this operation has been expressed using ordinary tuple 
construction, which is not possible here -- the returned context needs to contain a 
two-element vector $\alpha \xtimes \alpha$.

Finally, the \ident{lift} mapping is almost the same as in the flat version. It
operates on a singleton vector, which is equivalent to operating on a scalar as before.
The operation could easily be extended to a vector in a point-wise way, but we keep it simple
and perform sub-coeffecting separately on individual variables.

% --------------------------------------------------------------------------------------------------

\subsection{Examples}
~
\newpage
~


% --------------------------------------------------------------------------------------------------

\subsection{Semantics of structural caluculus}

\newcommand{\ctx}{\textit{ctx}}

\begin{figure*}[t]

\begin{equation*}
\begin{array}{ll}
 \sem{\coctx{x\!:\!\tau}{\alift{\cunit}} \vdash x : \tau }~\ctx =
   \ident{counit}_{\cunit}~\ctx & (\emph{var})
\\[0.5em]
\sem{\coctx{\Gamma}{\czero} \vdash c_i : \tau }~\ctx =
  \delta~(c_i) & (\emph{const})
\\[0.5em]
\sem{\coctx{\Gamma_1, \Gamma_2}{\aclrd{\textbf{r}\;\atimes\;(\cclrd{t} \,\aseq\, \aclrd{\textbf{s}})}} \vdash e_1~e_2 : \tau_2}~\ctx = & (\emph{app})\\[-0.25em]
  \hspace{2.5em}
  \begin{array}{l}  
  \kvd{let}~(\ctx_1, \ctx_2) = \ident{split}_{\aclrd{\textbf{r}}, \cclrd{t} \,\aseq\, \aclrd{\textbf{s}}}~\ctx\\[-0.25em]
  \kvd{in}~\sem{\coctx{\Gamma_1}{\aclrd{\textbf{r}}} \vdash e_1 : \tau_1 \xrightarrow{\cclrd{t}} \tau_2}~\ctx_1~
      (\ident{cobind}_{\cclrd{t}, \aclrd{\textbf{s}}}~\sem{\coctx{\Gamma_2}{\aclrd{\textbf{s}}} \vdash e_2 : \tau_1 }~\ctx_2)
  \end{array}
\\[1.5em]
\sem{\coctx{\Gamma}{\aclrd{\textbf{r}}} \vdash \lambda x.e : \tau_1 \xrightarrow{\cclrd{s}} \tau_2 }~\ctx = \lambda v.& (\emph{abs})\\[-0.25em]
\hspace{2em}\sem{\coctx{\Gamma, x\!:\!\tau_1}{\aclrd{\textbf{r}}\,\atimes\,\alift{\cclrd{s}}} \vdash e : \tau_2}
  ~(\ident{merge}_{\aclrd{\textbf{r}}, \alift{\cclrd{s}}}~(\ctx, v))
\\[0.5em]
\sem{\coctx{\Gamma,x \!:\! \tau_1}{\aclrd{\textbf{r}} \cons \alift{ \czero }} \vdash e : \tau}~\ctx = & (\emph{weak}) \\
  \hspace{2em}
  \kvd{let}~(\ctx_1, \_) = \ident{split}_{\aclrd{\textbf{r}}, \alift{\czero}}~\ctx~~\kvd{in}~~
  \sem{\coctx{\Gamma}{ \aclrd{\textbf{r}} } \vdash e : \tau}~\ctx_1
\\[0.5em]
\sem{\coctx{\Gamma_1,x\!:\!\tau_1,\Gamma_2}{\aclrd{\textbf{r}} \atimes \alift{\cclrd{s}} \atimes \aclrd{\textbf{q}}} \vdash e : \tau}~\ctx = & (\emph{sub})\\
  \hspace{2em}
  \sem{\coctx{\Gamma_1,x\!:\!\tau_1,\Gamma_2}{\aclrd{\textbf{r}} \atimes \alift{\cclrd{s'}} \atimes \aclrd{\textbf{q}}} \vdash e : \tau}~
  (\ident{nest}_{\aclrd{\textbf{r}}, \alift{\cclrd{s}}, \aclrd{\textbf{q}}}~\ident{lift}_{\cclrd{s}, \cclrd{s'}}~\ctx)
\\[0.5em]
\sem{\coctx{\Gamma_1,y\!:\!\tau_2,x\!:\!\tau_1,\Gamma_2}{\aclrd{\textbf{r}} 
        \atimes \alift{\cclrd{t},\cclrd{s}} \atimes \aclrd{\textbf{q}}} \vdash e : \tau}~\ctx = & (\emph{exch})\\
  \hspace{2em}
  \sem{\coctx{\Gamma_1,x\!:\!\tau_1,y\!:\!\tau_2,\Gamma_2}{\aclrd{\textbf{r}} \atimes \alift{\cclrd{s},\cclrd{t}} \atimes \aclrd{\textbf{q}}} \vdash e : \tau}~
  (\ident{nest}_{\aclrd{\textbf{r}}, \alift{\cclrd{t},\cclrd{s}}, \aclrd{\textbf{q}}}~\ident{swap}_{\cclrd{t},\cclrd{s}}~\ctx)
\\[0.5em]
\sem{\coctx{\Gamma_1,x\!:\!\tau_1,\Gamma_2}{\aclrd{\textbf{r}} 
        \atimes \alift{\cclrd{s} \,\cpar\, \cclrd{t}} \atimes \aclrd{\textbf{q}}} \vdash \subst{e}{z,y}{x} : \tau}~\ctx = & (\emph{contr})\\
  \hspace{2em}
  \sem{\coctx{\Gamma_1,y\!:\!\tau_1,z\!:\!\tau_1,\Gamma_2}{\aclrd{\textbf{r}} \atimes \alift{\cclrd{s},\cclrd{t}} \atimes \aclrd{\textbf{q}}} \vdash e : \tau}~
  (\ident{nest}_{\aclrd{\textbf{r}}, \alift{\cclrd{s}\,\cpar\,\cclrd{t}}, \aclrd{\textbf{q}}}~\ident{dup}_{\cclrd{s}, \cclrd{t}}~\ctx)
\end{array}
\end{equation*}
\\[0.5em]
{\small Assuming the following auxiliary definitions:}
\begin{equation*}
\hspace{-2em}
\begin{array}{l}
 \ident{swap}_{\cclrd{t},\cclrd{s}} ~:~ 
  \ctyp{\alift{\cclrd{t}, \cclrd{s}}}{(\alpha \xtimes \beta)} \rightarrow \ctyp{\alift{\cclrd{s}, \cclrd{t}}}{(\beta \xtimes \alpha)} \\[-0.25em]
\ident{swap}_{\cclrd{t},\cclrd{s}}~\ctx = \\
  \hspace{2em} \kvd{let}~(\ctx_1, \ctx_2) = \ident{split}_{\alift{\cclrd{t}}, \alift{\cclrd{s}}}~\ctx \\[-0.25em]
  \hspace{2em} \kvd{in}~\ident{merge}_{\alift{\cclrd{s}}, \alift{\cclrd{t}}}~(\ctx_2, \ctx_1)
 \\[0.5em]
\ident{nest}_{\aclrd{\textbf{r}}, \aclrd{\textbf{s}}, \aclrd{\textbf{s'}}, \aclrd{\textbf{t}}} ~:~
  (\ctyp{\aclrd{\textbf{s}}}{\beta} \rightarrow \ctyp{\aclrd{\textbf{s'}}}{\beta'}) \rightarrow
    \ctyp{\aclrd{\textbf{r}} \atimes \aclrd{\textbf{s}} \atimes \aclrd{\textbf{t}}}{(\alpha \xtimes \beta \xtimes \gamma)} \rightarrow
    \ctyp{\aclrd{\textbf{r}} \atimes \aclrd{\textbf{s'}} \atimes \aclrd{\textbf{t}}}{(\alpha \xtimes \beta' \xtimes \gamma)}\\[-0.25em]
\ident{nest}_{\aclrd{\textbf{r}}, \aclrd{\textbf{s}}, \aclrd{\textbf{s'}}, \aclrd{\textbf{t}}}~f~\ctx =\\
  \hspace{2em} \kvd{let}~(\ctx_1, \ctx') = \ident{split}_{\aclrd{\textbf{r}}, \aclrd{\textbf{s}}\atimes\aclrd{\textbf{t}}}~\ctx \\[-0.25em]
  \hspace{2em} \kvd{let}~(\ctx_2, \ctx_3) = \ident{split}_{\aclrd{\textbf{s}}, \aclrd{\textbf{t}}}~\ctx' \\[-0.25em] 
  \hspace{2em} \kvd{in}~\ident{merge}_{\aclrd{\textbf{r}}, \aclrd{\textbf{s'}}\atimes\aclrd{\textbf{t}}}~
        (\ctx_1, \ident{merge}_{\aclrd{\textbf{s'}}, \aclrd{\textbf{t}}}~(f~\ctx_2, \ctx_3))
\end{array}
\end{equation*}

\caption{Categorical semantics of the structural coeffect calculus}
\label{fig:struct-semantics}
\end{figure*}

~
\newpage

- variable access gets single-variable context and so it just projects the value
- application does not duplicate before splitting - it splits directly (to get two sub-contexts)
- abs is pretty much the same
- weakening just drops a value with ``ign'' annotation - we write this using pseuod-pattern matching, but it could be special operation in the category
- sub, exchange and contraction are quite similar - they extract and then merge
~ 
\newpage







\newcommand{\ctimes}{ \hat{\times} }
\newcommand{\cdupl}[0]{ \Delta } %{ \Upsilon }

\newcommand{\csprd}[0]{ \times }
\newcommand{\csvop}[0]{ \vee }
\newcommand{\cmerge}[0]{ \cpar }
\newcommand{\csplit}[0]{ \czip }

% alculus names and rule names
\newcommand{\clflt}{$\lambda_{\textnormal{\sffamily Cf}}$}
\newcommand{\clstr}{$\lambda_{\textnormal{\sffamily Cs}}$}

\newcommand{\ruleft}[1]{\textsf{\clflt\small{-T#1}}}
\newcommand{\rulefs}[1]{\textsf{\clflt\small{-S#1}}}
\newcommand{\rulest}[1]{\textsf{\clstr\small{-T#1}}}
\newcommand{\ruless}[1]{\textsf{\clstr\small{-S#1}}}
\newcommand{\rulessc}[1]{\textsf{\clstr\small{-SC#1}}}

\newcommand{\cobind}[1]{{#1}^\dagger}
\newcommand{\obj}[0]{obj}
\newcommand{\ccat}[0]{C}

\begin{figure*}[t]
\newcommand{\uu}[1]{\overline{#1}}
\begin{equation*}
\sem{\ctyp{r_1 \csprd \ldots \csprd r_n}{(x_1 : \tau_1 \times \ldots \times x_n : \tau_n)} \vdash e : \tau} : 
  \ctyp{r_1 \csprd \ldots \csprd r_n}{(\tau_1 \times \ldots \times \tau_n)} \rightarrow \tau
\end{equation*}
\begin{equation*}
\begin{array}{rcll}
\sem{\ctyp{0}{\Gamma} \vdash x_i : \tau_i} &=&
    \epsilon_0                                                                   &\quad$\ruless{var}$
\\[0.32em]
\sem{\ctyp{r}{\Gamma} \vdash \lambda x.e : \ctyp{s}{\tau_1} \rightarrow \tau_2} &=& 
    \Lambda (\sem{\ctyp{r \csprd s}{(\Gamma, x:\tau_1)} \vdash e : \tau_2} \circ \cmerge_{r, s})    &\quad$\ruless{fun}$
\\[0.32em]
\sem{\ctyp{s \csprd (r \csvop t)}{(\Gamma_1, \Gamma_2)} \vdash e_1~e_2 : \tau} &=& 
    (\lambda (\gamma_1, \gamma_2) \rightarrow
    \sem{\ctyp{s}{\Gamma_1} \vdash e_1 : \ctyp{r}{\tau_1} \rightarrow \tau_2}~\gamma_1~
    (\cobind{\sem{\ctyp{t}{\Gamma_2}\vdash e_2 : \tau_1}}{t,r}~\gamma_2)) \circ \csplit_{s,(r \vee t)}  &\quad$\ruless{app}$
\\[0.32em]
\sem{\ctyp{r'}{\Gamma'} \vdash e : \tau} &=&
    \sem{\ctyp{r}{\Gamma} \vdash e : \tau} \circ \sem{\ctxtrans{\ctyp{r'}{\Gamma'}}{\ctyp{r}{\Gamma}}}       &\quad$\ruless{ctx}$
\end{array}
\end{equation*}

\begin{equation*}
\begin{array}{rclll}
\sem{\ctxtrans{\ctyp{r'\csprd s}{(\Gamma_1', \Gamma_2)}}{\ctyp{r \csprd s}{(\Gamma_1, \Gamma_2)}}}
&=& \cmerge_{r, s} \circ (\sem{\ctxtrans{\ctyp{r'}{\Gamma_1'}}{\ctyp{r}{\Gamma_1}}} \times \ident{id})~\circ \csplit_{r', s}\quad&\quad\quad\quad& % spaces to align
\textnormal{\rulessc{nest}}
\\[0.8em]
\sem{\ctxtrans{\ctyp{r\csprd s}{(\Gamma_1, \Gamma_2)} }{ \ctyp{s\csprd r}{(\Gamma_2, \Gamma_1)}}}
&=& \cmerge_{s, r} \circ \ident{swap} \circ \csplit_{r, s}&&
\textnormal{\rulessc{exch}}
\\[0.8em]
\sem{\ctxtrans{\ctyp{r \csprd 1}{(\Gamma, ())}}{\ctyp{r}{\Gamma}}}
&=&\ident{fst} \circ \csplit_{r, 1}&&
\textnormal{\rulessc{empty}}
\\[0.8em]
\sem{\ctxtrans{ \ctyp{r\csprd 0}{(\Gamma, x:\tau)} }{ \ctyp{r}{\Gamma}}}
&=&\ident{fst} \circ \csplit_{r, 0}&&
\textnormal{\rulessc{weak}}
\\[0.8em]
\sem{\ctxtrans{\ctyp{r \csvop s}{(x:\tau)}}{
    \ctyp{r \csprd s}{(x:\tau, x:\tau)}}}
&=&\cmerge_{r, s} \circ \cdupl_{r, s}&&
\textnormal{\rulessc{contr}}
\\[0.8em]
\sem{\ctxtrans{\ctyp{r\csprd (s\csprd t)}{(\Gamma_1, (\Gamma_2, \Gamma_3))}}{
    \ctyp{(r\csprd s)\csprd t}{((\Gamma_1, \Gamma_2), \Gamma_3)}}}\;
&=& \cmerge_{r \csprd s, t} \circ (\cmerge_{r, s} \times \ident{id}) \circ \ident{assoc}_1 \circ
        (\ident{id} \times \csplit_{s, t}) \circ \csplit_{r, s \csprd t}&&
\textnormal{\rulessc{assoc}}
\\[0.8em]
\sem{\ctxtrans{\ctyp{r}{\Gamma}}{\ctyp{s}{\Gamma}}}
&=&
\iota_{r, s}
&&
\textnormal{\rulessc{sub}}
\end{array}
\end{equation*}


\caption{Categorical semantics for~\clstr}
\label{fig:structural-semantics}
\vspace{-0.5em}
\end{figure*}


% --------------------------------------------------------------------------------------------------


\section{Semantics of structural coeffects}
\label{sec:structural-semantics}

The semantics of structural coeffect calculus \clstr~can be defined similarly to the semantics of
\clflt. The most notable difference is that the structure of coeffect tag now mirrors the structure
of the variable context. Thus an expression $\ctyp{r \csprd s}{(\Gamma_1, \Gamma_2)} \vdash e : \tau$ 
is modelled as a function $\ctyp{r \csprd s}{(\Gamma_1 \ctimes \Gamma_2)} \rightarrow \tau$. 

As discussed in \ref{sec:structural-types}, the variable context $\Gamma$ in structural coeffect
system is not a simple finite product, but instead a binary tree. To model this, we do not use
ordinary products in the domain of the semantic function, but instead use a special 
constructor $\ctimes$. This way, we can guarantee that the variable structure corresponds to 
the tag structure. 

\subsection{Structural tagged comonads}
\label{sec:structural-tagged-comonads}

To model composition of functions, we reuse the definition of \emph{tagged comonads} from 
Section~\ref{sec:tagged-comonads} without any change. This means that composing morphisms
$T^r \tau_1 \rightarrow \tau_2$ with $T^s \tau_2 \rightarrow \tau_3$ still gives us a morphism
$T^{r \csvop s} \tau_1 \rightarrow \tau_3$ and we use the $\csvop$ operation to combine the 
context-requirements.

However, functions that do not exist in context have only a single input variable (with a 
single corresponding tag). To model complex variable contexts, we need two additional
operations that allow manipulation with the variable context. Similarly to the model of \clflt,
we also require operations that model duplication and sub-coeffecting:

\begin{definition}[Structural tagged comonad]
Given a structural coeffect tag structure $(S, \csprd, \csvop, 0, 1)$ a \emph{structural tagged comonad} is a 
tagged comonad over $(S, \csvop, 0)$ comprising of $T^r$, $\epsilon_0$ and
$\cobind{(-)}{r, s}$ together with a mapping $- \ctimes -$ from a pair of objects 
$\obj{\ccat} \times \obj{\ccat}$ to an object $\obj{\ccat}$ and families of mappings:
%
\begin{equation*}
\begin{array}{rcl}
\cmerge_{r, s} &:& T^r A \times T^s B \rightarrow T^{(r \csprd s)}(A \ctimes B)\\
\csplit_{r, s} &:& T^{(r \csprd s)}(A \ctimes B) \rightarrow T^r A \times T^s N\\
\end{array}
\end{equation*}
%
% Delta is still defineable
% ?? \cdupl_{r, s}  &:& T^{(r \csvop s)}A \rightarrow T^r A \times T^s A\\
%
And with a family of mappings $\iota_{r, s} : T^r A \rightarrow T^s A$ for all $r, s \in S$ such that $r \vee s = r$.
%
%  TODO: laws?
%
\end{definition}

The family of mappings $\iota_{r, s}$ is the same as for \emph{flat} coeffects
and it can still be used to define a family of mappings that represents \emph{duplicating}
of variables while splitting the additional coeffect tags:
%
\begin{equation*}
\begin{array}{l}
\cdupl_{r, s} : T^{(r \vee s)}A \rightarrow T^r A \times T^s A\\
\cdupl_{r, s}(\gamma) = (\iota_{(r\vee s), r}~\gamma, \iota_{(r\vee s), s}~\gamma)
\end{array}
\end{equation*}
%
The type of the $\cmerge_{r,s}$ operation looks similar to the one used for \emph{flat} coeffects,
but with two differences. Firstly, it combines tags using $\csprd$ instead of $\vee$, 
which corresponds to the fact that the variable context now consists of two parts (a tree node).
Secondly, to model the tree node, the resulting context is modelled as $A \ctimes B$
(instead of $A \times B$ as previously).

To model structural coeffects, we also need $\csplit_{r, s}$, which serves as the dual
of $\cmerge_{r, s}$. It represents \emph{splitting} of context containing multiple 
variables. The operation was not needed for \clflt, because there \emph{splitting} could
be defined in terms of \emph{duplication} provided by $\cdupl_{r, s}$. For \clstr,
the situation is different. The $\csplit_{r, s}$ operation takes a context annotated
with $r \csprd s$ that carries $A \ctimes B$.

Examples of \emph{structural tagged comonads} are shown in Section~\ref{sec:structural-tainting}.
Before looking at them, we finish our discussion of categorical semantics.

\paragraph{Categorical notes.}
The mapping $T^r$ can be extended to an endo-functor $\hat{T}^r$ in the same way as in
Section~\ref{sec:tagged-comonads}. However, we still cannot freely manipulate the 
variables in the context. Given a context modelled as $T^{r \csprd s}(A \ctimes B)$,
we can lift a morphism $f$ to $\hat{T}^{r \csprd s}(f)$, but we cannot manipulate the
variables, because $A \ctimes B$ is not a product and does not have projections $\pi_i$.

This also explains why $\csplit$ cannot be defined in terms of $\cdupl$. Even if we 
could apply $\cdupl_{r, s}$ on the input (if the tag $r \csprd s$ coincided
with tag $r \csvop s$) we would still not be able to obtain $T^r A$ from $T^r (A \ctimes B)$.

This restriction is intentional -- at the semantic level, it prevents manipulations
with the context that would break the correspondence between tag structure and the
product structure.

\subsection{Categorical semantics}
The categorical semantics of \clstr is shown in Figure~\ref{fig:structural-semantics}.
It uses the \emph{structural tagged comonad} structure introduced in the previous section,
together with the helper operation $\cdupl_{r, s}$ and the following simple helper 
operations:
\begin{equation*}
\begin{array}{rcl}
\ident{assoc} &=& \lambda (\delta_r, (\delta_s, \delta_t)) \rightarrow ((\delta_r, \delta_s), \delta_t)\\
\ident{swap}  &=& \lambda (\gamma_1, \gamma_2) \rightarrow (\gamma_2, \gamma_1)\\
f \times g &=& \lambda (x, y) \rightarrow (f~x, g~y)
\end{array}
\end{equation*}

When compared with the semantics of \clflt (Figure~\ref{fig:flat-semantics}), there is a number of 
notable differences. Firstly, the rule \ruless{var} is now interpreted as $\epsilon_0$ without 
the need for projection $\pi_i$. When accessing a variable, the context contains only the accessed
variable. The \ruless{fun} rule has the same structure -- the only difference is that we use the
$\csprd$ operator for combining context tags instead of $\vee$ (which is a result of the change of
type signature in $\cmerge_{r, s}$).

The rule \ruless{app} now uses the operation $\csplit_{s, (r \csvop t)}$ instead of $\cdupl_{s, (r \csvop t)}$,
which means that it splits the context instead of duplicating it. This makes the system more 
structural -- the expressions use disjunctive parts of the context -- and also explains why the
composed coeffect tag is $s \csprd (r \csvop t)$.

The only rule from \clflt~that was not syntax-directed (\rulefs{sub}) is now generalized to a number
of non-syntax-directed rules \rulessc{} that perform various manipulations with the context. The semantics
of $\sem{ \ctxtrans{\ctyp{r_1}{\Gamma_1}}{\ctyp{r_2}{\Gamma_2}} }$ is a function that, when given a context
$\ctyp{r_1}{\Gamma_1}$ produces a new context $\ctyp{r_2}{\Gamma_2}$. The semantics in \ruless{ctx} then 
takes a context, converts it to a new context which is compatible with the original expression $e$.
The context manipulation rules work as follows:

\begin{itemize}
\item The \rulessc{nest} and \rulessc{exch} rules use $\csplit_{r, s}$ to split the context into a product
  of contexts, then perform some operation with the contexts -- transform one and swap them, respectively.
  Finally, they re-construct a single context using $\cmerge_{r, s}$.

\item The \rulessc{empty} and \rulessc{weak} rules have the same semantics. They both split the context
  and discard one part (containing either an unused variable or an empty context).

\item If we interpreted \rulessc{contr} by applying functor $T^{r\csvop s}$ to a function that duplicates
  a variable, the resulting context would be $\ctyp{r \csvop s}(x:\tau, x:\tau)$, which would break the
  correspondence between coeffect tag and context variable structure. However, that interpretation would be
  incorrect, because we use $\ctimes$ instead of normal product for variable contexts. As a result, the
  rule has to be interpreted as a composition of $\cdupl_{r, s}$ and $\cmerge_{r, s}$, which also turns
  a tag $r \csvop s$ into $r \csprd s$.

\item The \rulessc{assoc} rule is similar to \rulessc{exch} in the sense that it de-constructs the context,
  manipulates it (using \ident{assoc}) and then re-constructs it.

\item Finally, the \rulessc{sub} rule interprets sub-coeffecting on the context associated with a single
  variable using the primitive natural transformation $\iota_{r, s}$.
\end{itemize}

\paragraph{Alternative: Separate variables.} As an alternative, we could model an expression by
attaching the context separately to individual variables. This an expression
$\ctyp{r \csprd s}{(\Gamma_1, \Gamma_2)} \vdash e : \tau$ would be modelled as 
$\ctyp{r}{\Gamma_1} \times \ctyp{s}{\Gamma_2} \rightarrow \tau$. However, this 
approach largely complicates the definition of application (where tag of all 
variables in a context is affected). Moreover, it makes it impossible to express
\clflt~in terms of \clstr~as discussed in Section~\ref{sec:generalization}.

\paragraph{Alternative: Without sub-coeffecting.} The semantics presented above uses the natural
transformation $\iota_{r, s}$, which represents sub-coeffecting, to define the duplication
operation $\cdupl_{r, s}$. However, structural coeffect calculus \clstr does not require 
sub-coeffecting in the same way as flat \clflt (where it is required for subject reduction).

This means that it is possible to define a variant of the system that does not have the \rulest{sub} 
typing rule. Then the semantics does not need the $\iota_{r, s}$ transformation, but instead,
the following natural transformation has to be provided:
%
\begin{equation*}
\cdupl_{r, s} : T^{(r \vee s)}A \rightarrow T^r A \times T^s A
\end{equation*}
%
This variant of the system could be used to define a system that ensures that all provided 
context is used and is not over-approximated. This difference is similar to the difference between
affine type systems (where a variable can be used at most once) and linear type systems 
(where a variable has to be used exactly once).


\newpage
~
\newpage
~























% ==================================================================================================
%
%                               ##      
%      ####                     ##      
%        ##                     ##      
%        ##   ##   ##  ## ###   ##  ##  
%        ##   ##   ##  ###  ##  ## ##   
%    ##  ##   ##   ##  ##   ##  ####    
%    ##  ##   ##  ###  ##   ##  ## ##   
%     ####     ### ##  ##   ##  ##  ##  
%                                      
% ==================================================================================================


% --------------------------------------------------------------------------------------------------

\subsection{Properties of reductions}
\label{sec:structural-metatheory}
Similarly to the flat version, the \clstr~calculus is defined abstractly. We cannot define its
operational meaning, because that will differ for every concrete application. For example,
when tracking array accesses, variables are interpreted as arrays and $\pastval{a}{n}$ denotes
access to a specified element. 

Just like previously, we can state general properties of the reductions. As the syntax of expressions
is the same for \clstr~as for \clflt, the substitution and reduction $\twoheadrightarrow_\beta$ 
are also the same and can be found in Figure~\ref{fig:flat-substred}.

The structural coeffect calculus \clstr~associates information with individual variables. This means
that when an expression requires certain context, we know from what scope it comes -- the
context must be provided by a scope that defines the associated variable, which is either a
lambda abstraction or global scope. This distinguishes the structural system from the flat system
where context could have been provided by any scope and the lambda rule allowed arbitrary 
splitting of context requirements between the two scopes (or declaration and caller site).

\paragraph{Internalized substitution.}
Before looking at properties of the evaluation, we consider let binding, which can be 
viewed as internalized substitution. The typing rule \rulest{let} can be derived from 
application and abstraction as follows. 

\begin{lemma}[Definition of let binding]
If $\ctyp{r}{\Gamma} \vdash (\lambda x.e_2)~e_1 : \tau_2$
then $\ctyp{r}{\Gamma} \vdash \kvd{let}~x=e_1~\kvd{in}~e_2 : \tau_2$.
\end{lemma}
\begin{proof}
The premises and conclusions of a typing derivation of $(\lambda x.e_2)~e_1$
correspond with the typing rule \rulest{let}:
\begin{equation*}
\inference[]
  {\inference
    {\ctyp{r \csprd s}{(\Gamma_1, v:\tau_1)} \vdash e_2 : \tau_2 & v \notin \Gamma_1}
    {\ctyp{r}{\Gamma_1} \vdash \lambda v.e_2 : \ctyp{s}{\tau_1} \rightarrow \tau_2 } &
   \ctyp{t}{\Gamma_2} \vdash e_1 : \tau_1 }
  {\ctyp{r \csprd (s \csvop t)}{(\Gamma_1, \Gamma_2)} \vdash (\lambda v.e_2)~e_1 : \tau_2}\qedhere
\end{equation*}
\end{proof}

The term $e_2$ which is substituted in $e_1$ is checked in a different variable
and coeffect context $\ctyp{t}{\Gamma_2}$. This is common in sub-structural systems where
a variable cannot be freely used repeatedly. The context $\Gamma_2$ is used in place of 
the variable that we are substituting for. The let binding captures substitution for a 
specific variable (the context is of a form $\ctyp{r \csprd s}{\Gamma, v:\tau}$).
For a general substitution, we need to define the notion of context with a hole.

\newcommand{\chole}[4]{\ctyp{#1}{#2}[#3|#4]}
\paragraph{Substitution and holes.}
In \clstr, the structure of the variable context is not a set, but a tree. When substituting
for a variable, we need to replace the variable in the context with the context of the
substituted expression. In general, this can occur anywhere in the tree. To formulate the
statement, we define contexts with holes, written $\Delta[-]$. Note that there is a hole
in the free variable context and in a corresponding part of the coeffect tag:

\begin{equation*}
\begin{array}{rcl}
\Delta[-] &::=& \ctyp{1}{()} \\
       & \sep & \ctyp{r}{(x:\tau)}\\
       & \sep & \ctyp{-}{(-)} \\
       & \sep & \ctyp{r_1 \csprd r_2}{(\Gamma_1, \Gamma_2)}\quad~~(\textnormal{where}~\ctyp{r_i}{\Gamma_i} \in \Delta[-])\\
\end{array}
\end{equation*}
%
Assuming we have a context with hole $\ctyp{r}{\Gamma} \in \Delta[-]$, the hole filling 
operation $\chole{r}{\Gamma}{r'}{\Gamma'}$ fills the hole in the variable context with 
$\Gamma'$ and the corresponding coeffect tag hole with $r'$. The operation is defined in 
Figure~\ref{fig:structural-hole-fill}. Using contexts with holes, we can now formulate the
general substitution lemma for \clstr.

\begin{figure}

\begin{equation*}
\begin{array}{rcl}
                              \chole{1}{()}{r'}{\Gamma'} &=& \ctyp{1}{()}\\
                        \chole{r}{(x:\tau)}{r'}{\Gamma'} &=& \ctyp{r}{(x:\tau)}\\
                             \chole{-}{(-)}{r'}{\Gamma'} &=& \ctyp{r'}{\Gamma'}\\
\chole{r_1 \csprd r_2}{(\Gamma_1, \Gamma_2)}{r'}{\Gamma'} &=& \ctyp{r_1' \csprd r_2'}{(\Gamma_1', \Gamma_2')}\\
&\multicolumn{2}{l} {
  \textnormal{where}~~\ctyp{r_i'}{\Gamma_i'} = \chole{r_i}{\Gamma_i}{r'}{\Gamma'}
}
\end{array}
\end{equation*}
\caption{The definition of hole filling operation for $\Delta[-]$}
\label{fig:structural-hole-fill}
\vspace{-2em}
\end{figure}

\begin{lemma}[Substitution Lemma]
\label{thm:structural-subst}
If $\chole{r}{\Gamma}{R}{v:\tau'} \vdash e : \tau$ and $\ctyp{S}{\Gamma'} \vdash e' : \tau'$
then $\chole{r}{\Gamma}{R \csvop S}{\Gamma'} \vdash \subst{e}{v}{e'} : \tau$.
\end{lemma} 
\begin{proof}
Proceeds by rule induction over $\vdash$ using the properties of structural coeffect tag structure
$(S, \csvop, 0, \csprd, 1)$ (see Appendix~\ref{app:structural-substitution}).
\end{proof}

\begin{theorem}[Subject reduction]
\label{thm:structural-reduction}
If $\ctyp{r}{\Gamma} \vdash e_1 : \tau$ and $e_1 \twoheadrightarrow_\beta e_2$ then 
$\ctyp{r}{\Gamma} \vdash e_2 : \tau$.
\end{theorem}
\begin{proof}
Direct consequence of Lemma~\ref{thm:structural-subst} (see Appendix~\ref{app:structural-reduction}).
\end{proof}

\paragraph{Local soundness and completeness.}
As with the previous calculus, we want to guarantee that the introduction and
elimination rules (\rulest{fun} and \rulest{app}) are appropriately strong. 
This can be done by showing \emph{local soundness} and \emph{local completeness},
which correspond to $\beta$-reduction and $\eta$-expansion. Former is a special
case of subject reduction and the latter is proved by a simple derivation:
%
\begin{theorem}[Local soundness]
If $\ctyp{r}{\Gamma} \vdash (\lambda x.e_2)~e_1 : \tau$ then
$\ctyp{r}{\Gamma} \vdash \subst{e_2}{x}{e_1} : \tau$.
\end{theorem}
\begin{proof}
Special case of subject reduction (Theorem~\ref{thm:structural-reduction}).
\end{proof}
%
\begin{theorem}[Local completeness]
If $\ctyp{r}{\Gamma} \vdash f : \ctyp{s}{\tau_1} \rightarrow \tau_2$ then
$\ctyp{r}{\Gamma} \vdash \lambda x.f x : \ctyp{s}{\tau_1} \rightarrow \tau_2$.
\end{theorem}
\begin{proof} The property is proved by the following typing derivation:
%
\begin{equation*}
\inference
  {\inference
     {\ctyp{r}{\Gamma} \vdash f : \ctyp{s}{\tau_1} \rightarrow \tau_2 &
      \ctyp{0}{(x:\tau_1)} \vdash x : \tau_1 }
     {\ctyp{r \csprd (s \csvop 0)}{(\Gamma, x:\tau_1)} \vdash f~x : \tau_2}}
  {\ctyp{r}{\Gamma} \vdash \lambda x.f x : \ctyp{s}{\tau_1} \rightarrow \tau_2 }
\qedhere
\end{equation*}
\end{proof}
%
In the last step, we use the \emph{lower bound} property of structural coeffect tag, which
guarantees that $s\csvop0 = s$. Recall that in \clflt, the typing derivation for
$\lambda x.f x$ required for local completeness was not the only possible derivation.
In the last step, it was possible to split the coeffect tag arbitrarily between the 
context and the function type. 

In the \clstr~calculus, this is not, in general, the case. The $\csprd$ operator is not required 
to be associative and to have units and so a unique splitting may exist.  For example, if we
define $\csprd$ as the operator of a \emph{free magma}, then it is invertible and for a given
$t$, there are unique $r$ and $s$ such that $t = r \csprd s$. However, if the $\csprd$ operation 
has additional properties, then there may be other possible derivation.

% ==================================================================================================

\section{Examples of structural coeffects}

\subsection{Example: Liveness analysis}
\label{sec:structural-neededness}

\subsection{Example: Data-flow (revisited) }
\label{sec:structural-tainting}

\todo{Also, consider additional language features that we consider for flat coeffects
  (mainly recursion and possibly conditionals)}
  
\section{Conclusions}

\todo{ (...) }

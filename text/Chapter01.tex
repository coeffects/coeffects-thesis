\chapter{Introduction} 
\label{ch:introduction} 

%----------------------------------------------------------------------------------------

Modern software applications behave differently depending on the environment or context 
in which they execute. They often run in increasingly rich environments that provide 
resources (e.g. database or GPS sensor) and are gradually more diverse (e.g. multiple 
versions of different mobile platforms). Web applications are split between client, server 
and mobile components; mobile applications must be aware of the context and of the platform 
while the ``internet of things'' makes the environments even more heterogeneous; 
applications that access rich data sources need to propagate security policies and provenance 
information about the data.

Writing such context-aware applications is a fundamental problem of modern software 
engineering. The state of the art relies on ad-hoc approaches -- using hand-written conditions 
or pre-processors for conditional compilation. Common problems that developers face include: 

\begin{itemize}
\item \textbf{System capabilities.} When writing code that is cross-compiled to multiple 
  targets (e.g. SQL, CUDA or JavaScript) the compilation often occurs at runtime and developers 
  have no guarantee that it will succeed until the program is executed.
  
\item \textbf{Platform versioning.} When developing application for multiple versions of a 
  system (e.g. Android), developers rely on lazy loading at runtime or use conditional compilation 
  using \ident{\#if}. The former delays errors to runtime, while the latter requires building all 
  possible configurations to discover simple compile-time errors.
  
\item \textbf{Resources \& data availability.} When programming applications that access 
  resources or data provided by the environment (e.g. specific database table, GPS sensor), 
  the program typically performs dynamic check for the resource availability. However, this 
  is not checked in any way -- we have no easy way to tell what happens when the resource is 
  not available (e.g. is there a fallback strategy or not?)

\item \textbf{Meta-programming.} In distributed client-server programming, we often need to 
  manipulate code using meta-programming techniques on the server and then pass a closed 
  expression to client, translated to another language (JavaScript). We need to be able to 
  guarantee that such expressions are well-formed before they are translated.
\end{itemize}

The issues discussed above pose an important problem to current and future state of computing.
In this thesis, we develop a unifying mathematical model that allows us to capture the properties
discussed above. We use the model in three different ways to capture different notions of 
context-dependence.

In the rest of the chapter, we first look at the distinguishing factor of two of the presented
coeffect systems (how they differ from other existing work). Then we look at a number of concrete 
problems that are discussed in greater details in later chapters. We use the examples to demonstrate 
the real-world problems that motivate our work. (Theoretical and other motivations are discussed 
later in Chapter~\ref{ch:related-work}).

%===================================================================================================

\section{Context and lambda abstraction}

Our work on context-aware programming languages connects two directions in existing research on the 
theory of programming languages. On one side, effect systems \cite{effects-gifford} and monadic 
computations \cite{monad-notions,monads-effects-marriage} provide a detailed and established method 
for tracking what effects programs have -- that is, how they affect the environment where they execute. 
On the other side, the work on comonadic notions of computations \cite{comonads-notions} shows how 
to use the mathematical dual of monads -- comonads -- to give categorical semantics of 
context-dependent computations.

Effect systems introduced track actions such as memory operations or communication. They are described 
by typing judgments of a form $\Gamma \vdash e : \alpha, \sigma$ where $\Gamma$ is the context of a 
program (typically available variables), $e$ is the expression (program) itself, $\alpha$ is the type of 
values returned by the program (e.g. integer or boolean) and $\sigma$ is a set of possible effects. The 
judgment states that, given the context $\Gamma$, an expression has a type $\alpha$ and can only perform 
effects specified by the set $\sigma$. Wadler and Thiemann \cite{monads-effects-marriage} explain how 
this shapes effect analysis of a lambda abstraction -- that is, how effect systems analyze the effects 
associated with a definition of a function:
%
\begin{quote} 
\emph{In the rule for abstraction, the effect is empty because evaluation immediately
returns the function, with no side effects. The effect on the function arrow
is the same as the effect for the function body, because applying the function will
have the same side effects as evaluating the body.}
\end{quote}
%
This means that, when a programmer defines a function, the system records that executing the 
function will perform the effects of the body of the function. However, simply defining a function
is an effect-free computation. Tate \cite{effects-producer-semantics} calls such effect systems
\emph{producer} effect systems and generalizes the idea of function to more general ``thunking'':
%
\begin{quote}
\emph{We will define an effect as a producer effect if all computations with that effect 
can be thunked as "pure" computations for a domain-specific notion of purity.}
\end{quote}
%
In contrast to the static analysis of (producer) effect systems, the analysis of 
\emph{context-dependence} does not match this pattern. In the systems we consider, lambda 
abstraction places requirements on both the \emph{call-site} (latent requirements) and the 
\emph{declaration-site} (immediate requirements), resulting in different syntactic properties. 
We informally discuss three examples first that demonstrate how contextual requirements propagate. 

%===================================================================================================

\section{Why coeffects matter}

This section gives three examples of context-dependent computations whose properties can be captured
by the tree calculi presented in this thesis. We look at an example of the \emph{flat coeffect calculus} 
(Chapter~\ref{ch:flat-coeffects}), \emph{structural coeffect calculus} (Chapter~\ref{ch:structural-coeffects})
and \emph{coeffect meta-language} (Chapter~\ref{ch:coeffect-metalanguage}).

%---------------------------------------------------------------------------------------------------

\subsection{Flat coeffect calculus}

The flat coeffect calculus associates a single piece of information with the context. As an example,
we look at a simple distributed programming language that includes the concept of \emph{resources}.
A resource may be accessed using a special construct $\kvd{access}~\ident{Res}$. The following 
example shows a function taht lists recent events -- it accesses the resource \ident{News} 
(representing a database) and a resource \ident{Clock} (with the current time):
%
\begin{equation*}
\begin{array}{l}
\kvd{let}~\ident{recentEvents} = \lambda () \rightarrow\\
\quad\kvd{let}~\ident{db} = \kvd{access}~\ident{News}~\kvd{in}\\
\quad\ident{query}~\ident{db}~\str{"SELECT * WHERE Date > \%1"}~(\kvd{access}~\ident{Clock})
\end{array}
\end{equation*}
%
Consider a scenario where the function is \emph{declared} on the server-side and then
transferred and \emph{executed} on the client-side. The resource \ident{News} represents a 
database that is only available on the server-side and so the function needs to keep a remote
reference to the server. However, the \ident{Clock} resource may (or may not) be available on the
client-side. If the resource is available on the client, then it may be re-bound and the function
will use the current client's data -- for example, to accommodate for time-zone changes.

This example demonstrates how lambda abstraction behaves for context-dependent computations. The 
context requirement of the function body is a set of resources $\{\ident{Clock}, \ident{News}\}$.
The context requirements are split between the \emph{declaration-site} and \emph{call-site}.
However, there are multiple possible splittings. The splitting $\{\ident{News}\} \cup \{\ident{Clock}\}$
models the case when the database is accessed from the server, but time is taken from the client,
while the splitting $\{\ident{Clock}, \ident{News}\} \cup \{\}$ corresponds to the case when both
resources are accessed from the server.

%---------------------------------------------------------------------------------------------------

\subsection{Structural coeffect calculus}

\newcommand{\pastval}[2]{\ident{#1}_{[#2]}}
\newcommand{\dnat}{\ident{stream}}

The calculus used in the previous section annotates the entire context with a single value --  such
as the set of required resources. However, sometimes it is desirable to annotate not the entire 
context, but individual variables of the context.

As an example, consider a language that allows us to get a value of a variable (representing
some changing data-source) \ident{x} versions back using the syntax $\pastval{a}{x}$. 
To track information about individual variables, we use a product-like operation $\times$ on tags 
to mirrors the product structure of variables. For example:
%
\begin{equation*}
\begin{array}{l}
(\ident{a}:\dnat, \ident{b}:\dnat) ~@~ 5 \times 10
  \vdash
    \pastval{a}{5}+\pastval{b}{10}: \ident{nat}
\end{array}
\end{equation*}
%
The annotations $5 \times 10$ corresponds to the free-variable context $\ident{a}, \ident{b}$, denoting
that we need at most 5 and 10 past values of \ident{a} and \ident{b}, respectively. If we substitute 
\ident{c} for both \ident{a} and \ident{b}, we need to combine the context-requirements and take the
maximum of the requirements of the individual variables:
%
\begin{equation*}
\begin{array}{l}
(\ident{c}:\dnat) ~@~ \ident{max}(5, 10)
  \vdash
    \pastval{c}{5}+\pastval{c}{10}: \ident{nat}
\end{array}
\end{equation*}
%
This version of the calculus removes the non-determinism of the lambda abstraction from the previous
version. As we associate information with individual variables, lambda abstraction creates a function
that requires the context requirements associated with the variable that is being abstracted over.

For example, if we wrapped the earlier example above in a function taking \ident{a} (and using 
\ident{b} from the declaration-site) then the function would have context requirements $5$ -- that
is the number associated with the variable \ident{a}.

%---------------------------------------------------------------------------------------------------

\subsection{Coeffect meta-language}

\todo{Give some example that uses the coeffects metalanguage style. This will probably be 
meta-programming with open expressions (similarly to what Pfenning and Nanevski do with 
contextual modal type theory).}

%---------------------------------------------------------------------------------------------------

\section{Why context matters}

We claimed earlier that context-dependent comptuations are becoming increasingly common and our
work is focused on annotating the (free-variable) context with additional information.
The importance of context can be also demonstrated by looking at a technology that focuses 
solely on making the context richer -- F\# type providers.

\todo{Say more about type providers -- they extend the context $\Gamma$ so that it can be 
lazily loaded and it can be huge. Potentially, it could be also annotated with additional
meta-data...}

%===================================================================================================

\section{Contributions}

\todo{We present three calculi that model common notions of context-dependence and can be used
as basis for developing context-aware programming langauges with static type systems.}

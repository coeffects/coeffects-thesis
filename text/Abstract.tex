%!TEX root = ../main.tex
\pdfbookmark[1]{Abstract}{Abstract} % Bookmark name visible in a PDF viewer

\begingroup
\let\clearpage\relax
\let\cleardoublepage\relax
\let\cleardoublepage\relax

\chapter*{Abstract} % Abstract name
The development of programming languages needs to reflect important changes in the way
programs execute. In recent years, this has included \eg~the development of parallel programming 
models (in reaction to the multi-core revolution) or improvements in data access technologies. 
This thesis is a response to another such revolution -- the diversification of devices and 
systems where programs run. 

The key point made by this thesis is the realization that execution environment or
\emph{context} is fundamental for writing modern applications and that programming 
languages should provide abstractions for programming with context and verifying how 
it is accessed. 

We identify a number of program properties that were not connected before, but model some notion 
of context. Our examples include tracking different execution platforms (and their versions) 
in cross-platform development, resources available in different execution environments (\eg~GPS
sensor on a phone and database on the server), but also more traditional notions such as 
variable usage (\eg~in liveness analaysis and linear logics) or past values in 
stream-based data-flow programming.

Our first contribution is the discovery of the connection between the above examples and
their novel presentation in the form of calculi (\emph{coeffect systems}). The presented type 
systems and formal semantics highlight the relationship between different notions of context. 
Our second and third contributions are two unified coeffect calculi that capture commonalities 
in the presented examples. In particular, our \emph{flat coeffect calculus} models languages 
with contextual properties of the execution environment and our \emph{structural coeffect 
calculus} models languages where the contextual properties are attached to the variable usage.

Although the focus of this thesis is on the syntactic properties of the presented 
systems, we also discuss their category-theoretical motivation. We introduce the notion of
an \emph{indexed} comonad (based on dualisation of the well-known monad structure) and show 
how they provide semantics of the two coeffect calculi. 

\endgroup			

\vfill
%!TEX root = ../main.tex

\chapter{Conclusions} 
\label{ch:conclusions} 

%---------------------------------------------------------------------------------------------------

How to make this practical?

codo notation / computation expressions

embedding algebraic things


%---------------------------------------------------------------------------------------------------

Both flat coeffect calculus and structural coeffect calculus (presented in the past two chapters)
use indexed comonads to define the semantics of the langauge. In this section, we follow the 
meta-language style and embed indexed comonads into the language -- the type constructor
$\ctyp{r}{\alpha}$ becomes a first-class value and we add language constructs corresponding to
primitive operations of the indexed comonad.

%===================================================================================================

\newcommand{\munit}{\mathsf{e}}
\newcommand{\mseq}[0]{ \oplus }


\section{Introduction}

\section{Meta-language}
\label{sec:conclusions-meta} 

\begin{figure}
\begin{equation*}
\tyrule{var}
  {x : \alpha \in \Gamma}
  {\Gamma \vdash x : \alpha }
\end{equation*}
\begin{equation*}
\tyrule{app}
  {\Gamma \vdash e_1 : \alpha \rightarrow \beta &
   \Gamma \vdash e_2 : \alpha }
  {\Gamma \vdash e_1~e_2 : \beta}
\end{equation*}
\begin{equation*}
\tyrule{abs}
  {\Gamma, x:\alpha \vdash e : \beta}
  {\Gamma \vdash \lambda x.e : \alpha \rightarrow \beta }
\end{equation*}
\begin{equation*}
\tyrule{letbox}
  {\Gamma \vdash e_1 : \ctyp{r \mseq s}{\alpha} & \Gamma, x : \ctyp{r}{\alpha} \vdash e_2 : \beta }
  {\Gamma \vdash \kvd{let~box}~x=e_1~\kvd{in}~e_2 : \ctyp{s}{\beta}}
\end{equation*}
\begin{equation*}
\tyrule{eval}
  {\Gamma \vdash e : \ctyp{\munit}{\alpha}}
  {\Gamma \vdash !e : \alpha}
\end{equation*}
\begin{equation*}
\tyrule{sub}
  {\Gamma \vdash e : \ctyp{s}{\alpha} }
  {\Gamma \vdash e : \ctyp{r}{\alpha} }
\;(s\leq r)  
\end{equation*}

\caption{Type system for the coeffect meta-language cml}
\label{fig:cml-types}
\end{figure}

% ==================================================================================================

\section{Related work}
This chapter is closely related to Contextual Modal Type Theory (CMTT) of Nanevski et al. However
they develop their language using model logic as a basis, while we use categorical foundations 
as the basis - leading to a different system.

\section{Also related work}

This paper follows the approaches of effect systems \cite{effects-gifford,effects-talpin-et-al,monads-effects-marriage}
and categorical semantics based on monads and comonads \cite{monad-notions,comonads-notions}. Syntactically,
\emph{coeffects} differ from \emph{effects} in that they model systems where $\lambda$-abstraction 
may split contextual requirements between the declaration-site and call-site.

Our \emph{indexed (monoidal) comonads} (\S\ref{sec:comonads}) fill the gap between (non-indexed)
\emph{(monoidal) comonads} of Uustalu and Vene \cite{comonads-notions}
and indexed monads of Atkey~\cite{monads-parameterised-notions}, Wadler and Thiemann
\cite{monads-effects-marriage}. Interestingly, \emph{indexed} comonads are \emph{more
general} than comonads, capturing more notions of context-dependence (\S\ref{sec:motivation}).

% --------------------------------------------------------------------------------------------------

\vspace{-1em}
\paragraph{Comonads and modal logics.}

Bierman and de Paiva \cite{logic-intuitionistic-modal} model the
$\square$ modality of an intuitionistic S4 modal logic using monoidal
comonads, which links our calculus to modal logics.  This link can be
materialized in two ways.

Pfenning et al. and Nanevski et al.  derive term languages using the Curry-Howard
correspondence~\cite{logic-modal-reconstruction,logic-intuitionistic-modal,logic-cmtt},
building a \emph{metalanguage} (akin to Moggi's monadic metalanguage
\cite{monad-notions}) that includes $\square$ as a type
constructor. For example, in \cite{logic-modal-reconstruction}, the
modal type $\Box \tau$ represents closed terms.
In contrast, the \emph{semantic} approach uses monads or comonads
\emph{only} as a semantics.  This has been employed by Uustalu and
Vene and (again) Moggi \cite{monad-notions,comonads-notions}.  We
follow the semantic approach.

Nanevski et al. extend an S4 term language to a \emph{contextual}
modal type theory (CMTT)~\cite{logic-cmtt}.
The \emph{context} is a set of variables required by a computation, which
makes CMTT useful for meta-programming and staged computations. Our contextual types are
indexed by a coeffect algebra, which is more general and can capture
variable contexts, but also integers, two-point lattices, \emph{etc.}.

The work on CMTT suggests two extensions to coeffects. The first is
developing the logical foundations. We briefly considered special cases
of our system that permits local soundness in \S\ref{sec:syntactic} and
local completeness can be treated similarly. The second problem is 
developing the coeffects \emph{metalanguage}. The use of coeffect algebras
would provide an additional flexibility over CMTT, allowing a wider range 
of applications.

\section{Summary}